\section{Lebesgue Measure}
Up to now, we have not really had any interesting examples. Sorry.
Of course, it is examples which motivated all the abstract theory we have developed so far.
The most important and motivating example is Lebesgue measure on the line $\R$ and in Euclidean space $\R^n$.

We will look at another example, the Hausdorff measure(s), in a later section.

\subsection{Boxes and Cubes}

\begin{definition}[Boxes and Cubes]
    \label{def:boxes_and_cubes}
A \textit{closed box} in $\R^n$ is a set of the form
\[ R = [a_1,b_1]\times [a_2,b_2] \times \cdots \times [a_n,b_n] \]
for some real numbers $a_j\leq b_j$ for $j=1,\ldots,n$.
The \textit{volume} of such a box is defined to be
\[ \vol(R) := \prod_{j=1}^n (b_j - a_j). \]
An \textit{open box} is a set of the form
\[ (a_1,b_1)\times (a_2,b_2) \times \cdots \times (a_n,b_n) \]
for some real numbers $a_j < b_j$ for $j=1,\ldots,n$.
A \textit{half-open} box is a set of the form
\[ [a_1,b_1)\times [a_2,b_2) \times \cdots \times [a_n,b_n) \]
for some real numbers $a_j \leq b_j$ for $j=1,\ldots,n$.
The volume of an open or half-open box is defined in the same way as for closed boxes.

\vspace{2mm}
\noindent A \textit{cube} is a special case of a box where all the side lengths are equal, i.e. $b_j - a_j = b_i - a_i$ for all $1\leq i,j\leq n$.
If $Q\subset \R^n$ is a cube with side length $\ell > 0$, then $\vol(Q) = \ell^n$.
\end{definition}

We remark that we allow degenerate closed and half-open boxes, where $a_j = b_j$ for some $j$.
Degenerate open boxes are empty, so we do not allow them.
Also, in our definition, boxes and cubes have sides parallel to the coordinate axes.

\begin{definition}[Almost Disjoint]
    Say that a collection of boxes $\{ R_j \}_{j\in J}$ in $\R^n$ is \textit{almost disjoint} if the interiors of the boxes are pairwise disjoint.
\end{definition}

Intuitively, a collection of boxes is almost disjoint if they only overlap on their boundaries.

\begin{lemma}
    \label{lem:almost_disjoint_boxes}

    Let $R\sub \R^n$ be a closed box, and let $\{ R_j \}_{j=1}^N$ be a finite collection of almost disjoint closed boxes in $\R^n$
    such that \[ R = \bigcup_{j=1}^N R_j. \]
    Then
    \[ \vol\left( \bigcup_{j=1}^N R_j \right) = \sum_{j=1}^N \vol(R_j). \]
\end{lemma}

\begin{proof}
    For each $1\leq j \leq N$, let the box $R_j$ be given by
    \[ R_j = [a_{j,1},b_{j,1}]\times [a_{j,2},b_{j,2}] \times \cdots \times [a_{j,n},b_{j,n}] \]
    where $a_{j,i} \leq b_{j,i}$ for $i=1,\ldots,n$.
    Then for each index $1\leq i \leq n$, we collect the endpoints of the $i$-th intervals of all the boxes into a set, 
    \[ S_i = \{ a_{j,i}, b_{j,i} : 1\leq j \leq N \} \]
    which is a finite set of real numbers, and therefore can be written as 
    \[  S_i = \{ c_{i,1}, c_{i,2}, \ldots, c_{i,m_i} \} \]
    where $c_{i,1} < c_{i,2} < \cdots < c_{i,m_i}$.
    
    For each $1\leq i \leq n$ and each $1\leq k \leq m_i$ define the interval
    \[ I_{i,k} := [c_{i,k-1}, c_{i,k}]. \]
    This gives a collection of closed boxes $\{ \tilde{R}_{k_1,k_2,\ldots,k_n} \}_{1\leq k_i \leq m_i, 1\leq i \leq n}$ where
    \[ \tilde{R}_{k_1,k_2,\ldots,k_n} := I_{1,k_1} \times I_{2,k_2} \times \cdots \times I_{n,k_n}. \]
    We note that there are at most $m_1 m_2 \cdots m_n$ such boxes, and that these boxes are almost disjoint because the interior of a box $\tilde{R}_{k_1,k_2,\ldots,k_n}$ is a product of open intervals
    $I_{1,k_1}^\circ \times I_{2,k_2}^\circ \times \cdots \times I_{n,k_n}^\circ$, and two such products of open intervals are either equal or disjoint by definition of endpoints $c_{i,j}$.

    Define index sets 
    \[ \mathcal{J} := \left\{ (k_1,k_2,\ldots,k_n) : \tilde{R}_{k_1,k_2,\ldots,k_n} \subseteq R \right\} \quad\text{and}\quad \mathcal{J}_j := \left\{ (k_1,k_2,\ldots,k_n) : \tilde{R}_{k_1,k_2,\ldots,k_n} \subseteq R_j \right\} \]
    for $1\leq j \leq N$.
    Then for each $1\leq j \leq N$ we claim that we have
    \[ R_j = \bigcup_{(k_1,k_2,\ldots,k_n) \in \mathcal{J}_j} \tilde{R}_{k_1,k_2,\ldots,k_n}. \]
    Fix $1\leq j \leq N$.
    The inclusion $\supseteq$ holds by definition of $\mathcal{J}_j$. To see that the inclusion $\subseteq$ holds, let $x\in R_j$; then 
    \[ x = (x_1,x_2,\ldots,x_n) \in [a_{j,1},b_{j,1}]\times [a_{j,2},b_{j,2}] \times \cdots \times [a_{j,n},b_{j,n}]. \]
    Since $a_{j,i}, b_{j,i} \in S_i$ for each $1\leq i \leq n$, there exist indices $1\leq k_i \leq m_i$ such that $x_i \in I_{i,k_i}$ for each $1\leq i \leq n$.
    Thus \[x\in \tilde{R}_{k_1,k_2,\ldots,k_n},\] and by definition of $\mathcal{J}_j$ we have $(k_1,k_2,\ldots,k_n) \in \mathcal{J}_j$.
    Therefore $x\in \bigcup_{(k_1,k_2,\ldots,k_n) \in \mathcal{J}_j} \tilde{R}_{k_1,k_2,\ldots,k_n}$. 
    Since $x\in R_j$ was arbitrary, we have proven the reverse inclusion $\supseteq$ and shown the claim.
    As a consequence of this claim and the fact that $R= \bigcup_{j=1}^N R_j$, we also have
    \[ \mathcal{J} = \bigcup_{j=1}^N \mathcal{J}_j. \tag{$*$} \]
    The fact that the boxes $\tilde{R}_{k_1,k_2,\ldots,k_n}$ are almost disjoint also implies that the sets $\{ \mathcal{J}_j \}_{j=1}^N$ are pairwise disjoint.

    We also claim that 
    \[ \vol(R_j) = \sum_{(k_1,k_2,\ldots,k_n) \in \mathcal{J}_j} \vol(\tilde{R}_{k_1,k_2,\ldots,k_n}) \tag{$**$} \]
    for each $1\leq j \leq N$.

    To see this, fix $1\leq j \leq N$. Then for each $1\leq i \leq n$ we have
    \[ a_{j,i} = c_{i,l_i} < c_{i,l_i+r_i} = b_{j,i} \]
    for some $1\leq l_i < l_i+r_i \leq m_i$ and some $r_i\geq 1$; in particular, the interval $[a_{j,i},b_{j,i}]$ is partitioned by the intervals $I_{i,l_i}, I_{i,l_i+1}, \ldots, I_{i,l_i+r_i}$.
    As a telescoping sum, we have
    \[ b_{j,i} - a_{j,i} = c_{i,l_i+r_i} - c_{i,l_i} = \sum_{s=l_i}^{l_i+r_i-1} (c_{i,s+1} - c_{i,s}) \]
    for each $1\leq i \leq n$. Thus, we obtain
    \begin{align*}
        \vol(R_j) &= \prod_{i=1}^n (b_{j,i} - a_{j,i}) = \prod_{i=1}^n \sum_{s=l_i}^{l_i+r_i-1} (c_{i,s+1} - c_{i,s}) \\
            &= \sum_{s_1=l_1}^{l_1+r_1-1} \sum_{s_2=l_2}^{l_2+r_2-1} \cdots \sum_{s_n=l_n}^{l_n+r_n-1} \prod_{i=1}^n (c_{i,s_i+1} - c_{i,s_i}) \\
            &= \sum_{s_1=l_1}^{l_1+r_1-1} \sum_{s_2=l_2}^{l_2+r_2-1} \cdots \sum_{s_n=l_n}^{l_n+r_n-1} \vol(\tilde{R}_{s_1,s_2,\ldots,s_n}) \\
            &= \sum_{(k_1,k_2,\ldots,k_n) \in \mathcal{J}_j} \vol(\tilde{R}_{k_1,k_2,\ldots,k_n}).
    \end{align*}
    where in the final equality we have used the fact that $\tilde{R}_{s_1,s_2,\ldots,s_n} \subseteq R_j$ if and only if $l_i \leq s_i \leq l_i+r_i-1$ for each $1\leq i \leq n$, which is equivalent to $(s_1,s_2,\ldots,s_n) \in \mathcal{J}_j$.
    We have thus shown the claim.

    We make a third and final claim that 
    \[ \vol\left(R \right) = \sum_{(k_1,k_2,\ldots,k_n) \in \mathcal{J}} \vol(\tilde{R}_{k_1,k_2,\ldots,k_n}). \tag{$\dagger$} \]
    To see this, we write $R$ as a box of \[ R = [a_1,b_1]\times [a_2,b_2]\times \cdots \times [a_n,b_n]. \]
    Then for each $1\leq i \leq n$ we have $a_i, b_i \in S_i$, and in fact it must be the case that $a_i = c_{i,1}$ and $b_i = c_{i,m_i}$.
    Thus, for each $1\leq i \leq n$ we have
    \[ b_i - a_i = c_{i,m_i} - c_{i,1} = \sum_{s=1}^{m_i-1} (c_{i,s+1} - c_{i,s}) \]
    and therefore
    \begin{align*}
        \vol(R) &= \prod_{i=1}^n (b_i - a_i) = \prod_{i=1}^n \sum_{s=1}^{m_i-1} (c_{i,s+1} - c_{i,s}) \\
            &= \sum_{s_1=1}^{m_1-1} \sum_{s_2=1}^{m_2-1} \cdots \sum_{s_n=1}^{m_n-1} \prod_{i=1}^n (c_{i,s_i+1} - c_{i,s_i}) \\
            &= \sum_{s_1=1}^{m_1-1} \sum_{s_2=1}^{m_2-1} \cdots \sum_{s_n=1}^{m_n-1} \vol(\tilde{R}_{s_1,s_2,\ldots,s_n}) \\
            &= \sum_{(k_1,k_2,\ldots,k_n) \in \mathcal{J}} \vol(\tilde{R}_{k_1,k_2,\ldots,k_n}).
    \end{align*}
    This proves the claim.

    Now we see that putting $(*)$, $(**)$, and $(\dagger)$ together yields
    \begin{align*}
        \vol(R) &= \sum_{(k_1,k_2,\ldots,k_n) \in \mathcal{J}} \vol(\tilde{R}_{k_1,k_2,\ldots,k_n}) \\
            &= \sum_{j=1}^N \sum_{(k_1,k_2,\ldots,k_n) \in \mathcal{J}_j} \vol(\tilde{R}_{k_1,k_2,\ldots,k_n}) \\
            &= \sum_{j=1}^N \vol(R_j).
    \end{align*}
    This completes the proof.
\end{proof}

\begin{lemma}
    \label{lem:covering_box_by_boxes}
    Let $R\sub \R^n$ be a closed box, and let $\{ R_j \}_{j=1}^N$ be a finite collection of closed boxes such that
    \[ R\sub \bigcup_{j=1}^N R_j. \] 
    Then
    \[ \vol(R) \leq \sum_{j=1}^N \vol(R_j). \]
\end{lemma}

    The idea is similar to the last proof.

\begin{proof}
    For each $1\leq j \leq N$, let the box $R_j$ be given by
    \[ R_j = [a_{j,1},b_{j,1}]\times [a_{j,2},b_{j,2}] \times \cdots \times [a_{j,n},b_{j,n}] \]
    where $a_{j,i} \leq b_{j,i}$ for $i=1,\ldots,n$.
    Then for each index $1\leq i \leq n$, we collect the endpoints of the $i$-th intervals of all the boxes into a set, 
    \[ S_i = \{ a_{j,i}, b_{j,i} : 1\leq j \leq N \} \]
    which is a finite set of real numbers, and therefore can be written as 
    \[  S_i = \{ c_{i,1}, c_{i,2}, \ldots, c_{i,m_i} \} \]
    where $c_{i,1} < c_{i,2} < \cdots < c_{i,m_i}$.
    
    For each $1\leq i \leq n$ and each $1\leq k \leq m_i$ define the interval
    \[ I_{i,k} := [c_{i,k-1}, c_{i,k}]. \]
    This gives a collection of closed boxes $\{ \tilde{R}_{k_1,k_2,\ldots,k_n} \}_{1\leq k_i \leq m_i, 1\leq i \leq n}$ where
    \[ \tilde{R}_{k_1,k_2,\ldots,k_n} := I_{1,k_1} \times I_{2,k_2} \times \cdots \times I_{n,k_n}. \]
    We note that there are at most $m_1 m_2 \cdots m_n$ such boxes, and that these boxes are almost disjoint because the interior of a box $\tilde{R}_{k_1,k_2,\ldots,k_n}$ is a product of open intervals
    $I_{1,k_1}^\circ \times I_{2,k_2}^\circ \times \cdots \times I_{n,k_n}^\circ$, and two such products of open intervals are either equal or disjoint by definition of endpoints $c_{i,j}$.

    Define index sets 
    \[ \mathcal{J} := \left\{ (k_1,k_2,\ldots,k_n) : \tilde{R}_{k_1,k_2,\ldots,k_n} \subseteq R \right\} \quad\text{and}\quad \mathcal{J}_j := \left\{ (k_1,k_2,\ldots,k_n) : \tilde{R}_{k_1,k_2,\ldots,k_n} \subseteq R_j \right\} \]
    for $1\leq j \leq N$.
    Then for each $1\leq j \leq N$ we claim that we have
    \[ R_j = \bigcup_{(k_1,k_2,\ldots,k_n) \in \mathcal{J}_j} \tilde{R}_{k_1,k_2,\ldots,k_n} \]
    and similarly \[ R = \bigcup_{(k_1,k_2,\ldots,k_n) \in \mathcal{J}} \tilde{R}_{k_1,k_2,\ldots,k_n}. \]
    The proof of this claim is identical to the proof of the same claim in the previous lemma.
    As a consequence of this claim and the fact that $R\sub \bigcup_{j=1}^N R_j$, we also have
    \[ \mathcal{J} \sub \bigcup_{j=1}^N \mathcal{J}_j. \tag{$*$} \]
    The main difference from the previous lemma is that the sets $\{ \mathcal{J}_j \}_{j=1}^N$ are not necessarily pairwise disjoint.
    Because the boxes $\{R_j\}_{j=1}^N$ are not necessarily almost disjoint, the a single box $\tilde{R}_{k_1,k_2,\ldots,k_n}$ may be contained in more than one of the boxes $R_j$.

    Now we see that 
    \begin{align*}
        \vol(R) &= \sum_{(k_1,k_2,\ldots,k_n) \in \mathcal{J}} \vol(\tilde{R}_{k_1,k_2,\ldots,k_n}) \\
            &\leq \sum_{j=1}^N \sum_{(k_1,k_2,\ldots,k_n) \in \mathcal{J}_j} \vol(\tilde{R}_{k_1,k_2,\ldots,k_n}) \\
            &= \sum_{j=1}^N \vol(R_j)
    \end{align*}
    where we have used the previous lemma and our claim in the first equality, the inclusion $(*)$ in the inequality, and the previous lemma and our claim again in the final equality. 
\end{proof}


\begin{exercise}
    \label{ex:open_set_covered_by_boxes}
    Let $U\sub \R^n$ be a nonempty open set.
    Show that there exists a countable collection of almost disjoint closed cubes $\{ Q_j \}_{j=1}^\infty$ such that
    \[ U = \bigcup_{j=1}^\infty Q_j. \]
\end{exercise}

\begin{proof}
    For this proof, we will use the dyadic cubes 
    \[ Q_{m,k} := \left\{ x = (x_1,\ldots,x_n)\in\R^n : 2^k m_j \leq x_j \leq 2^k (m_j+1) \text{ for } j=1,2,\ldots,n \right\} , \quad m\in \Z^n, k\in\Z. \]
    We say that $Q_{m,k}$ is a dyadic cube of generation $k$.
    It has edge length $2^k$ and volume $2^{kn}$, and corner point $2^k m\in 2^k \Z^n$.
    Notice that $Q_{m,0}$ is a half-open cube of edge length $1$ with corner point $m\in \Z^n$.
    Then $Q_{m,1}$ is a half-open cube of edge length $2$ with corner point $2m\in 2\Z^n$, and so on.
    Similarly, $Q_{m,-1}$ is a half-open cube of edge length $\frac{1}{2}$ with corner point $\frac{m}{2}\in \frac{1}{2}\Z^n$, and so on.
    The dyadic cubes of generation $k$ form are almost disjoint and their union is $\R^n$.

    We define the desired collection of cubes $\{ Q_j \}_{j=1}^\infty$ as follows.

    At the first step, let $\mathcal{Q}_{-1}$ be the set of all dyadic cubes of generation $-1$, which have edge length $\frac{1}{2}$.
    There are countably many such cubes, and we sort them into three sets based on their relationship to $U$ --- we define
    \begin{align*}
        \text{green}_1 &= \{ Q_{m,-1} : m\in \Z^n \text{ and } Q_{m,-1} \subset U \} \\
        \text{red}_1 &= \{ Q_{m,-1} : m\in \Z^n \text{ and } Q_{m,-1} \cap U = \emptyset \} \\
        \text{yellow}_1 &= \{ Q_{m,-1} : m\in \Z^n \text{ and } Q_{m,-1} \cap U \neq \emptyset, Q_{m,-1} \not\subset U \}.
    \end{align*}
    The green cubes are those which are completely contained in $U$, the red cubes are those which do not intersect $U$ at all and are contained in the complement of $U$, and the yellow cubes are those which intersect $U$ but are not completely contained in $U$.
    The set of green cubes $\text{green}_1$ and the set of yellow cubes $\text{yellow}_1$ cannot both be empty, since $U$ is nonempty and open.
    Let $\mathcal{G}_1 := \text{green}_1$ be the set of green cubes at step $1$.

    At the second step, we look at the yellow cubes from the first step.
    Each yellow cube $Q_{m,-1}$ evolves into $2^n$ dyadic cubes of generation $-2$, each with edge length $\frac{1}{4}$.
    We sort these new cubes into three sets based on their relationship to $U$ --- we define
    \begin{align*}
        \text{green}_2 &= \{ Q_{m,-2} : Q_{m,-2} \subset U \text{ and } Q_{m,-1} \subset \text{yellow}_1 \} \\
        \text{red}_2 &= \{ Q_{m,-2} : Q_{m,-2} \cap U = \emptyset \text{ and } Q_{m,-1} \subset \text{yellow}_1 \} \\
        \text{yellow}_2 &= \{ Q_{m,-2} : Q_{m,-2} \cap U \neq \emptyset, Q_{m,-2} \not\subset U \text{ and } Q_{m,-1} \subset \text{yellow}_1 \}.
    \end{align*}
    The set of green cubes $\text{green}_2$ and the set of yellow cubes $\text{yellow}_2$ cannot both be empty, since each yellow cube from the first step intersects $U$ and $U$ is open.
    Let $\mathcal{G}_2 := \mathcal{G}_1 \cup \text{green}_2$ be the set of green cubes at step $2$.

    We continue in this manner, where at step $k$ we look at the yellow cubes from step $k-1$, subdivide each of them into $2^n$ dyadic cubes of generation $-k$, and sort these new cubes into three sets based on their relationship to $U$.
    We define the sets $\text{green}_k$, $\text{red}_k$, and $\text{yellow}_k$ in the obvious way, and let $\mathcal{G}_k := \mathcal{G}_{k-1} \cup \text{green}_k$ be the set of green cubes at step $k$.
    This process results in a countable collection of green cubes $\mathcal{G} := \bigcup_{k=1}^\infty \mathcal{G}_k$.
    We claim that $\mathcal{G}$ is the desired collection of cubes, i.e
    \[ U = \bigcup_{Q\in \mathcal{G}} Q. \]

    To see this, first note that every cube in $\mathcal{G}$ is contained in $U$ by construction, so we have the inclusion $\supseteq$.
    To see the inclusion $\subseteq$, let $x\in U$.
    Then there exists an $N>0$ and a cube $Q_{m,-N}$ of generation $-N$ such that $x\in Q_{m,-N} \subset U$.
    Either this cube $Q_{m,-N}$ is in $\mathcal{G}$, or it is contained in a cube which is in $\mathcal{G}$.
    Therefore $x\in \bigcup_{Q\in \mathcal{G}} Q$, and since $x\in U$ was arbitrary, we have shown the inclusion $\sub$.

    Finally, we note that the cubes in $\mathcal{G}$ are almost disjoint, since they are dyadic cubes of various generations.
    Since $U$ was an arbitrary nonempty open set, this completes the proof.
\end{proof}

\begin{corollary}
    \label{cor:borel_generated_by_boxes}
    The Borel $\sigma$-algebra on $\R^n$ is generated by the collection of closed boxes in $\R^n$.
    Similarly, it is also generated by the collection of open boxes in $\R^n$, or by the collection of half-open boxes in $\R^n$.
\end{corollary}

\begin{proof}
    Let $\mathcal{A}$ be the smallest $\sigma$-algebra containing all closed (open, half-open) boxes in $\R^n$.
    By the previous exercise, every open set in $\R^n$ is a countable union of closed (open, half-open) boxes, so every open set is in $\mathcal{A}$.
    Since the Borel $\sigma$-algebra is the smallest $\sigma$-algebra containing all open sets, we have that the Borel $\sigma$-algebra is contained
    in $\mathcal{A}$.

    Conversely, the collection of closed (open, half-open) boxes is a subset of the Borel $\sigma$-algebra and is a $\pi$-system which generates $\mathcal{A}$, so we know that $\mathcal{A}$ is contained in the Borel $\sigma$-algebra.
\end{proof}

\newpage

\subsection{Lebesgue Measure on $\R^n$}

With the theory we have put in place, this should be pretty straightforward.


We want to define an outer measure $\mu$ on $\R^n$ which assigns to each box its volume.
This measure should also satisfy the following properties, to capture our intuition of volume:
\begin{itemize}
    \item (Translation Invariance) If $A\sub \R^n$ and $x\in \R^n$, then $\mu(A+x) = \mu(A)$, where $A+x = \{ a+x : a\in A \}$.
    \item (Homogeneity) If $A\sub \R^n$ and $t>0$, then $\mu(tA) = t^n\mu(A)$, where $tA = \{ ta : a\in A \}$.
    \item (Disjoint Additivity) If $A\sub \R^n$ and $B\sub \R^n$ are disjoint, then $\mu(A\cup B) = \mu(A) + \mu(B)$.
\end{itemize}
We will see that these properties are all satisfied by $\mu$-measurable sets with respect to the Lebesgue measure $\mu$ we construct.

\begin{definition}[Lebesgue Outer Measure]
    \label{def:lebesgue_outer_measure}
    The \emph{Lebesgue outer measure} $\mu$ on $\R^n$ is defined for each $E\sub \R^n$ by
    \[ \mu(E) := \inf \left\{ \sum_{j=1}^\infty \vol(Q_j) : E\sub \bigcup_{j=1}^\infty Q_j, \text{ each } Q_j \text{ is a closed cube} \right\} \]
    where $\vol(Q_j)$ is the volume of the cube $Q_j$.    
\end{definition}

That is, to define the Lebesgue outer measure of a set $E\sub \R^n$, we cover $E$ by a countable collection of closed cubes $\{ Q_j \}_{j=1}^\infty$ and take the infimum of the sums of the volumes of these cubes over all such covers.
We see that for each $E\sub \R^n$, the set over which we are taking the infimum in the definition of $\mu(E)$ is nonempty, 
so $\mu(E)$ is well-defined and takes values in $[0,\infty]$.

One can instead use open, half-open, or closed boxes instead of closed cubes in the definition, and the resulting outer measure will be the same.
One can even use dyadic cubes instead of arbitrary cubes or boxes.
The theory is the same in all these cases.
One can also cover by countably many open or closed balls instead of cubes or boxes, but it is nontrivial to show this gives the same theory (but it does).

What is crucial though is that we allow countably many cubes or boxes in the cover.
Allowing only finitely many cubes or boxes would give the theory of \emph{Jordan content}, instead of Lebesgue measure.

\vspace{3mm}

Let us begin by showing that the Lebesgue outer measure gives the correct value on boxes.

\begin{lemma}
    \label{lem:lebesgue_outer_measure_on_boxes}
    Let $R\sub \R^n$ be a box which is either closed, open, or half-open.
    Then $\mu(R) = \vol(R)$.
\end{lemma}

\begin{proof}

    \textit{Step 1}: We will first show that $\mu(Q) = \vol(Q)$ for every closed cube $Q\sub \R^n$.
    \vspace{2mm}
    
    Let $Q\sub \R^n$ be a cube of the form
    \[ Q = [a_1,b_1]\times [a_2,b_2] \times \cdots \times [a_n,b_n]. \]
    To see that $\mu(Q) = \vol(Q)$, first note that $\mu(Q) \leq \vol(Q)$ since $Q$ can be covered by itself.
    This inequality shows that equality holds if $\vol(Q) = 0$, so we may assume that $\vol(Q) > 0$.
    Let $\{ Q_j \}_{j=1}^\infty$ be a cover of $Q$ by closed cubes; let $\epsilon > 0$ be given.
    For each $j\geq 1$ we choose an open cube $S_{j,\epsilon}$ containing $Q_j$ such that $\vol(S_{j,\epsilon}) \leq (1+\epsilon)\vol(Q_j)$.
    Then $\{ S_{j,\epsilon} \}_{j=1}^\infty$ is an open cover of the compact set $Q$, so we can choose a finite subcover $\{ S_{j,\epsilon} \}_{j=1}^N$ of $Q$.
    Taking the closure of these cubes, we obtain a finite cover of $Q$ by closed cubes $\{ \overline{S_{j,\epsilon}} \}_{j=1}^N$.
    Then by \ref{lem:covering_box_by_boxes} we have
    \[ \vol(Q) \leq \sum_{j=1}^N \vol(\overline{S_{j,\epsilon}}) = \sum_{j=1}^N \vol(S_{j,\epsilon}) \leq (1+\epsilon) \sum_{j=1}^N \vol(Q_j). \]
    Since $\epsilon > 0$ was arbitrary, we have
    \[ \vol(Q) \leq \sum_{j=1}^\infty \vol(Q_j). \]
    Taking the infimum over all such covers $\{ Q_j \}_{j=1}^\infty$ of $Q$, we obtain $\vol(Q) \leq \mu(Q)$.
    Thus we have shown $\mu(Q) = \vol(Q)$ for every closed cube $Q\sub \R^n$.

    \vspace{2mm}
    \textit{Step 2}: We will now extend this to all boxes.
    \vspace{2mm}

    Now let $R\sub \R^n$ be a box which is either closed, open, or half-open.
    If $\vol(R) = 0$, then $\mu(R) = 0$ as $R$ can be covered by a degenerate cube of volume $0$.
    Thus we may assume that $\vol(R) > 0$.
    For convenience, we let
    \[ \overline{R} = [a_1,b_1]\times [a_2,b_2] \times \cdots \times [a_n,b_n] \]
    where $a_j< b_j$ for $j=1,\ldots,n$.

    Suppose that $\{Q_j\}_{j=1}^\infty$ is a cover of $R$ by closed cubes.
    Then for each $\varepsilon > 0$ and each $j\geq 1$ we can choose an open cube $S_{j,\epsilon}$ containing $Q_j$ such that $\vol(S_{j,\epsilon}) \leq (1+\epsilon)\vol(Q_j)$.
    Then $\{ S_{j,\epsilon} \}_{j=1}^\infty$ is an open cover of the compact set $\overline{R}$, so we can choose a finite subcover $\{ S_{j,\epsilon} \}_{j=1}^N$ of $\overline{R}$.
    Taking the closure of these cubes, we obtain a finite cover of $\overline{R}$ by closed cubes $\{ \overline{S_{j,\epsilon}} \}_{j=1}^N$.
    Then by \ref{lem:covering_box_by_boxes} we have
    \[ \vol(\overline{R}) \leq \sum_{j=1}^N \vol(\overline{S_{j,\epsilon}}) = \sum_{j=1}^N \vol(S_{j,\epsilon}) \leq (1+\epsilon) \sum_{j=1}^N \vol(Q_j) = (1+\epsilon) \sum_{j=1}^\infty \vol(Q_j). \]
    Since $\epsilon > 0$ was arbitrary, we have
    \[ \vol(\overline{R}) \leq \sum_{j=1}^\infty \vol(Q_j). \]
    Taking the infimum over all such covers $\{ Q_j \}_{j=1}^\infty$ of $R$, we obtain $\vol(\overline{R}) \leq \mu(R)$.
    Since $\vol(\overline{R}) = \vol(R)$, we have shown that $\vol(R) \leq \mu(R)$.

    To see the reverse inequality, fix $k\geq 1$; we let $\mathcal{Q}_k$ be given by
    \[ \mathcal{Q}_k = \left\{ [m_1/k, (m_1+1)/k] \times [m_2/k, (m_2+1)/k] \times \cdots \times [m_n/k, (m_n+1)/k] : m_i \in \mathbb{Z} \right\} \]
    which is a collection of almost disjoint closed cubes of side length $1/k$ that cover $\R^n$.
    We let $\mathcal{Q}_k^{(1)}$ be the subcollection of cubes in $\mathcal{Q}_k$ which are entirely contained in $R$, and we let $\mathcal{Q}_k^{(2)}$ be the subcollection of cubes in $\mathcal{Q}_k$ which intersect $R$ but are not entirely contained in $R$.
    Then both $\mathcal{Q}_k^{(1)}$ and $\mathcal{Q}_k^{(2)}$ are finite collections of cubes since $R$ is bounded, and we have \[R \sub\bigcup_{Q\in \mathcal{Q}_k^{(1)} \cup \mathcal{Q}_k^{(2)}} Q. \]
    See that for $1\leq i \leq n$, the number of intervals of length $1/k$ which fit inside $[a_i,b_i]$ is at most $N_i = \lfloor k(b_i - a_i) \rfloor$; hence the number of cubes in $\mathcal{Q}_k^{(1)}$ is at most $N_1 N_2 \cdots N_n$.
    Thus we have
    \[ \sum_{Q\in \mathcal{Q} } \vol(Q) \leq \left( \prod_{i=1}^n N_i \right) \frac{1}{k^n} \leq \prod_{i=1}^n \frac{\lfloor k(b_i - a_i) \rfloor}{k^n} \leq \prod_{i=1}^n (b_i - a_i) = \vol(R). \tag{$\star$}\]

    \vspace{2mm}
    We claim that the number of cubes in $\mathcal{Q}_k^{(2)}$ is bounded above by $Ck^{n-1}$ for some constant $C$ depending only on $R$ and $n$.
    \vspace{2mm}

    For a cube $Q\in \mathcal{Q}_k^{(2)}$, see that $Q$ intersects the boundary of $R$.
    Since each cube $Q\in \mathcal{Q}_k^{(2)}$ has side length $1/k$, we see that there exists $1\leq i \leq n$ such that the $i$-th closed interval $[m_i/k, (m_i+1)/k]$ defining $Q$ intersects $[a_i,b_i]$ but is not contained in $[a_i,b_i]$.
    Thus either $m_i/k < a_i < (m_i+1)/k$ or $m_i/k < b_i < (m_i+1)/k$.
    In either case, there are at most $2$ choices for $m_i$.

    Say $Q\in \mathcal{Q}$$_k^{(2)}$ is defined by the intervals $[m_j/k, (m_j+1)/k]$ for $j=1,\ldots,n$, 
    and that there is an $i$ such that $[m_i/k, (m_i+1)/k]$ intersects $[a_i,b_i]$ but is not contained in $[a_i,b_i]$.
    Then for each $j\neq i$, we have $a_j \leq (m_j+1)/k$ and $b_j \geq m_j/k$, so
    \[ m_j \in \{ \lceil ka_j \rceil - 1, \lceil ka_j \rceil, \ldots, \lfloor kb_j \rfloor \}. \]
    Thus for each $1\leq i\leq n$, there are at most $2$ choices for the integer $m_i$, and for each $j\neq i$ there are at most $\lfloor kb_j \rfloor - \lceil ka_j \rceil + 2 $ choices for the integer $m_j$.

    Thus the number of cubes in $\mathcal{Q}_k^{(2)}$ is at most
    \begin{align*}
         \sum_{i=1}^n \left(2 \,\prod_{j\neq i} \left( \lfloor kb_j \rfloor - \lceil ka_j \rceil + 2 \right) \right) &\leq 2\, \sum_{i=1}^n \prod_{j\neq i} (k(b_j - a_j) + 2) \\
            &\leq 2k^{n-1} \sum_{i=1}^n \prod_{j\neq i} ((b_j - a_j) + \frac{2}{k} ) \\ 
            &\leq 2k^{n-1} \sum_{i=1}^n \prod_{j\neq i} (b_j - a_j + 2) \\
            &=: C k^{n-1}
    \end{align*}
    which proves the claim.

    As a result, it follows that 
    \[ \sum_{Q\in \mathcal{Q}_k^{(2)}} \vol(Q) \leq C k^{n-1} \cdot \frac{1}{k^n} = \frac{C}{k}. \tag{$\star\star$} \]

    \vspace{2mm}

    Using ($\star$) and ($\star\star$), we have
    \[ \sum_{Q\in \mathcal{Q}_k^{(1)} \cup \mathcal{Q}_k^{(2)}} \vol(Q) \leq \vol(R) + \frac{C}{k}. \]
    Since $k\geq 1$ was arbitrary, we have shown that for every error $\epsilon > 0$, there exists a cover of $R$ by closed cubes whose total volume is at most $\vol(R) + \epsilon$.
    Taking the infimum over all such covers of $R$, we obtain $\mu(R) \leq \vol(R)$, finally proving that $\mu(R) = \vol(R)$.
\end{proof}

\begin{theorem}
    \label{thm:lebesgue_outer_measure}
    The Lebesgue outer measure $\mu$ is a metric outer measure on $\R^n$ which is finite on compact sets.
\end{theorem}

\begin{proof}
    \textit{Step 1}: 
    We will verify the three properties of an outer measure.
    \vspace{2mm}

    First see that $\mu(\emptyset) = 0$ because the empty set can be covered by a degenerate cube of volume $0$.
    Next, if $E\sub F\sub \R^n$, then any cover of $F$ by closed cubes is also a cover of $E$ by closed cubes, so
    \[ \mu(E) \leq \mu(F). \]
    Finally, if $E = \bigcup_{j=1}^\infty E_j$, then for $\epsilon > 0$ and
    for each $j$ we can choose a cover of $E_j$ by closed cubes $\{ Q_{j,k} \}_{k=1}^\infty$ such that
    \[ \sum_{k=1}^\infty \vol(Q_{j,k}) \leq \mu(E_j) + \frac{\epsilon}{2^j} \]
    Then $\{ Q_{j,k} : j,k\in \Z^+ \}$ is a cover of $E$ by closed cubes, so
    \begin{align*}
        \mu(E) &\leq \sum_{j=1}^\infty \sum_{k=1}^\infty \vol(Q_{j,k}) \\
            &\leq \sum_{j=1}^\infty \left( \mu(E_j) + \frac{\epsilon}{2^j} \right) \\
            &= \sum_{j=1}^\infty \mu(E_j) + \epsilon.
    \end{align*}
    Since $\epsilon > 0$ was arbitrary, we have
    \[ \mu(E) \leq \sum_{j=1}^\infty \mu(E_j). \]
    Thus $\mu$ is an outer measure on $\R^n$.

    \vspace{2mm}
    \textit{Step 2}: We will show that $\mu$ is finite on compact sets.
    \vspace{2mm}

    Now let $K\sub \R^n$ be compact.
    Then $K$ is closed and bounded by the Heine-Borel theorem, so $K$ is contained in some closed cube $Q$.
    Since $\mu(Q) \leq \vol(Q) < \infty$, we have $\mu(K) \leq \mu(Q) < \infty$.
    Thus $\mu$ is finite on compact sets.

    \vspace{2mm}
    \textit{Step 3}: We will show that $\mu$ is a metric outer measure.
    That is, if $E_1, E_2\sub \R^n$ are such that $\dist(E_1,E_2) > 0$, then
    \[ \mu(E_1\cup E_2) = \mu(E_1) + \mu(E_2). \]
    \vspace{2mm}

    Let $E_1, E_2\sub \R^n$ be such that $\dist(E_1,E_2) > 0$.
    Then by countable subadditivity of $\mu$, we have
    \[ \mu(E_1\cup E_2) \leq \mu(E_1) + \mu(E_2). \]
    To prove the reverse inequality, let $\delta$ be such that $0 < \delta < \dist(E_1,E_2)$. 
    Let $\epsilon > 0$ be arbitrary, and let $\{ Q_j \}_{j=1}^\infty$ be a cover of $E_1\cup E_2$ by closed cubes such that
    \[ \sum_{j=1}^\infty \vol(Q_j) < \mu(E_1\cup E_2) + \epsilon. \]
    After subdividing the cubes if necessary, we may assume that $\diam(Q_j) < \delta$ for each $j$.
    Then each cube $Q_j$ intersects at most one of the sets $E_1$ or $E_2$, since $\dist(E_1,E_2) > \delta$.
    Let 
    \[ J_i := \{ j \in \mathbb{N} : Q_j \cap E_i \neq \emptyset \} \]
    for $i=1,2$.
    Then $J_1 \cap J_2 = \emptyset$ and we have 
    \[ E_i \sub \bigcup_{j\in J_i} Q_j \]
    for $i=1,2$. By definition of the Lebesgue outer measure $\mu$, we have
    \begin{align*}
        \mu(E_1) + \mu(E_2) &\leq \sum_{j\in J_1} \vol(Q_j) + \sum_{j\in J_2} \vol(Q_j) \\
            &= \sum_{j\in J_1 \cup J_2} \vol(Q_j) \\
            &\leq \sum_{j=1}^\infty \vol(Q_j) \\
            &< \mu(E_1\cup E_2) + \epsilon.
    \end{align*}
    Since $\epsilon > 0$ was arbitrary, we have 
    \[ \mu(E_1) + \mu(E_2) \leq \mu(E_1\cup E_2) \]
    which proves that $\mu(E_1\cup E_2) = \mu(E_1) + \mu(E_2)$ as desired.

    Since $E_1, E_2\sub \R^n$ with $\dist(E_1,E_2) > 0$ were arbitrary, we have shown that $\mu$ is a metric outer measure on $\R^n$.
\end{proof}

\begin{corollary}[Unions of Almost Disjoint Boxes]
    \label{cor:unions_of_almost_disjoint_boxes}
    If $\{ Q_j \}_{j=1}^\infty$ is a countable collection of almost disjoint boxes in $\R^n$, then
    \[ \mu\left( \bigcup_{j=1}^\infty Q_j \right) = \sum_{j=1}^\infty \vol(Q_j). \]
\end{corollary}

\begin{proof}
    Let $\epsilon > 0$ be arbitrary.
    For each $j\geq 1$ we let $S_{j,\epsilon}$ be a closed box contained in $Q_j$ such that $\vol(Q_j) \leq \vol(S_{j,\epsilon}) + \epsilon/2^j$.
    Then for each $N\geq 1$ the collection of boxes $\{ S_{j,\epsilon} \}_{j=1}^N$ is disjoint, so that $\dist(S_{i,\epsilon}, S_{j,\epsilon}) > 0$ for $i\neq j$.
    Since $\mu$ is a metric outer measure by \ref{thm:lebesgue_outer_measure}, for each $N\geq 1$ we have
    \[ \mu\left( \bigcup_{j=1}^N S_{j,\epsilon} \right) = \sum_{j=1}^N \mu(S_{j,\epsilon}) = \sum_{j=1}^N \vol(S_{j,\epsilon}) \geq \sum_{j=1}^N \left(\vol(Q_j) - \frac{\epsilon}{2^j}\right) = \sum_{j=1}^N \vol(Q_j) - \epsilon \]
    where the second equality follows from Lemma \ref{lem:lebesgue_outer_measure_on_boxes}.
    Since $\bigcup_{j=1}^N S_{j,\epsilon} \sub \bigcup_{j=1}^\infty Q_j$ for each $N\geq 1$, monotonicity of $\mu$ gives
    \[ \mu\left( \bigcup_{j=1}^\infty Q_j \right) \geq \mu\left( \bigcup_{j=1}^N S_{j,\epsilon} \right) \geq \sum_{j=1}^N \vol(Q_j) - \epsilon, \]
    and taking the limit as $N\to\infty$ gives
    \[ \mu\left( \bigcup_{j=1}^\infty Q_j \right) \geq \sum_{j=1}^\infty \vol(Q_j) - \epsilon. \]
    Since $\epsilon > 0$ was arbitrary, we have shown that 
    \[ \mu\left( \bigcup_{j=1}^\infty Q_j \right) \geq \sum_{j=1}^\infty \vol(Q_j). \]

    The reverse inequality follows from the fact that $\{\overline{Q_j}\}_{j=1}^\infty$ is a cover of $\bigcup_{j=1}^\infty Q_j$ by closed boxes, so we have
    \[ \mu\left( \bigcup_{j=1}^\infty Q_j \right) \leq \sum_{j=1}^\infty \vol(\overline{Q_j}) = \sum_{j=1}^\infty \vol(Q_j). \]
\end{proof}

As another corollary of the fact that Lebesgue outer measure is a metric outer measure, we have the following.

\begin{corollary}[Borel Regularity of Lebesgue Measure]
    \label{cor:borel_regular_lebesgue_measure}
    The Lebesgue outer measure $\mu$ is a Borel regular outer measure on $\R^n$, and the Borel $\sigma$-algebra is contained in the $\sigma$-algebra of $\mu$-measurable sets.
    That is, every Borel set in $\R^n$ is Lebesgue measurable.
\end{corollary}

\begin{proof}
    Since $\mu$ is a metric outer measure by \ref{thm:lebesgue_outer_measure}, it is Borel regular by \ref{thm:caratheodory_criterion}.
\end{proof}

\begin{remark}
    Recall that the $\sigma$-algebra of $\mu$-measurable sets is denoted by $\mathcal{M}_\mu$, and sets in $\mathcal{M}_\mu$ are called $\mu$-measurable.
    In this case of Lebesgue outer measure $\mu$, sets in $\mathcal{M}_\mu$ are called \emph{Lebesgue measurable}.
\end{remark}

\begin{corollary}[Countable Disjoint Additivity]
    \label{cor:countable_disjoint_additivity_lebesgue}
    Let $\{ E_j \}_{j=1}^\infty$ be a countable collection of disjoint Lebesgue measurable sets in $\R^n$.
    Then \[ \mu\left( \bigcup_{j=1}^\infty E_j \right) = \sum_{j=1}^\infty \mu(E_j). \]
\end{corollary}

\begin{proof}
    This is just a special case of \ref{prop:sequences_of_measurable_sets}.
\end{proof}

Instead of just boxes, we can compute the Lebesgue outer measure of more interesting sets.

\begin{example}[The Cantor Set]
    \label{ex:cantor_set}
    Let's look at a classical example.
    The \emph{Cantor set} $C\sub [0,1]$ is constructed by starting with the interval $[0,1]$ and iteratively removing the open middle third of each remaining interval.
    That is, we start with $C_0 = [0,1]$.
    At the first step, we remove the open interval $(1/3, 2/3)$ of length $1/3$ to obtain
    \[ C_1 = [0,1/3] \cup [2/3,1]. \]
    At the second step, we remove the open intervals $(1/9,2/9)$ and $(7/9,8/9)$, each of length $1/9$ to obtain
    \[ C_2 = [0,1/9] \cup [2/9,1/3] \cup [2/3,7/9] \cup [8/9,1]. \]
    Continuing in this way, at the $n$-th step we remove $2^{n-1}$ disjoint open intervals, each of length $1/3^n$, to obtain $C_n$.
    By induction, we see that for each $n\in \Z^+$, the set $C_n$ consists of $2^n$ disjoint closed intervals, each of length $1/3^n$.

    The \textbf{Cantor set} is defined to be
    \[ C := \bigcap_{n=0}^\infty C_n. \]
    Clearly $C$ is closed, as the intersection of closed sets.
    It can also be shown that $C$ is uncountable and totally disconnected (i.e. the only connected subsets of $C$ are singletons).

    Moreover, we can compute the Lebesgue measure of $C$.
    For each $n\in \Z^+$, see that $C\sub C_n$ and thus $\mu(C) \leq \mu(C_n) = (2/3)^n$.
    Taking the limit as $n\to\infty$, we obtain $\mu(C) = 0$.
    Thus the Cantor set is an uncountable set of Lebesgue outer measure zero. \qed
\end{example}

At this point, we have shown that the Lebesgue outer measure $\mu$ is a Borel regular outer measure on $\R^n$ which is finite on compact sets, (so it can measure all Borel sets), and $\mu$ computes the correct value on boxes.
Thus, by \ref{cor:borel_reg_outer_measures_in_rn}, any measure which gives the same values on boxes as $\mu$ must agree with $\mu$ on all Borel sets.
In particular, there can be no other measure which gives the correct value on boxes and is defined on all Borel sets.

\begin{notation}[Lebesgue Measure]
    \label{not:lebesgue_measure}
    From now on, we will refer to the Lebesgue outer measure $\mu$ simply as the \emph{Lebesgue measure} on $\R^n$, and we will denote it by $\mathcal{L}^n$ instead of $\mu$.
    That is, for each $E\sub \R^n$, we write $\mathcal{L}^n(E)$ to denote the Lebesgue measure of $E$.
\end{notation}

We now prove the two remaining properties of Lebesgue measure we wanted: translation invariance and homogeneity.
\begin{proposition}[Translation Invariance and Homogeneity]
    \label{prop:translation_invariance_homogeneity}
    The Lebesgue measure $\mathcal{L}^n$ on $\R^n$ is translation invariant and homogeneous.
    That is, if $E\sub \R^n$ and $y\in \R^n$, then $\mathcal{L}^n(E+y) = \mathcal{L}^n(E)$, where $E+y = \{ x+y : x\in E \}$.
    Similarly, if $E\sub \R^n$ and $t>0$, then $\mathcal{L}^n(tE) = t^n\mathcal{L}^n(E)$, where $tE = \{ tx : x\in E \}$.
\end{proposition}

\begin{proof}
    \textit{Translation Invariance}:
    Let $E\sub \R^n$ and $y\in \R^n$.
    See that if $\{ Q_j \}_{j=1}^\infty$ is a cover of $E$ by closed cubes, then $\{ Q_j + y \}_{j=1}^\infty$ is a cover of $E+y$ by closed cubes, and $\vol(Q_j + y) = \vol(Q_j)$ for each $j$.
    Conversely, if $\{ S_j \}_{j=1}^\infty$ is a cover of $E+y$ by closed cubes, then $\{ S_j - y \}_{j=1}^\infty$ is a cover of $E$ by closed cubes, and $\vol(S_j - y) = \vol(S_j)$ for each $j$.
    Thus by definition of the Lebesgue measure $\mathcal{L}^n$, we have
    \begin{align*}
        \mathcal{L}^n(E+y) &= \inf \left\{ \sum_{j=1}^\infty \vol(Q_j) : E+y \sub \bigcup_{j=1}^\infty Q_j, \text{ each } Q_j \text{ is a closed cube} \right\} \\
            &= \inf \left\{ \sum_{j=1}^\infty \vol(S_j) : E \sub \bigcup_{j=1}^\infty S_j, \text{ each } S_j \text{ is a closed cube} \right\} \\
            &= \mathcal{L}^n(E).
    \end{align*}
    Since $E\sub \R^n$ and $y\in \R^n$ were arbitrary, we have shown that $\mathcal{L}^n$ is translation invariant.

    \textit{Homogeneity}:
    Let $E\sub \R^n$ and $t>0$.
    See that if $\{ Q_j \}_{j=1}^\infty$ is a cover of $E$ by closed cubes, then $\{ tQ_j \}_{j=1}^\infty$ is a cover of $tE$ by closed cubes, and $\vol(tQ_j) = t^n \vol(Q_j)$ for each $j$.
    Conversely, if $\{ S_j \}_{j=1}^\infty$ is a cover of $tE$ by closed cubes, then $\{ S_j / t \}_{j=1}^\infty$ is a cover of $E$ by closed cubes, and $\vol(S_j / t) = t^{-n} \vol(S_j)$ for each $j$.
    Thus by definition of the Lebesgue measure $\mathcal{L}^n$, we have
    \begin{align*}
        \mathcal{L}^n(tE) &= \inf \left\{ \sum_{j=1}^\infty \vol(Q_j) : tE \sub \bigcup_{j=1}^\infty Q_j, \text{ each } Q_j \text{ is a closed cube} \right\} \\
            &= \inf \left\{ \sum_{j=1}^\infty t^n \vol(S_j) : E \sub \bigcup_{j=1}^\infty S_j, \text{ each } S_j \text{ is a closed cube} \right\} \\
            &= t^n \mathcal{L}^n(E).
    \end{align*}
    Since $E\sub \R^n$ and $t>0$ were arbitrary, we have shown that $\mathcal{L}^n$ is homogeneous.
\end{proof}

The proofs of rotation invariance and scaling under linear transformations are more involved, and we will not prove them until after we have developed a theory of integration.
