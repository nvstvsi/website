\section{Fundamental Theorem of Calculus on $\R$}

In this section, we prove both versions of the Fundamental Theorem of Calculus for Lebesgue integrals on the real line.
We are not striving for maximum generalizty here, because that will come later.

\noindent In short section, we always let $[a,b]$ denoted a closed, bounded interval in $\R$ with $a < b$.

\subsection{Absolute Continuity}

\begin{definition}[Absolute Continuity]
    \label{def:absolute_continuity}
    A function $f:[a,b]\to\R$ is said to be \textit{absolutely continuous} on $[a,b]$ if for each $\epsilon > 0$, there exists a $\delta > 0$ such that for any finite collection of pairwise disjoint sub-intervals $\{(a_k,b_k)\}_{k=1}^N$ of $[a,b]$ with
    \[ \sum_{k=1}^N (b_k - a_k) < \delta, \]
    we have
    \[ \sum_{k=1}^N |f(b_k) - f(a_k)| < \epsilon. \]
    We define the set of all absolutely continuous functions on $[a,b]$ to be $AC([a,b])$.
\end{definition}

\begin{remark}[Absolute Continuity Implies Uniform Continuity]
    \label{rem:AC_implies_uniformly_continuous}
If $f$ is absolutely continuous on $[a,b]$, then it is uniformly continuous on $[a,b]$.

Let $\epsilon > 0$, and let $\delta > 0$ be such that for any finite collection of pairwise disjoint sub-intervals $\{(a_k,b_k)\}_{k=1}^N$ of $[a,b]$ with
\[ \sum_{k=1}^N (b_k - a_k) < \delta, \]
we have
\[ \sum_{k=1}^N |f(b_k) - f(a_k)| < \epsilon. \]
Then by choosing $N=1$, we have that for any sub-interval $(a_1,b_1)$ of $[a,b]$ with
\[ b_1 - a_1 < \delta, \]
we have
\[ |f(b_1) - f(a_1)| < \epsilon. \]
This is precisely the definition of uniform continuity.
\end{remark}

\begin{example}[Lipschitz Functions are Absolutely Continuous]
    \label{ex:lipschitz_functions_are_absolutely_continuous}
    If $f:[a,b]\to\R$ is Lipschitz with Lipschitz constant $L>0$, then for each $\epsilon > 0$, if we let $\delta := \epsilon / L$, then for any finite collection of pairwise disjoint sub-intervals $\{(a_k,b_k)\}_{k=1}^N$ of $[a,b]$ with
    \[ \sum_{k=1}^N (b_k - a_k) < \delta, \]
    we have
    \[ \sum_{k=1}^N |f(b_k) - f(a_k)| \leq \sum_{k=1}^N L |b_k - a_k| = L \sum_{k=1}^N (b_k - a_k) < L \delta = \epsilon. \]
    Thus $f$ is absolutely continuous on $[a,b]$.
\end{example}

\begin{example}[Not All Absolutely Continuous Functions are Lipschitz]
    \label{ex:not_all_absolutely_continuous_functions_are_lipschitz}
    For example, the function $f:[0,1]\to\R$ defined by $f(x) = \sqrt{x}$ is absolutely continuous on $[0,1]$ but not Lipschitz.

    ....
\end{example}

\begin{example}[The Cantor-Lebesgue Function is Not Absolutely Continuous]
    \label{ex:cantor_lebesgue_function_is_not_absolutely_continuous}
    The Cantor-Lebesgue function $f:[0,1]\to[0,1]$ is continuous and increasing, but not absolutely continuous.

    ....
\end{example}

\begin{exercise}[$AC$ is a Vector Space]
    \label{ex:AC_is_a_vector_space}
    Show that $AC([a,b])$ is a vector space over $\R$.
\end{exercise}
\begin{proof}
    First see that $0$, the zero function, is in $AC([a,b])$.
    Now let $f,g \in AC([a,b])$ and $\alpha\in \R\setminus\{0\}$.

    Let $\epsilon > 0$.
    Since $f$ and $g$ are absolutely continuous on $[a,b]$, there exist $\delta_1,\delta_2 > 0$ such that for any finite collection of pairwise disjoint sub-intervals $\{(a_k,b_k)\}_{k=1}^N$ of $[a,b]$ with
    \[ \sum_{k=1}^N (b_k - a_k) < \delta_1 \quad \text{and} \quad \sum_{k=1}^N (b_k - a_k) < \delta_2, \]
    we have
    \[ \sum_{k=1}^N |f(b_k) - f(a_k)| < \frac{\epsilon}{2|\alpha|} \quad \text{and} \quad \sum_{k=1}^N |g(b_k) - g(a_k)| < \frac{\epsilon}{2}. \]
    Then letting $\delta := \min\{\delta_1,\delta_2\}$, we have that for any finite collection of pairwise disjoint sub-intervals $\{(a_k,b_k)\}_{k=1}^N$ of $[a,b]$ with
    \[ \sum_{k=1}^N (b_k - a_k) < \delta, \]
    we have
    \begin{align*}
        \sum_{k=1}^N |(\alpha f + g)(b_k) - (\alpha f + g)(a_k)| &\leq \sum_{k=1}^N |\alpha| |f(b_k) - f(a_k)| + \sum_{k=1}^N |g(b_k) - g(a_k)| \\
            &< |\alpha| \frac{\epsilon}{2|\alpha|} + \frac{\epsilon}{2} = \epsilon.
    \end{align*}
    Thus $\alpha f + g \in AC([a,b])$.
    Therefore $AC([a,b])$ is a vector space over $\R$.
\end{proof}

\begin{exercise}[Product of $AC$ is $AC$]
    \label{ex:product_of_AC_is_AC}
    If $f,g \in AC([a,b])$, then $fg \in AC([a,b])$.
\end{exercise}
\begin{proof}
    Let $f,g \in AC([a,b])$.
    Since $f$ and $g$ are continuous on the compact set $[a,b]$, they are bounded.
    That is, there exist $M_1,M_2 > 0$ such that
    \[ |f(x)| \leq M_1 \quad\text{and}\quad |g(x)| \leq M_2 \quad\forall x\in[a,b]. \]

    Let $\epsilon > 0$.
    Since $f$ and $g$ are absolutely continuous on $[a,b]$, there exist $\delta_1,\delta_2 > 0$ such that for any finite collection of pairwise disjoint sub-intervals $\{(a_k,b_k)\}_{k=1}^N$ of $[a,b]$ with
    \[ \sum_{k=1}^N (b_k - a_k) < \delta_1 \quad \text{and} \quad \sum_{k=1}^N (b_k - a_k) < \delta_2, \]
    we have
    \[ \sum_{k=1}^N |f(b_k) - f(a_k)| < \frac{\epsilon}{2M_2} \quad \text{and} \quad \sum_{k=1}^N |g(b_k) - g(a_k)| < \frac{\epsilon}{2M_1}. \]
    Then letting $\delta := \min\{\delta_1,\delta_2\}$, we have that for any finite collection of pairwise disjoint sub-intervals $\{(a_k,b_k)\}_{k=1}^N$ of $[a,b]$ with
    \[ \sum_{k=1}^N (b_k - a_k) < \delta, \]
    we have
    \begin{align*}
        \sum_{k=1}^N |(fg)(b_k) - (fg)(a_k)| &= \sum_{k=1}^N |f(b_k)g(b_k) - f(a_k)g(a_k)| \\
            &= \sum_{k=1}^N |f(b_k)g(b_k) - f(a_k)g(b_k) + f(a_k)g(b_k) - f(a_k)g(a_k)| \\
            &\leq \sum_{k=1}^N |g(b_k)||f(b_k) - f(a_k)| + \sum_{k=1}^N |f(a_k)||g(b_k) - g(a_k)| \\
            &\leq M_2 \sum_{k=1}^N |f(b_k) - f(a_k)| + M_1 \sum_{k=1}^N |g(b_k) - g(a_k)| \\
            &< M_2 \frac{\epsilon}{2M_2} + M_1 \frac{\epsilon}{2M_1} = \epsilon.
    \end{align*}


\end{proof}

\begin{exercise}[Equivalent Definition of $AC$ Functions]
    \label{ex:equivalent_definition_of_AC_functions}
    Let $f:[a,b]\to\R$ be a function.
    Then $f$ is absolutely continuous on $[a,b]$ if and only if for each $\epsilon > 0$, there exists a $\delta > 0$ such that for any countable collection of pairwise disjoint sub-intervals $\{(a_k,b_k)\}_{k=1}^\infty$ of $[a,b]$ with
    \[ \sum_{k=1}^\infty (b_k - a_k) < \delta, \]
    we have
    \[ \sum_{k=1}^\infty |f(b_k) - f(a_k)| < \epsilon. \]
    Deduce that if $f$ is absolutely continuous on $[a,b]$, then $f$ maps sets of Lebesgue measure zero to sets of Lebesgue measure zero.
\end{exercise}

\begin{proof}
    ($\Rightarrow$) Suppose that $f$ is absolutely continuous on $[a,b]$.
    Let $\epsilon > 0$, and let $\delta > 0$ be such that for any finite collection of pairwise disjoint sub-intervals $\{(a_k,b_k)\}_{k=1}^N$ of $[a,b]$ with
    \[ \sum_{k=1}^N (b_k - a_k) < \delta, \]
    we have
    \[ \sum_{k=1}^N |f(b_k) - f(a_k)| < \frac{\epsilon}{2}. \]

    Now let $\{(a_k,b_k)\}_{k=1}^\infty$ be any countable collection of pairwise disjoint sub-intervals of $[a,b]$ with
    \[ \sum_{k=1}^\infty (b_k - a_k) < \delta. \]
    Then for each $N\in\Z^+$, we have
    \[ \sum_{k=1}^N (b_k - a_k) < \delta, \]
    so
    \[ \sum_{k=1}^N |f(b_k) - f(a_k)| < \frac{\epsilon}{2}. \]
    By taking the limit as $N\to\infty$ we see that
    \[ \sum_{k=1}^\infty |f(b_k) - f(a_k)| \leq \frac{\epsilon}{2} < \epsilon. \]
    
    Since the collection $\{(a_k,b_k)\}_{k=1}^\infty$ satisfying $\sum_{k=1}^\infty (b_k - a_k) < \delta$ was arbitrary, and $\epsilon > 0$ was arbitrary, we have shown that the function $f$ satisfies the condition in the statement of the exercise.
    Thus the forward direction is proven.

    \vspace{2mm}

    ($\Leftarrow$) The reverse direction is trivial since any finite collection of pairwise disjoint sub-intervals is also a countable collection of pairwise disjoint sub-intervals.
    
    \vspace{2mm}

    Suppose now that $f$ is increasing and absolutely continuous on $[a,b]$, and let $E\subseteq [a,b]$ be a set of Lebesgue measure zero.
    Let $\epsilon > 0$, and let $\delta > 0$ be such that for any countable collection of pairwise disjoint sub-intervals $\{(a_k,b_k)\}_{k=1}^\infty$ of $[a,b]$ with
    \[ \sum_{k=1}^\infty (b_k - a_k) < \delta, \]
    we have
    \[ \sum_{k=1}^\infty |f(b_k) - f(a_k)| < \epsilon. \]
    Since $E$ has Lebesgue measure zero, there exists a countable collection of pairwise disjoint open intervals $\{(a_k,b_k)\}_{k=1}^\infty$ of $[a,b]$ such that
    \[ E \subseteq \bigcup_{k=1}^\infty (a_k,b_k) \quad\text{and}\quad \sum_{k=1}^\infty (b_k - a_k) < \delta. \]
    Then
    \[ f(E) \subseteq f\left( \bigcup_{k=1}^\infty (a_k,b_k) \right) \subseteq \bigcup_{k=1}^\infty [f(a_k),f(b_k)] \]
    since $f$ is increasing, so
    \[ m(f(E)) \leq m\left( \bigcup_{k=1}^\infty [f(a_k),f(b_k)] \right) \leq \sum_{k=1}^\infty |f(b_k) - f(a_k)| < \epsilon. \]
    Since $\epsilon > 0$ was arbitrary, we have shown that $m(f(E)) = 0$.
\end{proof}

\begin{lemma}[$AC$ implies $BV$]
    \label{lem:AC_implies_BV}
    If $f:[a,b]\to\R$ is absolutely continuous on $[a,b]$, then it is of bounded variation on $[a,b]$ and its total variation function is absolutely continuous.
\end{lemma}
\begin{proof}
    Let $f:[a,b]\to\R$ be absolutely continuous on $[a,b]$.
    Then there exists $\delta > 0$ such that for any finite collection of pairwise disjoint sub-intervals $\{(a_k,b_k)\}_{k=1}^N$ of $[a,b]$ with
    \[ \sum_{k=1}^N (b_k - a_k) < \delta, \]
    we have
    \[ \sum_{k=1}^N |f(b_k) - f(a_k)| < 1. \]

    Then let $P$ be a partition \[a = x_0 < x_1 < \cdots < x_N = b \] 
    of $[a,b]$ such that $x_k - x_{k-1} < \delta$ for each $k=1,\ldots,N$.
    Then
    \[ V_{x_{k-1}}^{x_k}(f) = \sup \left\{ \sum_{j=1}^M |f(y_j) - f(y_{j-1})| : P' = \{ y_0 < y_1 < \cdots < y_M \} \text{ is a partition of } [x_{k-1},x_k] \right\} \leq 1 \]
    for each $k=1,\ldots,N$, so
    \[ V_a^b(f) = \sum_{k=1}^N V_{x_{k-1}}^{x_k}(f) \leq N < \infty. \]
    Thus $f$ is of bounded variation on $[a,b]$.

    \vspace{2mm}

    Now we will show that the function $x\longmapsto V_a^x(f)$ is absolutely continuous on $[a,b]$.
    Let $\epsilon > 0$. Since $f$ is absolutely continuous on $[a,b]$, there exists $\delta > 0$ such that for any finite collection of pairwise disjoint sub-intervals $\{(a_k,b_k)\}_{k=1}^N$ of $[a,b]$ with
    \[ \sum_{k=1}^N (b_k - a_k) < \delta, \]
    we have
    \[ \sum_{k=1}^N |f(b_k) - f(a_k)| < \frac{\epsilon}{2}. \] 

    Fix a finite collection of pairwise disjoint sub-intervals $\{(c_k,d_k)\}_{k=1}^M$ of $[a,b]$ with
    \[ \sum_{k=1}^M (d_k - c_k) < \delta, \]
    and for each $k=1,\ldots,M$, let $P_k = \{ c_k = y_0 < y_1 < \cdots < y_{N_k} = d_k \}$ be a partition of $[c_k,d_k]$.
    We have
    \[ \sum_{k=1}^M V( f|_{[c_k,d_k]}, P_k ) < \frac{\epsilon}{2} \]
    by choice of $\delta$.

    By taking the supremum over all partitions $P_k$ of $[c_k,d_k]$ for each $k=1,\ldots,M$, we have
    \[ \sum_{k=1}^M V_{c_k}^{d_k}(f) \leq \frac{\epsilon}{2} < \epsilon. \]
    It follows from additivity of the total variation function (see Proposition \ref{prop:additivity_of_total_variation}) that
    \[ V_{c_k}^{d_k}(f) = V_a^{d_k}(f) - V_a^{c_k}(f) \quad \forall k = 1,\ldots,M. \]
    Putting these two facts together, we have
    \[ \sum_{k=1}^M |V_a^{d_k}(f) - V_a^{c_k}(f)| = \sum_{k=1}^M V_{c_k}^{d_k}(f) < \epsilon. \]
    Since the collection $\{(c_k,d_k)\}_{k=1}^M$ satisfying $\sum_{k=1}^M (d_k - c_k) < \delta$ was arbitrary, and $\epsilon > 0$ was arbitrary, we have shown that the function $x\longmapsto V_a^x(f)$ is absolutely continuous on $[a,b]$.
\end{proof}

\begin{corollary}[Jordan Decomposition for $AC$ Functions]
    \label{cor:AC_is_difference_of_increasing_AC}
    If $f:[a,b]\to\R$ is absolutely continuous on $[a,b]$, then there exist increasing, absolutely continuous functions $f_1,f_2:[a,b]\to\R$ such that
    \[ f = f_1 - f_2. \]
\end{corollary}
\begin{proof}
    From the previous lemma \ref{lem:AC_implies_BV} and the Jordan decomposition theorem (Theorem \ref{thm:jordan_decomposition_theorem}),
    it follows that 
    \[ f(x) = (f(x) + V_a^x(f)) - V_a^x(f) \qquad \forall x \in [a,b] \]
    expresses $f$ as the difference of two increasing functions on $[a,b]$.
    The previous lemma also shows that $x\mapsto V_a^x(f)$ is absolutely continuous on $[a,b]$, and in Exercise \ref{ex:AC_is_a_vector_space} we showed that the difference of two absolutely continuous functions is absolutely continuous.
    Thus $f$ can be expressed as the difference of two increasing, absolutely continuous functions on $[a,b]$.
\end{proof}

The following Corollary is frequently used.
\begin{corollary}[$AC$ Functions are Differentiable Almost Everywhere]
    \label{cor:AC_functions_are_differentiable_almost_everywhere}
    If $f:[a,b]\to\R$ is absolutely continuous on $[a,b]$, then it is differentiable almost everywhere on $[a,b]$.
\end{corollary}
\begin{proof}
    Since $f$ is absolutely continuous on $[a,b]$, it is of bounded variation on $[a,b]$ by Lemma \ref{lem:AC_implies_BV}.
    Thus by Corollary \ref{cor:BV_functions_are_differentiable_a_e}, $f$ is differentiable almost everywhere on $[a,b]$.
\end{proof}

\subsection{Fundamental Theorem of Calculus}

\begin{lemma}[(Improved) Second Fundamental Theorem of Calculus]
    \label{lem:anti_derivative_is_AC}
    Let $f:[a,b]\to\R$ be an integrable function.
    Then the indefinite integral
    \[ F(x) = \int_a^x f(t)\,\dif t, \qquad x\in[a,b], \]
    is absolutely continuous on $[a,b]$ and has derivative $F'(x) = f(x)$ for almost every $x\in[a,b]$.
\end{lemma}
\noindent We already know $F$ is continuous on $[a,b]$, and that $F'(x)$ exists and equals $f(x)$ for almost every $ x\in[a,b]$ by Theorem \ref{thm:second_fundamental_theorem_of_calculus}.
The new conclusion here is that $F$ is absolutely continuous on $[a,b]$.

\begin{proof}

Let $\{ (a_k,b_k) \}_{k=1}^N$ be an arbitrary finite collection of pairwise disjoint sub-intervals of $[a,b]$.
Then we have
\begin{align*}
    \sum_{k=1}^N |F(b_k) - F(a_k)| &= \sum_{k=1}^N \left| \int_a^{b_k} f(t)\,dt - \int_a^{a_k} f(t)\,dt \right| \\
        &= \sum_{k=1}^N \left| \int_{a_k}^{b_k} f(t)\,dt \right| \\
        &\leq \sum_{k=1}^N \int_{a_k}^{b_k} |f(t)|\,dt = \int_{\bigcup_{k=1}^N (a_k,b_k)} |f(t)|\,dt. \\
\end{align*}

Let $\epsilon > 0$, and let $\delta > 0$ be such that if $E\subseteq [a,b]$ is a measurable set with $\mathcal{L}^1(E) < \delta$, then 
\[ \int_E |f(t)|\,dt < \epsilon. \]
(This is possible by \ref{lem:integral_on_small_sets_is_small} since $f$ is integrable on $[a,b]$.)
Then for each finite collection of pairwise disjoint sub-intervals $\{(a_k,b_k)\}_{k=1}^N$ of $[a,b]$ with
\[ \sum_{k=1}^N (b_k - a_k) < \delta, \]
we have
\[ \sum_{k=1}^N |F(b_k) - F(a_k)| \leq \int_{\bigcup_{k=1}^N (a_k,b_k)} |f(t)|\,dt < \epsilon. \]
Thus $F$ is absolutely continuous on $[a,b]$. 
\end{proof}

\begin{lemma}[Zero Derivative Implies Constant]
    \label{lem:zero_derivative_implies_constant}
    Let $f:[a,b]\to\R$ be an increasing absolutely continuous function on $[a,b]$.
    If $f'(x) = 0$ for almost every $x\in[a,b]$, then $f$ is constant on $[a,b]$.
\end{lemma}

\begin{proof}
    \textit{Step 1:} We assume that $f$ is increasing on $[a,b]$ and that $f'(x) = 0$ for almost every $x\in[a,b]$.
    \vspace{2mm}

    Since $f$ is increasing and uniformly continuous on the compact set $[a,b]$, its range must be the closed interval $[f(a),f(b)]$.
    We will show that the length of this interval is zero. Let 
    \[ E := \{ x\in [a,b] : f'(x)  = 0 \} \]
    so that $\mathcal{L}^1([a,b]\setminus E) = 0$ by assumption.
    By \ref{ex:equivalent_definition_of_AC_functions}, the absolute continuity of $f$ implies that the image of $[a,b]\setminus E$ under $f$ has Lebesgue measure zero.
    That is, 
    \[ \mathcal{L}^1(f([a,b]\setminus E)) = 0. \]

    \vspace{2mm}
    \textit{Step 2:} We claim that $f(E)$ also has Lebesgue measure zero.
    \vspace{2mm}

    Let $\epsilon > 0$. If $x_0\in E$ then $f'(x_0) = 0$, so by definition of the derivative, there exists a $\delta > 0$ such that for each $x\in [x_0,x_0 + \delta)$ we have
    \[  \frac{f(x) - f(x_0)}{x - x_0} < \epsilon \]
    which is equivalent to
    \[ f(x) - f(x_0) < \epsilon (x - x_0) \]
    which rearranges to
    \[ \epsilon x_0 - f(x_0) < \epsilon x - f(x). \]
    Thus $x_0$ is invisible from the right with respect to the continuous function $x\mapsto \epsilon x - f(x)$.
    Since $x_0\in E$ was arbitrary, we have shown that every point in $E$ is invisible from the right with respect to the continuous function $x\mapsto \epsilon x - f(x)$.

    By the Rising Sun Lemma (Lemma \ref{lem:rising_sun_lemma}), there is a countable collection of pairwise disjoint open intervals $\{(a_k,b_k)\}_{k=1}^\infty$ such that
    \[ E \subseteq \bigcup_{k=1}^\infty (a_k,b_k) \quad\text{and}\quad f(b_k) - \epsilon b_k \leq f(a_k) - \epsilon a_k \quad\forall k\in\Z^+. \]
    That is, \[ f(b_k) - f(a_k) \leq \epsilon (b_k - a_k) \quad\forall k\in\Z^+. \]
    But then we see that 
    \[ \sum_{k=1}^\infty (f(b_k) - f(a_k)) \leq \epsilon \sum_{k=1}^\infty (b_k - a_k) \leq \epsilon (b-a). \]
    Hence
    \[ \mathcal{L}^1(f(E)) \leq \mathcal{L}^1\left(f\left(\bigcup_{k=1}^\infty (a_k,b_k)\right)\right) \leq \mathcal{L}^1\left(\bigcup_{k=1}^\infty f((a_k,b_k))\right) \leq \sum_{k=1}^\infty (f(b_k) - f(a_k)) \leq \epsilon (b-a). \]
    Since $\epsilon > 0$ was arbitrary, we have shown that $\mathcal{L}^1(f(E)) = 0$.

    But now we see that 
    \[ \mathcal{L}^1(f([a,b])) = \mathcal{L}^1(f(E) \cup f([a,b]\setminus E)) \leq \mathcal{L}^1(f(E)) + \mathcal{L}^1(f([a,b]\setminus E)) = 0 + 0 = 0. \]
    Since $f([a,b]) = [f(a),f(b)]$ is a closed interval, it must be that $f(a) = f(b)$ and $f(x) = f(a)$ for all $x\in[a,b]$.

    \vspace{2mm}

    [It is instructive to note where we have used the fact that $f$ is increasing in the above argument --- 
    first of all, this allows us to remove the absolute value bars in the first inequality in step 2, and this is crucial for all the following computations.
    It is also worth noting that we used the absolute continuity of $f$ in step 1 deduce that the image of $\{ x : f'(x) = 0 \} $ has Lebesgue measure zero, and also in step 2 to apply the Rising Sun Lemma (which requires continuity).]
\end{proof}

\begin{theorem}[First Fundamental Theorem of Calculus for $AC$ Functions]
    \label{thm:fundamental_theorem_of_calculus_for_AC_functions}
    If $f:[a,b]\to\R$ is absolutely continuous on $[a,b]$, then
    \[ \int_a^b f'(t)\,\dif t = f(b) - f(a). \]
\end{theorem}

Note that we already know that $f'$ exists almost everywhere on $[a,b]$ and is integrable, since $AC$ functions are $BV$ functions (Lemma \ref{lem:AC_implies_BV}) and $BV$ functions are differentiable almost everywhere (Corollary \ref{cor:BV_functions_are_differentiable_a_e}) and have integrable derivatives.

\begin{proof}
    \textit{Step 1:}
    First we consider an increasing absolutely continuous function $f:[a,b]\to\R$.
    Then as we remarked above, $f'$ exists almost everywhere on $[a,b]$ and is integrable.
    We define a function $\Phi:[a,b]\to\R$ by
    \[ \Phi(x) := f(x) - \int_a^x f'(t)\,\dif t, \qquad\forall x \in [a,b]. \]
    Then becuase $f'$ is integrable on $[a,b]$, the function $x\mapsto \int_a^x f'(t)\,\dif t$ is absolutely continuous on $[a,b]$ by Lemma \ref{lem:anti_derivative_is_AC}, and hence $\Phi$ is absolutely continuous on $[a,b]$ as well since it is the difference of two absolutely continuous functions.

    See that $\Phi$ is increasing on $[a,b]$ since for all $x,y\in[a,b]$ with $x < y$, we have
    \begin{align*}
        \Phi(y) - \Phi(x) &= f(y) - f(x) - \int_a^y f'(t)\,\dif t + \int_a^x f'(t)\,\dif t \\
            &= f(y) - f(x) - \int_x^y f'(t)\,\dif t \\
            &\geq f(y) - f(x) - (f(y) - f(x)) = 0,
    \end{align*}
    where we have used Exercise \ref{ex:fundamental_theorem_of_calculus_inequality} in the last line. 
    [This is where we use the fact that $f$ is increasing.]

    We also see that $\Phi$ is differentiable almost everywhere on $[a,b]$ and that
    \[ \Phi'(x) = f'(x) - f'(x) = 0 \]
    for almost every $x\in[a,b]$, where we have used the fact that $f'$ exists almost everywhere on $[a,b]$, and that the derivative of the indefinite integral is the integrand almost everywhere (the Second Fundamental Theorem of Calculus \ref{thm:second_fundamental_theorem_of_calculus}).

    Therefore by Lemma \ref{lem:zero_derivative_implies_constant}, $\Phi$ is constant on $[a,b]$, so
    for each $x\in[a,b]$ we have $\Phi(x) = \Phi(a)$, which is equivalent to
    \[ f(x) - \int_a^x f'(t)\,\dif t = f(a) - \int_a^a f'(t)\,\dif t = f(a). \]
    Rearranging this gives
    \[ f(x) = f(a) + \int_a^x f'(t)\,\dif t \qquad\forall x\in[a,b]. \]
    In particular, letting $x = b$ gives
    \[ f(b) - f(a) = \int_a^b f'(t)\,\dif t. \]

    \vspace{2mm}

    \textit{Step 2:}
    Now let $f:[a,b]\to\R$ be an arbitrary absolutely continuous function on $[a,b]$.
    By Corollary \ref{cor:AC_is_difference_of_increasing_AC}, there exist increasing absolutely continuous functions $f_1,f_2:[a,b]\to\R$ such that
    \[ f = f_1 - f_2. \]
    Then by Step 1, we have
    \[ f_1(x) - f_1(a) = \int_a^x f_1'(t)\,\dif t \quad\text{and}\quad f_2(x) - f_2(a) = \int_a^x f_2'(t)\,\dif t \qquad\forall x\in [a,b]. \]
    Subtracting these two equations gives
    \[ f(x) - f(a) = (f_1(x) - f_2(x)) - (f_1(a) - f_2(a)) = \int_a^x f_1'(t)\,\dif t - \int_a^x f_2'(t)\,\dif t = \int_a^x (f_1'(t) - f_2'(t))\,\dif t \qquad\forall x\in[a,b]. \]
    But since $f = f_1 - f_2$, we have $f' = f_1' - f_2'$ almost everywhere on $[a,b]$, so
    \[ f(x) - f(a) = \int_a^x f'(t)\,\dif t \qquad \forall x\in [a,b]. \] 
    In particular, letting $x = b$ gives
    \[ f(b) - f(a) = \int_a^b f'(t)\,\dif t. \]
\end{proof}

\begin{corollary}[Integration by Parts for $AC$ Functions]
    \label{cor:integration_by_parts_for_AC_functions}
    If $f,g:[a,b]\to\R$ are absolutely continuous on $[a,b]$, then
    \[ \int_a^b f(t)g'(t)\,\dif t = f(b)g(b) - f(a)g(a) - \int_a^b f'(t)g(t)\,\dif t. \]
\end{corollary}
\begin{proof}
    Let $f,g:[a,b]\to\R$ be absolutely continuous on $[a,b]$.
    By exercise \ref{ex:product_of_AC_is_AC}, the product $fg$ is absolutely continuous on $[a,b]$.
    Then by Theorem \ref{thm:fundamental_theorem_of_calculus_for_AC_functions}, we have
    \[ (fg)(b) - (fg)(a) = \int_a^b (fg)'(t)\,\dif t. \]
    But by the product rule for derivatives, we have
    \[ (fg)'(t) = f'(t)g(t) + f(t)g'(t) \]
    for almost every $t\in[a,b]$, so
    \[ f(b)g(b) - f(a)g(a) = \int_a^b f'(t)g(t)\,\dif t + \int_a^b f(t)g'(t)\,\dif t. \]
    Rearranging this gives the desired result.
\end{proof}

This finishes our generalization of one-variable calculus to the Lebesgue integral.
