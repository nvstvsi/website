
\section{Fubini and Tonelli Theorems}

In this section, we prove the Fubini and Tonelli theorems, which allow us to compute integrals on product spaces as iterated integrals.
In a later section, we will prove a curvilinear version of this theorem for computing integrals called the \emph{Area Formula} and the \emph{Coarea Formula}.

We follow the treatment given by Axler.
\subsection{Products of Measures}

\begin{definition}[Product of $\sigma$-algebras]
    \label{def:product_sigma_algebra}
    Let $(X,\mathcal{A})$ and $(Y,\mathcal{B})$ be measurable spaces.
    We define the \textit{product} of the $\sigma$-algebras $\mathcal{A}$ and $\mathcal{B}$ as the smallest $\sigma$-algebra on $X \times Y$ containing all products $A \times B$ with $A \in \mathcal{A}$ and $B \in \mathcal{B}$, 
    and we denote this $\sigma$-algebra by $\mathcal{A} \otimes \mathcal{B}$.
    A \textit{measurable product} is a set of the form $A\times B$ with $A \in \mathcal{A}$ and $B \in \mathcal{B}$.
\end{definition}

\begin{definition}[Cross Sections of a Set]
    \label{def:cross_sections}
    Let $X$ and $Y$ be sets, and let $E \subseteq X \times Y$.
    For each $x \in X$, we define the \emph{cross section} of $E$ at $x$ as
    \[ E_x := \{ y \in Y : (x,y) \in E \} \subseteq Y. \]
    For each $y \in Y$, we define the \emph{cross section} of $E$ at $y$ as
    \[ E^y := \{ x \in X : (x,y) \in E \} \subseteq X. \]
\end{definition}

\begin{lemma}[Cross Sections of Measurable Products are Measurable]
    \label{lem:cross_section_of_measurable_product_is_measurable}
    Let $(X,\mathcal{A})$ and $(Y,\mathcal{B})$ be measurable spaces, and let $E = A \times B$ be a measurable product with $A \in \mathcal{A}$ and $B \in \mathcal{B}$.
    Then for each $x \in X$ we have $E_x \in \mathcal{B}$, and for each $y \in Y$ we have $E^y \in \mathcal{A}$.
\end{lemma}
\begin{proof}
    Let $\mathcal{E}$ be the collection of all subsets $E \subseteq X \times Y$ such that $E_x \in \mathcal{B}$ for each $x \in X$ and $E^y \in \mathcal{A}$ for each $y \in Y$.
    Then for all $A\in \mathcal{A}$ and all $B \in \mathcal{B}$, we have $(A \times B)_x = B$ if $x \in A$ and $(A \times B)_x = \emptyset$ otherwise, so $(A \times B)_x \in \mathcal{B}$ for each $x \in X$.
    Similarly, we have $(A \times B)^y = A$ if $y \in B$ and $(A \times B)^y = \emptyset$ otherwise, so $(A \times B)^y \in \mathcal{A}$ for each $y \in Y$.
    Thus $A \times B \in \mathcal{E}$ for all $A \in \mathcal{A}$ and $B \in \mathcal{B}$.

    See that $\mathcal{E}$ is closed under complements because for each $E\in \mathcal{E}$ we have
    \[ ((X\times Y)\setminus E)_x = Y \setminus E_x \in \mathcal{B} \]
    for all $x \in X$ and similarly
    \[ ((X\times Y)\setminus E)^y = X \setminus E^y \in \mathcal{A} \]
    for all $y \in Y$.

    Also see that $\mathcal{E}$ is closed under countable unions because for each sequence $\{E_j\}_{j=1}^\infty$ in $\mathcal{E}$ we have
    \[ \left( \bigcup_{j=1}^\infty E_j \right)_x = \bigcup_{j=1}^\infty (E_j)_x \in \mathcal{B} \]
    for all $x \in X$ and similarly
    \[ \left( \bigcup_{j=1}^\infty E_j \right)^y = \bigcup_{j=1}^\infty (E_j)^y \in \mathcal{A} \]
    for all $y \in Y$.

    Since $\mathcal{E}$ contains all measurable products and is a $\sigma$-algebra, we have $\mathcal{A} \otimes \mathcal{B} \subseteq \mathcal{E}$.
\end{proof}

\begin{definition}[Cross Section of a Function]
    \label{def:cross_section_of_function}
    Let $X$ and $Y$ be sets, and let $f : X \times Y \to [-\infty,\infty]$ be a function.
    For each $x \in X$, we define the \emph{cross section} of $f$ at $x$ as the function $f_x : Y \to [-\infty,\infty]$ given by
    \[ f_x(y) := f(x,y) \quad \text{for all } y \in Y. \]
    For each $y \in Y$, we define the \emph{cross section} of $f$ at $y$ as the function $f^y : X \to [-\infty,\infty]$ given by
    \[ f^y(x) := f(x,y) \quad \text{for all } x \in X. \]
\end{definition}
    Basically if $f$ is a function defined on a product $X\times Y$, then fixing $x\in X$ gives a function $f_x$ defined on $Y$, and fixing $y \in Y$ gives a function $f^y$ defined on $X$.

\begin{lemma}[Cross Sections of Measurable Functions are Measurable]
    \label{lem:cross_section_of_measurable_function_is_measurable}
    Let $(X,\mathcal{A})$ and $(Y,\mathcal{B})$ be measurable spaces, and let $f : X \times Y \to [-\infty,\infty]$ be an $(\mathcal{A} \otimes \mathcal{B})$-measurable function.
    Then for each $x \in X$, the cross section $f_x : Y \to [-\infty,\infty]$ is a $\mathcal{B}$-measurable function, and for each $y \in Y$, the cross section $f^y : X \to [-\infty,\infty]$ is an $\mathcal{A}$-measurable function.
\end{lemma}

\begin{proof}
    Let $D$ be a Borel subset of $[-\infty,\infty]$ and let $x\in X$.
    Then we have
    \begin{align*}
        y = (f_x)^{-1}(D) &\iff f_x(y) \in D \\
        &\iff f(x,y) \in D \\
        &\iff (x,y) \in f^{-1}(D).
        &\iff y \in (f^{-1}(D))_x
    \end{align*}
    which shows that \[ (f_x)^{-1}(D) = (f^{-1}(D))_x \in \mathcal{B}. \]
    Since $f$ is $(\mathcal{A} \otimes \mathcal{B})$-measurable, we have $f^{-1}(D) \in \mathcal{A} \otimes \mathcal{B}$, so by Lemma \ref{lem:cross_section_of_measurable_set_is_measurable} we have $(f^{-1}(D))_x \in \mathcal{B}$.
    Thus $f_x$ is $\mathcal{B}$-measurable.
    Since $x\in X$ was arbitrary, we have that $f_x$ is $\mathcal{B}$-measurable for each $x \in X$.

    The same idea shows that for each $y \in Y$, the cross section $f^y : X \to [-\infty,\infty]$ is an $\mathcal{A}$-measurable function.
\end{proof}

\subsection{Monotone Class Lemma}

\begin{definition}[Algebra (of Sets)]
    \label{def:algebra_of_sets}
    Let $X$ be a set.
    A collection $\mathcal{A}$ of subsets of $X$ is an \emph{algebra} if
    \begin{itemize}
        \item $\emptyset \in \mathcal{A}$,
        \item if $A \in \mathcal{A}$, then $X \setminus A \in \mathcal{A}$, and
        \item if $A,B \in \mathcal{A}$, then $A \cup B \in \mathcal{A}$.
    \end{itemize}
\end{definition}
Notice that an algebra is also closed under finite intersections, by DeMorgan's laws.

An algebra is like a $\sigma$-algebra, except that it is only closed under finite unions instead of countable unions.

\begin{lemma}[Finite Unions of Measurable Products Form an Algebra]
    \label{lem:finite_union_of_measurable_products_is_algebra}
    Let $(X,\mathcal{A})$ and $(Y,\mathcal{B})$ be measurable spaces, and let $\mathcal{E}$ be the collection of all finite unions of measurable products $A \times B$ with $A \in \mathcal{A}$ and $B \in \mathcal{B}$.
    Then $\mathcal{E}$ is an algebra on $X \times Y$.

    Moreover, every finite union of measurable products in $\mathcal{A}\otimes\mathcal{B}$ can be written as a countable union of disjoint measurable products in $\mathcal{A}\otimes\mathcal{B}$.
\end{lemma}
\begin{proof}
    It is clear that $\emptyset \in \mathcal{E}$ since $\emptyset = \emptyset \times \emptyset$.
    Furthermore it is clear that $\mathcal{E}$ is closed under finite unions by definition.
    The set $\mathcal{E}$ is also closed under finite intersections because if $(A_1 \times B_1) \cup (A_2 \times B_2) \cup \cdots \cup (A_m \times B_m)$ and $(C_1 \times D_1) \cup (C_2 \times D_2) \cup \cdots \cup (C_n \times D_n)$ are in $\mathcal{E}$, then
    \begin{align*}
        \bigcup_{j=1}^m (A_j \times B_j) \,\cap\, \bigcup_{k=1}^n (C_k \times D_k) &= \bigcup_{j=1}^m \bigcup_{k=1}^n \left( (A_j \times B_j) \cap (C_k \times D_k) \right) \\
            &= \bigcup_{j=1}^m \bigcup_{k=1}^n \left( (A_j \cap C_k) \times (B_j \cap D_k) \right) \in \mathcal{E}.
    \end{align*}

    To see that the set $\mathcal{E}$ is closed under compliments, let $A \in \mathcal{A}$ and $B \in \mathcal{B}$.
    Then \[ (X\times Y)\setminus (A\times B) = \left( (X \setminus A) \times Y \right) \cup \left( A \times (Y \setminus B) \right). \]
    Hence the compliment of $A\times B$ is in $\mathcal{E}$. 
    That is, the compliemnt of each measurable product is in $\mathcal{E}$.
    Since $\mathcal{E}$ is closed under finite unions, we see that the compliment of any finite union of measurable products is in $\mathcal{E}$ by DeMorgan's laws and the fact that $\mathcal{E}$ is closed under finite intersections.

    Therefore $\mathcal{E}$ is an algebra on $X \times Y$.

\vspace{2mm}

    For the second part, note that if $A\times B$ and $C \times D$ are measurable products, then
    \[ (A\times B)\cup (C\times D) = (A\times B)\cup(C\times (D\setminus B)) \cup (C\setminus A)\times(B\cap D) \]
    is a union of three disjoint measurable products.

    Now consider a finite union of measurable products
    \[ E = \bigcup_{j=1}^m (A_j \times B_j) \]
    with $A_j \in \mathcal{A}$ and $B_j \in \mathcal{B}$ for each $j=1,2,\ldots,m$.
    If there is some $j\in \{1,2,\ldots,m\}$ such that $A_j \times B_j$ is not disjoint from $\bigcup_{k \neq j} (A_k \times B_k)$, then we can replace $A_j \times B_j$ in the union with a union of three disjoint measurable products as above.
    Doing this for all such $j$ gives $E$ as a finite union of disjoint measurable products.
\end{proof}

\begin{definition}[Monotone Class]
    \label{def:monotone_class}
    Let $X$ be a set and let $\mathcal{M}$ be a collection of subsets of $X$.
    We say that $\mathcal{M}$ is a \emph{monotone class} if
    \begin{itemize}
        \item if $\{A_j\}_{j=1}^\infty$ is an increasing sequence of sets in $\mathcal{M}$, then $\bigcup_{j=1}^\infty A_j \in \mathcal{M}$, and
        \item if $\{A_j\}_{j=1}^\infty$ is a decreasing sequence of sets in $\mathcal{M}$, then $\bigcap_{j=1}^\infty A_j \in \mathcal{M}$.
    \end{itemize}
\end{definition}

Clearly every $\sigma$-algebra is a monotone class, since $\sigma$-algebras are closed under countable unions and countable intersections.
Monotone classes don't have to be algebras or $\sigma$-algebras, as the following example shows. 

\begin{example}[Monotone Class of Intervals]
    \label{ex:monotone_class_of_intervals}
    Let $\mathcal{M}$ be the collection of all intervals (closed, open, half-open, degenerate) in $\R$.
    Then $\mathcal{M}$ is a monotone class since increasing unions and decreasing intersections of intervals are intervals.
    
    However, $\mathcal{M}$ is not an algebra since it is not closed under finite unions.
\end{example}

\begin{exercise}[Smallest Monotone Class]
    \label{ex:smallest_monotone_class}
    Let $X$ be a set, and let $\mathcal{E}$ be a collection of subsets of $X$.
    Show that 
    \[ \mathcal{M}_\mathcal{E} := \bigcap \{ \mathcal{M} : \mathcal{M} \text{ is a monotone class containing } \mathcal{E} \} \]
    is the smallest monotone class containing $\mathcal{E}$.    
\end{exercise}

\begin{proof}
    There are two things to show here --- first that $\mathcal{M}_\mathcal{E}$ is a monotone class, and second that for each monotone class $\mathcal{M}$ containing $\mathcal{E}$, we have $\mathcal{M}_\mathcal{E} \subseteq \mathcal{M}$.

    To see that $\mathcal{M}_\mathcal{E}$ is a monotone class, let $\{A_j\}_{j=1}^\infty$ be an increasing sequence of sets in $\mathcal{M}_\mathcal{E}$.
    Then for each monotone class $\mathcal{M}$ containing $\mathcal{E}$, we have $A_j \in \mathcal{M}$ for each $j\in\Z^+$, so $\bigcup_{j=1}^\infty A_j \in \mathcal{M}$ since $\mathcal{M}$ is a monotone class.
    Since this is true for each monotone class $\mathcal{M}$ containing $\mathcal{E}$, we have $\bigcup_{j=1}^\infty A_j \in \mathcal{M}_\mathcal{E}$.
    A similar argument shows that if $\{A_j\}_{j=1}^\infty$ is a decreasing sequence of sets in $\mathcal{M}_\mathcal{E}$, then $\bigcap_{j=1}^\infty A_j \in \mathcal{M}_\mathcal{E}$.
    Thus $\mathcal{M}_\mathcal{E}$ is a monotone class.

    Now let $\mathcal{M}$ be a monotone class containing $\mathcal{E}$.
    Then by definition of $\mathcal{M}_\mathcal{E}$, we have $\mathcal{M}_\mathcal{E} \subseteq \mathcal{M}$.
\end{proof}

\begin{theorem}[Monotone Class Lemma]
    \label{thm:monotone_class_lemma}
    Let $X$ be a set, and let $\mathcal{A}$ be an algebra of subsets of $X$.
    Then the smallest monotone class containing $\mathcal{A}$ is the smallest $\sigma$-algebra containing $\mathcal{A}$.
\end{theorem}

\begin{proof}
    Let $\mathcal{M}_\mathcal{A}$ be the smallest monotone class containing $\mathcal{A}$.
    Since every $\sigma$-algebra is a monotone class, we have $\mathcal{M}_\mathcal{A} \subseteq \sigma(\mathcal{A})$, where $\sigma(\mathcal{A})$ is the smallest $\sigma$-algebra containing $\mathcal{A}$.
    To show the reverse inclusion, fix a set $A \in \mathcal{A}$ and let 
    \[ \mathcal{E}_A := \{ E \in \mathcal{M}_\mathcal{A} : A \cup E \in \mathcal{M}_\mathcal{A} \}. \]
    Then $\mathcal{A}\subseteq \mathcal{E}_A$ because $\mathcal{A}$ is closed under finite unions.

    We want to show that $\mathcal{E}_A$ is a monotone class.
    Let $\{E_j\}_{j=1}^\infty$ be an increasing sequence of sets in $\mathcal{E}_A$.
    Then for each $j\in\Z^+$ we have $A \cup E_j \in \mathcal{M}_\mathcal{A}$, so
    \[ A \cup \left( \bigcup_{j=1}^\infty E_j \right) = \bigcup_{j=1}^\infty (A \cup E_j) \in \mathcal{M}_\mathcal{A}. \]
    Thus $\bigcup_{j=1}^\infty E_j \in \mathcal{E}_A$.
    Since $\{E_j\}_{j=1}^\infty$ was arbitrary, we have shown that $\mathcal{E}_A$ is closed under countable increasing unions.
    Now let $\{E_j\}_{j=1}^\infty$ be a decreasing sequence of sets in $\mathcal{E}_A$.
    Then for each $j\in\Z^+$ we have $A \cup E_j \in \mathcal{M}_\mathcal{A}$, so
    \[ A \cup \left( \bigcap_{j=1}^\infty E_j \right) = \bigcap_{j=1}^\infty (A \cup E_j) \in \mathcal{M}_\mathcal{A}. \]
    Thus $\bigcap_{j=1}^\infty E_j \in \mathcal{E}_A$.
    Since $\{E_j\}_{j=1}^\infty$ was arbitrary, we have shown that $\mathcal{E}_A$ is closed under countable decreasing intersections.
    Therefore $\mathcal{E}_A$ is a monotone class containing $\mathcal{A}$, so $\mathcal{M}_\mathcal{A} \subseteq \mathcal{E}_A$ by Exercise \ref{ex:smallest_monotone_class}.
    Hence we have shown that $A \cup E \in \mathcal{M}_\mathcal{A}$ for each $E \in \mathcal{M}_\mathcal{A}$.

    Now define
    \[ \mathcal{F} := \{ F \in \mathcal{M}_\mathcal{A} : F\cup E \text{ for each } E \in \mathcal{M}_\mathcal{A} \}. \]
    Then $\mathcal{A} \subseteq \mathcal{F}$ since we have just shown that $A \cup E \in \mathcal{M}_\mathcal{A}$ for each $A \in \mathcal{A}$ and each $E \in \mathcal{M}_\mathcal{A}$.
    Again, one shows that $\mathcal{F}$ is a monotone class by a similar argument as above.
    As before, we have $\mathcal{M}_\mathcal{A} \subseteq \mathcal{F}$, so we have shown that $F \cup E \in \mathcal{M}_\mathcal{A}$ for each $E,F \in \mathcal{M}_\mathcal{A}$.

    We claim that $\mathcal{M}_\mathcal{A}$ is a $\sigma$-algebra.
    The previous paragraph shows that $\mathcal{M}_\mathcal{A}$ is closed under finite unions.
    If $\{E_j\}_{j=1}^\infty$ is a sequence of sets in $\mathcal{M}_\mathcal{A}$, then we can define a new sequence $\{F_j\}_{j=1}^\infty$ by
    \[ F_j := \bigcup_{k=1}^j E_k. \]
    Then $\{F_j\}_{j=1}^\infty$ is an increasing sequence of sets in $\mathcal{M}_\mathcal{A}$, so
    \[ \bigcup_{j=1}^\infty E_j = \bigcup_{j=1}^\infty F_j \in \mathcal{M}_\mathcal{A}. \]
    Thus $\mathcal{M}_\mathcal{A}$ is closed under countable unions.

    To see that $\mathcal{M}_\mathcal{A}$ is closed under complements, let
    \[ \mathcal{D} := \{ D \in \mathcal{M}_\mathcal{A} : D^c \in \mathcal{M}_\mathcal{A} \}. \]
    Then $\mathcal{A} \subseteq \mathcal{D}$ since $\mathcal{A}$ is closed under compliments.
    Once again, one shows that $\mathcal{D}$ is a monotone class by a similar argument as above.
    As before, we have $\mathcal{M}_\mathcal{A} \subseteq \mathcal{D}$, so we have shown that $D^c \in \mathcal{M}_\mathcal{A}$ for each $D \in \mathcal{M}_\mathcal{A}$.
    Therefore $\mathcal{M}_\mathcal{A}$ is closed under complements.
    This shows that $\mathcal{M}_\mathcal{A}$ is a $\sigma$-algebra containing $\mathcal{A}$, so $\sigma(\mathcal{A}) \subseteq \mathcal{M}_\mathcal{A}$.
\end{proof}

\subsection{Products of $\sigma$-finite Measures}

\begin{definition}[$\sigma$-finite]
    \label{def:sigma_finite}
    Let $(X,\mathcal{A},\mu)$ be a measure space.
    We say that $(X,\mathcal{A},\mu)$ is \emph{$\sigma$-finite} if there exists a countable collection of $\mu$-measurable sets $\{E_j\}_{j=1}^\infty$ such that $X = \bigcup_{j=1}^\infty E_j$ and $\mu(E_j) < \infty$ for each $j\in\Z^+$.

    \vspace{2mm}

    \noindent A subset $A\subseteq X$ is \emph{$\sigma$-finite} if there exists a countable collection of $\mu$-measurable sets $\{A_j\}_{j=1}^\infty$ such that $A = \bigcup_{j=1}^\infty A_j$ and $\mu(A_j) < \infty$ for each $j\in\Z^+$.
    
    \vspace{2mm}

    \noindent A function $f: X \to [-\infty,\infty]$ is \emph{$\sigma$-finite} if $f$ is $\mu$-measurable and the set $\{x \in X : f(x) \neq 0\}$ is $\sigma$-finite.
\end{definition}

\begin{lemma}[The Measure of Cross Sections is a Measurable Function]
    \label{lem:measure_of_cross_section_is_measurable}
    Let $(X,\mathcal{A},\mu)$ and $(Y,\mathcal{B},\nu)$ be $\sigma$-finite measure spaces. If $E \in \mathcal{A} \otimes \mathcal{B}$, then the function
    \[ x\longmapsto \nu(E_x) \]
    is $\mathcal{A}$-measurable, and the function
    \[ y\longmapsto \mu(E^y) \]
    is $\mathcal{B}$-measurable.
\end{lemma}
\begin{proof}
    If $E\in \mathcal{A} \otimes \mathcal{B}$, we know that $E_x\in \mathcal{B}$ for each $x\in X$.
    Thus the function $x \mapsto \nu(E_x)$ is well-defined on $X$.

    \vspace{2mm}
    \textit{Step 1:} Assume first that $\nu(Y) < \infty$.
    \vspace{2mm}

    Let 
    \[ \mathcal{M} := \{ E \in \mathcal{A}\otimes\mathcal{B} : x \mapsto \nu(E_x) \text{ is } \mathcal{A}\text{-measurable on } X \}. \]
    We want to show that $\mathcal{M} = \mathcal{A} \otimes \mathcal{B}$.

    If $A\in \mathcal{A}$ and $B \in \mathcal{B}$, then for each $x \in X$ we have
    \[ (A \times B)_x = \begin{cases} B & \text{if } x \in A \\ \emptyset & \text{if } x \notin A \end{cases} \]
    so that
    \[ \nu((A \times B)_x) = \begin{cases} \nu(B) & \text{if } x \in A \\ 0 & \text{if } x \notin A \end{cases} = \nu(B) \chi_A(x). \]
    In particular, the function $x\longmapsto \nu((A \times B)_x)$ is $\mathcal{A}$-measurable on $X$ since $\chi_A$ is $\mathcal{A}$-measurable.
    Therefore $A \times B \in \mathcal{M}$ for all $A \in \mathcal{A}$ and $B \in \mathcal{B}$.

    Let $\mathcal{E}$ be the collection of all finite unions of measurable products $A \times B$ with $A \in \mathcal{A}$ and $B \in \mathcal{B}$, 
    and let $E \in \mathcal{E}$.
    Then by Lemma \ref{lem:finite_union_of_measurable_products_is_algebra}, we can write $E$ as a finite union of disjoint measurable products $E = \bigcup_{j=1}^m E_j$.
    Then 
    \begin{align*}
        \nu(E_x) &= \nu\left( \left( \bigcup_{j=1}^m E_j \right)_x \right) \\
            &= \nu\left( \bigcup_{j=1}^m (E_j)_x \right) \\
            &= \sum_{j=1}^m \nu((E_j)_x)
    \end{align*}
    by disjoint additivity of $\nu$.
    Since each $E_j$ is a measurable product, we have $E_j \in \mathcal{M}$, so the function $x \mapsto \nu((E_j)_x)$ is $\mathcal{A}$-measurable on $X$ for each $j=1,2,\ldots,m$.
    Thus the function $x \mapsto \nu(E_x)$ is a finite sum of $\mathcal{A}$-measurable functions, so $x \mapsto \nu(E_x)$ is $\mathcal{A}$-measurable on $X$.
    Therefore $E \in \mathcal{M}$ for all $E \in \mathcal{E}$.
    Since $E\in \mathcal{E}$ was arbitrary, we have shown that $\mathcal{E} \subseteq \mathcal{M}$.

    Our next goal is to show that $\mathcal{M}$ is a monotone class on $X\times Y$.
    Let $\{E_j\}_{j=1}^\infty$ be an increasing sequence of sets in $\mathcal{M}$, i.e.
    \[ E_1 \subseteq E_2 \subseteq E_3 \subseteq \cdots. \]
    Then for each $x \in X$, we have
    \[ \left(\bigcup_{j=1}^\infty E_j\right)_x = \bigcup_{j=1}^\infty (E_j)_x \]
    which implies that
    \[ \nu\left( \left( \bigcup_{j=1}^\infty E_j \right)_x \right) = \nu\left( \bigcup_{j=1}^\infty (E_j)_x \right) = \lim_{j \to \infty} \nu((E_j)_x) \]
    since the sequence of sets $\{(E_j)_x\}_{j=1}^\infty$ is increasing.
    Since the pointwise limit of measurable functions is measurable, we have that the function $x \mapsto \nu\left( \left( \bigcup_{j=1}^\infty E_j \right)_x \right)$ is $\mathcal{A}$-measurable on $X$.
    Thus $\bigcup_{j=1}^\infty E_j \in \mathcal{M}$.
    This shows that $\mathcal{M}$ is closed under countable increasing unions.
    
    Now let $\{E_j\}_{j=1}^\infty$ be a decreasing sequence of sets in $\mathcal{M}$, i.e.
    \[ E_1 \supseteq E_2 \supseteq E_3 \supseteq \cdots. \]
    Then for each $x \in X$, we have
    \[ \left(\bigcap_{j=1}^\infty E_j\right)_x = \bigcap_{j=1}^\infty (E_j)_x \]
    which implies that
    \[ \nu\left( \left( \bigcap_{j=1}^\infty E_j \right)_x \right) = \nu\left( \bigcap_{j=1}^\infty (E_j)_x \right) = \lim_{j \to \infty} \nu((E_j)_x) \]
    since the sequence of sets $\{(E_j)_x\}_{j=1}^\infty$ is decreasing and $\nu(Y) < \infty$.
    Since the pointwise limit of measurable functions is measurable, we have that the function $x \mapsto \nu\left( \left( \bigcap_{j=1}^\infty E_j \right)_x \right)$ is $\mathcal{A}$-measurable on $X$.
    Thus $\bigcap_{j=1}^\infty E_j \in \mathcal{M}$.
    This shows that $\mathcal{M}$ is closed under countable decreasing intersections; hence $\mathcal{M}$ is a monotone class.

    We have shown that $\mathcal{M}$ is a monotone class that contains the algebra $\mathcal{E}$ of all finite unions of measurable products in $\mathcal{A}\times \mathcal[B]$.
    By the Monotone Class Lemma (Theorem \ref{thm:monotone_class_lemma}), the smallest $\sigma$-algebra containing $\mathcal{E}$ is the smallest monotone class containing $\mathcal{E}$, so we have
    $\mathcal{A}\otimes \mathcal{B} \subset \mathcal{M}$. 
    This completes the proof of the fact that $x \mapsto \nu(E_x)$ is $\mathcal{A}$-measurable on $X$ for each $E \in \mathcal{A} \otimes \mathcal{B}$ in the case that $\nu(Y) < \infty$.

    \vspace{2mm}
    \textit{Step 2:} Now we prove the general case.
    \vspace{2mm}

    Since $(Y,\mathcal{B},\nu)$ is $\sigma$-finite, there exists an increasing sequence $\{Y_j\}_{j=1}^\infty$ of $\nu$-measurable subsets of $Y$ such that $Y = \bigcup_{j=1}^\infty Y_j$ and $\nu(Y_j) < \infty$ for each $j\in\Z^+$.
    If $E \in \mathcal{A}\otimes \mathcal{B}$, then
    \[ \nu( E_x ) = \lim_{j \to \infty} \nu( (E \cap (X \times Y_j))_x ) \]
    for each $x \in X$ since $\{(E \cap (X \times Y_j))_x\}_{j=1}^\infty$ is an increasing sequence of sets whose union is $E_x$.
    By Step 1, for each $j\in\Z^+$ the function $x \mapsto \nu( (E \cap (X \times Y_j))_x )$ is $\mathcal{A}$-measurable on $X$ since $E \cap (X \times Y_j) \in \mathcal{A} \otimes \mathcal{B}$ and $\nu(Y_j) < \infty$.
    Since the pointwise limit of measurable functions is measurable, we have that the function $x \mapsto \nu(E_x)$ is $\mathcal{A}$-measurable on $X$.
    This completes the proof of the first part of the lemma.

    The proof of the second part is similar.
\end{proof}

At last, we are ready to define the product of two measures. 

\begin{definition}[Product of Measures]
    \label{def:product_measure}
    Let $(X,\mathcal{A},\mu)$ and $(Y,\mathcal{B},\nu)$ be $\sigma$-finite measure spaces.
    We define their \textit{product} $\mu \otimes \nu : \mathcal{A} \otimes \mathcal{B} \to [0,\infty]$ by
    \[ (\mu \otimes \nu)(E) = \int_X \int_Y \Chi_E(x,y) \, \dif \nu(y) \, \dif \mu(x). \]
\end{definition}

Notice that we need the $\sigma$-finiteness assumption to ensure that the function $x \mapsto \nu(E_x)$ is $\mathcal{A}$-measurable, so that the outer integral is well-defined.

\begin{lemma}
    \label{lem:product_of_measures_is_measure}
    Let $(X,\mathcal{A},\mu)$ and $(Y,\mathcal{B},\nu)$ be $\sigma$-finite measure spaces.
    Then the product $\mu \otimes \nu : \mathcal{A} \otimes \mathcal{B} \to [0,\infty]$ is a measure on $X \times Y$.
\end{lemma}

\begin{proof}
    Clearly we have $(\mu \otimes \nu)(\emptyset) = 0$ since $\chi_\emptyset(x,y) = 0$ for all $(x,y) \in X \times Y$.
    Now let $\{E_j\}_{j=1}^\infty$ be a sequence of disjoint sets in $\mathcal{A} \otimes \mathcal{B}$.
    Then we have
    \begin{align*}
        (\mu\otimes \nu)\left( \bigcup_{j=1}^\infty E_j \right) &= \int_X \int_Y \Chi_{\bigcup_{j=1}^\infty E_j}(x,y) \, \dif \nu(y) \, \dif \mu(x) \\
            &= \int_X \nu\left( \bigcup_{j=1}^\infty (E_j)_x \right) \, \dif \mu(x) \\
            &= \int_X \sum_{j=1}^\infty \nu((E_j)_x) \, \dif \mu(x) \\
            &= \sum_{j=1}^\infty \int_X \nu((E_j)_x) \, \dif \mu(x) \\
            &= \sum_{j=1}^\infty \int_X \int_Y \Chi_{E_j}(x,y) \, \dif \nu(y) \, \dif \mu(x) \\
            &= \sum_{j=1}^\infty (\mu \otimes \nu)(E_j)
    \end{align*}
    where the interchange of the sum and integral is justified by the Monotone Convergence Theorem (\ref{thm:monotone_convergence_theorem}).
    Thus $\mu \otimes \nu$ is countably additive, so $\mu \otimes \nu$ is a measure on $X \times Y$.
\end{proof}

\subsection{Tonelli's Theorem and Fubini's Theorem}

We have built up all the machinery we need to state and prove Tonelli's theorem and Fubini's theorem.

\begin{theorem}[Tonelli's Theorem]
    \label{thm:tonelli_theorem}
    Let $(X,\mathcal{A},\mu)$ and $(Y,\mathcal{B},\nu)$ be $\sigma$-finite measure spaces, and let $f: X \times Y \to [0,\infty]$ be a nonnegative $\mathcal{A} \otimes \mathcal{B}$-measurable function.
    Then
    \[ x \longmapsto \int_Y f(x,y) \, \dif\nu(y) \qquad\text{is a }\mu\text{-measurable function on }X, \]
    \[ y \longmapsto \int_X f(x,y) \, \dif\mu(x) \qquad\text{is a }\nu\text{-measurable function on }Y, \]
    and
    \[ \int_{X \times Y} f \, \dif(\mu \otimes \nu) = \int_X \int_Y f(x,y) \, \dif\nu(y) \, \dif\mu(x) = \int_Y \int_X f(x,y) \, \dif\mu(x) \, \dif\nu(y). \]
\end{theorem}

We want to notice a few things --- first, the spaces we integrate over must be $\sigma$-finite.
Second, the function $f$ must be nonnegative. Third is that we make \emph{no} integrability assumptions on $f$.
This is a very powerful theorem.

\begin{proof}
    \textit{Step 0:} First consider the special case where $f = \Chi_E$ for some $E \in \mathcal{A} \otimes \mathcal{B}$, and assume that $\mu(X) < \infty$ and $\nu(Y) < \infty$.
    \vspace{2mm}

    Let $E \in \mathcal{A} \otimes \mathcal{B}$ be arbitrary. 
    Then we have
    \[ \int_Y \Chi_E(x,y) \, \dif\nu(y) = \nu(E_x) \]
    for each $x \in X$, and similarly
    \[ \int_X \Chi_E(x,y) \, \dif\mu(x) = \mu(E^y) \]
    for each $y \in Y$.
    By Lemma \ref{lem:measure_of_cross_section_is_measurable}, the functions $x \mapsto \nu(E_x)$ and $y \mapsto \mu(E^y)$ are measurable.
    Thus, in this case, the first two conclusions of Tonelli's theorem hold.

    Now let
    \[ \mathcal{M} := \left\{ E \in \mathcal{A} \otimes \mathcal{B} : \int_X \int_Y \Chi_E(x,y) \, \dif\nu(y) \, \dif\mu(x) = \int_Y \int_X \Chi_E(x,y) \, \dif\mu(x) \, \dif\nu(y) \right\}. \]
    If $A \in \mathcal{A}$ and $B \in \mathcal{B}$, then we have $A \times B \in \mathcal{M}$ since
    \begin{align*}
        \int_X \int_Y \Chi_{A \times B}(x,y) \, \dif\nu(y) \, \dif\mu(x) &= \int_X \int_Y \Chi_A(x) \Chi_B(y) \, \dif\nu(y) \, \dif\mu(x) \\
            &= \int_X \Chi_A(x) \left( \int_Y \Chi_B(y) \, \dif\nu(y) \right) \, \dif\mu(x) \\
            &= \nu(B) \int_X \Chi_A(x) \, \dif\mu(x) \\
            &= \nu(B) \mu(A)
    \end{align*}
    and similarly the other integral equals $\mu(A) \nu(B)$.
    Thus $A \times B \in \mathcal{M}$ for all $A \in \mathcal{A}$ and $B \in \mathcal{B}$.

    Let $\mathcal{E}$ be the collection of all finite unions of measurable products $A \times B$ with $A \in \mathcal{A}$ and $B \in \mathcal{B}$.
    Then by Lemma \ref{lem:finite_union_of_measurable_products_is_algebra}, we can write an arbitrary $E \in \mathcal{E}$ as a finite union of disjoint measurable products $E = \bigcup_{j=1}^m E_j$.
    Then by disjoint additivity of the integral, we have
    \begin{align*}
        \int_X \int_Y \Chi_E(x,y) \, \dif\nu(y) \, \dif\mu(x) &= \int_X \int_Y \Chi_{\bigcup_{j=1}^m E_j}(x,y) \, \dif\nu(y) \, \dif\mu(x) \\
            &= \int_X \int_Y \sum_{j=1}^m \Chi_{E_j}(x,y) \, \dif\nu(y) \, \dif\mu(x) \\
            &= \sum_{j=1}^m \int_X \int_Y \Chi_{E_j}(x,y) \, \dif\nu(y) \, \dif\mu(x) \\
            &= \sum_{j=1}^m \int_Y \int_X \Chi_{E_j}(x,y) \, \dif\mu(x) \, \dif\nu(y) \\
            &= \int_Y \int_X \Chi_{\bigcup_{j=1}^m E_j}(x,y) \, \dif\mu(x) \, \dif\nu(y) \\
            &= \int_Y \int_X \Chi_E(x,y) \, \dif\mu(x) \, \dif\nu(y).
    \end{align*}
    Thus $E \in \mathcal{M}$.
    Since $E\in \mathcal{E}$ was arbitrary, we have shown that $\mathcal{E} \subseteq \mathcal{M}$.

    Now the Monotone Convergence Theorem (\ref{thm:monotone_convergence_theorem}) shows that $\mathcal{M}$ is closed under countable increasing unions.
    The Bounded Convergence Theorem (\ref{thm:bounded_convergence_theorem}) shows that $\mathcal{M}$ is closed under countable decreasing intersections --- note that \emph{this} is where we need the finiteness assumptions on $\mu(X)$ and $\nu(Y)$.
    Thus $\mathcal{M}$ is a monotone class containing the algebra $\mathcal{E}$ of all finite unions of measurable products in $\mathcal{A}\times \mathcal{B}$.
    By the Monotone Class Lemma (Theorem \ref{thm:monotone_class_lemma}), the smallest $\sigma$-algebra containing $\mathcal{E}$ is the smallest monotone class containing $\mathcal{E}$, so we have
    $\mathcal{A}\otimes \mathcal{B} \subset \mathcal{M}$. 
    That is, for each $E \in \mathcal{A} \otimes \mathcal{B}$ we have
    \[ \int_X \int_Y \Chi_E(x,y) \, \dif\nu(y) \, \dif\mu(x) = \int_Y \int_X \Chi_E(x,y) \, \dif\mu(x) \, \dif\nu(y). \]
    Since the integral on the left-hand side is precisely $(\mu \otimes \nu)(E)$ by Definition \ref{def:product_measure}, we have
    \[ (\mu \otimes \nu)(E) = \int_{X\times Y} \Chi_E \, \dif(\mu \otimes \nu) = \int_X \int_Y \Chi_E(x,y) \, \dif\nu(y) \, \dif\mu(x) = \int_Y \int_X \Chi_E(x,y) \, \dif\mu(x) \, \dif\nu(y). \]
    This completes the proof of Tonelli's theorem in the special case where $f = \Chi_E$ for some $E \in \mathcal{A} \otimes \mathcal{B}$ and $\mu(X) < \infty$ and $\nu(Y) < \infty$.

    \vspace{2mm}
    \textit{Step 1:} Now consider the case where $f = \Chi_E$ for some $E \in \mathcal{A} \otimes \mathcal{B}$, but we do not assume that $\mu(X) < \infty$ or $\nu(Y) < \infty$.
    \vspace{2mm}

    Since $(X,\mathcal{A},\mu)$ and $(Y,\mathcal{B},\nu)$ are $\sigma$-finite, there exist increasing sequences $\{X_j\}_{j=1}^\infty$ and $\{Y_k\}_{k=1}^\infty$ of measurable subsets of $X$ and $Y$, respectively, such that $X = \bigcup_{j=1}^\infty X_j$ and $Y = \bigcup_{k=1}^\infty Y_k$ and $\mu(X_j) < \infty$ and $\nu(Y_k) < \infty$ for each $j,k\in\Z^+$.
    
    Let $E \in \mathcal{A} \otimes \mathcal{B}$.
    For each $j,k \in \Z^+$, we apply the previous case to the finite measures $\mu\mres_{X_j}$ and $\nu\mres_{Y_k}$ to conclude that
    \[  \int_{X_j} \int_{Y_k} \Chi_E(x,y) \, \dif\nu(y) \, \dif\mu(x) = \int_{Y_k} \int_{X_j} \Chi_E(x,y) \, \dif\mu(x) \, \dif\nu(y). \]
    Now by the Monotone Convergence Theorem (\ref{thm:monotone_convergence_theorem}), we have
    \begin{align*}
        \int_X \int_Y \Chi_E(x,y) \, \dif\nu(y) \, \dif\mu(x) &= \lim_{j \to \infty} \lim_{k \to \infty} \int_{X_j} \int_{Y_k} \Chi_E(x,y) \, \dif\nu(y) \, \dif\mu(x) \\
            &= \lim_{j \to \infty} \lim_{k \to \infty} \int_{Y_k} \int_{X_j} \Chi_E(x,y) \, \dif\mu(x) \, \dif\nu(y) \\
            &= \int_Y \int_X \Chi_E(x,y) \, \dif\mu(x) \, \dif\nu(y).
    \end{align*}
    Now the definition of the product measure (Definition \ref{def:product_measure}) shows that
    \[ (\mu \otimes \nu)(E) = \int_X \int_Y \Chi_E(x,y) \, \dif\nu(y) \, \dif\mu(x). \]
    Putting this together, we have
    \[ \int_{X \times Y} \Chi_E \, \dif(\mu \otimes \nu) = \int_X \int_Y \Chi_E(x,y) \, \dif\nu(y) \, \dif\mu(x) = \int_Y \int_X \Chi_E(x,y) \, \dif\mu(x) \, \dif\nu(y). \]
    Since $E \in \mathcal{A} \otimes \mathcal{B}$ was arbitrary, this completes the proof of Tonelli's theorem in the case where $f = \Chi_E$ for some $E \in \mathcal{A} \otimes \mathcal{B}$.

    \vspace{2mm}
    \textit{Step 2:} Now consider the general case.
    \vspace{2mm}

    Let $f: X \times Y \to [0,\infty]$ be a nonnegative $\mathcal{A} \otimes \mathcal{B}$-measurable function.
    Then we define a sequence of nonnegative simple functions $\{s_k\}_{k=1}^\infty$ on $X \times Y$ by
    \[ s_k(x,y) := \begin{cases}
        \frac{m}{2^k}, & \text{if } f(x,y) < k \text{ and } \frac{m}{2^k} \leq f(x,y) < \frac{m+1}{2^k} \text{ for some } m = 0,1,2,\ldots,2^{2k}k - 1 \\
        k, & \text{if } f(x,y) \geq k \\
    \end{cases} \]
    for each $k\in\Z^+$.

    Note that by construction, we have 
    \[ 0 \leq s_k(x,y) \leq s_{k+1}(x,y) \leq f(x,y) \]
    for all $(x,y) \in X \times Y$ and $k\in\Z^+$, and
    \[ \lim_{k \to \infty} s_k(x,y) = f(x,y) \]
    for all $(x,y) \in X \times Y$.
    Notice that each $s_k$ is a finite linear combination of characteristic functions of measurable sets, so by the previous step we know Tonelli's theorem holds for each $s_k$.
    Now the Monotone Convergence Theorem (\ref{thm:monotone_convergence_theorem}) shows that
    \[ \int_Y f(x,y) \, \dif\nu(y) = \lim_{k \to \infty} \int_Y s_k(x,y) \, \dif\nu(y) \]
    for each $x \in X$. Thus the function 
    \[ x \longmapsto \int_Y f(x,y) \, \dif\nu(y) \]
    is the pointwise limit of a sequence of $\mathcal{A}$-measurable functions, so it is $\mathcal{A}$-measurable on $X$.
    Similarly the map
    \[ y \longmapsto \int_X f(x,y) \, \dif\mu(x) \]
    is $\mathcal{B}$-measurable on $Y$.

    Finally, another application of the Monotone Convergence Theorem (\ref{thm:monotone_convergence_theorem}) shows that
    \begin{align*}
        \int_{X \times Y} f \, \dif(\mu \otimes \nu) &= \lim_{k \to \infty} \int_{X \times Y} s_k \, \dif(\mu \otimes \nu) \\
            &= \lim_{k \to \infty} \int_X \int_Y s_k(x,y) \, \dif\nu(y) \, \dif\mu(x) \\
            &= \int_X \int_Y f(x,y) \, \dif\nu(y) \, \dif\mu(x)
    \end{align*}
    and similarly
    \[ \int_{X \times Y} f \, \dif(\mu \otimes \nu) = \int_Y \int_X f(x,y) \, \dif\mu(x) \, \dif\nu(y). \]
    This completes the proof of Tonelli's theorem.
\end{proof}

\begin{corollary}
    \label{cor:double_indexed_sum}
    Let $\{a_{i,j}\}_{i,j=1}^\infty$ be a doubly-indexed collection of nonnegative numbers.
    Then
    \[ \sum_{i,j=1}^\infty a_{i,j} = \sum_{i=1}^\infty \sum_{j=1}^\infty a_{i,j} = \sum_{j=1}^\infty \sum_{i=1}^\infty a_{i,j}. \]
\end{corollary}
\begin{proof}
    Apply Tonelli's theorem with $X = Y = \Z^+$, $\mu = \nu$ the counting measure, and $f(i,j) = a_{ij}$.
\end{proof}

\begin{theorem}[Fubini's Theorem]
    \label{thm:fubini_theorem}
    Let $(X,\mathcal{A},\mu)$ and $(Y,\mathcal{B},\nu)$ be $\sigma$-finite measure spaces, and let $f: X \times Y \to [-\infty,\infty]$ be an $\mathcal{A} \otimes \mathcal{B}$-integrable function.
    Then
    \[ x \longmapsto \int_Y f(x,y) \, \dif\nu(y) \qquad\text{is a }\mu\text{-measurable function on }X, \]
    \[ y \longmapsto \int_X f(x,y) \, \dif\mu(x) \qquad\text{is a }\nu\text{-measurable function on }Y, \]
    and \[ \int_Y |f(x,y)| \, \dif\nu(y) < \infty \text{ for } \mu\text{-almost every } x \in X \quad\text{ and }\quad \int_X |f(x,y)| \, \dif\mu(x) < \infty \text{ for }\nu\text{-almost every } y \in Y, \]
    and
    \[ \int_{X \times Y} f \, \dif(\mu \otimes \nu) = \int_X \int_Y f(x,y) \, \dif\nu(y) \, \dif\mu(x) = \int_Y \int_X f(x,y) \, \dif\mu(x) \, \dif\nu(y). \]
\end{theorem}

We remark a few things --- first, the spaces we integrate over must be $\sigma$-finite.
Second, the function $f$ can take on negative values, but we must assume that $f$ is integrable, i.e. that $\int_{X \times Y} |f| \, \dif(\mu \otimes \nu) < \infty$.
Third is that we have the same conclusion as Tonelli's theorem, but we also have that the iterated integrals of $|f|$ are finite almost everywhere.

\begin{remark}[Using Tonelli and Fubini in Practice]
    \label{rmk:using_tonelli_and_fubini_in_practice}
    Here's the trick. Assume that you have a function $f$ defined on a product space $X\times Y$ and you want to compute its integral.
    What you should do is apply Tonelli's theorem to $|f|$ and show that the integral $\int_{X \times Y} |f| \, \dif(\mu \otimes \nu)$ is finite by evaluating one of the iterated integrals
    given in Tonelli's theorem.
    This shows that $f$ is integrable on $X \times Y$.
    Then you can apply Fubini's theorem to $f$ to compute the integral $\int_{X \times Y} f \, \dif(\mu \otimes \nu)$ by evaluating either of the iterated integrals given in Fubini's theorem.

    In practice, this combo is referred to as the Fubini-Tonelli theorem or just Fubini's theorem.
\end{remark}

\begin{proof}
    Applying Tonelli's theorem (Theorem \ref{thm:tonelli_theorem}) to the nonnegative $\mathcal{A} \otimes \mathcal{B}$-measurable function $|f|$, we have
    \[ x \longmapsto \int_Y |f(x,y)| \, \dif\nu(y) \qquad\text{is a }\mu\text{-measurable function on }X, \]
    so the set \[ \left\{ x\in X : \int_Y |f(x,y)| \, \dif\nu(y) = \infty \right\} \]
    is in $\mathcal{A}$.
    Since $f$ is integrable on $X \times Y$, Tonelli's theorem and the assumption $f$ is integrable shows that
    \[ \int_{X \times Y} |f| \, \dif(\mu \otimes \nu) = \int_X \int_Y |f(x,y)| \, \dif\nu(y) \, \dif\mu(x) < \infty. \]
    Thus we must have
    \[ \mu\left( \left\{ x\in X : \int_Y |f(x,y)| \, \dif\nu(y) = \infty \right\} \right) = 0. \]

    Recall the definitions 
    \[ f^+ = \max(f,0) \quad\text{and}\quad f^- = \max(-f,0) \]
    which satisfy $f = f^+ - f^-$ and $|f| = f^+ + f^-$.
    Applying Tonelli's theorem to the nonnegative $\mathcal{A} \otimes \mathcal{B}$-measurable functions $f^+$ and $f^-$ shows that
    \[ x \longmapsto \int_Y f^+(x,y) \, \dif\nu(y) \qquad\text{and}\qquad x \longmapsto \int_Y f^-(x,y) \, \dif\nu(y) \]
    are $\mu$-measurable functions on $X$. 
    Since $f^+ \leq |f|$ and $f^- \leq |f|$, the sets \[ \left\{ x\in X : \int_Y f^+(x,y) \, \dif\nu(y) = \infty \right\} \quad\text{and}\quad \left\{ x\in X : \int_Y f^-(x,y) \, \dif\nu(y) = \infty \right\} \]
    must also have $\mu$-measure zero.
    The intersection of these two sets is \[ \left\{ x\in X : \int_Y |f(x,y)| \, \dif\nu(y) = \infty \right\}, \]
    which is exacly the set on which the integral $\int_Y f(x,y) \, \dif\nu(y)$ fails to be defined.
    Thus we have shown that
    \[ \int_Y |f(x,y)| \, \dif\nu(y) < \infty \text{ for } \mu\text{-almost every } x \in X. \]

    Now the function 
    \[ x \longmapsto \int_Y f(x,y) \, \dif\nu(y) = \int_Y f^+(x,y) \, \dif\nu(y) - \int_Y f^-(x,y) \, \dif\nu(y) \]
    which must be modified to take the valued $0$ on if the right side is of the form $\infty - \infty$ (but this happens on a set of $\mu$-measure zero, as we just showed), is the difference of two $\mu$-measurable functions, so it is $\mu$-measurable on $X$.
    We compute that
    \begin{align*}
        \int_{X\times Y} f \dif (\mu\otimes \nu) &= \int_{X \times Y} f^+ \, \dif(\mu \otimes \nu) - \int_{X \times Y} f^- \, \dif(\mu \otimes \nu) \\
            &= \int_X \int_Y f^+(x,y) \, \dif\nu(y) \, \dif\mu(x) - \int_X \int_Y f^-(x,y) \, \dif\nu(y) \, \dif\mu(x) \\
            &= \int_X \int_Y (f^+(x,y) - f^-(x,y)) \, \dif\nu(y) \, \dif\mu(x) \\
            &= \int_X \int_Y f(x,y) \, \dif\nu(y) \, \dif\mu(x)
    \end{align*}
    where we have used the definition of the integral of a real-valued function and Tonelli's theorem on $f^+$ and $f^-$.

    This proves the first iterated integral formula in Fubini's theorem.

    The assertions about the $\nu$-measurability of the function $y \mapsto \int_X f(x,y) \, \dif\mu(x)$ and the second iterated integral formula in Fubini's theorem follow by a symmetric argument.
\end{proof}

We summarize the technique described in Remark \ref{rmk:using_tonelli_and_fubini_in_practice} in the following result.
\begin{corollary}[The Fubini-Tonelli Theorem]
    \label{cor:fubini_tonelli_theorem}
    Let $(X,\mathcal{A},\mu)$ and $(Y,\mathcal{B},\nu)$ be $\sigma$-finite measure spaces, and let $f: X \times Y \to [-\infty,\infty]$ be an $\mathcal{A} \otimes \mathcal{B}$-measurable function.
    If either
    \[ \int_X \int_Y |f(x,y)| \, \dif\nu(y) \, \dif\mu(x) < \infty \quad\text{ or }\quad \int_Y \int_X |f(x,y)| \, \dif\mu(x) \, \dif\nu(y) < \infty , \]
    then $f$ is integrable on $X \times Y$, and
    \[ \int_{X \times Y} f \, \dif(\mu \otimes \nu) = \int_X \int_Y f(x,y) \, \dif\nu(y) \, \dif\mu(x) = \int_Y \int_X f(x,y) \, \dif\mu(x) \, \dif\nu(y). \]
\end{corollary}
\begin{proof}
    The function $|f|$ is nonnegative and $\mathcal{A} \otimes \mathcal{B}$-measurable, so we can apply Tonelli's theorem to $|f|$ to conclude that
    \[ \int_{X \times Y} |f| \, \dif(\mu \otimes \nu) = \int_X \int_Y |f(x,y)| \, \dif\nu(y) \, \dif\mu(x) = \int_Y \int_X |f(x,y)| \, \dif\mu(x) \, \dif\nu(y). \]
    If either of the iterated integrals on the right-hand side is finite, then $\int_{X \times Y} |f| \, \dif(\mu \otimes \nu) < \infty$, so $f$ is integrable on $X \times Y$.
    We can then apply Fubini's theorem to $f$ to conclude the desired result.
\end{proof}

\begin{exercise}[Area Under Graph Formula]
    \label{ex:area_under_graph_formula}
    Let $(X,\mathcal{A},\mu)$ be a $\sigma$-finite measure space and let $f: X \to [0,\infty]$ be a $\mu$-measurable function.
    Then \[ \int_X f \, d\mu = \int_0^\infty \mu(\{x \in X : t < f(x) \}) \, \dif t. \]
\end{exercise}

\begin{proof}
    Let $\mathcal{B}_\R$ be the Borel $\sigma$-algebra on $\R$.

    For each $k\in \Z^+$ let
    \[ E_k := \bigcup_{m=0}^{k^2-1} \left( f^{-1}\left( \left[ \frac{m}{k}, \frac{m+1}{k} \right) \right) \right) \quad\text{ and }\quad F_k := f^{-1}\left( [k, \infty) \right) \times(0,k). \]
    The for each $k\in\Z^+$ the set $E_k$ is a finite union of measurable products in $\mathcal{A} \times \mathcal{B}_\R$, and $F_k$ is a measurable product in $\mathcal{A} \times \mathcal{B}_\R$, so both $E_k$ and $F_k$ are in $\mathcal{A} \otimes \mathcal{B}_\R$.
    Since \[ \{ (x,t) \in X \times \R : 0 < t < f(x) \} = \bigcup_{k=1}^\infty (E_k \cup F_k), \]
    the set \(\{ (x,t) \in X \times \R : 0 < t < f(x) \}\) is in \(\mathcal{A} \otimes \mathcal{B}_\R\).

    The definition of the product measure $\mu \otimes \mathcal{L}^1$ then gives
    \begin{align*}
        (\mu \otimes \mathcal{L}^1) \left( \{ (x,t) \in X \times \R : 0 < t < f(x) \} \right) &= \int_X \int_\R \Chi_{\{ (x,t) : 0 < t < f(x) \}}(x,t) \, \dif \mathcal{L}^1(t) \, \dif \mu(x) \\
            &= \int_X \int_0^{f(x)} 1 \, \dif t \, \dif \mu(x) \\
            &= \int_X f(x) \, \dif \mu(x).
    \end{align*}
    This formula shows that the measure of the set $\{ (x,t) \in X \times \R : 0 < t < f(x) \}$ with respect to the product measure \(\mu \otimes \mathcal{L}^1 \) is equal to the integral of $f$ over $X$.
    That is, the ``area under the graph'' of $f$ is equal to the integral of $f$.

    On the other hand, Tonelli's theorem (Theorem \ref{thm:tonelli_theorem}) says that we chan interchange the order of integration to get
    \begin{align*}
        (\mu \otimes \mathcal{L}^1) \left( \{ (x,t) \in X \times \R : 0 < t < f(x) \} \right) &= \int_0^\infty \int_X \Chi_{\{ (x,t) : 0 < t < f(x) \}}(x,t) \, \dif \mu(x) \, \dif \mathcal{L}^1(t) \\
            &= \int_0^\infty \mu(\{ x \in X : t < f(x) \}) \, \dif t.
    \end{align*}
    Combinging the two formulas gives
    \[ \int_X f(x) \, \dif \mu(x) = \int_0^\infty \mu(\{ x \in X : t < f(x) \}) \, \dif t, \]
    which is the desired result.
\end{proof}

\subsection{Notation for Multiple Integrals in $\R^n$}

\begin{lemma}[Another Definition of Lebesgue Measure]
    \label{lem:another_definition_of_lebesgue_measure}
    For each $n>1$, the Lebesgue measure $\mathcal{L}^n$ on $\R^n$ is the product measure of $(\R^{n-1},\mathcal{L}^{n-1})$ and $(\R,\mathcal{L}^1)$, i.e.
    \[ \mathcal{L}^n = \mathcal{L}^{n-1} \otimes \mathcal{L}^1. \]
\end{lemma}
\begin{proof}
    Fix $n>1$. Then the measure spaces $(\R^{n-1},\mathcal{L}^{n-1})$ and $(\R,\mathcal{L}^1)$ are $\sigma$-finite since $\R^{n-1}$ and $\R$ can both be written as a countable union of finite measure sets.
    Thus the product measure $\mathcal{L}^{n-1} \otimes \mathcal{L}^1$ is well-defined on $\R^{n-1} \times \R = \R^n$.

    By Cor \ref{cor:borel_reg_outer_measures_in_rn}, we only need to show that $\mathcal{L}^n$ and $\mathcal{L}^{n-1} \otimes \mathcal{L}^1$ agree on all boxes in $\R^n$.
    Let $R = [a_1,b_1] \times [a_2,b_2] \times \cdots \times [a_n,b_n]$ be a box in $\R^n$.
    Then
    \begin{align*}
        (\mathcal{L}^{n-1} \otimes \mathcal{L}^1)(R) &= \int_{\R^{n-1}} \int_\R \Chi_R(x_1,x_2,\ldots,x_n) \, \dif \mathcal{L}^1(x_n) \, \dif \mathcal{L}^{n-1}(x_1,x_2,\ldots,x_{n-1}) \\
            &= \int_{\R^{n-1}} \int_{a_n}^{b_n} \Chi_{[a_1,b_1] \times [a_2,b_2] \times \cdots \times [a_{n-1},b_{n-1}]}(x_1,x_2,\ldots,x_{n-1}) \, \dif x_n \, \dif \mathcal{L}^{n-1}(x_1,x_2,\ldots,x_{n-1}) \\
            &= \int_{a_n}^{b_n} \dif x_n\,\cdot \int_{\R^{n-1}} \Chi_{[a_1,b_1] \times [a_2,b_2] \times \cdots \times [a_{n-1},b_{n-1}]}(x_1,x_2,\ldots,x_{n-1}) \, \dif \mathcal{L}^{n-1}(x_1,x_2,\ldots,x_{n-1}) \\
            &= (b_n - a_n) \int_{\R^{n-1}} \Chi_{[a_1,b_1] \times [a_2,b_2] \times \cdots \times [a_{n-1},b_{n-1}]}(x_1,x_2,\ldots,x_{n-1}) \, \dif \mathcal{L}^{n-1}(x_1,x_2,\ldots,x_{n-1}) \\
            &= (b_n - a_n) \mathcal{L}^{n-1}([a_1,b_1] \times [a_2,b_2] \times \cdots \times [a_{n-1},b_{n-1}]) \\
            &= (b_n - a_n) (b_1 - a_1)(b_2 - a_2) \cdots (b_{n-1} - a_{n-1}) \\
            &= (b_1 - a_1)(b_2 - a_2) \cdots (b_n - a_n) = \mathcal{L}^n(R)
    \end{align*}
    so we are done. 
\end{proof}

By this result and induction, if $n>1$ and we have an integer $1 \leq k < n$, then
\[ \mathcal{L}^n = \mathcal{L}^{n-k} \otimes \mathcal{L}^{k}. \]
Then we can apply Tonelli's theorem and Fubini's theorem to functions defined on $\R^n$ to get formulas such as
\[ \int_{\R^k \times \R^{n-k}} f \, \dif x = \int_{\R^k} \int_{\R^{n-k}} f(x_1,x_2) \, \dif \mathcal{L}^{n-k}(x_2) \, \dif \mathcal{L}^k(x_1) = \int_{\R^{n-k}} \int_{\R^k} f(x_1,x_2) \, \dif \mathcal{L}^k(x_1) \, \dif \mathcal{L}^{n-k}(x_2). \]
for integrable $f: \R^n\to [-\infty,\infty]$.
We remark that you should be careful in using the above formula, since we have $x_1 \in \R^k$ and $x_2 \in \R^{n-k}$ which is at odds with the usual notation.

To avoid confusion, we usually write $x \in \R^n$ as $x = (x_1,x_2,\ldots,x_n)$ where each $x_j \in \R$.
Then Fubini-Tonelli takes the form
\[ \int_{\R^n} f(x) \, \dif x = \int_{\R^k} \int_{\R^{n-k}} f(x_1,x_2,\ldots,x_n) \, \dif x_{k+1} \cdots \dif x_n \, \dif x_1 \cdots \dif x_k = \int_{\R^{n-k}} \int_{\R^k} f(x_1,x_2,\ldots,x_n) \, \dif x_1 \cdots \dif x_k \, \dif x_{k+1} \cdots \dif x_n. \]
for integrable $f: \R^n \to [-\infty,\infty]$, which we think is better (because it is more explicit and conforms to the usual notation), even if it is longer to write.

The next section will be devoted to examples and applications of Fubini-Tonelli theorems in $\R^n$.
