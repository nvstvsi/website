\documentclass[BV_Finite_Perimeter.tex]{subfiles}

\begin{document}
\section{The Hausdorff Outer Measure}

In this section, we explore our second concrete example of an outer measure, the Hausdorff outer measure.
This outer measure is defined on any metric space, but we are mostly interested in the case of $\R^n$. 

\vspace{5mm}

\noindent Throughout this section, let $X$ be a metric space with metric $d$, and we let
\[ \omega_\alpha :=  \frac{\pi^{\alpha/2}}{\Gamma\left( \frac{\alpha}{2} + 1 \right)} \]
for each $\alpha \geq 0$.

Recall that for an integer $n\geq 1$, the constant $\omega_n$ is the volume (Lebesgue outer measure) of the unit ball in $\R^n$ --- see Exercise \ref{ex:volume_of_n_dimensional_ball_formula}.

\subsection{Definition of the Hausdorff Outer Measure}

\begin{definition}[Hausdorff Outer Measure]
    \label{def:hausdorff_outer_measure}
    Let $\alpha \geq 0$.
    For each $\delta > 0$ and $A \subseteq X$, we define the \textit{$\delta$-approximated $\alpha$-dimensional Hausdorff outer measure} of $A$ by
    \[ \mathcal{H}^\alpha_\delta(A) := \omega_\alpha \inf \left\{ \sum_{j=1}^\infty \left( \frac{\diam(C_j)}{2}  \right)^\alpha :  A \sub \bigcup_{j=1}^\infty C_j, \,\text{and } \diam(C_j) \leq \delta \text{ for all } j\geq 1 \right\}. \]
    If there is no such cover of $A$ with sets of diameter less than $\delta$, we set $\mathcal{H}^\alpha_\delta(A) := \infty$.

    \vspace{2mm}

    \noindent The \textit{$\alpha$-dimensional Hausdorff outer measure} of $A$ is then defined by
    \[ \mathcal{H}^\alpha(A) := \lim_{\delta \to 0^+} \mathcal{H}^\alpha_\delta(A) \in [0, \infty]. \]
\end{definition}

If we are dealing with two metric spaces, we may add a subscript to clearly indicate which metric space we are considering the Hausdorff outer measure on, for instance $\mathcal{H}^\alpha_{\delta,X}$ or $\mathcal{H}^\alpha_X$.

\begin{remark}[Well-Definedness of the Hausdorff Outer Measure]
    \label{rmk:def_of_hausdorff_outer_measure}
    Quick sanity check: if $\delta_1 < \delta_2$, then $\mathcal{H}^\alpha_{\delta_1}(A) \geq \mathcal{H}^\alpha_{\delta_2}(A)$ because the infimum is taken over a smaller set of covers.
    Thus the function $\delta \mapsto \mathcal{H}^\alpha_\delta(A)$ is decreasing, so the limit as $\delta \to 0^+$ is well-defined (possibly infinite).

    Also we can without loss of generality restrict to covers by closed bounded sets --- for each set $C_j$, we have $\diam(C_j) = \diam(\overline{C_j})$, so we can replace $C_j$ by its closure without changing the diameter.
\end{remark}

\begin{theorem}
    \label{thm:hausdorff_outer_measure_is_outer_measure}
    Let $\alpha \geq 0$.
    Then $\mathcal{H}^\alpha$ is a Borel regular outer measure on $X$.
\end{theorem}

\begin{proof}
    \textit{Step 1:} Let $\delta > 0$.
    We claim that $\mathcal{H}^\alpha_\delta$ is an outer measure on $X$.
    \vspace{2mm}

    It is clear that $\mathcal{H}^\alpha_\delta(\emptyset) = 0$, since a set of arbitrarily small diameter covers the empty set.
    Let $A \sub B \sub X$.
    Then any cover of $B$ by sets of diameter at most $\delta$ is also a cover of $A$ by sets of diameter at most $\delta$, so $\mathcal{H}^\alpha_\delta(A) \leq \mathcal{H}^\alpha_\delta(B)$.
    Finally, let $\{A_j\}_{j=1}^\infty$ be a countable collection of subsets of $X$.
    For each $j \geq 1$, let $\{U_{j,k}\}_{k=1}^\infty$ be a cover of $A_j$ by sets of diameter at most $\delta$. Then $\{U_{j,k} : j,k \geq 1\}$ is a cover of $\bigcup_{j=1}^\infty A_j$ by sets of diameter at most $\delta$, so that
    \[ \mathcal{H}^\alpha_\delta\left( \bigcup_{j=1}^\infty A_j \right) \leq \omega_\alpha \sum_{j=1}^\infty \sum_{k=1}^\infty (\diam(U_{j,k})/2)^\alpha \]
    Taking the infimum over all such covers of each $A_j$ gives
    \[ \mathcal{H}^\alpha_\delta\left( \bigcup_{j=1}^\infty A_j \right) \leq \sum_{j=1}^\infty \mathcal{H}^\alpha_\delta(A_j). \]
    Thus $\mathcal{H}^\alpha_\delta$ is an outer measure on $X$.

    \vspace{2mm}
    \textit{Step 2:} Now we show that $\mathcal{H}^\alpha$ is an outer measure on $X$.
    \vspace{2mm}

    We have \[ \mathcal{H}^\alpha(\emptyset) = \lim_{\delta \to 0^+} \mathcal{H}^\alpha_\delta(\emptyset) = 0. \]
    Also if $A \sub B \sub X$, then $\mathcal{H}^\alpha_\delta(A) \leq \mathcal{H}^\alpha_\delta(B)$ for each $\delta > 0$, so taking limits as $\delta \to 0^+$ gives $\mathcal{H}^\alpha(A) \leq \mathcal{H}^\alpha(B)$.

    Finally, let $\{A_j\}_{j=1}^\infty$ be a countable collection of subsets of $X$.
    Then for each $\delta > 0$ we have
    \[ \mathcal{H}^\alpha_\delta\left( \bigcup_{j=1}^\infty A_j \right) \leq \sum_{j=1}^\infty \mathcal{H}^\alpha_\delta(A_j) \leq \sum_{j=1}^\infty \mathcal{H}^\alpha_\delta(A_j). \]
    Taking limits as $\delta \to 0^+$ gives
    \[ \mathcal{H}^\alpha\left( \bigcup_{j=1}^\infty A_j \right) \leq \sum_{j=1}^\infty \mathcal{H}^\alpha(A_j). \]

    \vspace{2mm}
    \textit{Step 3:} We claim that $\mathcal{H}^\alpha$ is a metric outer measure.
    \vspace{2mm}

    Let $A,B \sub X$ with $d(A,B) > 0$.
    Then choose $\delta > 0$ such that $\delta < \frac{1}{4} \dist(A,B)$. 
    Let $\{C_j\}_{j=1}^\infty$ be a cover of $A\cup B$ by sets of diameter at most $\delta$.
    
    Then we define collections of sets 
    \[ \mathcal{A} := \{ C_j : C_j \cap A \neq \emptyset \} \quad \text{and} \quad \mathcal{B} := \{ C_j : C_j \cap B \neq \emptyset \}. \]
    Then $\mathcal{A}$ is a cover of $A$ by sets of diameter at most $\delta$, and $\mathcal{B}$ is a cover of $B$ by sets of diameter at most $\delta$,
    and the fact that $A$ and $B$ are at least $4\delta$ apart implies that no set in $\mathcal{A}$ intersects any set in $\mathcal{B}$.
    Therefore
    \begin{align*}
        H^\alpha_\delta(A) + H^\alpha_\delta(B) &\leq \omega_\alpha \sum_{ C_j \in\mathcal{A} } \left( \frac{\diam(C_j)}{2} \right)^\alpha + \omega_\alpha \sum_{ C_j \in\mathcal{B} } \left( \frac{\diam(C_j)}{2} \right)^\alpha \\
            &\leq \omega_\alpha \sum_{j=1}^\infty \left( \frac{\diam(C_j)}{2} \right)^\alpha.
    \end{align*}
    Taking the infimum over all such covers $\{C_j\}_{j=1}^\infty$ of $A \cup B$ gives
    \[ \mathcal{H}^\alpha_\delta(A) + \mathcal{H}^\alpha_\delta(B) \leq \mathcal{H}^\alpha_\delta(A \cup B). \]
    Since this holds for all $\delta < \frac{1}{4} \dist(A,B)$, taking limits as $\delta \to 0^+$ gives
    \[ \mathcal{H}^\alpha(A) + \mathcal{H}^\alpha(B) \leq \mathcal{H}^\alpha(A \cup B). \]
    The reverse inequality follows from the fact that $\mathcal{H}^\alpha$ is an outer measure, so we conclude that
    \[ \mathcal{H}^\alpha(A \cup B) = \mathcal{H}^\alpha(A) + \mathcal{H}^\alpha(B). \]

    Since $A$ and $B$ were arbitrary sets with $\dist(A,B) > 0$, we conclude that $\mathcal{H}^\alpha$ is a metric outer measure on $X$.

    \vspace{2mm}

    Thus by the Carath\'eodory criterion (Theorem \ref{thm:caratheodory_criterion}), $\mathcal{H}^\alpha$ is a Borel outer measure on $X$.

    \vspace{2mm}
    \textit{Step 4:} Finally, we show that $\mathcal{H}^\alpha$ is a Borel regular outer measure.
    \vspace{2mm}

    Note that for each $C\subset X$, we have $\diam(C) = \diam(\overline{C})$.
    Thus for each $\delta > 0$ and $A \subseteq X$, we can modify the definition of $\mathcal{H}^\alpha_\delta(A)$ to take the infimum over covers by closed sets of diameter at most $\delta$.
    
    Now let $A \subseteq X$ be such that $\mathcal{H}^\alpha(A) < \infty$.
    Then for each $\delta > 0$, we have $\mathcal{H}^\alpha_\delta(A) < \infty$.
    For each $k\geq 1$, we choose a cover $\{C_{k,j}\}_{j=1}^\infty$ of $A$ by closed sets of diameter at most $1/k$ such that
    \[ \omega_\alpha \sum_{j=1}^\infty \left( \frac{\diam(C_{k,j})}{2} \right)^\alpha \leq \mathcal{H}^\alpha_{1/k}(A) + \frac{1}{k}. \]
    Then we define
    \[ B := \bigcap_{k=1}^\infty \bigcup_{j=1}^\infty C_{k,j} \]
    which is a Borel set.
    Also note that $A \sub B$ since $A \sub \bigcup_{j=1}^\infty C_{k,j}$ for each $k\geq 1$.
    Finally, we have
    \[ \mathcal{H}^\alpha_{1/k}(B) \leq \omega_\alpha \sum_{j=1}^\infty \left( \frac{\diam(C_{k,j})}{2} \right)^\alpha \leq \mathcal{H}^\alpha_{1/k}(A) + \frac{1}{k}, \qquad \forall k\geq 1 \]
    so taking limits as $k \to \infty$ gives
    \[ \mathcal{H}^\alpha(B) \leq \mathcal{H}^\alpha(A). \]
    Since $A \sub B$, we also have $\mathcal{H}^\alpha(A) \leq \mathcal{H}^\alpha(B)$, so $\mathcal{H}^\alpha(A) = \mathcal{H}^\alpha(B)$.
    That is, $A$ is contained in a Borel set $B$ with the same Hausdorff outer measure.

    Since $A$ was arbitrary, we conclude that $\mathcal{H}^\alpha$ is a Borel regular outer measure on $X$.
\end{proof}

\begin{remark}[What Hausdorff Measure do we use on Subsets?]
    \label{rmk:which_hausdorff_measure_on_subsets}
Ok lets clarify something. 
Say $(X,d)$ is a metric space, and let $\alpha \geq 0$.

\noindent Let $A\subset X$ be arbitrary.
Then we have three different things we could call the Hausdorff measure on $A$:
\begin{itemize}
    \item By considering $A$ as a subset of the metric space $(X,d)$, we get a metric space $(A,d|_{A\times A})$ and we can define the Hausdorff outer measure on this metric space.
        Call this measure $\mathcal{H}^\alpha_{A}$ to agree with our previous notation.
    \item By considering $A$ as a subset of the metric space $(X,d)$, we can define the Hausdorff outer measure on $X$ and then restrict this outer measure to $A$.
        Call this measure $\mathcal{H}^\alpha_{X}\mres A$, and remember that $\mathcal{H}^\alpha_{X}\mres A$ is still defined on all subsets of $X$, but only measures subsets of $A$ non-trivially.
    \item By considering $A$ as a subset of the metric space $(X,d)$, we can define the Hausdorff outer measure on $X$ and then consider the restriction of this measure to the measurable subsets of $A$.
        Call this measure $\mathcal{H}^\alpha_{X}|_{\mathcal{M}_A}$, where $\mathcal{M}_A$ is the $\sigma$-algebra of $\mathcal{H}^\alpha_X$-measurable subsets of $A$.
\end{itemize}
Which one should we use? We claim that the first and third options are literally the same.

Indeed, both $\mathcal{H}^\alpha_A$ and $\mathcal{H}^\alpha_X|_{\mathcal{M}_A}$ are Borel regular measures on the metric space $(A,d|_{A\times A})$, and they agree on all Borel subsets of $A$.
They actually agree on all subsets of $A$ though --- let $B\subset A$ be arbitrary; then Borel regularity of $\mathcal{H}^\alpha_A$ gives a Borel set $C\subset A$ with $B\subset C$ and $\mathcal{H}^\alpha_A(B) = \mathcal{H}^\alpha_A(C)$, and since $C$ is Borel in $A$, it is also $\mathcal{H}^\alpha_X$-measurable, so
\[ \mathcal{H}^\alpha_X|_{\mathcal{M}_A}(B) \leq \mathcal{H}^\alpha_X|_{\mathcal{M}_A}(C) = \mathcal{H}^\alpha_A(C) = \mathcal{H}^\alpha_A(B). \]
Similarly, Borel regularity of $\mathcal{H}^\alpha_X|_{\mathcal{M}_A}$ gives a Borel set $D\subset A$ with $B\subset D$ and $\mathcal{H}^\alpha_X|_{\mathcal{M}_A}(B) = \mathcal{H}^\alpha_X|_{\mathcal{M}_A}(D)$, and since $D$ is Borel in $A$, it is also Borel in $X$, so
\[ \mathcal{H}^\alpha_A(B) \leq \mathcal{H}^\alpha_A(D) = \mathcal{H}^\alpha_X|_{\mathcal{M}_A}(D) = \mathcal{H}^\alpha_X|_{\mathcal{M}_A}(B). \]
Since $B$ was an arbitrary subset of $A$, we conclude that $\mathcal{H}^\alpha_A$ and $\mathcal{H}^\alpha_X|_{\mathcal{M}_A}$ agree on all subsets of $A$.

Ok so we are down to two options - the first and second.
These are \emph{not} the same if $A$ is a proper subset of $X$, because the measure $\mathcal{H}^\alpha_X\mres A$ is defined on all subsets of $X$, while $\mathcal{H}^\alpha_A$ is only defined on subsets of $A$.
However, they do agree on all subsets of $A$.

To see this, let $B \sub A$ be arbitrary.
Then for each $\delta > 0$, any cover of $B$ by sets of diameter at most $\delta$ in $A$ is also a cover of $B$ by sets of diameter at most $\delta$ in $X$, so
\[ \mathcal{H}^\alpha_X\mres A(B) = \mathcal{H}^\alpha_X(B) \leq \mathcal{H}^\alpha_A(B). \]
Conversely, any cover of $B$ by sets of diameter at most $\delta$ in $X$ can be intersected with $A$ to get a cover of $B$ by sets of diameter at most $\delta$ in $A$, so
\[ \mathcal{H}^\alpha_A(B) \leq \mathcal{H}^\alpha_X\mres A(B). \]
Since $B$ was arbitrary, we conclude that $\mathcal{H}^\alpha_A$ and $\mathcal{H}^\alpha_X\mres A$ agree on all subsets of $A$.

\vspace{2mm}

In summary, if we want to consider the Hausdorff measure on a subset $A$ of a metric space $(X,d)$, we can either consider the Hausdorff measure on the metric space $(A,d|_{A\times A})$, or we can consider the restriction of the Hausdorff measure on $X$ to $A$.
Both options have their uses. For instance, we use the first option when stating a version of the Fubini-Tonelli Theorem for Hausdorff measures \ref{prop:fubini_tonelli_for_hausdorff_measures_on_orthogonal_subspaces}, and we will use the second option when dealing with Lipschitz domains and other geometric sets in the future. 

\end{remark}

\vspace{2mm}

\subsection{Properties of the Hausdorff Outer Measure}

Now that we have seen that $\mathcal{H}^\alpha$ is an outer measure, let's see what it's actually measuring.
We begin with the simplest case, $\alpha = 0$.

\begin{exercise}[$\mathcal{H}^0$ is the Counting Measure]
    \label{ex:H0_is_counting_measure}
    Let $(X,d)$ be a metric space.
    Show that $\mathcal{H}^0$ is the counting measure on $X$, i.e. that for each $A \sub X$,
    \[ \mathcal{H}^0(A) = \begin{cases}
        \#(A), & \text{if $A$ is finite}, \\
        \infty, & \text{if $A$ is infinite}.
    \end{cases} \]
\end{exercise}

\begin{proof}
    First see that 
    \[ \omega_0 := \frac{1}{\Gamma(1)} = 1 \]
    by Exercise \ref{ex:properties_of_gamma_function}.

    Let $(X,d)$ be an arbitrary metric space.
    It follows that $\mathcal{H}^0_\delta(\{x\}) = 1$ for each $x \in X$ and $\delta > 0$, so $\mathcal{H}^0(\{x\}) = 1$ for each $x \in X$.
    If $A \sub X$ is finite with $n$ elements, then $\mathcal{H}^0(A) = n$ by countable disjoint additivity and the fact that finite sets are closed, and hence Borel measurable.
    Finally, if $A \sub X$ is countably infinite, then $A$ is also Borel measurable, and we can write $A$ as a countable union of singletons, so $\mathcal{H}^0(A) = \infty$
    by countable disjoint additivity.
    If $A \sub X$ is uncountable, then there is a countable subset $B \sub A$, so $\mathcal{H}^0(A) \geq \mathcal{H}^0(B) = \infty$.
    Thus $\mathcal{H}^0$ is the counting measure on $X$.
\end{proof}

It is also easy to see that the Hausdorff outer measure is invariant under isometries.

\begin{lemma}[Isometry Invariance of the Hausdorff Outer Measure]
    \label{lem:isometry_invariance_of_hausdorff_outer_measure}
    Let $\alpha \geq 0$, and let $(X,d_X)$ and $(Y,d_Y)$ be metric spaces.
    If $f : X \to Y$ is an isometry, then
    \[ \mathcal{H}_Y^\alpha(f(A)) = \mathcal{H}_X^\alpha(A) \]
    for each $A \sub X$.
\end{lemma}

\begin{proof}
    Let $f : X \to Y$ be an isometry, meaning that
    \[ d_Y(f(x_1), f(x_2)) = d_X(x_1,x_2), \qquad \forall x_1,x_2 \in X. \]
    We will use the fact that if $C\sub X$, then $\diam(f(C)) = \diam(C)$, which follows immediately from the definitions --- see that
    \begin{align*}
        \diam(f(C)) &= \sup\{ d_Y(y_1, y_2) : y_1,y_2 \in f(C) \} = \sup\{ d_Y(f(x_1), f(x_2)) : x_1,x_2 \in C \} \\ 
            &= \sup\{ d_X(x_1,x_2) : x_1,x_2 \in C \} = \diam(C).
    \end{align*}

    Let $A\sub X$ be an arbitrary set.
    Fix $\delta > 0$.
    Then for each cover $\{C_j\}_{j=1}^\infty$ of $A$ by sets of diameter at most $\delta$, we have that $\{f(C_j)\}_{j=1}^\infty$ is a cover of $f(A)$ by sets of diameter at most $\delta$.
    Thus
    \[ \mathcal{H}^\alpha_{Y,\delta}(f(A)) \leq \omega_\alpha \sum_{j=1}^\infty \left( \frac{\diam(f(C_j))}{2} \right)^\alpha = \omega_\alpha \sum_{j=1}^\infty \left( \frac{\diam(C_j)}{2} \right)^\alpha. \]
    Taking the infimum over all such covers of $A$ gives
    \[ \mathcal{H}^\alpha_{Y,\delta}(f(A)) \leq \mathcal{H}^\alpha_{X,\delta}(A). \]
    Since $f$ is an isometry, the inverse $f^{-1}$ is also an isometry; so the same argument applied to $f^{-1}$ and the set $f(A)$ gives the reverse inequality,
    so $\mathcal{H}^\alpha_{Y,\delta}(f(A)) = \mathcal{H}^\alpha_{X,\delta}(A)$ for each $\delta > 0$.
    By taking limits as $\delta \to 0^+$ we see that $\mathcal{H}^\alpha_Y(f(A)) = \mathcal{H}^\alpha_X(A)$.
\end{proof}

With these two properties alone, there is not much we can do.
Our main interest is in the case of $\R^n$, where we can relate the Hausdorff outer measure to Lebesgue measure and volume.
To this end, we can show that Hausdorff measures on $\R^n$ are homogeneous with respect to scaling.

\begin{lemma}[Homogeneity of the Hausdorff Outer Measure]
    \label{lem:scaling_property_of_hausdorff_outer_measure}
    Let $\alpha \geq 0$.
    Then $\mathcal{H}^\alpha(tA) = t^\alpha \mathcal{H}^\alpha(A)$ for each $t > 0$ and $A \sub \R^n$.
\end{lemma}
\begin{proof}
    Let $A \sub \R^n$ and $t > 0$.
    Then for each cover $\{C_j\}_{j=1}^\infty$ of $A$ by sets of diameter at most $\delta$, we have that $\{tU_j\}_{j=1}^\infty$ is a cover of $tA$ by sets of diameter at most $t\delta$.
    Thus
    \[ \mathcal{H}^\alpha_{\delta}(tA) \leq \omega_\alpha \sum_{j=1}^\infty \left( \frac{\diam(tU_j)}{2} \right)^\alpha = \omega_\alpha \sum_{j=1}^\infty \left( t \frac{\diam(C_j)}{2} \right)^\alpha = t^\alpha \omega_\alpha \sum_{j=1}^\infty \left( \frac{\diam(C_j)}{2} \right)^\alpha. \]
    Taking the infimum over all such covers of $A$ gives
    \[ \mathcal{H}^\alpha_{\delta}(tA) \leq t^\alpha \mathcal{H}^\alpha_\delta(A). \]
    The reverse inequality follows by applying the same argument to $t^{-1}$ instead of $t$, so we conclude that
    \[ \mathcal{H}^\alpha_{\delta}(tA) = t^\alpha \mathcal{H}^\alpha_\delta(A) \qquad \forall t > 0, \, \delta > 0. \]
    Taking limits as $\delta \to 0^+$ gives $\mathcal{H}^\alpha(tA) = t^\alpha \mathcal{H}^\alpha(A)$.
\end{proof}

Now we can relate the Hausdorff outer measure to Lebesgue measure in the case $\alpha = 1$.
\begin{proposition}[$\mathcal{H}^1$ is Lebesgue Measure on $\R$]
    \label{prop:H1_is_lebesgue_on_R}
    The $1$-dimensional Hausdorff outer measure $\mathcal{H}^1$ on $\R$ coincides with the Lebesgue outer measure $\mathcal{L}^1$ on $\R$.
\end{proposition}
Thus $\mathcal{H}^1$ is measuring length in $\R$; we should expect that in more general metric spaces, $\mathcal{H}^1$ is measuring the length of curves.

\begin{proof}
    First note that $\omega_1 = 2$ by Exercise \ref{ex:volume_of_n_dimensional_ball_formula}.
    
    Let $A \sub \R$ be arbitrary, and let $\delta > 0$.
    Then we have
    \begin{align*}
        \mathcal{L}^1(A) &= \inf \left\{ \sum_{j=1}^\infty (b_j - a_j) : A \sub \bigcup_{j=1}^\infty [a_j, b_j] \right\} \\
            &\leq \inf \left\{ \sum_{j=1}^\infty (b_j - a_j) : A \sub \bigcup_{j=1}^\infty [a_j, b_j]\ \  \text{ and } b_j - a_j \leq \delta \text{ for all } j\geq 1 \right\} \\
            &= \inf \left\{ 2 \sum_{j=1}^\infty \left( \frac{\diam([a_j, b_j])}{2} \right) : A \sub \bigcup_{j=1}^\infty [a_j, b_j]\ \  \text{ and } \diam([a_j, b_j]) \leq \delta \text{ for all } j\geq 1 \right\} \\
            &= \mathcal{H}^1_\delta(A)
    \end{align*}
    since the diameter of the interval $[a_j, b_j]$ is $b_j - a_j$ and $\omega_1 = 2$.
    Because this holds for each $\delta > 0$, it follows that $\mathcal{L}^1(A) \leq \mathcal{H}^1(A)$.

    To prove the reverse inequality, for each $k\in \Z$ let $I_k := [k\delta, (k+1)\delta)$.
    Then $\{I_k\}_{k\in \Z}$ is a partition of $\R$ into intervals of length $\delta$; if $C\subseteq \R$ is an arbitrary set then 
    $\diam(C\cap I_k) \leq \delta$ for each $k\in \Z$ and
    \[ \sum_{k\in \Z} \diam(C\cap I_k) \leq \diam(C). \]
    Hence 
    \begin{align*}
        \mathcal{L}^1(A) &= \inf \left\{ \sum_{j=1}^\infty (b_j - a_j) : A \sub \bigcup_{j=1}^\infty [a_j, b_j] \right\} \\
            &\geq \inf \left\{ \sum_{j=1}^\infty \sum_{k\in \Z} \diam([a_j, b_j] \cap I_k) : A \sub \bigcup_{j=1}^\infty [a_j, b_j] \right\} \\
            &= \inf \left\{ 2 \sum_{j=1}^\infty \sum_{k\in \Z} \left( \frac{\diam([a_j, b_j] \cap I_k)}{2} \right) : A \sub \bigcup_{j=1}^\infty [a_j, b_j] \right\} \\
            &\geq \inf \left\{ 2 \sum_{m=1}^\infty \left( \frac{\diam(U_m)}{2} \right) : A \sub \bigcup_{m=1}^\infty U_m\ \  \text{ and } \diam(U_m) \leq \delta \text{ for all } m\geq 1 \right\} \\
            &= \mathcal{H}^1_\delta(A).
    \end{align*}
    Since this holds for each $\delta > 0$, it follows that $\mathcal{L}^1(A) \geq \mathcal{H}^1(A)$.

    Thus we conclude that $\mathcal{H}^1(A) = \mathcal{L}^1(A)$ for each $A \sub \R$.
    Since $A$ was arbitrary, we have shown that $\mathcal{H}^1$ coincides with $\mathcal{L}^1$ on $\R$.

    As a corollary of the proof above, we also see that for each $A \sub \R$ and each $\delta > 0$, we actually have $\mathcal{H}^1_\delta(A) = \mathcal{L}^1(A)$.
\end{proof}


\subsection{Hausdorff Dimension}

In this section, we explore the idea of Hausdorff dimension, which gets us closer to understanding what the Hausdorff outer measure is measuring.

\begin{lemma}
    \label{lem:hausdorff_outer_measure_zero}
    Let $(X, d)$ be a metric space and let $\alpha\geq 0$.
    If $A \sub X$ is such that $\mathcal{H}^\alpha_\delta(A) = 0 $ for some $\delta > 0$, then $\mathcal{H}^\alpha(A) = 0$.
\end{lemma}

\begin{proof}
    In the case that $\alpha = 0$, this follows immediately from Exercise \ref{ex:H0_is_counting_measure}, so we may assume that $\alpha > 0$.

    Let $A \subseteq X$ be such that $\mathcal{H}^\alpha_\delta(A) = 0$ for some $\delta > 0$.
    Fix $\epsilon > 0$.
    Then there exists a sequence of sets $\{C_j\}_{j=1}^\infty$ covering $A$ such that
    \[ \omega_\alpha \sum_{j=1}^\infty \left( \frac{\diam(C_j)}{2} \right)^\alpha < \epsilon. \]
    In particular, for each $j\geq 1$ we have
    \[ \diam(C_j) \leq 2\left( \frac{\epsilon}{\omega_\alpha} \right)^{\frac{1}{\alpha}} := \delta(\epsilon). \]
    This shows 
    \[ \mathcal{H}^\alpha_{\delta(\epsilon)}(A) \leq \epsilon. \]
    Since $\delta(\epsilon) \to 0$ as $\epsilon \to 0^+$, taking limits as $\epsilon \to 0^+$ gives $\mathcal{H}^\alpha(A) = 0$.
\end{proof}

The next lemma inspires the notion of Hausdorff dimension.
It says that if the Hausdorff measure of a set is finite for some dimension, then it is zero for all higher dimensions.
It also says that if the Hausdorff measure of a set is positive for some dimension, then it is infinite for all lower dimensions.
This works well with our intution --- a solid square has positive $2$-dimensional measure (area), but zero $3$-dimensional measure (volume), and infinite $1$-dimensional measure (length of curves needed to cover it).

\begin{lemma}
    \label{lem:hausdorff_outer_measure_finite_or_positive}
    Let $(X,d)$ be a metric space.
    Let $\alpha, \beta \geq 0$ with $\alpha < \beta$.
    \begin{enumerate}[(a)]
        \item If $A \subseteq X$ is such that $\mathcal{H}^\alpha(A) < \infty$, then $\mathcal{H}^\beta(A) = 0$.
        \item If $A \subseteq X$ is such that $\mathcal{H}^\beta(A) > 0$, then $\mathcal{H}^\alpha(A) = \infty$.
    \end{enumerate}
\end{lemma}
\begin{proof}
    \begin{enumerate}[(a)]
        \item Suppose $A\subseteq X$ is such that $\mathcal{H}^\alpha(A) < \infty$ and let $\delta>0$. 
            Then there exist a countable cover $\{C_j\}_{j=1}^\infty$ of $A$ by sets of diameter at most $\delta$ such that
            \[ \omega_\alpha \sum_{j=1}^\infty \left( \frac{\diam(C_j)}{2} \right)^\alpha \leq \mathcal{H}^\alpha_\delta(A) + 1 \leq \mathcal{H}^\alpha(A) + 1. \]
            As a result, we have 
            \begin{align*}
                \omega_\beta \sum_{j=1}^\infty \left( \frac{\diam(C_j)}{2}\right)^\beta &= \frac{\omega_\beta}{\omega_\alpha} 2^{\beta - \alpha} \omega_\alpha \sum_{j=1}^\infty \left( \frac{\diam(C_j)}{2} \right)^\alpha \left( \frac{\diam(C_j)}{2} \right)^{\beta - \alpha} \\
                    &\leq \frac{\omega_\beta}{\omega_\alpha} 2^{\beta - \alpha} \delta^{\beta - \alpha} (\mathcal{H}^\alpha(A) + 1).
            \end{align*}
            Since $\{C_j\}_{j=1}^\infty$ was an arbitrary cover of $A$ by sets of diameter at most $\delta$, we have
            \[ \mathcal{H}^\beta_\delta(A) \leq \frac{\omega_\beta}{\omega_\alpha} 2^{\beta - \alpha} \delta^{\beta - \alpha} (\mathcal{H}^\alpha(A) + 1). \]
            By taking $\delta \to 0^+$, we see that $\mathcal{H}^\beta(A) = 0$.
        
        \item This is just the contrapositive of part (a).
    \end{enumerate}
\end{proof}

\begin{definition}[Hausdorff Dimension]
    \label{def:hausdorff_dimension}
    Let $(X,d)$ be a metric space, and let $A \sub X$.
    The \textbf{Hausdorff dimension} of $A$ is defined as
    \[ \mathcal{H}_{\operatorname{dim}}(A) := \inf\{ \alpha \geq 0 : \mathcal{H}^\alpha(A) = 0 \} = \sup\{ \alpha \geq 0 : \mathcal{H}^\alpha(A) = \infty \}. \]
\end{definition}

With this definition, Lemma \ref{lem:hausdorff_outer_measure_finite_or_positive} shows that the Hausdorff dimension is well-defined; 
for $\alpha < \mathcal{H}_{\operatorname{dim}}(A)$ we have $\mathcal{H}^\alpha(A) = \infty$, and for $\alpha > \mathcal{H}_{\operatorname{dim}}(A)$ we have $\mathcal{H}^\alpha(A) = 0$.

\begin{example}[Finite and Countable Sets have Hausdorff Dimension Zero]
    \label{ex:hausdorff_dimension_of_finite_and_countable_sets}
    Let $(X,d)$ be a metric space, and let $A \sub X$ be finite or countably infinite.
    Then $\mathcal{H}_{\operatorname{dim}}(A) = 0$.
    
    If $A$ is finite with $n$ elements, then $\mathcal{H}^0(A) = n < \infty$, so $\mathcal{H}^\alpha(A) = 0$ for all $\alpha > 0$, and it follows that $\mathcal{H}_{\operatorname{dim}}(A) = 0$.
    If $A$ is countably infinite, then $\mathcal{H}^0(A) = \infty$ but we can write $A$ as a countable union of finite sets, and for each $\alpha > 0$ we have
    \begin{align*}
        \mathcal{H}^\alpha(A) &\leq \sum_{n=1}^\infty \mathcal{H}^\alpha(\{a_n\}) = \sum_{n=1}^\infty 0 = 0
    \end{align*}
    where $A = \{a_1, a_2, \ldots\}$ and we have used the previous argument for finite sets and countable subadditivity.
    Therefore $\mathcal{H}_{\operatorname{dim}}(A) = 0$.
\end{example}

The Hausdorff dimension can be non-integer, which is part of what makes it interesting.
You can test what the Hausdorff dimension ``should be'' by looking at the scaling behaviour of the set and comparing it to the scaling behaviour of $\mathcal{H}^\alpha$ from Lemma \ref{lem:scaling_property_of_hausdorff_outer_measure}.
\begin{example}[(How to Guess) the Hausdorff Dimension of the Cantor Set]
    \label{ex:hausdorff_dimension_of_cantor_set}
    The Cantor set $C \subset [0,1]$ has Hausdorff dimension $\frac{\log 2}{\log 3}$.

    We will not prove this fact, but let us describe how this result could have been guessed.
    Scaling the cantor set by a factor of $3$ produces two copies of itself --- the original Cantor set $C$ and a translated copy $2+C$ --- so we expect that the Hausdorff dimension $\alpha$ should satisfy
    \[ \mathcal{H}^\alpha (3C) = 2 \mathcal{H}^\alpha(C) \]
    by disjoint additivity, but scaling properties of the Hausdorff measure give
    \[ \mathcal{H}^\alpha(3C) = 3^\alpha \mathcal{H}^\alpha(C). \]
    Equating these two expressions for $\mathcal{H}^\alpha(3C)$ gives
    \[ 3^\alpha = 2 \implies \alpha = \frac{\log 2}{\log 3}. \]
    
    This \emph{not} a proof, but it is a good way to guess the answer.
    A full proof is more involved, and we will not give it here.
\end{example}

Later on, we will see that $\R^n$ has Hausdorff dimension $n$, and graphs of Lipschitz functions $\R^n \to \R^m$ have Hausdorff dimension $n$.
\subsection{Isodiametric Inequality, Equality of Hausdorff and Lebesgue Measure on $\R^n$}
In this section, we work exclusively in Euclidean space $\R^n$ with the standard Euclidean metric.

\begin{theorem}[Isodiametric Inequality]
    \label{thm:isodiametric_inequality}
    \[ \mathcal{L}^n(E) \leq \omega_n \left( \frac{\diam A}{2}\right)^n \]
    for each set $A \subseteq \R^n$.
\end{theorem}

This says that among all sets with a given diameter $\delta$, the $n$-dimensional volume is maximized by the $n$-dimensional ball with radius $\delta/2$.

\begin{proof}
    \textit{Step 0:} It suffices to prove the result for compact sets. 
    \vspace{2mm}
    
    Indeed, if $A \subseteq \R^n$ is a set with diameter $\diam A = \infty$, then the right-hand side is infinite, so the inequality holds trivially.
    If $\diam A < \infty$, then the set $\overline{A}$ is compact with $\diam \overline{A} = \diam A$, and $\mathcal{L}^n(A) \leq \mathcal{L}^n(\overline{A})$ by outer regularity of Lebesgue measure, so if the result holds for compact sets then
    \[ \mathcal{L}^n(A) \leq \mathcal{L}^n(\overline{A}) \leq \omega_n \left( \frac{\diam \overline{A}}{2} \right)^n = \omega_n \left( \frac{\diam A}{2} \right)^n. \]
    Thus we only need to prove the result for compact sets.

    \vspace{2mm}
    \textit{Step 1:} Steiner Symmetrization Definition.
    \vspace{2mm}

    For each compact set $A \subset \R^n$ and each $j\in \{1,2,\ldots,n\}$, we define the \textit{Steiner symmetrization} $S_j(A)$ of $A$ with respect to the hyperplane orthogonal to the $j$-th coordinate axis as follows: 
    for each $\xi \in \R^n$ such that $\xi_j = 0$, we define the line
    \[ \ell_{\xi,j} := \{ \xi + t e_j : t \in \R \} \]
    and let 
    \[ \pi_\xi : \ell_\xi \to \R, \quad \pi_\xi(\xi + t e_j) = t \]
    which projects points on the line $\ell_\xi$ onto $\R$ ---- then we set
    \[ S_j(A) := \bigcup_{ \{ \xi \in \R^n : \xi_j = 0 \text{ and } A \cap \ell_{\xi,j} \neq \emptyset \} } \left\{ \xi + t e_j : t \in \R, |t| \leq \frac{\mathcal{L}^1(\pi_\xi(A \cap \ell_{\xi,j}))}{2} \right\}. \]
    In other words, the set $S_j(A)$ is obtained by replacing each slice $A \cap \ell_{\xi,j}$ with a line segment segment; this line segment has the same length as the slice $A \cap \ell_{\xi,j}$, but is centered at the hyperplane $\{ x \in \R^n : x_j = 0 \}$.

    \vspace{2mm}
    \textit{Step 2:} We claim that for each $j\in \{1,2,\ldots,n\}$ and each compact set $A \subset \R^n$, the Steiner symmetrization $S_j(A)$ is compact.
    \vspace{2mm}

    \begin{proof}
    Let $A \subset \R^n$ be compact and fix $j \in \{1,2,\ldots,n\}$.
    We need to show that $S_j(A)$ is bounded and closed.

    Since $A$ is bounded, there exists $R > 0$ such that $A \sub B(0,R)$.
    Let $\xi \in \R^n$ with $\xi_j = 0$.
    If $\|\xi\| > R$, then the line $\ell_{\xi,j}$ does not intersect $B(0,R)$, and hence does not intersect $A$, so $A \cap \ell_{\xi,j} = \emptyset$.
    If $\|\xi\| \leq R$, then $ A \cap \ell_{\xi,j} \sub B(0,R) \cap \ell_{\xi,j}$,  and $B(0,R) \cap \ell_{\xi,j}$ is a line segment of length at most $2\sqrt{R^2 - \|\xi\|^2} \leq 2R$.

    Therefore the previous paragraph shows
     \[ S_j(A) \subseteq \{ \xi + t e_j : \|\xi\| \leq R, \xi_j = 0, |t| \leq R \} \]
    which is a bounded set, so $S_1(A)$ is bounded.

    To see that $S_j(A)$ is closed, we define a function
    \[ g_j : \{0 \} \times \R^{n-1} \to \R ,\qquad g_j(\xi) := \mathcal{L}^1(\pi_\xi(A \cap \ell_{\xi,j})). \]
    We claim that $g_j$ is upper semi-continuous at each point $\xi \in \{0\} \times \R^{n-1}$.

    To check this fix $\xi \in \{0\} \times \R^{n-1}$ and let $\epsilon > 0$ and note that Borel regularity of Lebesgue measure gives an open set $U \sub \R$ such that $\pi_\xi(A \cap \ell_{\xi,j}) \sub U$ and
    \[  \mathcal{L}^1(U) \leq \mathcal{L}^1(\pi_\xi(A \cap \ell_{\xi,j})) + \epsilon. \]
    Since $A$ is closed and $\pi_\xi^{-1}(U)$ is an open set which contains $A \cap \ell_{\xi,j}$, there exists $\delta > 0$ such that for each $\eta \in \{0\} \times \R^{n-1}$ with $|\eta - \xi| < \delta$, we have
    \[ A \cap \ell_{\eta,j} \sub \pi_\eta^{-1}(U). \]
    Thus for each such $\eta$, we have
    \[ g_j(\eta) = \mathcal{L}^1(\pi_\eta(A \cap \ell_{\eta,j})) \leq \mathcal{L}^1(U) \leq \mathcal{L}^1(\pi_\xi(A \cap \ell_{\xi,j})) + \epsilon = g_j(\xi) + \epsilon. \]
    This shows that $g_j$ is upper semi-continuous at $\xi$.
    Since $\xi$ was arbitrary, we conclude that $g_j$ is upper semi-continuous on $\{0\} \times \R^{n-1}$.

    Now let $\{x^{(k)}\}_{k=1}^\infty$ be a sequence in $S_j(A)$ which converges to some $x \in \R^n$.
    We need to show that $x \in S_j(A)$.
    For each $k\geq 1$, there exists $\xi^{(k)} \in \R^n$ with $\xi^{(k)}_j = 0$ such that
    \[ x^{(k)} = \xi^{(k)} + t_k e_j \]
    for some $t_k \in \R$ with
    \[ |t_k| \leq \frac{\mathcal{L}^1(\pi_{\xi^{(k)}}(A \cap \ell_{\xi^{(k)},j}))}{2} = \frac{g_j(\xi^{(k)})}{2}. \]
    Since $A$ is bounded, the sequence $\{\xi^{(k)}\}_{k=1}^\infty$ is bounded, so by passing to a subsequence if necessary we may assume that $\xi^{(k)} \to \xi$ for some $\xi \in \{0\} \times \R^{n-1}$.
    By upper semi-continuity of $g_j$, we have
    \[ \limsup_{k \to \infty} g_j(\xi^{(k)}) \leq g_j(\xi). \]
    Since $x_k \to x$, we have $t_k \to t$ for some $t \in \R$ such that
    \[ |t| = \lim_{k\to \infty} |t_k| \leq \limsup_{k\to \infty} \frac{g_j(\xi^{(k)})}{2} \leq \frac{g_j(\xi)}{2}. \]
    Therefore
    \[ x = \xi + t e_j \in S_j(A). \]
    Since $\{x^{(k)}\}_{k=1}^\infty$ was an arbitrary convergent sequence in $S_j(A)$, we conclude that $S_j(A)$ is closed.
    Thus $S_j(A)$ is compact.

    In particular, the set $S_j(A)$ has finite diameter.
    \end{proof}

    \vspace{2mm}
    \textit{Step 3:} We claim that for each $j\in \{1,2,\ldots,n\}$ and each compact set $A \subset \R^n$, we have
    \[ \diam(S_j(A)) \leq \diam A \quad \text{and} \quad \mathcal{L}^n(S_j(A)) = \mathcal{L}^n(A). \]
    \vspace{2mm}

    \begin{proof}
    Let $A \subset \R^n$ be compact and fix $j \in \{1,2,\ldots,n\}$.
    Let $\epsilon > 0$ be arbitrary; choose points $x,y \in S_j(A)$ such that
    \[ \diam(S_j(A)) \leq \|x - y\| + \epsilon \]
    which exist by the compactness of $S_j(A)$ from Step 2.

    We define
    \begin{align*}
        r_j^- &:= \inf \left\{ t\in \R : x_j + te_j \in A \right\} \\
        r_j^+ &:= \sup \left\{ t\in \R : x_j + te_j \in A \right\} \\
        s_j^- &:= \inf \left\{ t\in \R : y_j + te_j \in A \right\} \\
        s_j^+ &:= \sup \left\{ t\in \R : y_j + te_j \in A \right\}.
    \end{align*}
    By symmetry of interchanging $x$ and $y$ if necessary, we may assume without loss of generality that $s_j^+ - r_j^{-} \geq s_j^- - r_j^+$.
    Then we have
    \begin{align*}
        s_j^+ - r_j^- &\geq \frac{1}{2}(s_j^+ - r_j^-) + \frac{1}{2}(s_j^- - r_j^+) \\
            &= \frac{1}{2} \left( (s_j^+ - s_j^-) + (r_j^+ - r_j^-) \right) \\
            &= \frac{1}{2} \left( \mathcal{L}^1(\pi_{x - x_j e_j}(A \cap \ell_{x - x_j e_j, j})) + \mathcal{L}^1(\pi_{y - y_j e_j}(A \cap \ell_{y - y_j e_j, j})) \right) \\
            &= |x_j| + |y_j| \\
            &\geq |x_j - y_j|
    \end{align*}
    where we have used the definition of $S_j(A)$ in the third line, the fact that the line segment $\pi_{x - x_j e_j}(A \cap \ell_{x - x_j e_j, j})$ has length $2|x_j|$ and 
    and the line segment $\pi_{y - y_j e_j}(A \cap \ell_{y - y_j e_j, j})$ has length $2|y_j|$ in the fourth line, and the triangle inequality in the last line.
    
    Hence
    \begin{align*}
        (\diam(S_j(A)) - \epsilon)^2 &\leq \|x - y\|^2 \\
            &= \|\proj_{e_j^\perp}(x) - \proj_{e_j^\perp}(y) \|^2 + | x_j - y_j|^2 \\
            &\leq \|\proj_{e_j^\perp}(x) - \proj_{e_j^\perp}(y) \|^2 + (s_j^+ - r_j^-)^2 \\
            &= \|\proj_{e_j^\perp}(x) + r_j^- e_j - (\proj_{e_j^\perp}(y) + s_j^+ e_j) \|^2 \\
            &\leq (\diam A)^2
    \end{align*}
    where we have used the previous inequality in the third line, and the fact that the points $\proj_{e_j^\perp}(x) + r_j^- e_j$ and $\proj_{e_j^\perp}(y) + s_j^+ e_j$ are both in $A$ in because by definition of $r_j^-$ and $s_j^+$ and the fact that $A$ is compact in the last line.
    Thus \[\diam(S_j(A)) \leq \diam A + \epsilon. \]
    Since $\epsilon > 0$ was arbitrary, we conclude that $\diam(S_j(A)) \leq \diam A$.

    \vspace{2mm}

    Now we define a map 
    \[ f_j: \R^{n-1} \to \R, \quad f_j(\xi) := \frac{\mathcal{L}^1(\pi_{\xi}(A \cap \ell_{\xi,j}))}{2}. \]
    By Tonelli's Theorem (Theorem \ref{thm:tonellis_theorem}), the map $f_j$ is $\mathcal{L}^{n-1}$-measurable, and we have
    \[ \mathcal{L}^{n}(A) = \int_{\R^{n-1}} \mathcal{L}^1(\pi_{\xi}(A \cap \ell_{\xi,j})) \, d\mathcal{L}^{n-1}(\xi) = 2 \int_{\R^{n-1}} f_j(\xi) \, d\mathcal{L}^{n-1}(\xi). \]
    Using the definition of $S_j(A)$ we have
    \[ S_j(A) = \left\{ \xi + t e_j : \xi \in \R^n, \xi_j = 0, \text{ and } - \frac{f_j(\xi)}{2} \leq t \leq \frac{f_j(\xi)}{2} \right\} \]
    so another application of Tonelli's Theorem gives
    \begin{align*}
        \mathcal{L}^n(S_j(A)) &= \int_{\R^{n-1}} \mathcal{L}^1\left( \left\{ t \in \R : -\frac{f_j(\xi)}{2} \leq t \leq \frac{f_j(\xi)}{2} \right\} \right) d\mathcal{L}^{n-1}(\xi) \\
            &= \int_{\R^{n-1}} f_j(\xi) \, d\mathcal{L}^{n-1}(\xi) \\
            &= \mathcal{L}^n(A).
    \end{align*}

    \end{proof}

    \vspace{2mm}
    \textit{Step 4:} Conclusion of the proof.
    \vspace{2mm}

    Let $A \subset \R^n$ be compact.
    We define 
    \[ A_1 := S_{1}(A), \quad  A_2 := S_{2}(A_1) \ \ , \ \ \ldots\ \ , \ \  A_n := S_{n}(A_{n-1}). \]
    That is, $A_1$ is the Steiner symmetrization of $A$ with respect to the hyperplane orthogonal to $e_1$, $A_2$ is the Steiner symmetrization of $A_1$ with respect to the hyperplane orthogonal to $e_2$, and so on.
    We use the Steiner symmetrization $n$ times, once for each standard basis vector of $\R^n$, and find ourselves with a (nice) set $A_n$.

    \vspace{2mm}
    \textit{Claim:} The set $A_n$ is symmetric about the origin, i.e. $A_n = -A_n$.
    \vspace{2mm}

    Clearly $A_1$ is symmetric about the hyperplane $\{ e_1 \}^\perp$, so suppose inductively that $A_{k-1}$ is symmetric about each of the hyperplanes $\{ e_1 \}^\perp, \{ e_2 \}^\perp, \ldots, \{ e_{k-1} \}^\perp$ for some $2 \leq k \leq n$.
    Then $A_k$ is symmetric about the hyperplane $\{ e_k \}^\perp$ by definition of Steiner symmetrization. 
    To see that $A_k$ is symmetric about the other hyperplanes, let $1 \leq j \leq k-1$ be arbitrary, and let $R_j : \R^n \to \R^n$ be the reflection across the hyperplane $\{ e_j \}^\perp$, i.e.
    \[ R_j(x) := x - 2(x \cdot e_j)e_j. \]
    Since $A_{k-1}$ is symmetric about $\{ e_j \}^\perp$, we have $R_j(A_{k-1}) = A_{k-1}$.
    Thus for each $x\in \{ e_j \}^\perp$, we have
    \[ \mathcal{H}^1( A_k \cap \ell_{j,x} ) = \mathcal{H}^1( R_j(A_k) \cap \ell_{j,R_j(x)} ). \]
    which implies
    \[ \{ t : x + t e_k \in A_k \} = \{ t : R_j(x) + t e_k \in A_k \}. \]
    Therefore $A_k$ is symmetric about $\{ e_j \}^\perp$.
    By induction, $A_n$ is symmetric about each of the hyperplanes $\{ e_1 \}^\perp, \{ e_2 \}^\perp, \ldots, \{ e_n \}^\perp$, which implies that $A_n$ is symmetric about the origin.
    This proves the claim.

    \vspace{2mm}
    \textit{Claim:} We claim that \[ \mathcal{L}^n(A_n) \leq \omega_n \left( \frac{\diam(A_n)}{2} \right)^n. \]
    \vspace{2mm}

    For each $x \in A_n$, we have $-x \in A_n$ by the previous claim, so we have $\diam(A_n) \geq \| x - (-x) \| = 2\|x\|$.
    Therefore $A_n \sub B\left(0, \frac{\diam(A_n)}{2}\right)$, and we estimate
    \[ \mathcal{L}^n(A_n) \leq \mathcal{L}^n\left(B\left(0, \frac{\diam(A_n)}{2}\right)\right) = \omega_n \left( \frac{\diam(A_n)}{2} \right)^n. \]
    This proves the claim.

    \vspace{2mm}
    Now we complete the proof of the theorem.
    By the previous claim, we have
    \begin{align*}
        \mathcal{L}^n(A) &= \mathcal{L}^n( A_n ) \\
            &\leq \omega_n \left( \frac{\diam( A_n )}{2} \right)^n \\
            &\leq \omega_n \left( \frac{\diam(A)}{2} \right)^n.
    \end{align*}
    This completes the proof of the theorem.
\end{proof}

\begin{theorem}[Equality of Hausdorff and Lebesgue Measure on $\R^n$]
    \label{thm:hn_equals_lebesgue_measure_on_rn}
    \[ \mathcal{H}^n(A) = \mathcal{L}^n(A) \]
    for each set $A \subseteq \R^n$. 

    In short, $\mathcal{H}^n = \mathcal{L}^n$ on $\R^n$.
\end{theorem}

\begin{proof}
    \textit{Step 1:} We claim that $\mathcal{L}^n(A) \leq \mathcal{H}^n(A)$ for each set $A \subseteq \R^n$.
    \vspace{2mm}

    Let $A \subseteq \R^n$ be arbitrary and let $\delta > 0$.
    Let $\{C_j\}_{j=1}^\infty$ be a countable cover of $A$ by sets of diameter at most $\delta$.
    Then by the Isodiametric inequality (Theorem \ref{thm:isodiametric_inequality}), we have
    \begin{align*}
        \mathcal{L}^n(A) &\leq \mathcal{L}^n\left( \bigcup_{j=1}^\infty C_j \right) \leq \sum_{j=1}^\infty \mathcal{L}^n(C_j) \\
            &\leq \sum_{j=1}^\infty \omega_n \left( \frac{\diam(C_j)}{2} \right)^n.
    \end{align*}
    Taking the infimum over all such covers $\{C_j\}_{j=1}^\infty$ of $A$ gives
    \[ \mathcal{L}^n(A) \leq \mathcal{H}^n_\delta(A). \]
    Since $\delta > 0$ was arbitrary, taking limits as $\delta \to 0^+$ gives
    \[ \mathcal{L}^n(A) \leq \mathcal{H}^n(A). \]
    Since $A\subseteq \R^n$ was arbitrary, we have shown that $\mathcal{L}^n(A) \leq \mathcal{H}^n(A)$ for each $A \subseteq \R^n$.

    \vspace{2mm}
    \textit{Step 2:} We claim that $\mathcal{H}^n(A) = 0$ for each set $A \subseteq \R^n$ with $\mathcal{L}^n(A) = 0$.
    That is, we claim that $\mathcal{H}^n$ is absolutely continuous with respect to $\mathcal{L}^n$.
    \vspace{2mm}
    
    Assume that $A \subseteq \R^n$ is such that $\mathcal{L}^n(A) = 0$.
    Let $\delta > 0$ be arbitrary.
    Then we have
    \begin{align*}
        \mathcal{H}^n_\delta(A) &= \inf\left\{ \sum_{j=1}^\infty \omega_n \left( \frac{\diam(C_j)}{2} \right)^n : A \sub \bigcup_{j=1}^\infty B_j, \ \diam(B_j) \leq \delta \right\} \\
            &\leq \inf\left\{ \sum_{j=1}^\infty \omega_n \left( \frac{\diam(Q_j)}{2} \right)^n : A \sub \bigcup_{j=1}^\infty Q_j, \ \diam(Q_j) \leq \delta, \text{ each } Q_j \text{ is a cube} \right\} \\
            &= \inf\left\{ \sum_{j=1}^\infty \omega_n \left( \frac{\sqrt{n}}{2}\right)^n \mathcal{L}^n(Q_j) : A \sub \bigcup_{j=1}^\infty Q_j, \ \diam(Q_j) \leq \delta, \text{ each } Q_j \text{ is a cube} \right\} \\
            &\leq \omega_n \left( \frac{\sqrt{n}}{2}\right)^n \inf\left\{ \sum_{j=1}^\infty \mathcal{L}^n(Q_j) : A \sub \bigcup_{j=1}^\infty Q_j, \text{ each } Q_j \text{ is a cube} \right\} \\
            &= \omega_n \left( \frac{\sqrt{n}}{2}\right)^n \mathcal{L}^n(A) = 0.
    \end{align*}
    Since $\delta > 0$ was arbitrary, taking limits as $\delta \to 0^+$ gives
    \[ \mathcal{H}^n(A) = 0. \]
    This proves the claim.

    \vspace{2mm}
    \textit{Step 3:} We claim that $\mathcal{H}^n(A) \leq \mathcal{L}^n(A)$ for each set $A \subseteq \R^n$.
    \vspace{2mm}

    Let $A \subseteq \R^n$ be arbitrary. Fix $\epsilon , \delta > 0$. 
    Then here exist closed cubes $\{Q_j\}_{j=1}^\infty$ such that $\diam(Q_j) \leq \delta$ for each $j \geq 1$ and 
    \[ A \sub \bigcup_{j=1}^\infty Q_j \quad \text{and} \quad \sum_{j=1}^\infty \mathcal{L}^n(Q_j) \leq \mathcal{L}^n(A) + \epsilon. \]
    By \ref{lem:filling_open_set_with_balls}, for each $j \geq 1$ there exists a countable collection of closed balls $\{B_{j,k}\}_{k=1}^\infty$ such that $\diam(B_{j,k}) \leq \diam(Q_j)$ for each $k \geq 1$,
    and \[ \bigcup_{k=1}^\infty B_{j,k} \subset Q_j^\circ, \quad\text{and}\quad \mathcal{L}^n\left(Q_j^\circ \setminus \bigcup_{k=1}^\infty B_{j,k}\right) = 0. \]
    Since $\mathcal{L}^n(Q_j) = \mathcal{L}^n(Q_j^\circ)$ for each $j \geq 1$, we have
    \[ \mathcal{L}^n \left( Q_j \setminus \bigcup_{k=1}^\infty B_{j,k} \right) = 0. \]
    By Step 2, we know that
    \[ \mathcal{H}^n \left( Q_j \setminus \bigcup_{k=1}^\infty B_{j,k} \right) = 0. \]
    Therefore
    \begin{align*}
        \mathcal{H}^n_\delta(A) &\leq \mathcal{H}^n_\delta\left( \bigcup_{j=1}^\infty Q_j \right) &&\text{since } A \subseteq \bigcup_{j=1}^\infty Q_j\\
            &\leq \sum_{j=1}^\infty \mathcal{H}^n_\delta(Q_j) &&\text{by countable subadditivity}\\
            &= \sum_{j=1}^\infty \mathcal{H}^n_\delta\left( \bigcup_{k=1}^\infty B_{j,k} \right) &&\text{since }\  \mathcal{H}^n\left(Q_j\setminus \bigcup_{k=1}^\infty B_{j,k}\right) = 0\\
            &\leq \sum_{j,k=1}^\infty \mathcal{H}^n_\delta(B_{j,k}) &&\text{by countable subadditivity}\\
            &\leq \sum_{j,k=1}^\infty \omega_n \left( \frac{\diam(B_{j,k})}{2} \right)^n &&\text{ by definition of } \mathcal{H}^n_\delta\\
            &= \sum_{j,k=1}^\infty \mathcal{L}^n(B_{j,k}) &&\text{since } B_{j,k} \text{ are balls}\\
            &= \sum_{j=1}^\infty \mathcal{L}^n\left( \bigcup_{k=1}^\infty B_{j,k} \right) &&\text{by countable disjoint additivity}\\
            &= \sum_{j=1}^\infty \mathcal{L}^n(Q_j) &&\text{since } \ \bigcup_{k=1}^\infty B_{j,k} \subset Q_j^\circ \\
            &\leq \mathcal{L}^n(A) + \epsilon && \text{by choice of } \{Q_j\}_{j=1}^\infty.
    \end{align*}
    Phew. This shows
    \[ \mathcal{H}^n_\delta(A) \leq \mathcal{L}^n(A) + \epsilon. \]
    Since $\epsilon > 0$ was arbitrary, we have
    \[ \mathcal{H}^n_\delta(A) \leq \mathcal{L}^n(A). \]
    Since $\delta > 0$ was arbitrary, taking limits as $\delta \to 0^+$ gives
    \[ \mathcal{H}^n(A) \leq \mathcal{L}^n(A). \]
    Since $A \subseteq \R^n$ was arbitrary, we have shown that $\mathcal{H}^n(A) \leq \mathcal{L}^n(A)$ for each $A \subseteq \R^n$.

    \vspace{2mm}
    Combining Steps 1 and 3, we conclude that $\mathcal{H}^n(A) = \mathcal{L}^n(A)$ for each set $A \subseteq \R^n$.
    We remark that the proof actually shows that $\mathcal{L}^n = \mathcal{H}^n = \mathcal{H}^n_\delta$ on $\R^n$ for each $\delta > 0$.
\end{proof}

\begin{lemma}[Hausdorff Dimension Upper Bound of $\R^n$]
    \label{lem:hausdorff_dimension_upper_bound_Rn}
    Let $n \in \N$ and let $\alpha > n$.
    Then $\mathcal{H}^\alpha(\R^n) = 0$.
\end{lemma}

\begin{proof}
    Fix $\alpha > n$ and let $k\geq 1$ be an arbitrary integer.
    The unit cube $[0,1]^n$ can be written as the almost disjoint union of $k^n$ cubes of side length $1/k$ and thus diameter $\sqrt{n}/k$.
    Therefore
    \[ \mathcal{H}^\alpha_{ \sqrt{n}/k }([0,1]^n) \leq \sum_{j=1}^{k^n} \omega_\alpha \left(\frac{\sqrt{n}}{2k}\right)^\alpha < \omega_\alpha \sum_{j=1}^{k^n} \frac{\sqrt{n}^\alpha}{k^\alpha} = \omega_\alpha \sqrt{n}^\alpha k^{n-\alpha}. \]
    Since $\alpha > n$, we have $n - \alpha < 0$, so taking limits as $k \to \infty$ gives
    \[ \mathcal{H}^\alpha([0,1]^n) = 0. \]
    Since $\R^n$ can be written as a countable union of translates of $[0,1]^n$, countable subadditivity and invariance of Hausdorff measure under isometries implies that
    \[ \mathcal{H}^\alpha(\R^n) = 0. \]
\end{proof}

\begin{exercise}[Hausdorff Dimension of $\R^n$]
    \label{ex:hausdorff_dimension_of_rn}
    Show that $\mathcal{H}_{\operatorname{dim}}(\R^n) = n$.
\end{exercise}
\begin{proof}
    We have $\mathcal{H}^n(\R^n) = \mathcal{L}^n(\R^n) = +\infty$ by Theorem \ref{thm:hn_equals_lebesgue_measure_on_rn}, so $\mathcal{H}_{\operatorname{dim}}(\R^n) \geq n$.
    By Lemma \ref{lem:hausdorff_dimension_upper_bound_Rn}, we have $\mathcal{H}^\alpha(\R^n) = 0$ for each $\alpha > n$, so $\mathcal{H}_{\operatorname{dim}}(\R^n) \leq n$.
    Therefore $\mathcal{H}_{\operatorname{dim}}(\R^n) = n$.
\end{proof}

\begin{exercise}[Hausdorff Dimension of Linear Subspaces]
    \label{ex:hausdorff_dimension_of_lin_subspace}
    Let $V$ be a linear or affine subspace of $\R^n$ with $\dim(V) = k \leq n$.
    Show that $\mathcal{H}_{\operatorname{dim}}(V) = k$.
\end{exercise}
\begin{proof}
    Let $V$ be a linear or affine subspace of $\R^n$ with $\dim(V) = k \leq n$.
    Then there exists an isometry $f : \R^k \to V$.
    By isometry invariance of Hausdorff dimension (Exercise \ref{ex:isometry_invariance_of_hausdorff_dimension}), we have
    \[ \mathcal{H}_{\operatorname{dim}}(V) = \mathcal{H}_{\operatorname{dim}}(\R^k). \]
    By Exercise \ref{ex:hausdorff_dimension_of_rn}, we have $\mathcal{H}_{\operatorname{dim}}(\R^k) = k$.
    Therefore $\mathcal{H}_{\operatorname{dim}}(V) = k$.
\end{proof}

Be really careful because true equations like $\mathcal{L}^k \otimes \mathcal{L}^{n-k} = \mathcal{L}^n$
and the above theorem make it tempting to write things like
\[ \mathcal{H}^k \otimes \mathcal{H}^{n-k} = \mathcal{H}^n \]
which is \emph{not} true.
In particular, without specifying which Hausdorff measures we are using on the left hand side, this should be interpreted as the $\mathcal{H}^k$ measure on $\R^n$ and the measure $\mathcal{H}^{n-k}$ on $\R^n$, which is not what we want.
This has the product measure $\mathcal{H}^k \otimes \mathcal{H}^{n-k}$ being defined on $\R^n \times \R^n$, which is not what we want.

What you can deduce from the equation $\mathcal{L}^k \otimes \mathcal{L}^{n-k} = \mathcal{L}^n$ and the above theorem is that
\[ \mathcal{H}^k_{\R^k} \otimes \mathcal{H}^{n-k}_{\R^{n-k}} = \mathcal{H}^n. \]
Notice how the product measure is now defined on $\R^k \times \R^{n-k} = \R^n$, which is what we want.
Be careful. Add a subscript to the Hausdorff measures if it helps you keep things straight. 

\subsection{Fubini-Tonelli Theorem for Hausdorff Measures}
We work in Euclidean space $\R^n$ with the standard Euclidean metric.

The following is a version of the Fubini-Tonelli Theorem for Hausdorff measures on orthogonal subspaces of $\R^n$.
This is used often, but is not stated explicitly in many references, so we include a proof here for completeness.
\begin{proposition}[Fubini-Tonelli for Hausdorff Measures on Orthogonal Subspaces]
    \label{prop:fubini_tonelli_for_hausdorff_measures_on_orthogonal_subspaces}
    Let $1 \leq k \leq n-1$ be an integer, and let $L$ be a $k$-dimensional affine subspace of $\R^n$, i.e.
    \[ L = a + V \]
    for some $a \in \R^n$ and some $k$-dimensional linear subspace $V$ of $\R^n$.
    Then for each non-negative or integrable function $f : \R^n \to [-\infty, \infty]$, we have
    \[ \int_{\R^n} f \, d\mathcal{H}^n = \int_L \left( \int_{V^\perp} f(x + y) \, d\mathcal{H}^{n-k}_{V^\perp}(y) \right) d\mathcal{H}^k_L(x) = \int_{V^\perp} \left( \int_L f(x + y) \, d\mathcal{H}^k_L(x) \right) d\mathcal{H}^{n-k}_{V^\perp}(y). \] 
\end{proposition}
When I asked Dr. Koch, he said ``obviously'' this is true by isometry invariance and Fubini-Tonelli for Lebesgue measure, so let's write that out carefully.
It's not as easy as he thought, because of the subtlety in Remark \ref{rmk:which_hausdorff_measure_on_subsets} about which Hausdorff measure we are using on the subsets $L$ and $V^\perp$ of $\R^n$.

\begin{proof}
    Since $L$ and $V^\perp$ are subsets of the metric space $\R^n$, we can consider the Hausdorff measures $\mathcal{H}^k_L$ on the metric space $L$ and $\mathcal{H}^{n-k}_{V^\perp}$ on the metric space $V^\perp.$
    Let $Q \in O(n)$ be an orthogonal transformation such that
    \[  Q(\R^k \times \{0\}) = V \quad \text{and} \quad Q(\{0\} \times \R^{n-k}) = V^\perp. \]
    Notice that for each $x\in \R^n$ there exists a unique $(z,t)\in \R^k \times \R^{n-k} = \R^n$ such that
    \[ x - a = Q(z,t) \]
    which implies that
    \[ x = (a + Q(z,0)) + Q(0,t) \in L + V^\perp.\]

    Now let $f : \R^n \to [-\infty, \infty]$ be a non-negative or integrable function.
    Then by translation invariance \ref{ex:translation_invariance_of_lebesgue_integral} and isometry invariance of the Lebesgue integral \ref{ex:linear_change_of_variables}, we have
        \[\int_{\R^n} f \, \dif\mathcal{L}^n = \int_{\R^n} f(a + Q(\cdot)) \,\dif \mathcal{L}^n. \]
    Then by Fubini-Tonelli for Lebesgue measure this becomes
    \[ \int_{\R^n} f \dif \mathcal{L}^n = \int_{\R^k} \left( \int_{\R^{n-k}} f(a + Q(z,t)) \, \dif \mathcal{L}^{n-k}(t) \right) \dif \mathcal{L}^k(z). \tag{$\dagger$} \]

    \vspace{2mm}

    See that the map 
    \[ \R^{n-k} \to  V^\perp, \quad t \mapsto Q(0,t) \]
    is an isometry between the metric spaces $\R^{n-k}$ and $V^\perp$ and has inverse $V^\perp \to \R^{n-k}, \, y \mapsto \operatorname{pr}_2 (Q^{-1}(y))$,
    where $\operatorname{pr}_2 : \R^k \times \R^{n-k} \to \R^{n-k}$ is the projection onto the second factor, i.e.
    \[ \operatorname{pr}_2(z,t) = t \qquad \forall (z,t) \in \R^k \times \R^{n-k}. \]
    Thus by Isometry Invariance of Hausdorff Measure (Lemma \ref{lem:isometry_invariance_of_hausdorff_outer_measure}) and the fact that $\mathcal{H}^{n-k}_{\R^{n-k}} = \mathcal{L}^{n-k}$, we have
    \[ \int_{\R^{n-k}} g(t) \dif\mathcal{L}^{n-k}(t) = \int_{V^\perp} g\left( \operatorname{pr}_2 (Q^{-1}(y)) \right) \dif\mathcal{H}^{n-k}_{V^\perp}(y) \tag{$\star$}\]
    for each non-negative or integrable function $g : \R^{n-k} \to [-\infty, \infty]$.

    Similarly, the map
    \[ \R^k \to L, \quad z \mapsto a + Q(z,0) \]
    is an isometry between the metric spaces $\R^k$ and $L$ and has inverse $L \to \R^k, \, x \mapsto \operatorname{pr}_1 (Q^{-1}(x - a))$,
    where $\operatorname{pr}_1 : \R^k \times \R^{n-k} \to \R^k$ is the projection onto the first factor, i.e.
    \[ \operatorname{pr}_1(z,t) = z \qquad \forall (z,t) \in \R^k \times \R^{n-k}. \]
    Thus by Isometry Invariance of Hausdorff Measure (Lemma \ref{lem:isometry_invariance_of_hausdorff_outer_measure}) and the fact that $\mathcal{H}^k_{\R^k} = \mathcal{L}^k$, we have
    \[ \int_{\R^k} h(z) \dif\mathcal{L}^k(z) = \int_{L} h\left( \operatorname{pr}_1(Q^{-1}(x - a)) \right) \dif\mathcal{H}^k_{L}(x) \tag{$\star\star$}\]
    for each non-negative or integrable function $h : \R^k \to [-\infty, \infty]$.

    Now for each $z\in \R^k$, define a function
    \[ g_z:\R^{n-k} \to [-\infty, \infty], \quad g_z(t) := f(a + Q(z,t)) = f\left( (a+Q(z,0)) + Q(0,t) \right) \]
    so that for each $y\in V^\perp$, we have
    \[  y = Q(0,t) \text{ for some } t \in \R^{n-k} \implies \operatorname{pr}_2(Q^{-1}(y)) = t \]
    which implies that
    \[ g_z\left( \operatorname{pr}_2 (Q^{-1}(y))\right) = f\left( (a+Q(z,0)) + Q\left(0, \operatorname{pr}_2 (Q^{-1}(y))\right) \right) = f((a+Q(z,0)) + y) \qquad \forall y \in V^\perp. \]
    Hence for each fixed $z\in \R^k$, we can use $(\star)$ on the function $g_z$ to get
    \[ \int_{\R^{n-k}} f(a + Q(z,t)) \dif\mathcal{L}^{n-k}(t) = \int_{V^\perp} f((a+Q(z,0)) + y) \dif\mathcal{H}^{n-k}_{V^\perp}(y). \tag{$\heartsuit$}\]

    Now define a function
    \[ h:\R^k \to [-\infty, \infty], \quad h(z) := \int_{V^\perp} f((a+Q(z,0)) + y) \dif\mathcal{H}^{n-k}_{V^\perp}(y) \]
    so that for all $x \in L$ we have
    \[ x\in L = a + V \implies x - a \in V \implies Q^{-1}(x - a) \in \R^k \times \{0\} \]
    so that we have
    \begin{align*}
        h\left( \operatorname{pr}_1(Q^{-1}(x - a)) \right) &= \int_{V^\perp} f\left( (a+Q(\operatorname{pr}_1(Q^{-1}(x - a)),0)) + y \right) \dif\mathcal{H}^{n-k}_{V^\perp}(y) \\
            &= \int_{V^\perp} f( a + (x-a) + y) \dif\mathcal{H}^{n-k}_{V^\perp}(y) \\
            &= \int_{V^\perp} f(x + y) \dif\mathcal{H}^{n-k}_{V^\perp}(y).
    \end{align*}
    Since this is true for each $x\in L$, see that $(\star\star)$ applied to the function $h$ implies
    \[ \int_{\R^k} \left( \int_{V^\perp} f((a+Q(z,0)) + y) \dif\mathcal{H}^{n-k}_{V^\perp}(y) \right) \dif\mathcal{L}^k(z) = \int_{L} \left( \int_{V^\perp} f(x + y) \dif\mathcal{H}^{n-k}_{V^\perp}(y) \right) \dif\mathcal{H}^k_{L}(x). \tag{$\clubsuit$}\]

    Finally we can put it all together
    \begin{align*}
        \int_{\R^n} f \, \dif\mathcal{H}^n &= \int_{\R^n} f \, \dif\mathcal{L}^n &&\text{ since } \mathcal{H}^n = \mathcal{L}^n\\
            &= \int_{\R^k} \left( \int_{\R^{n-k}} f(a + Q(z,t)) \, \dif \mathcal{L}^{n-k}(t) \right) \dif \mathcal{L}^k(z) &&\text{ by } (\dagger)\\
            &= \int_{\R^k} \left( \int_{V^\perp} f((a+Q(z,0)) + y) \dif\mathcal{H}^{n-k}_{V^\perp}(y) \right) \dif\mathcal{L}^k(z) &&\text{ by } (\heartsuit)\\
            &= \int_{L} \left( \int_{V^\perp} f(x + y) \dif\mathcal{H}^{n-k}_{V^\perp}(y) \right) \dif\mathcal{H}^k_{L}(x) &&\text{ by } (\clubsuit)
    \end{align*}
    as desired. 

    The other equality follows by a symmetric argument.
\end{proof}


\begin{exercise}[Cavaleri's Principle]
    \label{ex:cavalieris_principle}
    Let $A\subseteq \R^n$ be a measurable set and let $1 \leq k \leq n-1$ be an integer.
    Then
    \[ \mathcal{L}^n(A) = \int_{\R^k} \mathcal{H}^{n-k}(A \cap (\{x\} \times \R^{n-k})) \, d\mathcal{L}^k(x) = \int_{\R^{n-k}} \mathcal{H}^k(A \cap (\R^k \times \{y\})) \, d\mathcal{L}^{n-k}(y). \]
\end{exercise}
\begin{proof}
    See that
    \begin{align*} 
        \mathcal{L}^n(A) &= \mathcal{H}^n(A) = \int_{\R^n} \Chi_A \, \dif\mathcal{H}^n &&\text{ since } \mathcal{H}^n = \mathcal{L}^n\\
            &= \int_{\R^k \times \{0\}} \left( \int_{\{0\} \times \R^{n-k}} \Chi_A((x,y)) \, \dif\mathcal{H}^{n-k}_{\{0\} \times \R^{n-k}}(0,y) \right) d\mathcal{H}^k_{\R^k \times \{0\}}(x,0) &&\text{ by Proposition \ref{prop:fubini_tonelli_for_hausdorff_measures_on_orthogonal_subspaces}}\\
            &= \int_{\R^k} \left( \int_{\R^{n-k}} \Chi_A(x,y) \, \dif\mathcal{H}^{n-k}_{\R^{n-k}}(y) \right) d\mathcal{H}^k_{\R^k}(x) &&\text{ isometry invariance of Hausdorff measures }\\
            &= \int_{\R^k} \mathcal{H}^{n-k}(A \cap (\{x\} \times \R^{n-k})) \, \dif\mathcal{H}^k_{\R^k}(x) \\
            &= \int_{\R^k} \mathcal{H}^{n-k}(A \cap (\{x\} \times \R^{n-k})) \, \dif\mathcal{L}^k(x) &&\text{ since } \mathcal{H}^k_{\R^k} = \mathcal{L}^k
    \end{align*}
    and similarly the other way around. 
\end{proof}

\end{document}