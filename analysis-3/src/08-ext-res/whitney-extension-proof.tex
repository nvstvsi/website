
\section{Proof of Whitney's $C^m$ Extension Theorem}

In this section we present the proof of Whitney's $C^m$ Extension Theorem.

\begin{lemma}[Modulus of Continuity for Whitney Jets]
    \label{lem:modulus_of_continuity_for_whitney_jet}
    A \textit{modulus of continuity} is a function $\omega: [0,\infty) \to [0,\infty)$ which is increasing, concave, and satisfies $\omega(0) = 0$.

    Let $A \subseteq \R^n$ be a nonempty compact set, and let $f^\bullet \in \mathcal{W}^m(A)$ be a Whitney $m$-jet on $A$.
    Then there exists a modulus of continuity $\omega$ such that for each multi-index $\alpha$ with $|\alpha| \leq m$, we have
    \[ \left\| (R^m_a f^\bullet)^{(\alpha)} (x) \right\| \leq \omega\left( \|x-a\|\right) \cdot \|x-a\|^{m-|\alpha|} \quad \forall\, x,a\in A \]
    and also satisfying 
    \[ \omega(t) = \omega(\diam A) \qquad \forall \, t\geq \diam A, \]
    and \[ \|f^\bullet\|_{\mathcal{W}^m(A)} = \| f^\bullet \|_{J^m(A)} + \omega(\diam A). \]
\end{lemma}

    \begin{proof}
        Define $\rho: [0,\infty) \to [0,\infty)$ by
        \[ \rho(t) := \sup \left\{ \frac{\| (R^m_a f^\bullet)^{(\alpha)} (x) \|}{\|x-a\|^{m-|\alpha|}} : a,x \in A, 0 < \|x-a\| \leq t, |\alpha| \leq m \right\} \]
        if $t > 0$, and $\rho(0) := 0$.
        Note that $\rho$ is finite-valued since $f^\bullet \in \mathcal{W}^m(A)$.
        Then $\rho$ is clearly increasing and 
        \[ \lim_{t\to 0^+} \rho(t) = 0 = \rho(0) \]
        so $\rho$ is continuous at $0$.
        Also if $t \geq \diam A$, then $\rho(t) = \rho(\diam A)$, so $\rho$ is constant on $[\diam A,\infty)$.
        (We note that $\rho$ is not necessarily concave or continuous on $(0,\infty)$.)

        Now define 
        \[ \operatorname{conv}(\rho) := \operatorname{conv}\left( [0,\infty) \cup \graph (\rho) \right) \]
        to be the closed convex hull of the graph of $\rho$ together with the nonnegative $t$-axis.
        Since $\operatorname{conv}(\rho)$ is convex and $\rho$ is non-negative, 
        see that for each $t \geq 0$, the set $\{ s : (t,s) \in \operatorname{conv}(\rho) \}$ is a closed (possibly degenerate) interval of the form $[0,\omega(t)]$ for some $\omega(t) \geq \rho(t) \geq 0$.

        This defines a function $\omega$ on $[0,\infty)$ by $\omega(t) := \sup \{ s : (t,s) \in \operatorname{conv}(\rho) \}$.

        Let us check the required properties of $\omega$.
        By construction, $\operatorname{conv}(\rho)$ is a closed subset of the upper half-plane, so $\omega(t) \geq 0$ for each $t \geq 0$.
        Thus $\omega: [0,\infty) \to [0,\infty)$ is well-defined.

        Now the only point in $[0,\infty) \cup \graph(\rho)$ with $t=0$ is $(0,0)$, so 
        the only point in $\operatorname{conv}(\rho)$ with $t=0$ is $(0,0)$, which implies $\omega(0) = 0$.
        
        We also see that $\omega$ is concave.
        Let $t_1,t_2 \geq 0$ and $\lambda \in [0,1]$ be arbitrary.
        Then the points $(t_1,\omega(t_1))$ and $(t_2,\omega(t_2))$ are in the convex set $\operatorname{conv}(\rho)$, so we have
        \[ (\lambda t_1 + (1-\lambda) t_2, \lambda \omega(t_1) + (1-\lambda) \omega(t_2)) \in \operatorname{conv}(\rho). \]
        But then the definition of $\omega(\lambda t_1 + (1-\lambda)t_2)$ implies that
        \[ \lambda \omega(t_1) + (1-\lambda) \omega(t_2) \leq \omega(\lambda t_1 + (1-\lambda) t_2). \]
        Since $t_1,t_2 \in [0,\infty)$ and $\lambda \in [0,1]$ were arbitrary, we conclude that $\omega$ is concave.

        Now $\omega: [0,\infty) \to [0,\infty)$ is a concave function with $\omega(0) = 0$, so we know from Analysis 1 that $\omega$ is increasing and continuous on $[0,\infty)$.

        Suppose towards a contradiction that there exists $t_0 > \diam A$ such that $\omega(t_0) > \rho(\diam A)$.
        Then we define the function 
        \[ \hat{\omega}: [0,\infty)\to [0,\infty), \quad \hat{\omega} := \begin{cases}
            \omega(t) &\text{if } 0 \leq t \leq \diam A, \\
            \omega(\diam A) &\text{if } t \geq \diam A.
        \end{cases} \]
        Note that $\hat{\omega}$ is a continuous function since $\hat{\omega}|_{[0,\diam A]} = \omega|_{[0,\diam A]}$ is continuous and $\hat{\omega}(\diam A) = \omega(\diam A)$.
        Also note that $\hat{\omega}$ is concave because it is the pointwise minimum of the concave function $\omega$ and the constant (concave) function $\omega(\diam A)$.

        That is, $\hat{\omega}$ is a continuous increasing concave function with $\hat{\omega}(0) = 0$ and $\hat{\omega}(t) = \rho(\diam A)$ for each $t \geq \diam A$.
        (In reality, $\hat{\omega}$ is literally equal to $\omega$, but it is easier to define a new function $\hat{\omega}$ than to prove directly that $\omega$ has this last property.)

        \vspace{2mm}
        The final thing to prove is the remainder estimate and the norm identity.
        \vspace{2mm}

        Let $\alpha$ be a multi-index with $|\alpha| \leq m$.
        Then by definition of $\rho$ we see that
        \[ \| (R^m_a f^\bullet)^{(\alpha)}(x) \| \leq \rho(\|x-a\|) \cdot \|x-a\|^{m-|\alpha|} \leq \hat{\omega}(\|x-a\|) \cdot \|x-a\|^{m-|\alpha|} \qquad\forall\, x,a\in A \]
        as desired.
        Also 
        \begin{align*}
            \| f^\bullet \|_{\mathcal{W}^m(A)} &= \| f^\bullet \|_{J^m(A)} + \sup\left\{ \frac{\| (R^m_a f^\bullet)^{(\alpha)}(x) \|}{\|x-a\|^{m-|\alpha|}} : x,a \in A, x\neq a, |\alpha| \leq m \right\} \\
                &= \| f^\bullet \|_{J^m(A)} + \sup \left\{ \frac{\| (R^m_a f^\bullet)^{(\alpha)}(x) \|}{\|x-a\|^{m-|\alpha|}} : x,a \in A, 0 < \|x-a\| \leq \diam A, |\alpha| \leq m \right\} \\
                &= \| f^\bullet \|_{J^m(A)} + \sup_{t \in (0,\diam A]} \rho(t) \\
                &= \| f^\bullet \|_{J^m(A)} + \rho(\diam A) \\
                &= \| f^\bullet \|_{J^m(A)} + \hat{\omega}(\diam A)
        \end{align*}
        where we have used that $A$ is compact in the second equality, and that $\rho$ is increasing in the third equality, and that $\hat{\omega}(\diam A) = \rho(\diam A)$ in the last equality.
        This finishes the proof of the claim.
    \end{proof}

We are now ready to prove Whitney's $C^m$ Extension Theorem.
We warn you that this is one of the \emph{longest} proofs in the book, and it is quite technical.

\begin{proof}[Proof of Whitney's $C^m$ Extension Theorem in the case $A \subset \R^n$ is compact]
    Assume that $A \subset \R^n$ is a nonempty compact set, and let $\{Q_j\}_{j=1}^\infty$ be the Whitney decomposition of $A^c$ given by Lemma \ref{lem:whitney_decomposition}. 
    Fix $0 < \varepsilon < \frac{1}{4}$ and for each $j \in \Z^+$, let $Q_j^* := (1+\varepsilon) Q_j$ be the cube with the same center as $Q_j$ but with side length $(1+\varepsilon)$ times that of $Q_j$. 
    Let $\{ \psi_j \}_{j=1}^\infty$ be the Whitney partition of unity subordinate to the open cover $\{Q_j^*\}_{j=1}^\infty$ of $A^c$ given by Lemma \ref{lem:whitney_partition_of_unity}.

    For each $j \in \Z^+$, choose a point $a_j \in A$ such that 
    \[ \dist(Q_j,A) = \dist(Q_j,a_j). \]
    For each jet $f^\bullet \in \mathcal{J}^m(A)$, we define the function $\mathcal{E}_m(f^\bullet) : \R^n \to \R$ by
    \[ \mathcal{E}_m(f^\bullet)(x) = \begin{cases}
    f^{(0)}(x) &\text{if } x \in A, \\
    \displaystyle \sum_{j=1}^\infty \psi_j(x) \cdot T^m_{a_j} f^\bullet (x) &\text{if } x \in A^c.
    \end{cases} \]
    This defines a linear map $\mathcal{E}_m$ from $\mathcal{J}^m(A)$ to functions on $\R^n$. 

    \vspace{2mm}
    \textit{Claim 1:}
    We claim that if $f^\bullet \in J^m(A)$, then $\mathcal{E}_m(f^\bullet)$ is $C^\infty$ on $A^c$.

    \begin{proof}[Proof of Claim 1]
        See that for each $j \in \Z^+$, at most $12^n$ other cubes from the Whitney decomposition can intersect $Q_j^*$ by Lemma \ref{lem:whitney_decomposition} and \ref{cor:whitney_cover_dilation}.
        Hence there are only finitely many terms in the sum defining $\mathcal{E}_m(f^\bullet)$ on $Q^*_j$ which are nonzero, and each of these terms is a smooth function on $Q_j^*$, so $\mathcal{E}_m(f^\bullet)$ is smooth on $Q_j^*$.

        Since $A^c = \bigcup_{j=1}^\infty Q_j^*$ by Lemma \ref{lem:whitney_decomposition}, we conclude that $\mathcal{E}_m(f^\bullet)$ is smooth on $A^c$.
    \end{proof}

    Now we assume that $f^\bullet \in \mathcal{W}^m(A)$, and let $f := \mathcal{E}_m(f^\bullet)$.
    We want to show that $f \in C^m(\R^n)$.

    \vspace{2mm}
    By Lemma \ref{lem:modulus_of_continuity_for_whitney_jet}, there exists a modulus of continuity $\omega$ such that for each multi-index $\alpha$ with $|\alpha| \leq m$, we have
    \[ \left\| (R^m_a f^\bullet)^{(\alpha)} (x) \right\| \leq \omega\left( \|x-a\|\right) \cdot \|x-a\|^{m-|\alpha|} \quad \forall\, x,a\in A \]
    and also satisfying 
    \[ \omega(t) = \omega(\diam A) \qquad \forall \, t\geq \diam A, \]
    and \[ \|f^\bullet\|_{\mathcal{W}^m(A)} = \| f^\bullet \|_{J^m(A)} + \omega(\diam A). \]

    Now we fix a cube $Q\subset \R^n$ such that $A \subset Q$ for the rest of the proof.
    
    \vspace{2mm}
    \textit{Claim 2:}
    For each multi-index $\alpha$ with $|\alpha| \leq m$, we define
        \[ \underline{\partial^{\alpha}} f(x) := \begin{cases}
            D^\alpha f(x) &\text{if } x \in A^c, \\
            f^{(\alpha)}(x) &\text{if } x \in A
        \end{cases} \]
    which is the candidate for the $\alpha$ partial derivative of $f$ on $\R^n$.

    We claim that there exists a constant $C > 0$ which only depends on $m$, $n$, and $\sup_{y\in Q} \dist(y,A)$ such that if $\alpha$ is a multi-index with $|\alpha| \leq m$, 
    and if $x \in Q$ and $a \in A$, then we have
    \[ \left| {\underline{\partial^{\alpha}}} f(x) - D^\alpha T^m_a f^\bullet(x) \right| \leq C \cdot \omega\left( \|x-a\|\right) \cdot \|x-a\|^{m-|\alpha|}. \tag{$\mathwitch$}\]

    For the moment, we will assume Claim 2 and use it to finish the proof of the theorem. 
    We claim that $(\mathwitch)$ implies that $f \in C^m(\R^n)$ and that there is a constant $c > 0$ depending only on $m$, $n$, and $\sup_{y\in Q} \dist(y,A)$ such that 
    \[ \| f \|_{C^m(Q)} \leq c \,\| f^\bullet \|_{\mathcal{W}^m(A)}. \tag{$\bigpumpkin$}\]
    Recall that the estimate $(\bigpumpkin)$ is the required continuity estimate for the extension operator $\mathcal{E}_m$.

    \begin{proof}[Proof that $(\mathwitch)$ implies $f \in C^m(\R^n)$]
        Fix a multi-index $\alpha$ with $|\alpha| < m$. 
        For each $j \in \{ 1,2 ,\ldots,n\}$, we let $e_j$ be the $j$-th standard basis vector in $\R^n$.
        If $a \in A$ and $x \in Q \setminus A$, then
        \[ \underline{\partial^{\alpha}} f(a) = f^{(\alpha)}(a) = D^\alpha T^m_a f^\bullet(a) \]
        so that 
        \begin{align*}
            \left| \underline{\partial^{\alpha}} f(x) - \underline{\partial^{\alpha}} f(a) - \sum_{j=1}^n \underline{\partial^{\alpha + e_j}} f(a) (x_j - a_j) \right| &= \left| \underline{\partial^{\alpha}} f(x) - D^\alpha T^m_a f^\bullet(a) - \sum_{j=1}^n \underline{\partial^{\alpha + e_j}} f(a) (x_j - a_j) \right| \\
                &\leq \left| \underline{\partial^{\alpha}} f(x) -  D^\alpha T^m_a f^\bullet(x) \right| \ + \\
                &\qquad\qquad + \left|  D^\alpha T^m_a f^\bullet(x) -  D^\alpha T^m_a f^\bullet(a) - \sum_{j=1}^n \underline{\partial^{\alpha + e_j}} f(a) (x_j - a_j) \right| \\
                &\leq C\cdot \omega\left( \|x-a\|\right) \cdot \|x-a\|^{m-|\alpha|} \ +\\
                &\qquad\qquad + \left|  D^\alpha T^m_a f^\bullet(x) -  D^\alpha T^m_a f^\bullet(a) - \sum_{j=1}^n D^{\alpha + e_j} T^m_a f^\bullet(a) (x_j - a_j) \right| \\ 
                &= o(\|x-a\|^{m-|\alpha|}) + o(\|x-a\|) \\
                &= o(\|x-a\|)
        \end{align*}
        where we have used the estimate $(\mathwitch)$ in the second inequality, and the fact that $T^m_a f^\bullet$ is a polynomial of degree at most $m$ in the second to last equality, and finally the fact that $|\alpha| < m$ in the last equality.

        If $a\in A$ and $x\in A$, then we have
        \begin{align*}
            \left| \underline{\partial^{\alpha}} f(x) - \underline{\partial^{\alpha}} f(a) - \sum_{j=1}^n \underline{\partial^{\alpha + e_j}} f(a) (x_j - a_j) \right| &= \left| D^{\alpha}T^m_a f^\bullet (x) - D^\alpha T^m_a f^\bullet(a) - \sum_{j=1}^n \underline{\partial^{\alpha + e_j}} f(a) (x_j - a_j) \right| \\
                &= \left| D^{\alpha}T^m_a f^\bullet (x) - D^\alpha T^m_a f^\bullet(a) - \sum_{j=1}^n D^{\alpha + e_j} T^m_a f^\bullet(a) (x_j - a_j) \right| \\
                &= o(\|x-a\|).
        \end{align*}
        where we have used the fact that $T^m_a f^\bullet$ is a polynomial of degree at most $m$.

        In either case, we have shown that for each multi-index $\alpha$ with $|\alpha| < m$ and each $a\in A$ we have
        \[ \left| \underline{\partial^{\alpha}} f(x) - \underline{\partial^{\alpha}} f(a) - \sum_{j=1}^n \underline{\partial^{\alpha + e_j}} f(a) (x_j - a_j) \right| = o(\|x-a\|) \quad \text{as } x \to a, \]
        and as a result, we conclude that $\underline{\partial^{\alpha}} f$ is differentiable at $a$ with derivative $\underline{\partial^{\alpha + e_j}} f(a) = f^{(\alpha + e_j)}(a)$ for each $j \in \{1,2,\ldots,n\}$.
        Since $a \in A$ was arbitrary, we conclude that $\underline{\partial^{\alpha}} f$ is differentiable at each point of $A$ with derivative $\underline{\partial^{\alpha + e_j}} f(a) = f^{(\alpha + e_j)}(a)$ for each $j \in \{1,2,\ldots,n\}$; hence $\underline{\partial^{\alpha}} f$ is $C^1$ on $\R^n$. 

        Now taking $\alpha = 0$ shows that for each $j\in \{1,2,\ldots,n\}$ the function $\underline{\partial_j} f$ is continuous, and the above argument combined with Claim 1 shows that $\underline{\partial_j} f$ is the $j^{\text{th}}$ partial derivative of $f$ on $\R^n$. 
        Therefore $f \in C^1(\R^n)$ and $\underline{\partial_j} f = D_j f$ for each $j \in \{1,2,\ldots,n\}$.

        In a similar way, we then see that for each multi-index $\alpha$ with $|\alpha| = 2$, the function $\underline{\partial^{\alpha}} f$ is continuous; writing
        $\alpha = e_i + e_j$ for some $i,j \in \{1,2,\ldots,n\}$, we see that $\underline{\partial^{\alpha}} f$ is the $i^\text{th}$ partial derivative of $\underline{\partial_j} f$, 
        and hence $\underline{\partial^{\alpha}} f$ is the $\alpha = e_i + e_j$ partial derivative of $f$ on $\R^n$.
        Since $\alpha$ was an arbitrary multi-index with $|\alpha| = 2$, we conclude that $f \in C^2(\R^n)$ and $\underline{\partial^{\alpha}} f = D^\alpha f$ for each multi-index $\alpha$ with $|\alpha| = 2$.

        Bootstrapping this argument, we conclude that for each multi-index $\alpha$ with $|\alpha| \leq m$ 
        the function $\underline{\partial^{\alpha}} f$ is continuous and is the $\alpha$ partial derivative of $f$ on $\R^n$; because this is true for each multi-index $\alpha$ with $|\alpha| < m$, we conclude that $f \in C^m(\R^n)$.
    \end{proof}

    \begin{proof}[Proof that $(\mathwitch)$ and $f \in C^m(\R^n)$ implies $(\bigpumpkin)$]
        Let $\alpha$ be a multi-index with $|\alpha| \leq m$, and let $x \in Q$ be arbitrary.
        Since $A$ is compact there is a point $a \in A$ such that $\dist(x,A) = \|x-a\|$.
        Set 
        \[ \lambda := \sup_{y \in Q} \dist(y,A). \]

        Then, as shown above we have $D^\alpha f(x) = \underline{\partial^{\alpha}} f(x)$, and for each $a \in A$ we have
        \[ D^\alpha (T^m_a f^\bullet)(x) = \sum_{|\beta|\leq m - |\alpha|} \frac{f^{(\alpha+\beta)}(a)}{\beta!}(x-a)^\beta = \sum_{|\beta| \leq m - |\alpha|} \frac{D^{\alpha+\beta} f(a)}{\beta!}(x-a)^\beta \tag{$\dagger$}\]
        where the first equality follows from the second computation in the proof of \ref{ex:remainder_jet_properties}, 
        and the second equality follows from the fact that $f^{(\alpha+\beta)}(a) = D^{\alpha+\beta} f(a)$ for all multi-indices $\beta$ with $|\beta| \leq m - |\alpha|$, which was shown in the previous step.
        
        Then we estimate
        \begin{align*}
            | D^\alpha f(x) | &\leq \left| D^\alpha f(x) - D^\alpha T^m_a f^\bullet(x) \right| + \left| D^\alpha T^m_a f^\bullet(x) \right| && \text{by triangle inequality}\\
                &= \left| \underline{\partial^{\alpha}} f(x) - D^\alpha T^m_a f^\bullet(x) \right| + \left| \sum_{|\beta| \leq m - |\alpha|} \frac{D^{\alpha+\beta} f(a)}{\beta!}(x-a)^\beta \right| && \text{ by } (\dagger)\\
                &\leq C \cdot \omega\left( \|x-a\| \right) \cdot \|x-a\|^{m-|\alpha|} + \sum_{|\beta| \leq m - |\alpha|}  \frac{\left|D^{\alpha+\beta} f(a)\right|}{\beta!}\left|(x-a)^\beta\right| &&\text{by }(\mathwitch) \text{ and the triangle inequality}\\
                &\leq C \cdot \omega\left( \lambda \right) \cdot \lambda^{m-|\alpha|} + \sum_{|\beta| \leq m - |\alpha|}  \frac{\left|D^{\alpha+\beta} f(a)\right|}{\beta!} \|x-a\|^{|\beta|} &&\text{since } \omega \text{ is increasing and } \|x-a\| \leq \lambda\\
                &\leq C \cdot \omega\left( \diam A \right) \cdot \lambda^{m-|\alpha|} + \sum_{|\beta| \leq m - |\alpha|}  \frac{\|f^\bullet\|_{J^m(A)}}{\beta!} \lambda^{|\beta|} \\
                &= C \left( \| f^\bullet \|_{\mathcal{W}^m(A)} - \| f^\bullet \|_{J^m(A)} \right) \cdot \lambda^{m-|\alpha|}  + \left( \sum_{|\beta| \leq m - |\alpha|}  \frac{1}{\beta!} \lambda^{|\beta|} \right) \cdot \|f^\bullet\|_{J^m(A)} \\
                &=: C \left( \| f^\bullet \|_{\mathcal{W}^m(A)} - \| f^\bullet \|_{J^m(A)} \right) \cdot \lambda^{m-|\alpha|}  + c_{m,\lambda} \cdot \|f^\bullet\|_{J^m(A)} \\
                &= C \lambda^{m-|\alpha|} \cdot \| f^\bullet \|_{\mathcal{W}^m(A)} + (c_{m,\lambda} - C \lambda^{m-|\alpha|}) \cdot \|f^\bullet\|_{J^m(A)} \\
        \end{align*}
        where $C$ is the constant from $(\mathwitch)$ and $c_{m,\lambda} := \sum_{|\beta| \leq m - |\alpha|}  \frac{1}{\beta!} \lambda^{|\beta|}$.
        Now let 
        \[ C' := \max \left\{ C, \frac{c_{m,\lambda}}{\lambda^{m-|\alpha|}} \right \} \]
        so that 
        \[ C \leq C' \quad\text{and}\quad \frac{c_{m,\lambda}}{\lambda^{m-|\alpha|}} \leq C' \]
        which implies
        \[ c_{m,\lambda} - C' \lambda^{m-|\alpha|} \leq 0. \]
        Then using the exact same argument as before but with $C'$ in place of $C$, we see that
        \begin{align*}
            |D^\alpha f(x)| &\leq C' \lambda^{m-|\alpha|} \cdot \| f^\bullet \|_{\mathcal{W}^m(A)} + (c_{m,\lambda} - C' \lambda^{m-|\alpha|}) \cdot \|f^\bullet\|_{J^m(A)} \\
                &\leq C' \lambda^{m-|\alpha|} \cdot \| f^\bullet \|_{\mathcal{W}^m(A)}.
        \end{align*}
        Setting 
        \[ c := C' \lambda^{m-|\alpha|} = \max \left\{ C \lambda^{m-|\alpha|}, c_{m,\lambda} \right\} \]
        we conclude that
        \[ |D^\alpha f(x)| \leq c \| f^\bullet \|_{\mathcal{W}^m(A)}. \]
        Since $\alpha$ was an arbitrary multi-index with $|\alpha| \leq m$ and $x\in Q$ was arbitrary, we conclude that
        \[ \max_{|\alpha| \leq m} \sup_{x \in Q} |D^\alpha f(x)| \leq c \| f^\bullet \|_{\mathcal{W}^m(A)} \]
        which says that 
        \[ \| f \|_{C^m(Q)} \leq c \,\| f^\bullet \|_{\mathcal{W}^m(A)} \]
        as desired in $(\bigpumpkin)$.
    \end{proof}

    It remains to prove Claim 2.
    For this, we make a subclaim which will be used in the proof of Claim 2.

    \vspace{2mm}
    \textit{Subclaim 2a.}
    We claim that if $a,b \in A$ and $\alpha$ is a multi-index with $|\alpha| \leq m$, then
    \[ \left| D^\alpha T^m_a f^\bullet(x) - D^\alpha T^m_b f^\bullet (x) \right| \leq 4^{m-|\alpha|} e^n \cdot \omega(\|a-b\|) \cdot\left( \|x-a\|^{m-|\alpha|} + \|x-b\|^{m-|\alpha|} \right) \qquad \forall\, x\in \R^n. \]

    \begin{proof}[Proof of Subclaim 2a.]
        Let $a,b \in A$ and $\alpha$ be a multi-index with $|\alpha| \leq m$.

        We will use three properties about jets and their formal derivatives and Taylor polynomials.
        Namely, for each multi-index $\beta$ with $|\beta| \leq m$, we have
        \[ f^{(\beta)}(a) = D^\beta(T^m_a f^\bullet)(a) \tag{i}\]  
        and
        \[ D^\beta(T^m_a f^\bullet - T^m_b f^\bullet)(a) = (R^m_b f^\bullet)^{(\beta)}(a) \tag{ii} \]
        and
        \[ f^{(\beta)}(b)(x-b)^\beta = D^\beta(T^m_b f^\bullet)(a)\cdot (x-a)^\beta, \tag{iii}\]
        which were proven in Exercise \ref{ex:more_formal_properties}.

        Then as a result of these properties, for each $x\in \R^n$ we have 
        \begin{align*}
            T^m_a f^\bullet(x) - T^m_b f^\bullet(x) &= \sum_{|\beta| \leq m} \frac{f^{(\beta)}(a)}{\beta!}(x-a)^\beta - \sum_{|\beta| \leq m} \frac{f^{(\beta)}(b)}{\beta!}(x-b)^\beta &&\text{ by definition of formal Taylor polynomial}\\
                &= \sum_{|\beta| \leq m} \frac{(x-a)^\beta}{\beta!} D^\beta(T^m_a f^\bullet)(a) \, - \sum_{|\beta| \leq m} \frac{(x-a)^\beta}{\beta!} D^\beta(T^m_b f^\bullet)(a) &&\text{ by properties (i) and (iii)}\\
                &= \sum_{|\beta| \leq m} \frac{(x-a)^\beta}{\beta!} \left( D^\beta(T^m_a f^\bullet)(a) - D^\beta(T^m_b f^\bullet)(a) \right) \\
                &= \sum_{|\beta| \leq m} \frac{(x-a)^\beta}{\beta!} D^\beta(T^m_a f^\bullet - T^m_b f^\bullet)(a) \\
                &= \sum_{|\beta| \leq m} \frac{(x-a)^\beta}{\beta!} (R^m_b f^\bullet)^{(\beta)}(a) &&\text{ by property (ii)}.
        \end{align*}

        As a result, we see that 
        \[ D^\alpha(T^m_a f^\bullet)(x) - D^\alpha(T^m_b f^\bullet)(x) = \sum_{|\beta| \leq m-|\alpha|} \frac{(x-a)^\beta}{\beta!} (R^m_a f^\bullet)^{(\beta+\alpha)}(a) \qquad \forall \, x\in \R^n. \]
        Hence if $x\in \R^n$ is such that $\|x-a\| \leq \|x-b\|$, then we can estimate
        \begin{align*}
            \left| D^\alpha T^m_a f^\bullet(x) - D^\alpha T^m_b f^\bullet (x) \right| &\leq
                \sum_{|\beta| \leq m-|\alpha|} \frac{\|x-a\|^{|\beta|}}{\beta!} \cdot \left\| (R^m_a f^\bullet)^{(\beta+\alpha)}(a) \right\| \\
                &\leq \sum_{|\beta| \leq m-|\alpha|} \frac{\|x-a\|^{|\beta|}}{\beta!} \cdot \|a-b\|^{m-|\alpha|-|\beta|} \cdot \omega(\|a-b\|) \\
                &= \sum_{|\beta|\leq m - |\alpha|} \frac{1}{\beta!} \cdot \|x-a\|^{|\beta|} \cdot \|a-b\|^{m-|\alpha|-|\beta|} \cdot \omega(\|a-b\|). &&(\star)
        \end{align*}
        Then for each $x\in \R^n$ such that $\|x-a\| \leq \|x-b\|$, and each multi-index $\beta$ with $|\beta| \leq m-|\alpha|$, we have
        \begin{align*}
            \|x-a\|^{|\beta|} \|a-b\|^{m-|\alpha|-|\beta|} &\leq \|x-a\|^{|\beta|}\left( \|x-a\| + \|a-b\| \right)^{m-|\alpha|-|\beta|} \\
                &\leq (\|x-a\|+\|a-b\|)^{m-|\alpha|} \\
                &\leq 2^{m-|\alpha|} \cdot \left( \|x-a\|^{m-|\alpha|} + \|a-b\|^{m-|\alpha|} \right) \\
                &\leq 2^{m-|\alpha|} \cdot \left( \|x-a\|^{m-|\alpha|} + \left( \|a-x\|+ \|x-b\| \right)^{m-|\alpha|}\right) \\
                &\leq 2^{m-|\alpha|} \cdot \left( \|x-a\|^{m-|\alpha|} + (2\|x-b\|)^{m-|\alpha|} \right) \\
                &\leq 4^{m-|\alpha|} \cdot \left( \|x-a\|^{m-|\alpha|} + \|x-b\|^{m-|\alpha|} \right) \\
        \end{align*}
        Returning to the estimate $(\star)$, we conclude that
        \begin{align*}
            \left| D^\alpha T^m_a f^\bullet(x) - D^\alpha T^m_b f^\bullet (x) \right| &\leq 
                4^{m-|\alpha|} \cdot \sum_{|\beta| \leq m-|\alpha|} \frac{1}{\beta!} \cdot \omega(\|a-b\|) \cdot \left( \|x-a\|^{m-|\alpha|} + \|x-b\|^{m-|\alpha|} \right) \\
                &\leq 4^{m-|\alpha|} \left( \sum_{\beta\in \N^n} \frac{1}{\beta!} \right) \cdot \omega(\|a-b\|) \cdot \left( \|x-a\|^{m-|\alpha|} + \|x-b\|^{m-|\alpha|} \right) \\
                &\leq 4^{m-|\alpha|} \left( \prod_{j=1}^n \sum_{k=0}^\infty \frac{1}{k!} \right) \cdot \omega(\|a-b\|) \cdot \left( \|x-a\|^{m-|\alpha|} + \|x-b\|^{m-|\alpha|} \right) \\
                &\leq 4^{m-|\alpha|} e^n \cdot \omega(\|a-b\|) \cdot \left( \|x-a\|^{m-|\alpha|} + \|x-b\|^{m-|\alpha|} \right).
        \end{align*}

        Similarly, if $x\in \R^n$ is such that $\|x-b\| \leq \|x-a\|$, then we can estimate
        \[ \left| D^\alpha T^m_a f^\bullet(x) - D^\alpha T^m_b f^\bullet (x) \right| \leq 4^{m-|\alpha|} e^n \cdot \omega(\|a-b\|) \cdot \|x-b\|^{m-|\alpha|}. \]
        Putting these two estimates together, we conclude that for each $x\in \R^n$ we have
        \[ \left| D^\alpha T^m_a f^\bullet(x) - D^\alpha T^m_b f^\bullet (x) \right| \leq 4^{m-|\alpha|} e^n \cdot \omega(\|a-b\|) \cdot\left( \|x-a\|^{m-|\alpha|} + \|x-b\|^{m-|\alpha|} \right) \]
        as stated in the subclaim.

    \end{proof}


    Now we prove $(\mathwitch)$ in Claim 2 using two cases. 

    \vspace{2mm}

    \noindent Let $\alpha$ be a multi-index with $|\alpha| \leq m$, and let $x \in Q$ and $a \in A$ be arbitrary.
    First consider the case where $x \in A$.

    \begin{proof}[Proof of $(\mathwitch)$ when $x \in A$.]
        If $x \in A$, then $\underline{\partial^{\alpha}} f(x) = f^{(\alpha)}(x)$ and we can estimate
        \begin{align*}
            \left| {\underline{\partial^{\alpha}}} f(x) - D^\alpha T^m_a f^\bullet(x) \right|
                &= \left| f^{(\alpha)}(x) - T^{m-|\alpha|}_a D^\alpha f^{\bullet}(x) \right| \\
                &= \left| (R^{m-|\alpha|}_a f^{\bullet + (\alpha)}) (x) \right| \\
                &\leq \omega\left( \|x-a\|\right) \cdot \|x-a\|^{m-|\alpha|} 
        \end{align*}
        where the first equality follows from Exercise \ref{ex:formal_derivative_of_formal_taylor_polynomial}, 
        and the final inequality follows from Lemma \ref{lem:modulus_of_continuity_for_whitney_jet} applied to the $m-|\alpha|$-jet $D^\alpha f^\bullet$.
        This shows that $(\mathwitch)$ holds for each $x \in A$.
    \end{proof}

    \begin{proof}[Proof of $(\mathwitch)$ when $x \in A^c$.]
        (NOTE: We have \emph{not} yet used the definition of the extension operator $\mathcal{E}_m$ in this proof, and the 
        consequences of the Whitney partition of unity and the Whitney decomposition, so this is where we will use these tools.)
        
    Assume that $x\in Q\setminus A$. Then by definition of $f = \mathcal{E}_m(f^\bullet)$ and the fact that $\{\psi_j\}_{j=1}^\infty$ is a partition of unity on $A^c$,
    we have
    \[ f(x) - T^m_a f^\bullet(x) = \sum_{j=1}^\infty \psi_j(x) T^m_{a_j} f^\bullet(x) - \sum_{j=1}^\infty \psi_j(x) T^m_a f^\bullet(x) = \sum_{j=1}^\infty \psi_j(x) \left( T^m_{a_j} f^\bullet(x) - T^m_a f^\bullet(x) \right). \]
    Hence we have
    \begin{align*}
        D^\alpha f(x) - D^\alpha T^m_a f^\bullet(x) &= \sum_{\beta \leq \alpha} \binom{\alpha}{\beta} \sum_{j=1}^\infty D^{\beta} \psi_j(x) \cdot D^{\alpha - \beta} \left( T^m_{a_j} f^\bullet(x) - T^m_a f^\bullet(x) \right) 
    \end{align*}
    by the Leibniz rule, which implies that
    \begin{align*}
        \left| D^\alpha f(x) - D^\alpha T^m_a f^\bullet(x) \right| &\leq
            \sum_{\beta \leq \alpha} \binom{\alpha}{\beta} \left| \sum_{j=1}^\infty D^{\beta} \psi_j(x) \cdot D^{\alpha - \beta} \left( T^m_{a_j} f^\bullet(x) - T^m_a f^\bullet(x) \right) \right|.
    \end{align*}
    For each multi-index $\beta$ with $\beta \leq \alpha$, we let 
    \[ S_\beta(x) := \sum_{j=1}^\infty D^\beta \psi_j(x) \cdot D^{\alpha - \beta} \left( T^m_{a_j} f^\bullet(x) - T^m_a f^\bullet(x) \right) \]
    so that the previous equation reads
    \[ \left| D^\alpha f(x) - D^\alpha T^m_a f^\bullet(x) \right| \leq \sum_{\beta \leq \alpha} \binom{\alpha}{\beta} |S_\beta(x)|. \tag{$\heartsuit$}\]

    First we estimate $|S_0(x)|$.
    If $j \geq 1$ is such that $x \in Q^*_j$, then we have
    \begin{align*}
        \| x-a_j \| &\leq \diam Q^*_j + \dist(Q_j^*, A) \\
            &\leq \frac{5}{4} \,\diam Q_j + \frac{3}{4} \,\dist(Q_j, A) &&\text{ since $Q_j^*$ is the dilation of $Q_j$ by a factor of } 1+\varepsilon < \frac{5}{4} \\
            &\leq 2 \diam Q_j + \dist(Q_j, A) \\
            &\leq 8 \dist(Q_j, A) + \dist(Q_j, A) &&\text{by Lemma \ref{lem:whitney_decomposition}} \\
            &\leq 9 \|x-a\|
    \end{align*}
    where the first inequality follows from the triangle inequality,
    and the final inequality follows from the fact that $x \in Q_j$ and $a \in A$.
    
    Hence if $j \geq 1$ is such that $x \in Q_j^*$, then we have
    \[ \|a-a_{j}\| \leq \|a-x\| + \|x-a_{j}\| \leq \|a-x\| + 9 \|x-a\| = 10 \|x-a\| \]
    which implies 
    \[ \omega(\|a-a_{j}\|) \leq 10 \omega(\|x-a\|) \]
    since $\omega$ is concave and increasing.

    If $j \geq 1$ is such that $x \notin Q_j^*$, then $\psi_j(x) = 0$ and so this term does not contribute to the sum defining $S_0(x)$.

    Therefore we have 
    \begin{align*}
    |S_0(x)| &\leq \sum_{ \{ j\, : \,x \in Q_j^* \} } 1 \cdot \left| D^\alpha(T^m_{a_j} f^\bullet - T^m_a f^\bullet)(x) \right| \\
        &\leq C_{m,n} \cdot \sum_{\{ j\, : \,x \in Q_j^* \}} \omega(\|a-a_j\|) \left( \|x-a\|^{m-|\alpha|} + \|x-a_j\|^{m-|\alpha|} \right) && \text{by Subclaim 2a} \\
        &\leq C_{m,n} \cdot \sum_{\{ j\, : \,x \in Q_j^* \}} 10 \omega(\|x-a\|) \left( \|x-a\|^{m-|\alpha|} + (9 \|x-a\|)^{m-|\alpha|} \right) \\
        &\leq C_{m,n} \cdot 2^n \cdot 10(9^{m-|\alpha|} + 1) \cdot \omega(\|x-a\|) \cdot \|x-a\|^{m-|\alpha|} &&\text{by Corollary \ref{cor:whitney_cover_dilation}}
    \end{align*}
    where $C_{m,n} = 4^m e^n$ is the constant from Subclaim 2a for the case of the zero multi-index.
    Thus there is a constant $C_0$ which only depends on $m$ and $n$ such that
    \[ |S_0(x)| \leq C_0 \cdot \omega(\|x-a\|) \cdot \|x-a\|^{m-|\alpha|}. \tag{$\clubsuit$}\]

    Next let $\beta$ be a multi-index with $\beta \leq \alpha$ and $|\beta| \geq 1$, and consider $|S_\beta(x)|$.
    Then see that for each $b \in A$ we have
    \begin{align*}
        S_\beta (x) &= \sum_{j=1}^\infty D^\beta \psi_j(x) \cdot D^{\alpha - \beta} \left( T^m_{a_j} f^\bullet(x) - T^m_a f^\bullet(x) \right) \\
            &= \sum_{j=1}^\infty D^\beta \psi_j(x) \cdot \left( D^{\alpha - \beta} T^m_{a_j} f^\bullet(x) - D^{\alpha - \beta} T^m_b f^\bullet(x) + D^{\alpha - \beta} T^m_b f^\bullet(x) - D^{\alpha - \beta} T^m_a f^\bullet(x) \right) \\
            &= \sum_{j=1}^\infty D^\beta \psi_j(x) \cdot \left( D^{\alpha - \beta} T^m_{a_j} f^\bullet(x) - D^{\alpha - \beta} T^m_b f^\bullet(x) \right) + \sum_{j=1}^\infty D^\beta \psi_j(x) \cdot \left( D^{\alpha - \beta} T^m_b f^\bullet(x) - D^{\alpha - \beta} T^m_a f^\bullet(x) \right) \\
            &= \sum_{j=1}^\infty D^\beta \psi_j(x) \cdot \left( D^{\alpha - \beta} T^m_{a_j} f^\bullet(x) - D^{\alpha - \beta} T^m_b f^\bullet(x) \right) + 0 \\
            &= \sum_{j=1}^\infty D^\beta \psi_j(x) \cdot \left( D^{\alpha - \beta} T^m_{a_j} f^\bullet(x) - D^{\alpha - \beta} T^m_b f^\bullet(x) \right)
    \end{align*}
    since $\sum_j \psi_j \equiv 1$ on $A^c$ and hence $\sum_j D^\beta \psi_j \equiv 0$ on $A^c$ for each multi-index $\beta$ with $|\beta| \geq 1$.

    Now choose a point $b\in A$ such that $\dist(x,A) = \|x-b\|$.
    Then if $j \geq 1$ is such that $x \in Q_j^*$, then as before we have $\|x-a_j\| \leq 9 \|x-b\|$ so that 
    \[ \|a_j - b\| \leq \|a_j - x\| + \|x-b\| \leq 10 \|x-b\| \]
    and 
    \[ \omega(\|a_j - b\|) \leq 10 \omega(\|x-b\|). \]
    Hence we can estimate
    \begin{align*}
        |S_\beta(x)| &\leq \sum_{ \{j\,:\, x \in Q_j^* \} }
            \left| D^\beta \psi_j(x) \cdot D^{\alpha - \beta} \left( T^m_{a_j} f^\bullet - T^m_b f^\bullet \right)(x) \right| \\
            &\leq \sum_{ \{j\,:\, x \in Q_j^* \} } \left| D^\beta \psi_j(x) \right| \cdot
                \left( 4^{m-|\alpha-\beta|}e^n \omega(\|a_j - b\|) \left( \|x-a_j\|^{m-|\alpha - \beta|} + \|x-b\|^{m-|\alpha - \beta|} \right) \right)
                && \quad \text{ by Subclaim 2a} \\
            &\leq C_{m,n,\beta} \cdot \sum_{ \{j\,:\, x \in Q_j^* \} } \left| D^\beta \psi_j(x) \right| \cdot 10 \omega(\|x-b\|) \left( (9\|x-b\|)^{m-|\alpha - \beta|} + \|x-b\|^{m-|\alpha -\beta|} \right) 
                && \quad\text{ where } C_{m,n,\beta} := 4^{m-|\alpha-\beta|} e^n \\
            &\leq C_{m,n,\beta} \cdot 10(9^{m-|\alpha - \beta|} + 1) 
                \cdot \omega(\|x-b\|)\|x-b\|^{m-|\alpha -\beta|} \sum_{ \{j\,:\, x \in Q_j^* \} } \left| D^\beta \psi_j(x) \right| \\
            &\leq C_{m,n,\beta} \cdot 10(9^{m-|\alpha - \beta|} + 1) \cdot \omega(\|x-b\|)\|x-b\|^{m-|\alpha - \beta|} \cdot C_\beta \sum_{ \{ j\,:\, x\in Q^*_j \} } (\diam Q_j)^{-|\beta|}
    \end{align*}
    where the last inequality holds by the property (iv) of the Whitney partition of unity \ref{prop:whitney_partition_of_unity}, and the constant $C_\beta$ only depends only on $\beta$ and $n$.

    Letting
    \[ \hat{C}_{\beta} := C_{m,n,\beta} \cdot 10(9^{m-|\alpha - \beta|} + 1) \cdot C_\beta \]
    to make things more managable, we have shown that 
    \[ |S_\beta(x)| \leq \hat{C}_{\beta} \cdot \omega(\|x-b\|)\|x-b\|^{m-|\alpha-\beta|} \sum_{ \{ j\,:\, x\in Q^*_j \} } (\diam Q_j)^{-|\beta|}. \]
    Now fix some $l \in \Z^+$ such that $x \in Q_l$. 
    Then for each $j\geq 1$ such that $x\in Q_j^*$, the cubes $Q_j$ and $Q_l$ must be adjacent, as shown in the proof of Corollary \ref{cor:whitney_cover_dilation}.
    Hence there are at most $2^n$ terms in the sum $\sum_{ \{ j\,:\, x\in Q^*_j \} } (\diam Q_j)^{-|\beta|}$, 
    as shown in \ref{cor:whitney_cover_dilation},
    and each term is at most $4^{|\beta|}(\dist(Q_l, A))^{-|\beta|}$ by Lemma \ref{lem:whitney_decomposition} (b) and (c).

    Hence we have 
    \begin{align*}
        |S_\beta(x)| &\leq 2^n \cdot 4^{|\beta|} \cdot \hat{C}_{\beta} \cdot \omega(\|x-b\|)\|x-b\|^{m-|\alpha-\beta|} \cdot (\dist(Q_l, A))^{-|\beta|} \\
            &= 2^n \cdot 4^{|\beta|} \cdot \hat{C}_{\beta} \cdot \omega\left( \dist(x,A)\right) \left(\dist(x,A)\right)^{m-|\alpha-\beta|} 
                \cdot \left( \dist(x,A) \right)^{-|\beta|} \\
    \end{align*} 
    because $\|x-b\| = \dist(x,A)$ and $\dist(Q_l, A) \geq \dist(x,A)$ by virtue of $x \in Q_l$.
    Since $\beta \leq \alpha$, we see that $m-|\alpha-\beta| = m-|\alpha| + |\beta|$ and thus we have
    \[ |S_\beta(x)| \leq 2^n \cdot 4^{|\beta|} \cdot \hat{C}_{\beta} \cdot \omega\left( \dist(x,A)\right) \left(\dist(x,A)\right)^{m-|\alpha|}. \]
    Letting $C'_\beta := 2^n \cdot 4^{|\beta|} \cdot \hat{C}_{\beta}$, we have shown that
    \[ |S_\beta(x)| \leq C'_\beta \cdot \omega\left( \dist(x,A)\right) \left(\dist(x,A)\right)^{m-|\alpha|}. \tag{$\spadesuit$}\]

    Putting this all together, we have 
    \begin{align*}
        | D^\alpha f(x) - D^\alpha T^m_a f^\bullet(x) | &\leq
            \sum_{\beta \leq \alpha} \binom{\alpha}{\beta} |S_\beta(x)| && \text{ by }(\heartsuit) \\
            &= |S_0(x)| + \sum_{\substack{\beta \leq \alpha, \\ |\beta| \geq 1}} \binom{\alpha}{\beta} |S_\beta(x)| \\
            &\leq C_0 \cdot \omega(\|x-a\|) \cdot \|x-a\|^{m-|\alpha|} + \sum_{\substack{\beta \leq \alpha, \\ |\beta| \geq 1}} \binom{\alpha}{\beta} C'_\beta \cdot\omega\left( \dist(x,A)\right) \left(\dist(x,A)\right)^{m-|\alpha|} &&\text{ by }(\clubsuit) \text{ and }(\spadesuit) \\
            &= C \cdot \omega\left( \dist(x,A)\right) \left(\dist(x,A)\right)^{m-|\alpha|} 
    \end{align*}
    where we have set
    \[ C := C_0 + \sum_{\substack{\beta \leq \alpha, \\ |\beta| \geq 1}} \binom{\alpha}{\beta} C'_\beta. \]
    Because $x\in A^c$, we have $\underline{\partial^\alpha} f(x) = D^\alpha f(x)$, and thus we have shown $(\mathwitch)$ in the case where $x \in A^c$.
    \end{proof}

\end{proof}

\begin{proof}[Proof of Whitney's Extension Theorem in the general case]
    Let $A \subseteq \R^n$ be a closed set, and let $f^\bullet\in \mathcal{W}^m(A)$ be a Whitney $m$-jet on $A$.
    


    partition of unity argument
    
\end{proof}

    We did it.
    That is like, \emph{the}, coolest theorem.
    I have been wanting to go through that full proof since I learned about the result as an undergrad. 

It turns out that even more can be said about the extension operator in terms of continuity.
If things are properly defined, 
it can be shown that a small modification of the operator $\mathcal{E}_m$ we constructed is a continuous map
from the Banach space of $s$-H\"older continuous Whitney jets on a compact subset to the Banach space $C^{m,s}(\R^n)$
of $C^m$ functions on $\R^n$ with $s$-H\"older continuous $m^{\text{th}}$ derivatives.
The details are in Stein's book \textit{Singular Integrals and Differentiability Properties of Functions}.

The natural follow up question is the following: Is there a continuous extension operator for jets of infinite order, satisfying a 
Whitney condition? 

Grothendieck showed that such an extension operator, with the natural definitions in place, cannot exist on a general closed set --- his example shows that
the singleton $\{0\}$ is a closed set for which no such extension operator can exist.
See Bierstone's paper \textit{Differentiable Functions} in the Bulletin of the Brazilian Mathematical Society for details.
