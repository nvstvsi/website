

\section{The Radon-Nikodym Theorem}

The Radon-Nikodym Theorem is a fundamental result in measure theory that generalizes the concept of derivatives to measures.

First we need a lemma.

\begin{lemma}
    \label{lem:radon_nik_lemma}
    Let $(X,\mathcal{A},\mu)$ be a measure space, and let $\lambda$ be another measure on $(X,\mathcal{A})$ that is absolutely continuous with respect to $\mu$ and is not the zero measure.
    Then there exists a positive number $\epsilon > 0$ and a measurable set $A_0 \in \mathcal{A}$ such that
    $\mu(A_0) > 0$ and $A_0$ is positive with respect to the signed measure $\lambda - \epsilon \mu$, i.e.
    \[ \lambda( A ) - \epsilon \mu( A ) \geq 0 \quad\text{ for all } A \in \mathcal{A} \text{ such that } A \subseteq A_0. \]
\end{lemma}
\begin{proof}
    Let $N \in \mathbb{Z}^+$ be arbitrary; consider the signed measure $\lambda - \frac{1}{N} \mu$.
    Then let 
    \[ X = A^+_N \cup A_N^- \]
    be a Hahn decomposition (\ref{thm:hahn_decomposition}) for this signed measure.

    Let 
    \[ A^+ = \bigcap_{N=1}^\infty A_N^+ \quad\text{ and }\quad A^- = \bigcap_{N=1}^\infty A_N^-. \]
    Then for each $N \in \Z^+$ we have \[ \lambda(A^-) \leq \frac{1}{N} \mu(A^-) \]
    which implies that $\lambda(A^-) = 0$. 
    Hence we must have $\lambda(A^+) > 0$ since $\lambda$ is not the zero measure.
    But then absolute continuity of $\lambda$ with respect to $\mu$ implies that $\mu(A^+) > 0$.
    By \ref{properties_of_signed_measures}, we see that
    \[ \lim_{N \to \infty} \mu(A_N^+) = \mu(A^+) \]
    because the sets $A_N^+$ are decreasing and $A^+ = \bigcap_{N=1}^\infty A_N^+$.
    Thus there exists some $N \in \Z^+$ such that $\mu(A_N^+) > 0$, and we may take $A_0 = A_N^+$.

    That is, for this choice of $N$, we have
    \[ \lambda(A) - \frac{1}{N} \mu(A) \geq 0 \quad\text{ for all } A \in \mathcal{A} \text{ such that } A \subseteq A_0. \]
    The proof is complete.
\end{proof}

\begin{theorem}[Radon-Nikodym Theorem for Signed Measures]
    \label{thm:radon_nikodym_signed_measure}
    Let $(X,\mathcal{A},\mu)$ be a measure space, and let $\nu$ be a signed measure on $(X,\mathcal{A})$ that is absolutely continuous with respect to $\mu$.
    Then there exists a unique $f \in L^1(X,\mu)$ such that
    \[ \nu(A) = \int_A f \,\dif \mu \quad\text{ for all } A \in \mathcal{A}. \]
\end{theorem}

Thus our primary example is essentially the only example.
After the proof, we will discuss how this is a generalization of the fundamental theorem of calculus.

The more popular proof is by John von Neumann using Hilbert spaces, but we will present the more elementary proof found in Kolmogorov and Fomin.
\begin{proof}
    If $\nu$ is the zero measure, then $f = 0$ works, and uniqueness $\mu$-almost everywhere is clear.
    Thus we may assume that $\nu$ is not the zero measure. We let
    \[ \mathcal{K} := \left\{ f\in L^1(X,\mu) : \int_A f\,\dif\mu \leq \nu(A) \ \ \text{ for all }\  A \in \mathcal{A} \right\} \]
    and let
    \[ M := \sup \left\{ \int_X f \,\dif \mu : f \in \mathcal{K} \right\}. \]
    Then there exists a sequence of functions $\{ f_k \}_{k=1}^\infty \subseteq \mathcal{K}$ such that
    \[ \lim_{k\to\infty} \int_X f_k \,\dif \mu = M. \]
    For each $n \in \Z^+$ we define
    \[ g_n := \max \{ f_1, f_2, \ldots, f_n \}. \]
    Then we obtain an increasing sequence $\{ g_n \}_{n=1}^\infty$ of measurable functions on $X$ and we claim that
    \[ g_n \in \mathcal{K} \quad\text{ for all } n \in \Z^+. \]

    To see this, fix $n\in \Z^+$ and let $A \in \mathcal{A}$ be arbitrary.
    Then there are disjoint sets $A_{n,1},A_{n,2},\ldots,A_{n,n} \in \mathcal{A}$ such that
    \[ A = \bigcup_{k=1}^n A_{n,k} \quad\text{ and }\quad g_n(x) = f_k(x) \text{ for all } x \in A_{n,k}. \]
    Thus we have
    \begin{align*}
        \int_A g_n \,\dif \mu &= \int_{\bigcup_{k=1}^n A_{n,k}} g_n\,\dif \mu = \sum_{k=1}^n \int_{A_{n,k}} f_k \,\dif \mu \\
        &\leq \sum_{k=1}^n \nu(A_{n,k}) = \nu(A)
    \end{align*}
    since $f_k \in \mathcal{K}$ for all $k \in \{1,2,\ldots,n\}$.
    Since $A\in \mathcal{A}$ was arbitrary, this shows $g_n \in \mathcal{K}$.
    Since $n \in \Z^+$ was arbitrary, the claim is proved.

    Since $\{ g_n \}_{n=1}^\infty$ is an increasing sequence of measurable functions in $\mathcal{K}$ we must have
    \[ \lim_{n\to\infty} \int_X g_n \,\dif \mu = M. \]
    We check this --- by virtue of $\{ g_n \}_{n=1}^\infty \subseteq \mathcal{K}$, the inequality $\lim_{n\to\infty} \int_X g_n \,\dif \mu \leq M$ is clear.
    If we had a strict inequality $\lim_{n\to\infty} \int_X g_n \,\dif \mu < M$, then monotonicity of the integral would imply that
    \[ \lim_{k\to\infty} \int_X f_k \,\dif \mu \leq \lim_{n\to\infty} \int_X g_n \,\dif \mu < M, \]
    contradicting our choice of $\{ f_k \}_{k=1}^\infty$.
    Thus we have equality as desired.

    We now define
    \[ f(x) := \lim_{n\to\infty} g_n(x) = \sup_{n\in \Z^+} g_n(x) \]
    for all $x \in X$.
    Then $f$ is a measurable function on $X$, as a limit of measurable functions, and the Bounded Convergence Theorem \ref{thm:bounded_convergence_theorem} implies that $f\in L^1(X,\mu)$ and
    \[ \int_X f \,\dif \mu = \int_X \lim_{n\to\infty} g_n \,\dif \mu = \lim_{n\to\infty} \int_X g_n \,\dif \mu = M < \infty. \]
    
    We now show that $f$ is the required function.

    We define a set function $\lambda : \mathcal{A} \to \R$ by
    \[ \lambda(A) := \nu(A) - \int_A f \,\dif \mu \quad\text{ for all } A \in \mathcal{A}. \]
    Then $\lambda$ is a signed measure on $(X,\mathcal{A})$.
    Since we must have $f \in \mathcal{K}$, it follows that $\lambda$ is a nonnegative measure, i.e. $\lambda(A) \geq 0$ for all $A \in \mathcal{A}$.
    We claim that $\lambda$ is the zero measure.

    Assume towards a contradiction that this is not the case, and there is a set $E\in \mathcal{A}$ for which $\lambda(E) > 0$.
    Then by Lemma \ref{lem:radon_nik_lemma} there exists $\epsilon > 0$ and a measurable set $A_0 \in \mathcal{A}$ such that $\mu(A_0) > 0$ and
    \[ \lambda(A) - \epsilon \mu(A) \geq 0 \quad\text{ for all } A \in \mathcal{A} \text{ such that } A \subseteq A_0. \]
    That is, 
    \[ \epsilon \mu(A) \leq \lambda(A) \quad \text{ for all } A \in \mathcal{A} \text{ such that } A \subseteq A_0. \]
    We define
    \[ h(x) := f(x) + \epsilon \chi_{A_0}(x) \qquad \forall x \in X. \]
    Then for each $A \in \mathcal{A}$ we have
    \begin{align*}
        \int_A h \,\dif \mu &= \int_A f\,\dif \mu + \int_A \epsilon \chi_{A_0} \,\dif \mu = \int_A f \,\dif \mu + \epsilon \mu(A \cap A_0) \\
            &\leq \nu(A) - \lambda(A) + \lambda(A \cap A_0) = \nu(A)
    \end{align*}
    which shows that $h \in \mathcal{K}$.
    However, we also have
    \[ \int_X h \,\dif \mu = \int_X f \,\dif \mu + \epsilon \mu(A_0) > M \]
    since $\mu(A_0) > 0$.
    This contradicts the definition of $M$ as a supremum, and implies that $\lambda$ is the zero measure.
    As a result, we have
    \[ \nu(A) = \int_A f \,\dif \mu \quad\text{ for all } A \in \mathcal{A}. \]

    It remains to show that the integrable function $f$ is unique in $L^1(X,\mu)$.
    Recall that $L^1(X,\mu)$ is defined as the set of equivalence classes of functions that are equal $\mu$-almost everywhere, so we need to show that if $f^\star \in L^1(X,\mu)$ is another function such that
    \[ \nu(A) = \int_A f^\star \,\dif \mu \quad\text{ for all } A \in \mathcal{A}, \]
    then $f = f^\star$ $\mu$-almost everywhere.

    We let
    \[ E := \{ x \in X : f(x) \neq f^\star(x) \} \]
    and see that
    \[ E = \bigcup_{n=1}^\infty \left( \left\{ x\in X : f(x) - f^\star(x) > \frac{1}{n} \right\} \cup \left\{ x \in X : f^\star(x) - f(x) > \frac{1}{n} \right\} \right). \]
    For each $n \in \Z^+$ Chebyshev's inequality \ref{lem:chebyshevs_inequality} implies that
    \[ \mu\left( \left\{ x\in X: f(x) - f^\star(x) > \frac{1}{n} \right\} \right) \leq n \cdot\int_{ \left\{ x\,:\, f(x) - f^\star(x) > \frac{1}{n} \right\} } |f - f^\star| \,\dif \mu = 0 \]
    and similarly
    \[ \mu\left( \left\{ x\in X: f^\star(x) - f(x) > \frac{1}{n} \right\} \right) = 0. \]
    Thus we have $\mu(E) = 0$, as the countable union of $\mu$ measure zero sets has $\mu$ measure zero.
    This shows that $f = f^\star$ $\mu$-almost everywhere, and the proof is complete.
\end{proof}

\begin{corollary}[Radon-Nikodym Theorem for Vector Measures]
    \label{cor:radon_nikodym_vector_measure}
    Let $(X,\mathcal{A},\mu)$ be a measure space, and let $\nu$ be a vector measure on $(X,\mathcal{A})$ taking values in $\R^m$ that is absolutely continuous with respect to $\mu$.
    Then there exists a unique function $f \in L^1(X,\mu;\R^m)$ such that
    \[ \nu(A) = \int_A f \,\dif \mu \quad\text{ for all } A \in \mathcal{A}. \]
\end{corollary}
\begin{proof}
    For each $k \in \{1,2,\ldots,m\}$ we define the signed measure $\nu_k : \mathcal{A} \to \R$ by
    \[ \nu_k(A) := \langle \nu(A), e_k \rangle \quad\text{ for all } A \in \mathcal{A} \]
    where $e_1, e_2, \ldots, e_m$ is the standard basis for $\R^m$.
    Then for each $k \in \{1,2,\ldots,m\}$ the signed measure $\nu_k$ is absolutely continuous with respect to $\mu$ since 
    \[ |\nu_k(A)| = | \langle \nu(A), e_k \rangle | \leq \| \nu(A) \| \cdot \| e_k \| = \| \nu(A) \| \]
    for all $A \in \mathcal{A}$.
    Thus for each $k \in \{1,2,\ldots,m\}$ we may apply Theorem \ref{thm:radon_nikodym_signed_measure} to obtain a unique $f_k \in L^1(X,\mu)$ such that
    \[ \nu_k(A) = \int_A f_k \,\dif \mu \quad\text{ for all } A \in \mathcal{A}. \]
    We then define the function $f : X \to \R^m$ by
    \[ f(x) := \sum_{k=1}^m f_k(x) e_k \quad\text{ for all } x \in X. \]
    Then $f \in L^1(X,\mu;\R^m)$ since each $f_k \in L^1(X,\mu)$, and for each $A \in \mathcal{A}$ we have
    \begin{align*}
        \int_A f \,\dif \mu &= \int_A \sum_{k=1}^m f_k(x) e_k \,\dif \mu(x) = \sum_{k=1}^m \left( \int_A f_k(x) \,\dif \mu(x) \right) e_k \\
            &= \sum_{k=1}^m \nu_k(A) e_k = \sum_{k=1}^m \langle \nu(A), e_k \rangle e_k = \nu(A)
    \end{align*}
    which shows existence.
    Uniqueness follows from the uniqueness in Theorem \ref{thm:radon_nikodym_signed_measure} applied to each component.
\end{proof}

By combining the Lebesgue Decomposition Theorem \ref{thm:lebesgue_decomposition_theorem} and the Radon-Nikodym Theorem for $\R^m$-valued vector measures, we obtain the following fundamental result.
\begin{corollary}[Lebesgue-Radon-Nikodym Theorem]
    \label{cor:lebesgue_radon_nikodym_theorem}
    Let $(X,\mathcal{A},\mu)$ be a measure space, and let $\nu$ be a vector measure on $(X,\mathcal{A})$ taking values in $\R^m$.
    Then there exist unique vector measures $\nu_{\text{ac}}$ and $\nu_{\text{s}}$ on $(X,\mathcal{A})$ such that
    \begin{enumerate}
        \item $\nu = \nu_{\text{ac}} + \nu_{\text{s}}$,
        \item $\nu_{\text{ac}} \ll \mu$,
        \item $\nu_{\text{s}} \perp \mu$,
    \end{enumerate}
    and there exists a unique function $f \in L^1(X,\mu;\R^m)$ such that
    \[ \nu(A) = \int_A f \,\dif \mu + \nu_{\text{s}}(A) \quad\text{ for all } A \in \mathcal{A}. \]
\end{corollary}

\begin{proof}
    The existence and uniqueness of the measures $\nu_{\text{ac}}$ and $\nu_{\text{s}}$ follow from the Lebesgue Decomposition Theorem \ref{thm:lebesgue_decomposition_theorem}.
    Since $\nu_{\text{ac}} \ll \mu$, the Radon-Nikodym Theorem for Vector Measures \ref{cor:radon_nikodym_vector_measure} implies that there exists a unique function $f \in L^1(X,\mu;\R^m)$ such that
    \[ \nu_{\text{ac}}(A) = \int_A f \,\dif \mu \quad\text{ for all } A \in \mathcal{A}. \]
    Thus we have
    \begin{align*}
        \nu(A) &= \nu_{\text{ac}}(A) + \nu_{\text{s}}(A) \\
            &= \int_A f \,\dif \mu + \nu_{\text{s}}(A)
    \end{align*}
    for all $A \in \mathcal{A}$, and the proof is complete.
\end{proof}

\begin{definition}
    \label{def:radon_nikodym_derivative}
    Let $(X,\mathcal{A},\mu)$ be a measure space, and let $\nu$ be a vector measure on $(X,\mathcal{A})$ that is absolutely continuous with respect to $\mu$.
    The \textit{Radon-Nikodym derivative} of $\nu$ with respect to $\mu$ is the unique function $f \in L^1(X,\mu;\R^m)$ such that
    \[ \nu(A) = \int_A f \,\dif \mu \quad\text{ for all } A \in \mathcal{A}. \]
    We denote the Radon-Nikodym derivative by $D_\mu \nu$ or $\pd{\nu}{\mu}$.
\end{definition}

Some people also write $ \nu = \mu \mres f $ or $\dif \nu = f \,\dif \mu$ to denote this relationship, but we will not use this notation.

With our preferred notation, the Radon-Nikodym theorem states that if $\nu$ is a signed measure absolutely continuous with respect to $\mu$, then
\[ \pd{\nu}{\mu} \in L^1(X,\mu) \quad\text{ and }\quad \nu(A) = \int_A \pd{\nu}{\mu} \,\dif \mu \text{ for all } A \in \mathcal{A} \]
which shows how it can be viewed as a generalization of the fundamental theorem of calculus.

\begin{corollary}[$\pd{\nu}{|\nu|}$ has Unit Norm]
    \label{cor:radon_nikodym_derivative_unit_norm}
    Let $(X,\mathcal{A},\mu)$ be a measure space, and let $\nu$ be a vector measure on $(X,\mathcal{A})$ taking values in $\R^m$ that is absolutely continuous with respect to $\mu$.
    Then the Radon-Nikodym derivative $\pd{\nu}{|\nu|} : X \to \R^m$ satisfies
    $ \left\| \pd{\nu}{|\nu|}(x) \right\| = 1$
    for $|\nu|$-almost every $x \in X$.
\end{corollary}

\begin{proof}
    Since $\nu \ll |\nu|$, the Radon-Nikodym Theorem for Vector Measures \ref{cor:radon_nikodym_vector_measure} implies that there exists a unique function $f \in L^1(X,|\nu|;\R^m)$ such that
    \[ \nu(A) = \int_A f \,\dif |\nu| \quad\text{ for all } A \in \mathcal{A}. \]
    By \ref{prop:total_variation_of_vector_measure_from_L1_function}, we have
    \[ |\nu|(A) = \int_A \| f(x) \| \,\dif |\nu|(x) \quad\text{ for all } A \in \mathcal{A}. \]
    In particular, taking $A = X$ gives
    \[ |\nu|(X) = \int_X \| f(x) \| \,\dif |\nu|(x). \]
    Thus we must have $\| f(x) \| = 1$ for $|\nu|$-almost every $x \in X$.
    By definition, $f$ is the Radon-Nikodym derivative $\pd{\nu}{|\nu|}$, so we are done.
\end{proof}

In the case that we have two $\sigma$-finite measures, we have the following version of the Radon-Nikodym theorem.
\begin{corollary}[Radon-Nikodym Theorem for $\sigma$-Finite Measures]
    \label{cor:radon_nikodym_finite_sigma_finite}
    Let $(X,\mathcal{A},\mu)$ be a $\sigma$-finite measure space, and let $\nu$ be a $\sigma$-finite measure on $(X,\mathcal{A})$ that is absolutely continuous with respect to $\mu$.
    Then there exists a unique measurable function $f : X \to [0,\infty)$ such that
    \[ \nu(A) = \int_A f \,\dif \mu \quad\text{ for all } A \in \mathcal{A}. \]
\end{corollary}

Notice that we need \emph{both} measures to be $\sigma$-finite, and we lose the conclusion of integrability of $f$ in this case.
\begin{proof}
    In the case that $\nu$ is a finite measure, then $\nu$ is also a signed measure, and we may apply Theorem \ref{thm:radon_nikodym_signed_measure} to obtain the desired result.

    Now assume that $\nu$ is a $\sigma$-finite measure.
    Then there exists a disjoint sequence of sets $\{ X_k \}_{k=1}^\infty \subseteq \mathcal{A}$ such that
    \[ X = \bigcup_{k=1}^\infty X_k \quad\text{ and }\quad \nu(X_k) < \infty \quad\text{ for all } k \in \Z^+. \]
    For each $k \in \Z^+$ we define the measure $\nu \mres X_k$ which is a finite measure on $(X,\mathcal{A})$.
    Then $\nu \mres X_k$ is absolutely continuous with respect to $\mu$ since $\nu$ is absolutely continuous with respect to $\mu$.
    Thus by the finite measure case, there exists a unique function $f_k \in L^1(X,\mu)$ such that
    \[ \nu(A \cap X_k) = \int_{A \cap X_k} f_k \,\dif \mu \quad\text{ for all } A \in \mathcal{A}. \]
    We then define the function $f : X \to \R$ by
    \[ f(x) := f_k(x) \]
    if $x\in X$ is such that $x \in X_k$.
    More formally, we have
    \[ f(x) := \sum_{k=1}^\infty f_k(x) \chi_{X_k}(x) \quad\text{ for all } x \in X. \]
    Then $f$ is a measurable function on $X$, as a countable sum of measurable functions.
    
    We compute that for each $A \in \mathcal{A}$ we have
    \begin{align*}
        \int_A f \,\dif \mu &= \sum_{k=1}^\infty \int_{A \cap \left( X_k \setminus \bigcup_{j=1}^{k-1} X_j \right)} f_k \,\dif \mu \\
            &= \sum_{k=1}^\infty \nu\left( A \cap \left( X_k \setminus \bigcup_{j=1}^{k-1} X_j \right) \right) = \nu(A)
    \end{align*}
    which shows existence.
    
    Now suppose that $f^\star : X \to [0,\infty)$ is another measurable function such that
    \[ \nu(A) = \int_A f^\star \,\dif \mu \quad\text{ for all } A \in \mathcal{A}. \]
    Then for each $k \in \Z^+$ we have
    \[ \nu(A \cap X_k) = \int_{A \cap X_k} f^\star \,\dif \mu \quad\text{ for all } A \in \mathcal{A}. \]
    By uniqueness in the finite measure case, it follows that $f_k = f^\star$ $\mu$-almost everywhere on $X_k$ for each $k \in \Z^+$.
    Thus $f = f^\star$ $\mu$-almost everywhere on $X$, and the proof is complete.
\end{proof}

\begin{corollary}[Lebesgue-Radon-Nikodym Theorem for $\sigma$-Finite Measures]
    \label{cor:lebeague_radon_nikodym_theorem_sigma_finite}
    Let $(X,\mathcal{A},\mu)$ be a $\sigma$-finite measure space, and let $\nu$ be a $\sigma$-finite measure on $(X,\mathcal{A})$.
    Then there exist unique measures $\nu_{\text{ac}}$ and $\nu_{\text{s}}$ on $(X,\mathcal{A})$ such that
    $\nu = \nu_{\text{ac}} + \nu_{\text{s}}$,
    $\nu_{\text{ac}} \ll \mu$,
    and $\nu_{\text{s}} \perp \mu$,
    and there exists a unique measurable function $f : X \to [0,\infty)$ such that
    \[ \nu(A) = \int_A f \,\dif \mu + \nu_{\text{s}}(A) \quad\text{ for all } A \in \mathcal{A}. \]
\end{corollary}

The proof is the same as that of Corollary \ref{cor:lebesgue_radon_nikodym_theorem} using Corollary \ref{cor:radon_nikodym_finite_sigma_finite} in place of Theorem \ref{thm:radon_nikodym_signed_measure}.

\begin{exercise}[Properties of the Radon-Nikodym Derivative]
    \label{ex:properties_of_radon_nikodym_derivative}
    Let $(X,\mathcal{A})$ be a measurable space, and let $\mu$ and $\lambda$ be measures on $(X,\mathcal{A})$, and let $\nu$ and $\sigma$ be vector measures on $(X,\mathcal{A})$ taking values in $\R^m$.
    The Radon-Nikodym derivatives satisfy the following properties:
    \begin{enumerate}
    \item If $\nu \ll \mu$ and $\sigma \ll \mu$, then
        \[ \pd{ (\nu + \sigma) }{\mu} = \pd{\nu}{\mu} + \pd{\sigma}{\mu}. \]
    \item If $\lambda \ll \mu$ and $g \in L^1(X,\lambda)$, then
        \[ \int_X g\,\dif \lambda = \int_X g \pd{\lambda}{\mu} \,\dif \mu. \]
    \item If $\nu \ll \lambda$ and $\lambda \ll \mu$, then
        \[ \pd{\nu}{\mu} = \pd{\nu}{\lambda} \pd{\lambda}{\mu}. \]
    \item If $\nu$ and $\mu$ are mutually absolutely continuous, then
        \[ \pd{\nu}{\mu} = \frac{1}{\left( \pd{\mu}{\nu} \right)} \]
    \end{enumerate}
\end{exercise}

\begin{proof}
    \begin{enumerate}
        \item Assume that $\nu \ll \mu$ and $\sigma \ll \mu$.
            Then for each $A \in \mathcal{A}$ we have
            \begin{align*}
                (\nu + \sigma)(A) &= \nu(A) + \sigma(A) = \int_A \pd{\nu}{\mu} \,\dif \mu + \int_A \pd{\sigma}{\mu} \,\dif \mu \\
                    &= \int_A \left( \pd{\nu}{\mu} + \pd{\sigma}{\mu} \right) \,\dif \mu
            \end{align*}
            which shows the desired result by uniqueness of the Radon-Nikodym derivative.

        \item Assume that $\lambda \ll \mu$.
            Then for each $A \in \mathcal{A}$ we have
            \[ \lambda(A) = \int_A 1 \,\dif \lambda = \int_A 1 \cdot \pd{\lambda}{\mu} \,\dif \mu = \int_X \Chi_A \pd{\lambda}{\mu} \,\dif \mu \]
            and \[ \lambda(A) = \int_X \Chi_A \,\dif \lambda \]
            so the result is true for characteristic functions.
            By the linearity of the integral, we see that if $s = \sum_{j = 1}^n a_j \Chi_{A_j}$
            is a simple function on $X$, then
            \[ \int_X s \,\dif \lambda = \sum_{j=1}^n a_j \lambda(A_j) = \int_X s \pd{\lambda}{\mu} \,\dif \mu. \]
            Thus the result is true for simple functions.

            Now let $g\in L^1(X,\lambda)$ be arbitrary.
            Then there exists a sequence of simple functions $\{ s_k \}_{k=1}^\infty$ on $X$ such that
            \[ \lim_{k\to\infty} \int_X | g - s_k | \,\dif \lambda = 0. \]
            By the Dominated Convergence Theorem \ref{thm:dominated_convergence_theorem}, we have
            \begin{align*}
                \int_X g \,\dif \lambda &= \int_X \lim_{k\to\infty} s_k \,\dif \lambda = \lim_{k\to\infty} \int_X s_k \,\dif \lambda \\
                    &= \lim_{k\to\infty} \int_X s_k \pd{\lambda}{\mu} \,\dif \mu = \int_X \lim_{k\to\infty} s_k \pd{\lambda}{\mu} \,\dif \mu = \int_X g \pd{\lambda}{\mu} \,\dif \mu
            \end{align*}
            which shows the desired result.

        \item Assume that $\nu \ll \lambda$ and $\lambda \ll \mu$.
            Then for each $A \in \mathcal{A}$ we have
            \[ \nu(A) = \int_A \pd{\nu}{\lambda} \,\dif \lambda = \int_A \pd{\nu}{\lambda} \pd{\lambda}{\mu} \,\dif \mu \]
            where the second equality follows from part (2).
            By uniqueness of the Radon-Nikodym derivative, we obtain the desired result.

        \item Assume that $\nu$ and $\mu$ are mutually absolutely continuous.
            Then for each $A \in \mathcal{A}$ we have
            \[ \mu(A) = \int_A \pd{\mu}{\nu} \,\dif \nu = \int_A \pd{\mu}{\nu} \pd{\nu}{\mu} \,\dif \mu \]
            where the second equality follows from part (2).
            Since we also have \[ \mu(A) = \int_A 1 \,\dif \mu, \]
            for each $A\in \mathcal{A}$, uniqueness of the Radon-Nikodym derivative implies that
            \[ 1 = \pd{\mu}{\nu} \pd{\nu}{\mu} \quad\text{ $\mu$-almost everywhere, } \]
            which shows the desired result.
    \end{enumerate}
\end{proof}

This finishes our discussion of the abstract Radon-Nikodym theorem.