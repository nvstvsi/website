\section{Besicovich Covering Theorem}

\subsection{Introduction}

In this section, we present the Besicovich Covering Theorem.
It is similar in spirit to the Vitali Covering Theorem we proved earlier, but has different hypotheses and conclusions.
The Besicovich Covering Theorem is particularly useful in differentiation theory and geometric measure theory, as it allows us to cover sets in $\R^n$ with families of balls in a controlled manner.

The Vitali Covering Theorem says that given an arbitrary collection of balls with uniformly bounded radii, we can extract a disjoint subcollection such that if we scale these balls by a factor of $5$, they cover the same set as the original collection.
This is a powerful obeservation, but does not really help us unless we have a way to control $\mu(5B)$ in terms of $\mu(B)$ for a measure $\mu$ --- this is known as a \textit{doubling condition}, and is not satisfied by all measures.
The Besicovich Covering Theorem, on the other hand, allows us to cover a set using a bounded number of disjoint subcollections of balls from the original collection, without needing to scale the balls.
This is particularly useful when working with measures that do not satisfy a doubling condition.

It is the Besicovich Covering Theorem that allows us to extend the Lebesgue Differentiation Theorem to Radon measures on $\R^n$.

\subsection{Besicovich Covering Theorem}

\begin{theorem}[Besicovich Covering Theorem]
    \label{thm:besicovich_covering_theorem}
    For each $n\in \Z^+$, there exists an integer constant $M_n \in \Z^+$ such that the following holds.

    Let $A \subseteq \R^n$ be a nonempty set, and let $\mathcal{B}$ be a collection of closed balls in $\R^n$ such that for each $a \in A$, there exists a ball $B \in \mathcal{B}$ centered at $a$ and the diameters are uniformly bounded, i.e.
    \[ \sup \{ \operatorname{diam}(B) : B \in \mathcal{B} \} < \infty. \]
    Then there exist subcollections $\mathcal{B}_1, \mathcal{B}_2, \ldots, \mathcal{B}_{M_n} \subseteq \mathcal{B}$ such that
    \begin{enumerate}[(i)]
        \item for each $1 \leq i \leq M_n$, the balls in $\mathcal{B}_i$ are disjoint, and
        \item we have \[A \subseteq \bigcup_{ i = 1 }^{ M_n } \bigcup_{ B \in \mathcal{B}_i } B. \]
    \end{enumerate}
\end{theorem}
The conclusion is saying that within each subcollection, the balls do not overlap, but different subcollections may have overlapping balls; moreover, the number of subcollections needed to cover the set of centers is bounded by a constant depending only on the dimension.
Our proof will give little indication on what the best possible value of $M_n$ is; we will only show that such a constant exists.

The proof is pretty damn long, and apparently at least one popular account has a major flaw in its proof --- beware of \emph{Geometric Integration Theory} by Krantz and Parks; they make a subtle claim (which they take as self-evident, and not even identified as something to be proved) that is actually false.
We follow the proof by Evans and Gariepy in \emph{Measure Theory and Fine Properties of Functions}.

\begin{proof}
    We begin by assuming that the set of centers $A$ is bounded; we will remove this assumption in the very last step of the proof.

\vspace{2mm}
\textit{Step 1:} First Greedy Algorithm.
\vspace{2mm}

    Start by selecting a ball $B_1$ from the collection $\mathcal{B}$ with radius satisfying
    \[ \operatorname{radius}(B_1) > \frac{3}{4} \sup \{ \operatorname{radius}B : B \in \mathcal{B} \}. \]
    Set $r_1 := \operatorname{radius}(B_1)$ and let $a_1$ be the center of $B_1$.
    Then let $A_2 := A \setminus B_1$, and if $A_2$ is nonempty, select a ball $B_2 \in \mathcal{B}$ centered at a point $a_2\in A_2$ with radius $r_2$ satisfying
    \[ r_2 > \frac{3}{4} \sup \{ r : \overline{B}(a,r) \in \mathcal{B}, a\in A_2 \}. \]
    Continuing in this manner, for each $j \in \Z^+$, we define
    \[ A_{j} := A \setminus \bigcup_{ k = 1 }^{ j - 1 } B_k, \]
    and if $A_j$ is nonempty, we select a ball $B_j\in \mathcal{B}$ centered at a point $a_j \in A_j$ with radius $r_j$ satisfying
    \[ r_j > \frac{3}{4} \sup \{ r : \overline{B}(a,r) \in \mathcal{B}, a\in A_j \} \]

    If $A_j \neq \emptyset$ for all $j \in \Z^+$, then we obtain an infinite sequence of balls $\{ B_j \}_{ j = 1 }^{ \infty }$, each selected from $\mathcal{B}$; in this case we set $J = \infty$. 
    If at some step $N \in \Z^+$ we have $A_N = \emptyset$, then the process terminates and we obtain a finite sequence of balls $\{ B_j \}_{ j = 1 }^{ N - 1 }$, and in this case we set $J = N - 1$.

\vspace{2mm}
\textit{Step 2:} Properties of the First Greedy Algorithm.
\vspace{2mm}

    \textit{Claim 1:} If $j > k$, then $ r_j \leq \frac{4}{3} r_k $.

    \begin{proof}[Proof of Claim 1]
        Assume $j > k$. Then $a_j \in A_k$, so we have
        \[ r_k \geq \frac{3}{4} \sup\{ r: \overline{B}(a,r) \in \mathcal{B}, a\in A_k \} \geq \frac{3}{4} r_j \]
        as claimed.
    \end{proof}

    \textit{Claim 2:} The collection of balls $\{ \frac{1}{3} B_j \}_{ j = 1 }^{ J }$ is disjoint.

    \begin{proof}[Proof of Claim 2]
        Let $1 \leq i < j \leq J$. Then $a_j \notin B_i$, so we have
        \[ \| a_i - a_j \| \geq r_i = \frac{r_i}{3} + \frac{2r_i}{3} \geq \frac{r_i}{3} + \frac{2}{3}\frac{3}{4}r_j = \frac{r_i}{3} + \frac{1}{2}r_j > \frac{r_i}{3} + \frac{r_j}{3} \]
        which shows that the distance between the centers of $B_i$ and $B_j$ is greater than the sum of the radii of the balls $\frac{1}{3} B_i$ and $\frac{1}{3} B_j$, and thus these two balls do not intersect.
    \end{proof}

    \textit{Claim 3:} If $J = \infty$, then $r_j \to 0$ as $j \to \infty$.

    \begin{proof}[Proof of Claim 3]
        By the previous claim, we know that the balls $\{ \frac{1}{3} B_j \}_{ j = 1 }^{ \infty }$ are disjoint.
        Since the set of centers $A$ is bounded, we must have $r_j \to 0$ as $j \to \infty$.
    \end{proof}

    \textit{Claim 4:} We have $A \subseteq \bigcup_{ j = 1 }^{ J } B_j$.

    \begin{proof}[Proof of Claim 4]
        If $J$ is finite, then the greedy algorithm in Step 1 terminated because $A_{J+1} = \emptyset$, which implies that
        \[ A \subseteq \bigcup_{ j = 1 }^{ J } B_j. \]
        Thus we assume that $J = \infty$. If $a \in A$ then by assumption there exists $r > 0$ and a closed ball $\overline{B}(a,r) \in \mathcal{B}$.
        By Claim 3, there exists $j \in \Z^+$ such that $r_j < \frac{3}{4}r$ --- in particular, this implies that $a \in \bigcup_{ i=1 }^{ j-1 } B_i$; otherwise, we would have selected a ball centered at $a$ with radius at least $r$ at step $j$ of the greedy algorithm, 
        contradicting the choice of $r_j$.
    \end{proof}

    Fix an integer $k > 1$ for the remainder of this step and the next step.
    Define
    \[ I_k := \{ j : 1\leq j < k, B_j \cap B_k \neq \emptyset \} \quad\text{ and }\quad K_k := I_k \cap \{ j : r_j \leq 3r_k \}. \]

    \textit{Claim 5:} We have $\#K_k \leq 20^n$.

    In words, we are claiming that there are at most $20^n$ balls from the first greedy algorithm that intersect $B_k$ and have radius at most $3r_k$.

    \begin{proof}[Proof of Claim 5]
        Let $j\in K_k$. Then $B_j \cap B_k \neq \emptyset$ and $r_j \leq 3r_k$.
        For each point $x\in \overline{B}(a_j, \frac{r_j}{3})$ see that 
        \begin{align*}
            \| x - a_k \| &\leq \| x - a_j \| + \| a_j - a_k \| \leq \frac{r_j}{3} + r_j + r_k \\
                &= \frac{4}{3} r_j + r_k \leq 4r_k  + r_k = 5r_k
        \end{align*}
        which shows that
        \[ B\left( a_j, \frac{r_j}{3} \right) \subseteq \overline{B}(a_k, 5r_k). \]
        By Claim 2, the balls $\{ \overline{B}(a_j, \frac{r_j}{3}) \}_{ j \in K }$ are disjoint, so we have
        \begin{align*}
            5^n \omega_n r_k^n &= \mathcal{L}^n\big( \overline{B}(a_k, 5r_k) \big) \\
                &\geq \sum_{ j \in K } \mathcal{L}^n\left( B\left( a_j, \frac{r_j}{3} \right) \right) &&\text{by disjointness} \\
                &= \sum_{ j \in K_k } \omega_n \left( \frac{r_j}{3} \right)^n \\
                &\geq \sum_{ j \in K_k } \omega_n \left( \frac{r_k}{4} \right)^n &&\text{by Claim 1 and } r_j \leq 3r_k \text{ for each }j\in K_k \\
                &= \#K_k \cdot \omega_n \left( \frac{r_k}{4} \right)^n.
        \end{align*}
        Dividing both sides by $\omega_n \left( \frac{r_k}{4} \right)^n$ gives
        \[ \#K_k \leq 5^n \cdot 4^n = 20^n \]
        as claimed.
    \end{proof}

\vspace{2mm}
\textit{Step 3:} We now bound the number of elements in $I_k\setminus K_k$.
\vspace{2mm}

    Let $i,j \in I_k\setminus K_k$ with $i \neq j$.
    Then $1\leq i,j < k$, both $B_i\cap B_k \neq \emptyset$ and $B_j \cap B_k \neq \emptyset$, and both $r_i > 3r_k$ and $r_j > 3r_k$.

    For distinct $i,j \in I_k\setminus K_k$, we let $\theta_{i,j} \in [0,\pi]$ be the angle between the vectors $a_i - a_k$ and $a_j - a_k$.
    Since we know $i,j < k$, we know that $a_k \notin B_i \cup B_j$ which implies $r_i < \| a_i - a_k \|$ and $r_j < \| a_j - a_k \|$.
    Since $B_i \cap B_k \neq \emptyset$ and $B_j \cap B_k \neq \emptyset$, we must have $\| a_i - a_k \| < r_i + r_k$ and $\| a_j - a_k \| < r_j + r_k$.
    Finally without loss of generality we may assume that $\| a_i - a_k \| \leq \| a_j - a_k \|$.
    In summary, 
    \[\begin{cases}
        3r_k < r_i < \| a_i - a_k \| < r_i + r_k, \\
        3r_k < r_j < \| a_j - a_k \| < r_j + r_k, \\
        \| a_i - a_k \| \leq \| a_j - a_k \|.
    \end{cases}\]

    \vspace{2mm}
    \textit{Claim 6:} For distinct $i,j \in I_k\setminus K_k$, if $\cos \theta_{i,j} > \frac{5}{6}$, then $a_i \in B_j$.
    \begin{proof}[Proof of Claim 6]
        We prove the contrapositive. Assume that $a_i \notin B_j$.
        Then we consider two possibilities --- either $\| a_i - a_j \| \geq \| a_j - a_k \|$ or $\| a_i - a_j \| < \| a_j - a_k \|$.
        If $\| a_i - a_j \| \geq \| a_j - a_k \|$, then the Law of Cosines gives
        \begin{align*}
            \cos \theta_{i,j} &= \frac{\| a_i - a_k \|^2 + \| a_j - a_k \|^2 - \| a_i - a_j \|^2}{2 \| a_i - a_k \| \| a_j - a_k \|} \\
                &\leq \frac{ \| a_i - a_k \|^2}{ 2 \| a_i - a_k \| \| a_j - a_k \| } \\
                &= \frac{\| a_i - a_k \|}{2 \| a_j - a_k \|} \leq \frac{1}{2} < \frac{5}{6}
        \end{align*}
        as desired.
        If instead $\| a_i - a_j \| < \| a_j - a_k \|$, then $r_j < \| a_i - a_j \|$ since $a_i \notin B_j$, and thus 
        \begin{align*}
            \cos \theta_{i,j} &= \frac{\| a_i - a_k \|^2 + \| a_j - a_k \|^2 - \| a_i - a_j \|^2}{2 \| a_i - a_k \| \| a_j - a_k \|} \\
                &= \frac{\| a_i - a_k \|^2}{2 \| a_i - a_k \| \| a_j - a_k \|} + \frac{( \|a_j - a_k\| - \|a_i - a_j\| )( \|a_j - a_k\| + \|a_i - a_j\|)}{2 \| a_i - a_k \| \| a_j - a_k \|} \\
                &= \frac{\| a_i - a_k \|}{2 \| a_j - a_k \|} + \frac{((\|a_j - a_k\| - \|a_i - a_j\|)(\|a_j - a_k\| + \|a_i - a_j\|))}{2 \| a_i - a_k \| \| a_j - a_k \|} \\
                &\leq \frac{ 1 }{ 2 } + \frac{ ( \|a_j - a_k\| - \|a_i - a_j\| )(2\|a_j - a_k\|) }{ 2\| a_i - a_k \| \| a_j - a_k \| } \\
                &= \frac{1}{2} + \frac{ r_j + r_k - r_j }{ r_i } = \frac{1}{2} + \frac{ r_k }{ r_i } < \frac{1}{2} + \frac{1}{3} = \frac{5}{6}
        \end{align*}

    \end{proof}

    \textit{Claim 7:} If $a_i \in B_j$, then \[ 0 \leq \| a_i - a_j \| + \| a_i - a_k \| - \| a_j - a_k \| \leq \| a_j - a_k \| \cdot \frac{8}{3}\left( 1 - \cos\theta_{i,j} \right) \]
    \begin{proof}[Proof of Claim 7]
        Assume that $a_i \in B_j$.
        Then we have
        \begin{align*}
            0 &\leq \frac{\|a_i - a_j\| + \|a_i - a_k\| - \|a_j - a_k\|}{\|a_j - a_k\|} \\
                &\leq \frac{\|a_i - a_j\| + \|a_i - a_k\| - \|a_j - a_k\|}{\|a_j - a_k\|} \cdot \frac{\|a_i - a_j\| - \|a_i - a_k\| + \|a_j - a_k\|}{\|a_i - a_j\|} \\
                &= \frac{\|a_i - a_j\|^2 - (\|a_i - a_k\| - \|a_j - a_k\|)^2}{\|a_j - a_k\| \|a_i - a_j\|} \\
                &= \frac{\|a_i - a_k\|^2 - 2\|a_i - a_k\| \|a_j - a_k\|\cos \theta_{i,j} + \|a_j - a_k\|^2 - \|a_i -a_k\|^2 + 2\|a_i - a_k\|\|a_j - a_k\| - \|a_j - a_k\|^2}{ \|a_j - a_k\| \|a_i - a_j\| } \\
                &= \frac{2\|a_i - a_k\| \|a_j - a_k\| (1 - \cos\theta_{i,j})}{\|a_j - a_k\| \|a_i - a_j\|} \\
                &= \frac{2\|a_i - a_k\| (1 - \cos\theta_{i,j})}{\|a_i - a_j\|} \\
                &\leq \frac{2(r_i + r_k)(1 - \cos\theta_{i,j})}{r_i} \\
                &\leq \frac{2\left( 1 + \frac{1}{3} \right) r_i (1 - \cos\theta_{i,j})}{r_i} \\
                &= \frac{8}{3} (1 - \cos\theta_{i,j}).
            \end{align*}
        Hence 
        \[ 0 \leq \| a_i - a_j \| + \| a_i - a_k \| - \| a_j - a_k \| \leq \| a_j - a_k \| \cdot \frac{8}{3}\left( 1 - \cos\theta_{i,j} \right) \]
        as desired.
    \end{proof}

    \textit{Claim 8:} If $a_i \in B_j$, then $\cos \theta_{i,j} \leq \frac{61}{64}$.
    \begin{proof}[Proof of Claim 8]
        Since $a_i \in B_j$ and $a_j \notin B_i$, we have $r_i < \| a_i - a_j \| < r_j$.
        Because $i < j$, we have $r_j \leq \frac{4}{3} r_i$ by Claim 1. Therefore
        \begin{align*}
            \| a_i - a_j \| + \| a_i - a_k \| - \| a_j - a_k \| &\geq r_i + r_i - r_j - r_k \\
                &\geq \frac{3}{4} r_j + \frac{3}{4} r_j - r_j - \frac{1}{3} r_j \\
                &= \frac{1}{2} r_j - r_k \geq \frac{1}{6} r_j \\
                &= \frac{1}{6} \frac{3}{4} \left( r_j + \frac{1}{3} r_j \right) \\
                &\geq \frac{1}{8} (r_j + r_k) \\
                &\geq \frac{1}{8} \| a_j - a_k \|
        \end{align*}
        so by Claim 7 we have
        \[ \frac{1}{8} \| a_j - a_k \| \leq \| a_i - a_j \| + \| a_i - a_k \| - \| a_j - a_k \| \leq \| a_j - a_k \| \cdot \frac{8}{3}\left( 1 - \cos\theta_{i,j} \right). \]
        Hence \[ \frac{3}{64} \leq 1 - \cos\theta_{i,j} \]
        which gives the desired result.
    \end{proof}

    \textit{Claim 9:} There exists a constant $L_n$ depending only on $n$ such that $\#(I_k\setminus K_k) \leq L_n$.
    \begin{proof}[Proof of Claim 9]
        See that for distinct $i,j \in I_k\setminus K_k$, then either $a_i \notin B_j$ or $a_i \in B_j$, so we have either
        $\cos \theta_{i,j} \leq \frac{5}{6}$ or $\cos \theta_{i,j} \leq \frac{61}{64}$ by Claims 6 and 8.
        In either case, we have $\cos \theta_{i,j} \leq \frac{61}{64}$.
        Thus the angle $\theta_{i,j}$ between the vectors $a_i - a_k$ and $a_j - a_k$ satisfies
        \[ \theta_{i,j} \geq \arccos\left( \frac{61}{64} \right) > 0. \]

        Now fix a radius $r_0 > 0$ so small that if $x \in \S^{n-1}$ and $y,z\in B(x,r_0)$ then the angle between $y$ and $z$ is less than $\arccos\left( \frac{61}{64} \right)$.
        Since $\S^{n-1}$ is compact, we can cover it with finitely many balls of radius $r_0$ with centers in $\S^{n-1}$; by the well-ordering principle, there exists a smallest number $L_n$ of such balls needed to cover $\S^{n-1}$.
        Then $\partial B_k$ can be covered by $L_n$ balls of radius $r_0 r_k$ with centers on $\partial B_k$.
        By the previous paragraph, if $i,j \in I_k\setminus K_k$ with $i \neq j$, then the angle between $a_i - a_k$ and $a_j - a_k$ is at least $\arccos\left( \frac{61}{64} \right)$, so the
        points $\frac{a_i - a_k}{\| a_i - a_k \|}$ and $\frac{a_j - a_k}{\| a_j - a_k \|}$ lie in different balls of radius $r_0$ on $\S^{n-1}$.
        Therefore, each ball of radius $r_0 r_k$ on $\partial B_k$ contains at most one of the points $a_i$ for $i \in I_k\setminus K_k$.
        Hence we have $\#(I_k\setminus K_k) \leq L_n$ as desired.
    \end{proof}

\vspace{2mm}
\textit{Step 4:} We now finish the proof in the case that $A$ is bounded.
\vspace{2mm}

    We set $C_n := 20^n + L_n + 1$, and see that by claims 5 and 9, we have
    \[ \#I_k = \#K_k + \#(I_k\setminus K_k) \leq 20^n + L_n < C_n \]
    for each integer $k > 1$.
    We define a function $\sigma : \Z^+ \to \{ 1, 2, \ldots, C_n \}$ as follows ---
    \begin{itemize}
        \item if $1 \leq j \leq C_n$, set $\sigma(j) := j$;
        \item if $k \geq C_n$, we can inductively define $\sigma(k+1)$ by noting that the previous estimate shows that
            \[ \#\{ j : 1\leq j\leq k , B_j \cap B_{k+1} \neq \emptyset \} = \# I_k < C_n \]
            so there exists $m\in \{1,2,\ldots,C_n\}$ such that $B_{k+1}\cap  B_j = \emptyset$ for all $j \in \{1,2,\ldots,k\}$ with $\sigma(j) = m$; we set $\sigma(k+1) := m$.
    \end{itemize}

    Now for each integer $1 \leq m \leq C_n$, define
    \[ \mathcal{B}_m := \{ B_j : j \in \Z^+, \sigma(j) = m \}. \]
    By definition of the function $\sigma$, the balls in each collection $\mathcal{B}_m$ are disjoint --- indeed, if $B_i, B_j \in \mathcal{B}_m$ with $i < j$ and $\sigma(i) = \sigma(j) = m$, then 
    we must have $B_i \cap B_j = \emptyset$ by the definition of $\sigma(j)$.
    Finally, by Claim 4, we have
    \[ A \subseteq \bigcup_{ j = 1 }^{ J } B_j = \bigcup_{ m = 1 }^{ C_n } \bigcup_{ B \in \mathcal{B}_m } B \]
    which completes the proof in the case that $A$ is bounded.
    
\vspace{2mm}
\textit{Step 5:} Finally we assume that $A$ is unbounded.
\vspace{2mm}

Assume that $A$ is unbounded, for each integer $l \in \Z^+$ define
\[ A_l := A \cap \left\{ x \in \R^n : 3(l-1) < \frac{\| x \|}{\sup_{B\in \mathcal B} \diam B} < 3l \right\} \]
and \[ \mathcal{B}^l := \{ \overline{B}(a,r) \in \mathcal{B} : a\in A_l \}. \]
Then for each $l \in \Z^+$, the set $A_l$ is bounded, so we may apply the reasoning from Steps 1 to 4 to the set $A_l$ and the collection $\mathcal{B}^l$ to obtain subcollections $\mathcal{B}^l_1, \mathcal{B}^l_2, \ldots, \mathcal{B}^l_{M_n} \subseteq \mathcal{B}^l$ such that
\begin{enumerate}[(i)]
    \item for each $1 \leq m \leq C_n$, the balls in $\mathcal{B}^l_m$ are disjoint, and
    \item we have \[ A_l \subseteq \bigcup_{ m = 1 }^{ C_n } \bigcup_{ B \in \mathcal{B}^l_m } B. \]
\end{enumerate}
For each integer $1 \leq m \leq C_n$, define
\[ \mathcal{B}_j := \bigcup_{l = 1}^\infty \mathcal{B}^{2l-1}_j \]
and \[ \mathcal{B}_{j+C_n} := \bigcup_{l = 1}^\infty \mathcal{B}^{2l}_j. \]
Then for each $1 \leq j \leq 2C_n$, the balls in $\mathcal{B}_j$ are disjoint --- if $B, B' \in \mathcal{B}_j$ with $B \neq B'$, then there exists $l,l' \in \Z^+$ such that $B \in \mathcal{B}^l_j$ and $B' \in \mathcal{B}^{l'}_j$; 
then if $l = l'$, we have $B \cap B' = \emptyset$ since the balls in $\mathcal{B}^l_j$ are disjoint, and if $l \neq l'$, then by construction of the sets $A_l$ and the collections $\mathcal{B}^l_j$, we have $B \cap B' = \emptyset$ as well.

Finally, by construction of the sets $A_l$ for $l \in \Z^+$, we have
\begin{align*}
    A &= \bigcup_{ l = 1 }^{ \infty } A_l \\
        &\subseteq \bigcup_{ l = 1 }^{ \infty } \bigcup_{ m = 1 }^{ C_n } \bigcup_{ B \in \mathcal{B}^l_m } B \\
        &= \bigcup_{ m = 1 }^{ C_n } \bigcup_{ l = 1 }^{ \infty } \bigcup_{ B \in \mathcal{B}^l_m } B \\
        &= \bigcup_{ m = 1 }^{ C_n } \bigcup_{ B \in \mathcal{B}_m \cup \mathcal{B}_{m+C_n} } B \\
        &= \bigcup_{ j = 1 }^{ 2C_n } \bigcup_{ B \in \mathcal{B}_j } B
\end{align*}
The proof is now complete upon setting $M_n := 2C_n$.
\end{proof}

\subsection{Applications}

Next we would like to mimic some results that we had for the Lebesgue measure on $\R^n$, but now for Borel regular outer measures on $\R^n$ which are finite on compact subsets.

\begin{lemma}[Filling Finite Measure Sets with Balls]
    \label{lem:filling_finite_measure_sets_with_balls}
    Let $\mu$ be a Borel regular outer measure on $\R^n$ which is finite on compact subsets, and let $A\subset \R^n$ be such that $\mu(A) < \infty$.
    Let $\mathcal{B}$ be a collection of closed balls such that the center of each ball in $\mathcal{B}$ belongs to $A$, and 
    \[ \inf\{ r : \overline{B}(a,r) \in \mathcal{B}, a\in A \} = 0. \]
    Then there is a countable subcollection of disjoint balls $\{ B_j \}_{ j = 1 }^{ \infty } \subseteq \mathcal{B}$ such that
    \[ \mu\left( A \setminus \bigcup_{ j = 1 }^{ \infty } B_j \right) = 0. \]
\end{lemma}
Note that the set $A$ in the statement of the lemma is not assumed to be $\mu$-measurable.

\begin{proof}
    \textit{Step 1:}
    Consider the family of closed balls 
    \[ \mathcal{B}^{1} := \{ B\in \mathcal{B} : \diam B \leq 1 \} \]
    which covers $A$ because if $a\in A$, then there exists a ball $\overline{B}(a,r) \in \mathcal{B}$ with radius $r \leq 1$ by the assumption on $\mathcal{B}$.
    By the Besicovich Covering Theorem (Theorem \ref{thm:besicovich_covering_theorem}), there exists subcollections $\mathcal{B}_1, \mathcal{B}_2, \ldots, \mathcal{B}_{M_n} \subseteq \mathcal{B}^{1}$ such that
    each collection only contains disjoint balls and
    \[ A \subseteq \bigcup_{ j = 1 }^{ M_n } \bigcup_{ B \in \mathcal{B}_j } B. \]
    Thus
    \[ \mu(A) \leq \sum_{ j = 1 }^{M_n} \mu\left( A \cap \bigcup_{ B \in \mathcal{B}_j } B \right) \]
    so there is some $1 \leq j \leq M_n$ such that
    \[ \mu\left( A \cap \bigcup_{ B \in \mathcal{B}_j } B \right) \geq \frac{1}{M_n} \mu(A). \]
    Therefore by \ref{prop:sequences_of_measurable_sets}, there exist finitely many disjoint balls $B_1, B_2, \ldots, B_{N_1} \in \mathcal{B}_j$ such that
    \[ \mu\left( A \cap \bigcup_{k=1}^{N_1} B_k \right) \geq \frac{1}{2M_n}\mu(A). \]
    Since $\bigcup_{k=1}^{N_1} B_k$ is a measurable set, we have
    \[ \mu(A) = \mu\left( A \cap \bigcup_{k=1}^{N_1} B_k \right) + \mu\left( A \setminus \bigcup_{k=1}^{N_1} B_k \right). \]
    which implies that
    \begin{align*}
        \mu\left( A \setminus \bigcup_{k=1}^{N_1} B_k \right) &= \mu(A) - \mu\left( A \cap \bigcup_{k=1}^{N_1} B_k \right) \\
            &\leq \mu(A) - \frac{1}{2M_n} \mu(A) = \left( 1 - \frac{1}{2M_n} \right) \mu(A).
    \end{align*}

    \vspace{2mm}

    \textit{Step 2:}
    Now the collection of closed balls
    \[ \mathcal{B}^2 := \left\{ B\in \mathcal{B}^1 : B \cap \left(\bigcup_{k=1}^{N_1} B_k\right) = \emptyset \right\} \]
    clearly covers the set $A \setminus \bigcup_{k=1}^{N_1} B_k$, and we also have
    \[ \inf\left\{ r : \overline{B}(a,r) \in \mathcal{B}^2, a\in A \setminus \bigcup_{k=1}^{N_1} B_k \right\} = 0 \]
    for each $a \in A \setminus \bigcup_{k=1}^{N_1} B_k$ by the assumption on $\mathcal{B}$.
    Repeating the above argument with the set $A \setminus \bigcup_{k=1}^{N_1} B_k$ and the collection $\mathcal{B}^2$, we obtain finitely many disjoint balls $B_{N_1+1}, B_{N_1+2}, \ldots, B_{N_2} \in \mathcal{B}^2$ such that
    \[ \mu\left( A \setminus \bigcup_{k=1}^{N_2} B_k \right) \leq \left( 1 - \frac{1}{2M_n} \right) \mu\left( A \setminus \bigcup_{k=1}^{N_1} B_k \right) \leq \left( 1 - \frac{1}{2M_n} \right)^2 \mu(A). \]
    Continuing in this manner, for each $k > 1$ we let $\mathcal{B}^k$ be the collection of closed balls in $\mathcal{B}^1$ which are disjoint from $\bigcup_{j=1}^{N_{k-1}} B_j$, and we obtain finitely many disjoint balls $B_{N_{k-1}+1}, B_{N_{k-1}+2}, \ldots, B_{N_k} \in \mathcal{B}^k$ such that
    \[ \mu\left( A \setminus \bigcup_{j=1}^{N_k} B_j \right) \leq \left( 1 - \frac{1}{2M_n} \right)^k \mu(A). \]
    Finally, taking the limit as $k \to \infty$, we have
    \[ \mu\left( A \setminus \bigcup_{ j = 1 }^{ \infty } B_j \right) = \lim_{ k \to \infty } \mu\left( A \setminus \bigcup_{j=1}^{N_k} B_j \right) \leq \lim_{ k \to \infty } \left( 1 - \frac{1}{2M_n} \right)^k \mu(A) = 0 \]
    as desired.
\end{proof}

\begin{corollary}[More on Filling Open Sets with Balls]
    \label{cor:more_on_filling_open_sets_with_balls}
    Let $\mu$ be a Borel regular outer measure on $\R^n$ which is finite on compact subsets, and let $\mathcal{B}$ be a nonempty collection of closed balls.
    Let $A$ be the set of centers of balls in $\mathcal{B}$, and assume that $\mu(A) < \infty$ and that
    \[ \inf\{ r : \overline{B}(a,r) \in \mathcal{B}, a\in A \} = 0 \]
    for each $a \in A$.
    
    Then for each open set $U\subseteq \R^n$ there exists a countable subcollection of disjoint balls $\{ B_j \}_{ j = 1 }^{ \infty } \subseteq \mathcal{B}$ such that
    \[ \bigcup_{j=1}^\infty B_j \subset U \]
    and \[ \mu\left( (A\cap U) \setminus \bigcup_{j=1}^\infty B_j \right) = 0. \]    
\end{corollary}
\begin{proof}
    Define the collection of closed balls
    \[ \mathcal{B}_U := \{ B\in \mathcal{B} : B \subset U \}. \]
    Then $\mathcal{B}_U$ covers the set $A\cap U$, and for each $a \in A\cap U$, we have
    \[ \inf\{ r : \overline{B}(a,r) \in \mathcal{B}_U, a\in A\cap U \} = 0 \]
    by the assumption on $\mathcal{B}$.
    Since $\mu(A\cap U) < \infty$, we may apply Lemma \ref{lem:filling_finite_measure_sets_with_balls} to the set $A\cap U$ and the collection $\mathcal{B}_U$ to obtain a countable subcollection of disjoint balls $\{ B_j \}_{ j = 1 }^{ \infty } \subseteq \mathcal{B}_U$ such that
    \[ \mu\left( (A\cap U) \setminus \bigcup_{ j = 1 }^{ \infty } B_j \right) = 0 \]
    as desired.
\end{proof}

\begin{definition}[Maximal Function for Borel Regular Measures]
    \label{def:maximal_function_besicovich_version}
    Let $\mu$ be a Borel regular outer measure on $\R^n$ which is finite on compact subsets.
    For each locally integrable function $f \in L^1_{\text{loc}}(\R^n,\mu)$, we define the \textit{maximal function} $M_\mu f : \R^n \to [0,\infty]$ by
    \[ M_\mu f(x) := \sup_{ r > 0 } \frac{1}{ \mu(\overline{B}(x,r)) } \int_{ \overline{B}(x,r) } |f| \,\dif \mu \]
    for each $x \in \R^n$, where we interpret the expression $\frac{1}{0}$ as $\infty$ and use the convention that $\infty\cdot 0 = 0$.
\end{definition}

\begin{remark}[About the Definition of the Maximal Function]
    \label{rmk:locally_integrable_function_besicovich_version}
    Recall that $f$ is said to be locally integrable with respect to $\mu$ if for each compact set $K \subseteq \R^n$, we have
    \[ \int_K |f(y)| \,\dif \mu(y) < \infty. \]
    We require that $\mu$ is finite on compact subsets to ensure that the fraction $\frac{1}{\mu(B)}$ is not zero. 
    The locally integrability of $f$ with respect to $\mu$ ensures that the integral $\int_B |f(y)| \,\dif \mu(y)$ is finite for each closed ball $B \subseteq \R^n$, so that the expression defining $M_\mu f(x)$ is well-defined (possibly infinite) for each $x \in \R^n$.
\end{remark}

Here we mimic the definition of the centered Hardy-Littlewood maximal function from before, but now using a general Borel regular outer measure $\mu$ on $\R^n$ which is finite on compact sets.

\begin{exercise}[Weak $(1,1)$ Inequality for the Maximal Function]
    \label{ex:weak_11_inequality_for_maximal_function_besicovich_version}
    Let $\mu$ be a Borel regular outer measure on $\R^n$ which is finite on compact subsets. Then there exists a constant $C_n > 0$ depending only on $n$ such that for each $f \in L^1_{\text{loc}}(\R^n,\mu)$ and each $t > 0$, we have
    \[ \mu\left( \{ x \in \R^n : M_\mu f(x) > t \} \right) \leq \frac{C_n}{t} \int_{ \R^n } |f| \,\dif \mu. \]
\end{exercise}

I am not $100 \%$ sure if this statement and the following proof is correct, I am maybe $50 \%$ sure.
I cannot find another reference --- the Wikipedia page on Besicovich's Covering Theorem is where I found the statement of the result, but their proof begins with the incorrect statement that $Mf$ is lower semicontinuous.
(To see that $Mf$ is not necessarily lower semicontinuous, consider the measure $\mu = \delta_0$ (the Dirac delta at $0$) on $\R$, and the function $f = \chi_{\{0\}}$; then $Mf(0) = 1$, but for each $x \neq 0$, we have $Mf(x) = 0$, so $Mf$ is not lower semicontinuous at $0$.)
Because the first statement is false (and not even acknowledged as something to be proved), I find it hard to trust the rest of their proof; their proof also uses a different version of Besicovich than the one I have stated.

I also had several false starts by trying to invoke Besicovich at the wrong time in this proof.
We will not use this result later, so it is not a big deal if the proof is wrong, but I would appreciate any feedback on this proof.

\begin{proof}
    Let $f \in L^1_{\text{loc}}(\R^n,\mu)$. If $\int_{ \R^n } |f| \,\dif \mu = \infty$, then the inequality is trivial since the right-hand side is infinite.
    Thus we assume that \[  \int_{ \R^n } |f| \,\dif \mu < \infty \]
    i.e. that $\| f \|_{ L^1(\R^n,\mu) } < \infty$.

    For each $t > 0$, define the set
    \[ E_t := \{ x \in \R^n : M_\mu f(x) > t \}. \]
    Let $K \subseteq E_t$ be a compact set.
    Then for each $x \in K$, by definition of $E_t$ there exists a radius $r_x > 0$ such that
    \[ \frac{1}{ \mu(\overline{B}(x,r_x)) } \int_{ \overline{B}(x,r_x) } |f| \,\dif \mu > t. \]
   
    [ It is tempting to say two false statments here, based on the following observation --- for each $x \in K$, by definition of $E_t$ there exists a radius $r_x > 0$ such that
    \[ \frac{1}{ \mu(\overline{B}(x,r_x)) } \int_{ \overline{B}(x,r_x) } |f| \,\dif \mu > t. \]

    The \textit{first mistake} is to state that compactness of $K$ implies that $\displaystyle \sup_{x\in K} r_x < \infty$. This is false; take a moment to think of an example.

    The \textit{second mistake} is more subtle, and was my attempt at rectifying the first mistake --- I was tempted to get an open cover $\{ B(x,r_x) : x \in K \}$ of $K$ and then use compactness to extract a finite subcover to get a finite collection of balls covering $K$ ; this finite subcover then \emph{does} have uniformly bounded diameters, since there are only finitely many balls.
    However, the center of these balls is no longer guaranteed to be all of $K$, so we cannot apply Besicovich to this finite collection of balls with $K$ as the set of centers. 
    This point is subtle, but important. ]

    For each integer $N\in\Z^+$, define
    \[ K_N := \left\{ x\in K : \exists 0 < r \leq N \,\text{ such that }\, \frac{1}{\mu(\overline{B}(x,r))} \int_{ \overline{B}(x,r) } |f| \,\dif \mu > t \right\} \]
    so that $K = \bigcup_{ N = 1 }^{ \infty } K_N$. 
    Fix $N \in \Z^+$.
    Then for each $x \in K_N$, there exists a radius $0 < r_x \leq N$ such that 
    \[  \frac{1}{ \mu(\overline{B}(x,r_x)) } \int_{ \overline{B}(x,r_x) } |f| \,\dif \mu > t. \]
    Thus the collection of closed balls
    \[ \mathcal{B}_N := \{ \overline{B}(x,r_x) : x \in K_N \} \]
    has $K_N$ as the set of centers, and the set of diameters of balls in $\mathcal{B}_N$ is bounded above by $2N$.
    Therefore we may apply the Besicovich Covering Theorem (Theorem \ref{thm:besicovich_covering_theorem}) to the set $K_N$ and the collection $\mathcal{B}_N$ to see that there exists disjoint subcollections $\mathcal{B}_{N,1}, \mathcal{B}_{N,2}, \ldots, \mathcal{B}_{N,C_n} \subseteq \mathcal{B}_N$ such that
    \[ K_N \subseteq \bigcup_{ j = 1 }^{ C_n } \bigcup_{ B \in \mathcal{B}_{N,j} } B \]
    where for each $1 \leq j \leq C_n$, the balls in $\mathcal{B}_{N,j}$ are disjoint.
    We remark that the constant $C_n$ depends only on $n$ and not on $N$, $K$, $f$ or anything else.
    Therefore, we have
    \begin{align*}
        \mu(K_N) &\leq \mu\left( \bigcup_{ j = 1 }^{ C_n } \bigcup_{ B \in \mathcal{B}_{N,j} } B \right) \\
            &\leq \sum_{ j = 1 }^{ C_n } \mu\left( \bigcup_{ B \in \mathcal{B}_{N,j} } B \right) \\
            &= \sum_{ j = 1 }^{ C_n } \sum_{ B \in \mathcal{B}_{N,j} } \mu(B) &&\text{by disjointness} \\
            &< \sum_{ j = 1 }^{ C_n } \sum_{ B \in \mathcal{B}_{N,j} } \frac{1}{t} \int_B |f| \,\dif \mu &&\text{by choice of balls in } \mathcal{B}_N \\
            &= \frac{1}{t} \sum_{ j = 1 }^{ C_n } \sum_{ B \in \mathcal{B}_{N,j} } \int_B |f| \,\dif \mu \\
            &\leq \frac{1}{t} \sum_{ j = 1 }^{ C_n } \int_{ \R^n } |f| \,\dif \mu && \text{by disjointness of balls in each } \mathcal{B}_{N,j} \\
            &= \frac{C_n}{t} \int_{ \R^n } |f| \,\dif \mu.
    \end{align*}
    Since $K = \bigcup_{ N = 1 }^{ \infty } K_N$ and this is an increasing union, by \ref{prop:sequences_of_measurable_sets} we have
    \[ \mu(K) = \lim_{N \to\infty} \mu(K_N) \leq \frac{C_n}{t} \int_{ \R^n } |f| \,\dif \mu. \]
    Since $K \subseteq E_t$ was an arbitrary compact subset, Borel regularity of $\mu$ implies that
    \[ \mu(E_t) = \sup\{ \mu(K) : K \subseteq E_t , K \text{ compact} \} \leq \frac{C_n}{t} \int_{ \R^n } |f| \,\dif \mu \]
    as desired.
\end{proof}