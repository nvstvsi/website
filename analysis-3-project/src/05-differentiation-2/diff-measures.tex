\section{Densities and the Radon-Nikodym Theorem}

In this section, we will finish the goal stated in the previous section of expressing the Radon-Nikodym derivative as a limit of ratios of measures.

\subsection{Lebesgue-Besicovich Differentiation Theorem}

First we want to show that densities can be used to prove a generalization of the Lebesgue Differentiation Theorem \ref{thm:lebesgue_differentiation_theorem}.

\begin{theorem}[Lebesgue-Besicovich Differentiation Theorem]
    \label{thm:lebesgue_besicovich_differentiation_theorem}
    Let $\mu$ be a Borel regular outer measure on $X$ which is open $\sigma$-finite and has the symmetric Vitali property.
    Let $f: X \to \R$ be a locally integrable function with respect to $\mu$, i.e. for each $x \in X$ there exists $r > 0$ such that
    \[ \int_{\overline{B}(x,r)} |f| \,\dif \mu < \infty. \]
    Then for $\mu$-almost every $x \in X$, we have
    \[ f(x) = \lim_{r \to 0^+} \frac{1}{\mu\big( \overline{B}(x,r) \big)} \int_{\overline{B}(x,r)} f \,\dif \mu. \]
\end{theorem}

\begin{proof}
    Assume first that $f \geq 0$.
    We define $\nu : 2^X \to [0,\infty]$ by
    \[ \nu(A) := \int_A f \,\dif \mu \]
    for each Borel set $A \subseteq X$, and 
    \[ \nu(E) := \inf\{ \nu(A) : A\subseteq X \text{ is a Borel set with } E\subseteq A \}. \]
    We claim that $\nu$ is a well-defined Borel regular outer measure on $X$.

    \begin{proof}[Proof of Claim]  
        See that $\nu$ is well-defined, since if $A\subseteq X$ is a Borel set, then clearly
        \[ \inf\left\{ \nu(\hat{A}) : \hat{A} \subseteq X \text{ is a Borel set with } A \subseteq \hat{A} \right\} \leq \nu(A) \]
        since $A$ is one such Borel set containing itself; also for each Borel set $\hat{A} \subseteq X$ with $A \subseteq \hat{A}$, we have
        \[ \nu(\hat{A}) = \int_{\hat{A}} f \,\dif \mu \geq \int_A f \,\dif \mu = \nu(A) \]
        by monotonicity of the Lebesgue integral (Proposition \ref{prop:properties_of_the_lebesgue_integral} (iii)).
        Therefore if $A \subseteq X$ is a Borel set, then
        \[ \nu(A) = \inf\{ \nu(\hat{A}) : \hat{A}\subseteq X \text{ is a Borel set with } A \subseteq \hat{A} \} \]
        which shows that $\nu$ is well-defined.

        To see that $\nu$ is a Borel regular outer measure, first note that $\nu(\emptyset) = 0$ and if $\{ A_j \}_{j=1}^\infty$ is a countable collection of subsets of $X$, then
        \[  \nu\left(\bigcup_{j=1}^\infty A_j\right) = \int_{\bigcup_{j=1}^\infty A_j} f \,\dif \mu \leq \sum_{j=1}^\infty \int_{A_j} f \,\dif \mu = \sum_{j=1}^\infty \nu(A_j). \]
        Therefore $\nu$ is an outer measure on $X$.

        Now let $A \subseteq X$ be a Borel set. For an arbitrary subset $S \subseteq X$, we let $C\subset X$ be a Borel set such that $S \subseteq C$ and $\nu(S) = \nu(C)$.
        Then
        \[ \nu(S) = \nu(C) = \int_C f \,\dif \mu = \int_{C\cap A} f \,\dif \mu + \int_{C\cap A^c} f \,\dif \mu = \nu(C\cap A) + \nu(C\cap A^c) \geq \nu(S\cap A) + \nu(S\cap A^c) \]
        which proves that $A$ is $\nu$-measurable as $S$ was an arbitrary subset of $X$.
        Thus $\nu$ is a Borel outer measure on $X$.
        The fact that $\nu$ is Borel regular follows from the definition of $\nu$ as the infimum over Borel sets.
        Therefore $\nu$ is a Borel regular outer measure on $X$.
    \end{proof}

    For each $j,k \in \Z^+$, we define the sets
    \[ A_{j,k} := f^{-1}\left( \left[ \frac{k-1}{j}, \frac{k}{j} \right] \right) \]
    and observe that for each $j,k \in \Z^+$, the set $A_{j,k}$ is $\mu$-measurable since $f$ is measurable; hence, by the Density Theorem \ref{thm:density_theorem_general}, for $\mu$-almost every $x \in A_{j,k}$ we have
    \[ \lim_{r\to 0^+} \frac{\mu( \overline{B}(x,r) \cap A_{j,k} )}{\mu( \overline{B}(x,r) )} = 1 \tag{$\diamondsuit$}\]
    By the Upper Density Theorem \ref{thm:upper_density_theorem}, for $\mu$-almost every $x \in A_{j,k}$ we have
    \[ \lim_{r\to 0+} \frac{\nu( \overline{B}(x,r) \setminus A_{j,k} )}{\mu( \overline{B}(x,r) )} = 0. \tag{$\clubsuit$}\]
    
    \vspace{2mm}

    Fix $j,k \in \Z^+$. Let $x \in A_{j,k}$ be a point where both $(\diamondsuit)$ and $(\clubsuit)$ hold, and let $r > 0$ be small enough so that $\int_{\overline{B}(x,r)} |f| \,\dif \mu < \infty$.
    Then we have the chain of inequalities
    \begin{align*}
        \mu( \overline{B}(x,r) \cap A_{j,k} ) \cdot\frac{k-1}{j} &\leq \int_{ \overline{B}(x,r) \cap A_{j,k} } f \,\dif \mu = \nu( \overline{B}(x,r) \cap A_{j,k} ) \\
            &= \int_{ \overline{B}(x,r) } f\,\dif\mu - \nu( \overline{B}(x,r) \setminus A_{j,k} ) \\
            &\leq \mu(\overline{B}(x,r)\cap A_{j,k}) \cdot \frac{k}{j}. \qquad (\heartsuit)
    \end{align*}
    Extracting the first and thirs terms of this chain of inequalities, we have
    \begin{align*}
        \frac{k-1}{j} &\leq \frac{1}{\mu( \overline{B}(x,r) \cap A_{j,k})} \left( \int_{\overline{B}(x,r)} f\,\dif\mu - \nu( \overline{B}(x,r)\setminus A_{j,k} ) \right) \\
            &= \frac{ \mu( \overline{B}(x,r) )}{ \mu( \overline{B}(x,r) \cap A_{j,k} )} \cdot \left( \frac{1}{ \mu( \overline{B}(x,r) )} \int_{ \overline{B}(x,r) } f \,\dif \mu - \frac{ \nu( \overline{B}(x,r) \setminus A_{j,k}) }{ \mu( \overline{B}(x,r) )} \right). \qquad (\heartsuit_1)
    \end{align*}
    by monotonicity of $\mu$.
    Taking limits as $r \to 0^+$ in $(\heartsuit_1)$ and using $(\diamondsuit)$ and $(\clubsuit)$, we see that
    \[ \frac{k-1}{j} \leq \liminf_{r\to 0^+} \frac{1}{\mu( \overline{B}(x,r) )} \int_{ \overline{B}(x,r) } f \,\dif \mu. \tag{$\spadesuit_1$}\]
    
    Similarly we take the third and fourth terms of the chain of inequalities $(\heartsuit)$ to see that
    \begin{align*}
        \frac{k}{j} &\geq \frac{1}{\mu( \overline{B}(x,r) \cap A_{j,k})} \left( \int_{ \overline{B}(x,r) } f \,\dif \mu - \nu( \overline{B}(x,r) \setminus A_{j,k} ) \right) \\
            &= \frac{ \mu( \overline{B}(x,r) ) }{ \mu( \overline{B}(x,r) \cap A_{j,k} ) } \cdot \left( \frac{1}{ \mu( \overline{B}(x,r) ) } \int_{ \overline{B}(x,r) } f \,\dif \mu - \frac{ \nu( \overline{B}(x,r) \setminus A_{j,k} ) }{ \mu( \overline{B}(x,r) ) } \right). \qquad (\heartsuit_2)
    \end{align*}
    Taking limits as $r \to 0^+$ in $(\heartsuit_2)$ and using $(\diamondsuit)$ and $(\clubsuit)$, we see that
    \[ \frac{k}{j} \geq \limsup_{r\to 0^+} \frac{1}{\mu( \overline{B}(x,r) )} \int_{ \overline{B}(x,r) } f \,\dif \mu. \tag{$\spadesuit_2$}\]
    Combining $(\spadesuit_1)$ and $(\spadesuit_2)$, we have
    \[ \frac{k-1}{j} \leq \liminf_{r\to 0^+} \frac{1}{\mu( \overline{B}(x,r) )} \int_{ \overline{B}(x,r) } f \,\dif \mu \leq \limsup_{r\to 0^+} \frac{1}{\mu( \overline{B}(x,r) )} \int_{ \overline{B}(x,r) } f \,\dif \mu \leq \frac{k}{j}. \]
    Since $\frac{k-1}{j} \leq f \leq \frac{k}{j}$ on $A_{j,k}$, we finally estimate
    \begin{align*}
        f(x) - \frac{1}{j} &\leq \frac{k-1}{j} \\
            &\leq \liminf_{r\to 0^+} \frac{1}{\mu( \overline{B}(x,r) )} \int_{ \overline{B}(x,r) } f \,\dif \mu \\
            &\leq \limsup_{r\to 0^+} \frac{1}{\mu( \overline{B}(x,r) )} \int_{ \overline{B}(x,r) } f \,\dif \mu \\
            &\leq \frac{k}{j} \leq f(x) + \frac{1}{j}.
    \end{align*}
    Since $\mu$-almost every $x \in A_{j,k}$ satisfies both $(\diamondsuit)$ and $(\clubsuit)$, we have shown that
    \[ f(x) - \frac{1}{j} \leq \liminf \frac{1}{\mu( \overline{B}(x,r) )} \int_{ \overline{B}(x,r) } f \,\dif \mu \leq \limsup \frac{1}{\mu( \overline{B}(x,r) )} \int_{ \overline{B}(x,r) } f \,\dif \mu \leq f(x) + \frac{1}{j} \]

    For each $j \in \Z^+$ and for each $x\in X$ there exists $k \in \Z^+$ such that $x \in A_{j,k}$.
    Thus for each $j\in \Z^+$, for $\mu$-almost every $x \in X$ we have
    \[ f(x) - \frac{1}{j} \leq \liminf_{r\to 0^+} \frac{1}{\mu( \overline{B}(x,r) )} \int_{ \overline{B}(x,r) } f \,\dif \mu \leq \limsup_{r\to 0^+} \frac{1}{\mu( \overline{B}(x,r) )} \int_{ \overline{B}(x,r) } f \,\dif \mu \leq f(x) + \frac{1}{j}. \]
    Taking $j \to \infty$, we conclude that for $\mu$-almost every $x \in X$,
    \[ f(x) = \lim_{r\to 0^+} \frac{1}{\mu( \overline{B}(x,r) )} \int_{ \overline{B}(x,r) } f \,\dif \mu. \]
\end{proof}

\begin{corollary}[Lebesgue Points]
    \label{cor:lebesgue_points}
    Let $\mu$ be a Borel regular outer measure on $X$ which is open $\sigma$-finite and has the symmetric Vitali property, and let $1 \leq p < \infty$.
    Let $f: X\to \R$ be a locally $p$-integrable function with respect to $\mu$, i.e. for each $x \in X$ there exists $r > 0$ such that
    \[ \int_{\overline{B}(x,r)} |f|^p \,\dif \mu < \infty. \]
    Then for $\mu$-almost every $x \in X$, we have
    \[ \lim_{r \to 0^+} \frac{1}{\mu\big( \overline{B}(x,r) \big)} \int_{\overline{B}(x,r)} |f(y) - f(x)|^p \,\dif \mu(y) = 0. \tag{$\dag$}\]
    Points $x \in X$ satisfying $(\dag)$ are called \textit{Lebesgue points} of $f$ with respect to $\mu$.
\end{corollary}

\begin{proof}
    Let $\{ q_j \}_{j=1}^\infty$ be an enumeration of the rational numbers.
    Then for each $j \in \Z^+$, the function $|f - q_j|^p$ is a locally integrable function with respect to $\mu$, so by the Lebesgue-Besicovich Differentiation Theorem \ref{thm:lebesgue_besicovich_differentiation_theorem}, for $\mu$-almost every $x \in X$ we have
    \[ |f(x) - q_j|^p = \lim_{r\to 0^+} \frac{1}{\mu( \overline{B}(x,r) )} \int_{ \overline{B}(x,r) } |f(y) - q_j|^p \,\dif \mu(y). \]

    Thus there exists a set $A\subset X$ with $\mu(A) = 0$ such that for each $j \in \Z^+$ and for each $x \in X \setminus A$, we have
    \[ |f(x) - q_j|^p = \lim_{r\to 0^+} \frac{1}{\mu( \overline{B}(x,r) )} \int_{ \overline{B}(x,r) } |f(y) - q_j|^p \,\dif \mu(y) \]
    for each $j \in \Z^+$.

    Now let $x \in X \setminus A$ be arbitrary, and let $\varepsilon > 0$. Choose $j \in \Z^+$ such that 
    \[ | f(x) - q_j|^p < \frac{\varepsilon}{2^p}. \]
    Then 
    \begin{align*}
        \limsup_{r\to 0^+} \frac{1}{\mu( \overline{B}(x,r) )} \int_{ \overline{B}(x,r) } |f(y) - f(x)|^p \,\dif \mu(y) &\leq \limsup_{r\to 0^+} \frac{2^{p-1}}{\mu( \overline{B}(x,r) )} \int_{ \overline{B}(x,r) } |f(y) - q_j|^p + |q_j - f(x)|^p \,\dif \mu(y) \\
            &= 2^{p-1} \Big( \limsup_{r\to 0^+} \frac{1}{\mu( \overline{B}(x,r) )} \int_{ \overline{B}(x,r) } |f(y) - q_j|^p \,\dif \mu(y) \\
            &\quad\qquad + \, \limsup_{r\to 0^+} \frac{1}{\mu( \overline{B}(x,r) )} \int_{ \overline{B}(x,r) } |q_j - f(x)|^p \,\dif \mu(y) \Big) \\
            &= 2^{p-1} \left( |f(x) - q_j|^p + |q_j - f(x)|^p \right) \\
            &= 2^p |f(x) - q_j|^p < \varepsilon.
    \end{align*}
    Since $\varepsilon > 0$ was arbitrary, we conclude that
    \[ \lim_{r \to 0^+} \frac{1}{\mu( \overline{B}(x,r) )} \int_{ \overline{B}(x,r) } |f(y) - f(x)|^p \,\dif \mu(y) = 0. \]
    Since this holds for each $x \in X \setminus A$ and $\mu(A) = 0$, we have shown that for $\mu$-almost every $x \in X$, $x$ is a Lebesgue point of $f$ with respect to $\mu$.
\end{proof}

\begin{notation}[Averaged Integral]
    \label{not:avg_integral}
    With $\mu$ and $f$ as in Corollary \ref{cor:lebesgue_points}, we will sometimes use the notation
    \[ \fint_{\overline{B}(x,r)} f \,\dif \mu := \frac{1}{\mu( \overline{B}(x,r) )} \int_{ \overline{B}(x,r) } f \,\dif \mu \]
    to denote the \textit{averaged integral} of $f$ over the closed ball $\overline{B}(x,r)$ with respect to the outer measure $\mu$.

    More generally, if $A \subseteq X$ is a $\mu$-measurable set with $\mu(A) > 0$  then we write
    \[ \fint_A f \,\dif \mu := \frac{1}{\mu(A)} \int_A f \,\dif \mu \]
    to denote the averaged integral of $f$ over the set $A$ with respect to the outer measure $\mu$.
\end{notation}

\subsection{The Lebesgue-Radon-Nikodym Theorem via Densities}

We begin with a result that is related to the Lebesgue Decomposition Theorem \ref{thm:lebsegue_decomposition}.

\begin{lemma}[Absolute Continuity Lemma]
    \label{lem:lower_densities_absolute_continuity}
    Let $\mu$ and $\mu_0$ be Borel regular outer measures on $X$, and assume that $(X,\mu)$ is $\sigma$-finite.
    Then there is a Borel set $Z \subseteq X$ such that $\mu_0(Z) = 0$ and $\mu\mres (X\setminus Z)$ is absolutely continuous with respect to $\mu_0$. 
    The Borel set $Z$ is unique up to $\mu$-measure zero, and as a result the Borel regular outer measure $\mu\mres (X\setminus Z)$ is uniquely determined.

    \vspace{2mm}

    \noindent The measure $\mu\mres (X\setminus Z)$ is called the \textit{absolutely continuous part} of $\mu$ with respect to $\mu_0$.
\end{lemma}

\begin{remark}[Relation to the Lebesgue Decomposition]
    \label{rem:lower_densities_absolute_continuity_lebesgue_decomposition}
    Recall that, considering $\mu$ and $\mu_0$ as measures on the measurable space $(X,\mathcal{B}(X))$, by the Lebesgue Decomposition Theorem \ref{thm:lebsegue_decomposition} there exist unique measures $\mu_{ac}$ and $\mu_s$ on $(X,\mathcal{B}(X))$ such that $\mu = \mu_{ac} + \mu_s$, where $\mu_{ac}$ is absolutely continuous with respect to $\mu_0$ and $\mu_s$ is singular with respect to $\mu_0$.
    By uniqueness of the absolutely continuous part of $\mu$ with respect to $\mu_0$ in Lemma \ref{lem:lower_densities_absolute_continuity}, we see that the measure $\mu\mres (X\setminus Z)$ in Lemma \ref{lem:lower_densities_absolute_continuity} coincides with the absolutely continuous part $\mu_{ac}$ of $\mu$ with respect to $\mu_0$ in the Lebesgue Decomposition Theorem \ref{thm:lebsegue_decomposition}.
\end{remark}

\begin{proof}
    \textit{Step 1:} First we assume that $\mu(X) < \infty$.
    \vspace{2mm}

    We define the set
    \[ \mathcal{A} := \{ A \subseteq X : \mu_0(A) = 0 \} \]
    and let \[ \alpha := \sup\{ \mu(A) : A \in \mathcal{A} \}. \]
    Choose a sequence $\{ A_j \}_{j=1}^\infty \subset \mathcal{A}$ such tha $\lim_{j\to \infty} \mu(A_j) = \alpha$.
    Then we have $\bigcup_{j=1}^\infty A_j \in \mathcal{A}$ since $\mu_0$ is countably subadditive.
    We claim that
    \[ Z := \bigcup_{j=1}^\infty A_j \]
    is the desired Borel set.

    To see this, assume that $S \subseteq X$ is such that $\mu_0(S) = 0$.
    Then by Borel regularity of $\mu_0$, there exists a Borel set $A \in \mathcal{A}$ such that $S \subseteq A$ and
    \[ \mu(S\setminus Z) \leq \mu(A\setminus Z) = \mu( Z \cup (A\setminus Z)) - \mu(Z) \leq \alpha - \alpha = 0 \]
    That is, \[ \mu\mres(X\setminus Z)(S) = 0. \]
    Since $S$ was an arbitrary Borel set with $\mu_0(S) = 0$, we have shown that $\mu\mres (X\setminus Z)$ is absolutely continuous with respect to $\mu_0$.

    To see the uniqueness of $Z$ up to $\mu$-measure zero, suppose that $\hat{Z} \subseteq X$ is another Borel set such that $\mu_0(\hat{Z}) = 0$ and $\mu\mres (X\setminus \hat{Z})$ is absolutely continuous with respect to $\mu_0$.
    Then we have
    \[ \mu(Z \setminus \hat{Z}) = \mu\mres (X\setminus \hat{Z})(Z \setminus \hat{Z}) = 0 \]
    since $\mu_0(Z \setminus \hat{Z}) = 0$.
    Similarly, we have $\mu(\hat{Z} \setminus Z) = 0$.
    Thus $\mu(Z \triangle \hat{Z}) = 0$, which proves the uniqueness of $Z$ up to $\mu$-measure zero.
    As a result, the Borel regular outer measure $\mu\mres (X\setminus Z)$ is uniquely determined.

    \vspace{2mm}
    \textit{Step 2:} Now we assume that $(X,\mu)$ is $\sigma$-finite.
    \vspace{2mm}

    Then there is an increasing sequence of Borel sets $\{ A_j \}_{j=1}^\infty$ such that $X = \bigcup_{j=1}^\infty A_j$ and $\mu(A_j) < \infty$ for each $j \in \Z^+$.
    For each $j \in \Z^+$, by Step 1 there exists a Borel set $Z_j \subseteq A_j$ such that $\mu_0(Z_j) = 0$ and $\mu\mres (A_j \setminus Z_j)$ is absolutely continuous with respect to $\mu_0$.
    Let \[ Z := \bigcup_{j=1}^\infty Z_j. \]
    Then $Z \subseteq X$ is a Borel set with $\mu_0(Z) = 0$.
    We claim that $\mu\mres (X\setminus Z)$ is absolutely continuous with respect to $\mu_0$.
    To see this, let $A \subseteq X\setminus Z$ be a Borel set with $\mu_0(A) = 0$.
    Then for each $j \in \Z^+$, we have
    \[ \mu\mres (X\setminus Z)(A \cap A_j) = \mu\mres (A_j \setminus Z_j)(A \cap A_j) = 0 \]
    since $\mu\mres (A_j \setminus Z_j)$ is absolutely continuous with respect to $\mu_0$.
    Thus by countable sub-additivity of $\mu\mres (X\setminus Z)$, we have
    \[ \mu\mres (X\setminus Z)(A) = \mu\mres (X\setminus Z)\left( \bigcup_{j=1}^\infty (A \cap A_j) \right) \leq \sum_{j=1}^\infty \mu\mres (X\setminus Z)(A \cap A_j) = 0. \]
    This proves that $\mu\mres (X\setminus Z)$ is absolutely continuous with respect to $\mu_0$.

    The uniqueness of the Borel set $Z$ up to $\mu$-measure zero follows from the uniqueness of $Z_j$ up to $\mu$-measure zero for each $j \in \Z^+$ established in Step 1.
    Thus the Borel regular outer measure $\mu\mres (X\setminus Z)$ is uniquely determined.
\end{proof}

We can also define lower densities in a similar manner to upper densities. The following two definitions and comparison lemma are the last ingredients to prepare before the proof of the Lebesgue-Radon-Nikodym Theorem via Densities.

\begin{definition}[Lower Densities]
    \label{def:lower_density_of_measure_wrt_measure}
    Let $\mu$ and $\mu_0$ be Borel regular outer measures on $X$, and assume that $\mu_0$ is locally finite. 
    Let
    \[ U_{\mu_0} := \{ x\in X : \mu_0(\overline{B}(x,r)) = 0 \text{ for some } r > 0 \} \quad\text{ and }\quad U_{\mu} := \{ x\in X : \mu(\overline{B}(x,r)) = 0 \text{ for some } r > 0 \}. \]
    We define the \textit{lower density} of $\mu$ with respect to $\mu_0$ by
    \[ \Theta_*^{\mu_0}(\mu,x) := \begin{cases}
        \displaystyle\liminf_{r \to 0^+} \frac{\mu(\overline{B}(x,r))}{\mu_0(\overline{B}(x,r)),} & \text{ if } x \in X \setminus (U_{\mu_0} \cup U_{\mu}), \\
        \infty, & \text{ if } x \in U_{\mu_0} \setminus U_{\mu}, \\
        0, & \text{ if }x \in U_{\mu}.
    \end{cases} \]
\end{definition}

\begin{definition}[Density of a Measure with respect to another Measure]
    \label{def:density_of_measure_wrt_measure}
    Let $\mu$ and $\mu_0$ be Borel regular outer measures on $X$, and assume that $\mu_0$ is locally finite. 
    If $x \in X$ is such that $\Theta^{*\mu_0}(\mu,x) = \Theta_*^{\mu_0}(\mu,x) \in [0,\infty]$, then we define the \textit{density} of $\mu$ with respect to $\mu_0$ at $x$ as the common value
    \[ \Theta^{\mu_0}(\mu,x) := \Theta^{*\mu_0}(\mu,x) = \Theta_*^{\mu_0}(\mu,x). \]
    If $x\in X$ is such that $\Theta_{*}^{\mu_0}(\mu,x) < \Theta^{*\mu_0}(\mu,x)$, then $\Theta^{\mu_0}(\mu,x)$ is undefined.
\end{definition}

We can also give an alternate definition of the density of a measure with respect to another measure, which more closely resembles the definitions of upper and lower densities.
\begin{remark}[Alternate Definition of Density of a Measure with respect to another Measure]
    \label{rem:density_of_measure_wrt_measure_defn}
    Letting $\mu,\mu_0$ be as in Definition \ref{def:density_of_measure_wrt_measure}, let 
    \[ U_{\mu_0} := \{ x \in X : \mu(\overline{B}(x,r)) = 0 \text{ for some } r > 0 \} \quad \text{ and } \quad U_\mu := \{ x \in X : \mu_0(\overline{B}(x,r)) = 0 \text{ for some } r > 0 \}. \]
    Then we see that 
    \[ \Theta^{\mu_0}(\mu,x) = \begin{cases}
        \displaystyle\lim_{r \to 0^+} \frac{\mu(\overline{B}(x,r))}{\mu_0(\overline{B}(x,r))}, & \text{ if } x \in X \setminus (U_{\mu_0} \cup U_\mu) \text{ and this limit exists}, \\
        \infty, & \text{ if } x \in U_{\mu_0} \setminus U_\mu, \\
        0, & \text{ if } x \in U_\mu,
    \end{cases} \]
    and is undefined otherwise.
\end{remark}

\begin{lemma}[Comparison Lemma for Lower Densities]
    \label{lem:lower_density_comparison_lemma}
    Let $\mu$ and $\mu_0$ be Borel regular outer measures on $X$, and assume that both $\mu$ and $\mu_0$ are open $\sigma$-finite.
    Let $t > 0$ and let $A \subseteq X$ be such that $\Theta_*^{\mu_0}(\mu,x) \leq t$ for each $x \in A$.
    \begin{enumerate}[(a)]
        \item If $\mu$ has the Symmetric Vitali Property, then $\mu(A) \leq \mu_0(A) \cdot t$.
        \item If $\mu_0$ has the Symmetric Vitali Property, then with $Z$ as in Lemma \ref{lem:lower_densities_absolute_continuity}, we have $\mu(A \setminus Z) \leq \mu_0(A) \cdot t$.
    \end{enumerate}
    
\end{lemma}

\begin{proof}
    \textit{Step 1:} First we assume that $\mu_0(X) < \infty$ and $\mu(A) < \infty$, and we prove (a).
    \vspace{2mm}

    First we observe that $A \subset X \setminus U_{\mu_0}$ since $\Theta_*^{\mu_0}(\mu,x) = \infty$ for each $x \in U_{\mu_0}$.
    Fix $\tau > t$. Since $\mu_0$ is Borel regular and $(X,\mu_0)$ is open $\sigma$-finite, Theorem \ref{thm:borel_reg_implies_inner/outer_reg} implies that there exists an open set $U \supseteq A$ such that
    \[ \mu_0(U) \leq \mu_0(A) + \tau - t. \]
    We define
    \[ \mathcal{B} := \{ \overline{B}(x,r) : x\in A \text{ and } \mu(\overline{B}(x,r)) < \tau \mu_0(\overline{B}(x,r)) \} \]
    and notice that for each $x \in A$, the definition of lower density implies that for each $\varepsilon > 0$ there exists $0< r_x < \varepsilon$ such that
    \[ \mu(\overline{B}(x,r_x)) < \tau \mu_0(\overline{B}(x,r_x)). \]
    Thus for each $x \in A$, there exists a sequence $\{ r_{x,j} \}_{j=1}^\infty$ such that $r_{x,j} \to 0$ as $j \to \infty$ and $\overline{B}(x,r_{x,j}) \in \mathcal{B}$ for each $j \in \Z^+$.
    As a result, we see that
    \[ \inf\{ r: \overline{B}(x,r) \in \mathcal{B} \} = 0 \]
    for each $x \in A$.

    Since $\mu$ has the Symmetric Vitali Property, there exists a countable collection of disjoint closed balls $\{ B_j \}_{j=1}^\infty \subset \mathcal{B}$ such that
    \[ \mu\left( A \setminus \bigcup_{j=1}^\infty B_j \right) = 0 \]
    and then also $\mu(B_j) < \tau \mu_0(B_j)$ for each $j \in \Z^+$ since $B_j \in \mathcal{B}$.
    Summing over $j \in \Z^+$ gives
    \begin{align*}
        \mu(A) &\leq \mu\left( A \setminus \bigcup_{j=1}^\infty B_j \right) + \mu\left( \bigcup_{j=1}^\infty B_j \right) \\
            &= 0 + \sum_{j=1}^\infty \mu(B_j) \\
            &< \tau \sum_{j=1}^\infty \mu_0(B_j) \\
            &= \tau \mu_0\left( \bigcup_{j=1}^\infty B_j \right) && \text{by disjointness of } \{ B_j \}_{j=1}^\infty \\
            &\leq \tau \mu_0(U) \\
            &\leq \tau ( \mu_0(A) + \tau - t ).
    \end{align*}
    Letting $\tau \to t^+$ gives the desired result
    \[ \mu(A) \leq \mu_0(A) \cdot t. \]

    \vspace{2mm}
    \textit{Step 2:} Now we assume that both $\mu$ and $\mu_0$ are open $\sigma$-finite, and we prove (a).
    \vspace{2mm}

    Then there exist countable collections of open sets $\{ U_k \}_{k=1}^\infty$ and $\{ V_k \}_{k=1}^\infty$ such that $X = \bigcup_{k=1}^\infty U_k = \bigcup_{k=1}^\infty V_k$ and $\mu(U_k) < \infty$ and $\mu_0(V_k) < \infty$ for each $k \in \Z^+$.
    By considering the collection 
    \[ \{ U_j \cap V_k : j,k\in \Z^+ \} \]
    of open set, see that $ X = \bigcup_{j,k=1}^\infty (U_j \cap V_k)$ and $\mu(U_j \cap V_k) < \infty$ and $\mu_0(U_j \cap V_k) < \infty$ for each $j,k \in \Z^+$.
    Thus be re-labeling if necessary, we may assume there is an increasing collection of open sets $\{ W_k \}_{k=1}^\infty$ such that $X = \bigcup_{k=1}^\infty W_k$ and $\mu(W_k) < \infty$ and $\mu_0(W_k) < \infty$ for each $k \in \Z^+$.

    For each $k \in \Z^+$ we consider the measures $\mu \mres W_k$ and $\mu_0 \mres W_k$ which are Borel regular outer measures on $X$ satisfying $(\mu \mres W_k)(X) = \mu(W_k) < \infty$ and $(\mu_0 \mres W_k)(X) = \mu_0(W_k) < \infty$.
    Then by Step 1, we have
    \[ (\mu \mres W_k)(A) \leq (\mu_0 \mres W_k)(A) \cdot t \]
    for each $k \in \Z^+$.
    That is, \[ \mu(A \cap W_k) \leq \mu_0(A \cap W_k) \cdot t \qquad \forall \,k \in \Z^+. \]
    By using the fact that $\{ W_k \}_{k=1}^\infty$ is an increasing collection of sets whose union is $X$, we have
    \[ \mu(A) = \lim_{k\to \infty} \mu(A \cap W_k) \leq \lim_{k\to \infty} \mu_0(A \cap W_k) \cdot t = \mu_0(A) \cdot t \]
    by Proposition \ref{prop:sequences_of_measurable_sets}.
    This proves (a).

    \vspace{2mm}
    \textit{Step 3:} Finally, we prove (b).
    \vspace{2mm}

    Let $Z \subseteq X$ be the Borel set given by Lemma \ref{lem:lower_densities_absolute_continuity}, so that $\mu \mres(X\setminus Z)$ is absolutely continuous with respect to $\mu_0$.
    Thus the Symmetric Vitali Property for $\mu_0$ implies that $\mu\mres (X\setminus Z)$ also has the Symmetric Vitali Property, so we apply the result of (a) to the set $A \setminus Z$ and the measure $\tilde{\mu}$, instead of $A$ and $\mu$, to see that
    \[ (\mu\mres(X\setminus Z))(A\setminus Z) = \mu(A\setminus Z) \leq \mu_0(A\setminus Z) \cdot t. \]
    Since $\mu_0(Z) = 0$, we have $\mu_0(A\setminus Z) = \mu_0(A)$, and the above inequality implies
    \[ \mu(A\setminus Z) \leq \mu_0(A\setminus Z) \cdot t = \mu_0(A) \cdot t \]
    which completes the proof of (b).
\end{proof}

At this point, we are ready to show how these densities relate to the differentiation theorems in the previous chapter.

\begin{theorem}[Differentiation of Measures]
    \label{thm:differentiation_of_measures}
    Let $\mu$ and $\mu_0$ be Borel regular outer measures on $X$, which are both open $\sigma$-finite.
    \begin{enumerate}[(a)]
        \item If $\mu$ has the symmetric Vitali property, then there is a Borel set $S \subseteq X$ with $\mu(S) = 0$ such that for every $x \in X\setminus S$, the density $\Theta^{\mu_0}(\mu,x)$ is defined.
        \item If $\mu_0$ has the symmetric Vitali property, then there is a Borel set $S\subseteq X$ with $\mu_0(S) = 0$ such that for every $x \in X\setminus S$, the density $\Theta^{\mu_0}(\mu,x)$ is defined.
    \end{enumerate}
    In either case, the function 
    \[ (X\setminus S) \ni x\longmapsto \Theta^{\mu_0}(\mu,x) \]
    is Borel measurable.
\end{theorem}

\begin{proof}[Proof of (a)]
    We will show (a) first, so assume that $\mu$ has the symmetric Vitali property.

    \vspace{2mm}
    \textit{Step 1:} First we assume that $\mu_0(X) < \infty$ and $\mu(A) < \infty$, and we prove (a).
    \vspace{2mm}
    
    First see that if $A\subseteq X$ is a subset and $a,b > 0$ are positive real numbers such that 
    \[ \Theta_*^{\mu_0}(\mu,x) < a \quad\text{ and }\quad b < \Theta^{*\mu_0}(\mu,x) \qquad \forall\, x\in A\]
    then \[ \mu(A) \leq a \mu_0(A) \quad\text{ and }\quad \mu_0(A) \cdot b \leq \mu(A) \]
    by the Comparison Lemmas \ref{lem:comparison_lemma_general} and \ref{lem:lower_density_comparison_lemma}.

    In particular, if $0 < a < b$ and 
    \[ E_{a,b} := \{ x\in X: \Theta_*^{\mu_0}(\mu,x) < a < b < \Theta^{*\mu_0}(\mu,x) \} \]
    then we have
    \[ \frac{\mu(E_{a,b})}{a} \leq \mu_0(E_{a,b}) \leq \frac{\mu(E_{a,b})}{b} \]
    which implies that \[ \mu_0(E_{a,b}) = \mu(E_{a,b}) = 0. \]
    Because 
    \[ \{ x\in X : \Theta_*^{\mu_0}(\mu,x) < \Theta^{*\mu_0}(\mu,x) \} \subseteq \bigcup_{\substack{ a,b\in \Q^+ \\ a < b }}  E_{a,b} \]
    we deduce that $\Theta_*^{\mu_0}(\mu,x) < \Theta^{*\mu_0}(\mu,x)$ only on a set which has both $\mu$-measure and $\mu_0$-measure zero.
    As a result the density $\Theta^{\mu_0}(\mu,x)$ is defined for $\mu$-almost every $x \in X \setminus U_{\mu_0}$.
    Since $\Theta^{\mu_0}(\mu,x)$ is also defined in $[0,\infty]$ for each $x \in U_{\mu}$ by \ref{rem:density_of_measure_wrt_measure_defn}, we conclude that $\Theta^{\mu_0}(\mu,x)$ is defined for $\mu$-almost every $x \in X$.
    By using Borel regularity of $\mu$, we may therefore choose a Borel set $S \subseteq X$ with $\mu(S) = 0$ such that for every $x \in X\setminus S$, the density $\Theta^{\mu_0}(\mu,x)$ is defined.

    We now show that thedensity function 
    \[ (X\setminus S) \ni x \longmapsto \Theta^{\mu_0}(\mu,x) \]
    is Borel measurable.

    For each fixed $ r > 0 $, the functions
    \[ x\longmapsto \mu(B(x,r)), \qquad x\longmapsto \mu_0(B(x,r)) \]
    defined on $X \setminus (S \cup U_{\mu_0} \cup U_{\mu})$ are upper semi-continuous, and hence Borel measurable.
    Thus their ratio
    \[ x\longmapsto \frac{\mu(B(x,r))}{\mu_0(B(x,r))} \]
    is also Borel measurable on $X \setminus (S \cup U_{\mu_0} \cup U_{\mu})$.
    Therefore 
    \[ \Theta^{\mu_0}(\mu,x) = \lim_{j\to \infty} \frac{\mu(B(x,\frac{1}{j}))}{\mu_0(B(x,\frac{1}{j}))} \]
    is a Borel measurable function on $X \setminus (S \cup U_{\mu_0} \cup U_{\mu})$, as the pointwise limit of a sequence of Borel measurable functions.

    Since $\Theta^{\mu_0}(\mu,x) = \infty$ on $U_{\mu_0} \setminus U_{\mu}$ and $\Theta^{\mu_0}(\mu,x) = 0$ on $U_{\mu}$, and the sets $U_{\mu_0}$ and $U_{\mu}$ are both open (by \ref{ex:set_where_measure_vanishes_are_open}), 
    so we conclude that $\Theta^{\mu_0}(\mu,x)$ is Borel measurable on all of $X \setminus S$.
    
    \vspace{2mm}
    \textit{Step 2:} Now we assume that both $\mu$ and $\mu_0$ are open $\sigma$-finite, and we prove (a).
    \vspace{2mm}

    Then there exist countable collections of open sets $\{ U_k \}_{k=1}^\infty$ and $\{ V_k \}_{k=1}^\infty$ such that $X = \bigcup_{k=1}^\infty U_k = \bigcup_{k=1}^\infty V_k$ and $\mu(U_k) < \infty$ and $\mu_0(V_k) < \infty$ for each $k \in \Z^+$.
    By considering the collection 
    \[ \{ U_j \cap V_k : j,k\in \Z^+ \} \]
    of open sets, see that $ X = \bigcup_{j,k=1}^\infty (U_j \cap V_k)$ and $\mu(U_j \cap V_k) < \infty$ and $\mu_0(U_j \cap V_k) < \infty$ for each $j,k \in \Z^+$.
    Thus be re-labeling if necessary, we may assume there is an increasing collection of open sets $\{ W_k \}_{k=1}^\infty$ such that $X = \bigcup_{k=1}^\infty W_k$ and $\mu(W_k) < \infty$ and $\mu_0(W_k) < \infty$ for each $k \in \Z^+$.

    For each $k \in \Z^+$ we consider the measures $\mu \mres W_k$ and $\mu_0 \mres W_k$ which are Borel regular outer measures on $X$ satisfying $(\mu \mres W_k)(X) = \mu(W_k) < \infty$ and $(\mu_0 \mres W_k)(X) = \mu_0(W_k) < \infty$.
    Then by Step 1, there exists a Borel set $S_k \subseteq W_k$ with $(\mu \mres W_k)(S_k) = \mu(S_k) = 0$ such that for every $x \in W_k \setminus S_k$, the density $\Theta^{\mu_0 \mres W_k}(\mu \mres W_k,x)$ is defined.
    Let \[ S := \bigcup_{k=1}^\infty S_k. \]
    Then $S \subseteq X$ is a Borel set with $\mu(S) = 0$.
    We claim that for every $x \in X\setminus S$, the density $\Theta^{\mu_0}(\mu,x)$ is defined.
    To see this, let $x \in X\setminus S$ be arbitrary.
    Then there exists $k_0 \in \Z^+$ such that $x \in W_{k_0}$.
    Since $x \notin S$, we have $x \notin S_{k_0}$, so the density $\Theta^{\mu_0 \mres W_{k_0}}(\mu \mres W_{k_0},x)$ is defined.
    By the definitions of densities, we see that
    \[ \Theta^{\mu_0}(\mu,x) = \Theta^{\mu_0 \mres W_{k_0}}(\mu \mres W_{k_0},x) \]
    is also defined.
    Since $x \in X\setminus S$ was arbitrary, we have shown that for every $x \in X\setminus S$, the density $\Theta^{\mu_0}(\mu,x)$ is defined.

    Lastly, see that the density function $\Theta^{\mu_0}(\mu,x)$ is Borel measurable on $X\setminus S$ by taking the pointwise limit of the Borel measurable functions $\Theta^{\mu_0 \mres W_k}(\mu \mres W_k,x)$ on each $W_k \setminus S_k$ as in Step 1.
\end{proof}

\begin{proof}[Proof of (b)]
    Now we prove (b), so assume that $\mu_0$ has the symmetric Vitali property.
    By \ref{cor:upper_density_finite}, we know that
    \[ \Theta^{*\mu_0}(\mu,x) < \infty \qquad \text{ for } \mu_0\text{-almost every } x \in X. \]
    By \ref{lem:lower_densities_absolute_continuity}, there is a Borel set $Z \subseteq X$ such that $\mu_0(Z) = 0$ and $\hat{\mu} := \mu \mres (X\setminus Z)$ is absolutely continuous with respect to $\mu_0$.
    Since $\hat{\mu}$ is absolutely continuous with respect to $\mu_0$, it is clear that $\hat{\mu}$ also has the Symmetric Vitali Property. 
    Now we use the argument of part (a) applied to the measures $\hat{\mu}$ to see that
    \[ \mu_0(E_{a,b}) = \mu(E_{a,b} \setminus Z) = 0 \]
    where $E_{a,b}$ is as defined in the proof of part (a).
    Thus the density $\Theta^{\mu_0}(\mu,x)$ is defined for $\mu_0$-almost every $x \in X \setminus Z$, and by our first remark in this proof, it is finite for $\mu_0$-almost every $x \in X$.
    By using Borel regularity of $\mu_0$, we may therefore choose a Borel set $S \subseteq X$ with $\mu_0(S) = 0$ such that for every $x \in X\setminus S$, the density $\Theta^{\mu_0}(\mu,x)$ is defined and finite.

    The Borel measurability of the density function
    \[ (X\setminus S) \ni x \longmapsto \Theta^{\mu_0}(\mu,x) \]
    follows from the same argument as in part (a).
\end{proof}

\begin{theorem}[The Lebesgue-Radon-Nikodym Theorem]
    \label{thm:radon_nikodym_theorem_via_densities}
    Let $\mu,\mu_0$ be Borel regular outer measures on $X$, which are both open $\sigma$-finite, and suppose that $\mu_0$ has the symmetric Vitali property.
    \begin{enumerate}[(i)]
        \item If $\mu \ll \mu_0$, then $\mu$ also has the symmetric Vitali property, and
            \[ \mu(A) = \int_A \Theta^{\mu_0}(\mu,x) \,\dif \mu_0(x) \]
            for each Borel set $A \subseteq X$.
        \item In general, there exists a Borel set $Z\subseteq X$ such that $\mu_0(Z) = 0$ and
            \[ \mu(A) = \int_A \Theta^{\mu_0}(\mu,x) \,\dif \mu_0(x) + (\mu\mres Z)(A) \]
            for each Borel set $A \subseteq X$.
        \item If $\mu$ also has the symmetric Vitali property, then the set $Z$ in part (b) can be taken to be 
            \[ Z = \{ x \in X : \Theta^{\mu_0}(\mu,x) = \infty \}. \]
    \end{enumerate}
\end{theorem}

\begin{remark}[Interpretation of the Lebesgue-Radon-Nikodym Theorem via Densities]
    \label{rem:interpretation_of_radon_nikodym_theorem_via_densities}
Basically this says that if you have Borel regular outer measures $\mu$ and $\mu_0$ on $X$ which are open $\sigma$-finite and such that $\mu_0$ has the symmetric Vitali property, then the Radon-Nikodym derivative of $\mu$ with respect to $\mu_0$ can be computed as a limit of ratios of measures at $\mu$-almost every point.

More precisely, using our notation for the Radon-Nikodym derivative, (i) says that if $\mu \ll \mu_0$, then
\[ D_{\mu_0}\mu(x) = \Theta^{\mu_0}(\mu,x) = \lim_{r \to 0^+} \frac{\mu\big( \overline{B}(x,r) \big)}{\mu_0\big( \overline{B}(x,r) \big)} \]
for $\mu_0$-almost every $x \in X$.

\begin{proof}[Sanity Check]
[ Sanity check --- note that $\Theta^{\mu_0}(\mu,x)$ exists for $\mu_0$-almost every $x \in X$ by Theorem \ref{thm:differentiation_of_measures}.
To justify the second equality computing $\Theta^{\mu_0}(\mu,x)$ as a limit, we use the alternate definition \ref{rem:density_of_measure_wrt_measure_defn} of density of a measure with respect to another measure.
Since
\[ \Theta^{\mu_0}(\mu,x) = \begin{cases}
        \displaystyle\lim_{r \to 0^+} \frac{\mu(\overline{B}(x,r))}{\mu_0(\overline{B}(x,r))}, & \text{ if } x \in X \setminus (U_{\mu_0} \cup U_\mu) \text{ and this limit exists}, \\
        \infty, & \text{ if } x \in U_{\mu_0} \setminus U_\mu, \\
        0, & \text{ if } x \in U_\mu,
\end{cases} \]
and $\mu_0 (U_{\mu_0}) = 0$, so the second branch only occurs on a $\mu_0$-measure zero set.

If $x \in U_\mu \setminus U_{\mu_0}$, then $ \mu_0(\overline{B}(x,r)) > 0$ for each $r > 0$ but $\mu(\overline{B}(x,r)) = 0$ for some $r > 0$, so the limit in the first branch is $0$ which agrees with the third branch (for such $x$).
Since the set $U_{\mu_0}$ has $\mu_0$-measure zero, we conclude that for every $x \in X \setminus U_{\mu_0}$ where $\Theta^{\mu_0}(\mu,x)$ is defined, we have
\[ \Theta^{\mu_0}(\mu,x) = \lim_{r \to 0^+} \frac{\mu\big( \overline{B}(x,r) \big)}{\mu_0\big( \overline{B}(x,r) \big)}. \]
Finally, use Theorem \ref{thm:differentiation_of_measures} to see that $\Theta^{\mu_0}(\mu,x)$ is defined for $\mu_0$-almost every $x \in X$, so the above equality holds for $\mu_0$-almost every $x \in X$ as well. ]
\end{proof}

\vspace{2mm}

Also (ii) says that in general, we can decompose $\mu$ into an absolutely continuous part with respect to $\mu_0$ and a singular part with respect to $\mu_0$ using these densities, i.e.
\begin{align*}
    \mu &= \ \, \mu_{ac} \ \ + \ \ \ \mu_s \\
        &= D_{\mu_0}\mu + (\mu\mres Z)
\end{align*}
and the absolutely continuous part and the singular part are unique, by the Lebesgue Decomposition Theorem \ref{thm:lebsegue_decomposition}.
Furthermore the absolutely continuous part $\mu_{ac}$ can be computed pointwise using the densities $\Theta^{\mu_0}(\mu,x) = \lim_{r \to 0^+} \frac{\mu(\overline{B}(x,r))}{\mu_0(\overline{B}(x,r))}$ for $\mu_0$-almost every $x \in X$ as in the previous paragraph.
\end{remark}

First we prove the statements of the theorem in the case that $\mu(X) < \infty$ and $\mu_0(X) < \infty$.
The general case where $\mu$ and $\mu_0$ are open $\sigma$-finite then follows by a standard exhaustion argument as in previous results.

\begin{proof}[Proof when $\mu(X) < \infty$ and $\mu_0(X) < \infty$.]

    We assume that both $\mu(X) < \infty$ and $\mu_0(X) < \infty$.
    \vspace{2mm}

    (i). Assume that $\mu \ll \mu_0$.
    Then it is clear that $\mu$ has the Symmetric Vitali Property since $\mu_0$ does.

    Since $\mu_0$ has the Symmetric Vitali Property, there is a Borel set $S \subseteq X$ with $\mu_0(S) = 0$ such that for every $x \in X\setminus S$, the density $\Theta^{\mu_0}(\mu,x)$ is defined by Theorem \ref{thm:differentiation_of_measures}.
    For each Borel set $A \subset X \setminus S$ we define
    \[ \nu(A) := \int_A \Theta^{\mu_0}(\mu,x) \,\dif \mu_0(x) \]
    and for an arbitrary set $E \subseteq X$, we define
    \[ \nu(E) := \inf \{  \nu(A) : A\text{ is a Borel set such that } A \supseteq E \}. \]
    The same argument as in the first step of the proof of \ref{lem:lower_densities_absolute_continuity} shows that $\nu$ is a well-defined Borel regular outer measure on $X$.

    Let $A \subseteq X$ be an arbitrary Borel set.
    If $0 < a < b$ and \[ A_{(a,b)} := \{ x \in A : a < \Theta^{\mu_0}(\mu,x) < b \}, \] then we have
    \[ a \mu_0 (A_{a,b}) \leq \nu(A_{a,b}) \leq b \mu_0 (A_{a,b}) \]
    by definition of $\nu$, while the Comparison Lemmas \ref{lem:comparison_lemma_general} and \ref{lem:lower_density_comparison_lemma} imply that
    \[ a \mu_0 (A_{a,b}) \leq \mu(A_{a,b}) \leq b \mu_0 (A_{a,b}) \]
    and as a result we see that
    \[ \frac{a}{b} \mu_0 (A_{a,b}) \leq \nu(A_{a,b}) \leq \frac{b}{a} \mu_0 (A_{a,b}). \]
    Since $a$ and $b$ are arbitrary positive real numbers with $a < b$, we conclude that
    \[ \nu(A) = \mu(A). \]
    Since $A \subset X$ was an arbitrary Borel set, we have proven (i).

    \vspace{2mm}

    (ii). Now assume that we are in the general case, where we allow sets $A \subset X$ with $\mu(A) > 0$ and $\mu_0(A) = 0$.
    By \ref{lem:lower_densities_absolute_continuity}, there exists a Borel set $Z \subseteq X$ such that $\mu_0(Z) = 0$ and $\hat{\mu} := \mu \mres (X\setminus Z)$ is absolutely continuous with respect to $\mu_0$.
    Applying part (i) to the measures $\hat{\mu}$ and $\mu_0$, we see that
    \[ \hat{\mu}(A) = \int_A \Theta^{\mu_0}(\hat{\mu},x) \,\dif \mu_0(x) \qquad\text{for each Borel set } A \subseteq X\]
    which means that 
    \[ \mu(A \setminus Z) = \int_{A \setminus Z} \Theta^{\mu_0}(\hat{\mu},x) \,\dif \mu_0(x) \qquad\text{for each Borel set} A \subset X. \]
    As a result, for each Borel set $A \subseteq X$, we have
    \[ \mu(A) = \mu(A \setminus Z) + \mu(A\cap Z) = \int_{A \setminus Z} \Theta^{\mu_0}(\hat{\mu},x) \,\dif \mu_0(x) + \mu(A\cap Z) = \int_A \Theta^{\mu_0}(\hat{\mu},x) \,\dif \mu_0(x) + (\mu\mres Z)(A) \]
    as desired. 
    
    \vspace{2mm}

    (iii). Finally, assume that $\mu$ also has the Symmetric Vitali Property.
    Then the density $\Theta^{\mu_0}(\mu,x)$ is defined for $\mu$-almost every $x \in X$ by Theorem \ref{thm:differentiation_of_measures} (and also for $\mu_0$-almost every $x \in X$ by the same theorem).
    Also see that if
    \[ A \subseteq \{ x \in X\setminus U_{\mu_0} : \Theta^{\mu_0}(\mu,x) < \infty \} \]
    then the Comparison Lemma for Lower Densities \ref{lem:lower_density_comparison_lemma} shows that 
    \[ \mu_0(A) = 0 \implies \mu(A) = 0. \]
    Therefore we can apply part (i) to the measure $\mu \mres (X\setminus \{ x : \Theta^{\mu_0}(\mu,x) = \infty \})$ which is absolutely continuous with respect to $\mu_0$ (as we just showed) to see that
    \[ \mu(A \setminus \{ x : \Theta^{\mu_0}(\mu,x) = \infty \}) = \int_{A \setminus \{ x : \Theta^{\mu_0}(\mu,x) = \infty \}} \Theta^{\mu_0}(\mu,x) \,\dif \mu_0(x) \]
    for each Borel set $A \subseteq X$.

    Thus if we let
    \[ Z := \{ x \in X : \Theta^{\mu_0}(\mu,x) = \infty \}, \]
    then $Z \subseteq X$ is a Borel set with $\mu_0(Z) = 0$ and for each Borel set $A \subseteq X$, we have
    \[ \mu(A) = \int_A \Theta^{\mu_0}(\mu,x) \,\dif \mu_0(x) + (\mu\mres Z)(A) \]
    as desired.
\end{proof}

\begin{proof}[Proof when both $\mu$ and $\mu_0$ are open $\sigma$-finite.]
    Assume now that both $\mu$ and $\mu_0$ are open $\sigma$-finite.
    Then there exist countable collections of open sets $\{ U_k \}_{k=1}^\infty$ and $\{ V_k \}_{k=1}^\infty$ such that $X = \bigcup_{k=1}^\infty U_k = \bigcup_{k=1}^\infty V_k$ and $\mu(U_k) < \infty$ and $\mu_0(V_k) < \infty$ for each $k \in \Z^+$.
    By considering the collection 
    \[ \{ U_j \cap V_k : j,k\in \Z^+ \} \]
    of open sets, see that $ X = \bigcup_{j,k=1}^\infty (U_j \cap V_k)$ and $\mu(U_j \cap V_k) < \infty$ and $\mu_0(U_j \cap V_k) < \infty$ for each $j,k \in \Z^+$.
    Thus be re-labeling if necessary, we may assume there is an increasing collection of open sets $\{ W_k \}_{k=1}^\infty$ such that $X = \bigcup_{k=1}^\infty W_k$ and $\mu(W_k) < \infty$ and $\mu_0(W_k) < \infty$ for each $k \in \Z^+$.

    For each $k \in \Z^+$ we consider the measures $\mu \mres W_k$ and $\mu_0 \mres W_k$ which are Borel regular outer measures on $X$ satisfying $(\mu \mres W_k)(X) = \mu(W_k) < \infty$ and $(\mu_0 \mres W_k)(X) = \mu_0(W_k) < \infty$.
    Then by the previous proof, there exists a Borel set $Z_k \subseteq W_k$ such that $(\mu_0 \mres W_k)(Z_k) = \mu_0(Z_k) = 0$ and
    \[ (\mu \mres W_k)(A) = \int_A \Theta^{\mu_0 \mres W_k}(\mu \mres W_k,x) \,\dif (\mu_0 \mres W_k)(x) + ((\mu \mres W_k)\mres Z_k)(A) \]
    for each Borel set $A \subseteq X$.
    Let \[ Z := \bigcup_{k=1}^\infty Z_k. \]
    Then $Z \subseteq X$ is a Borel set with $\mu_0(Z) = 0$.

    See that for each Borel set $A \subseteq X$, we have
    \begin{align*}
        \mu(A) &= \lim_{k\to \infty} (\mu \mres W_k)(A) \\
            &= \lim_{k\to \infty} \left( \int_A \Theta^{\mu_0 \mres W_k}(\mu \mres W_k,x) \,\dif (\mu_0 \mres W_k)(x) + ((\mu \mres W_k)\mres Z_k)(A) \right) \\
            &= \lim_{k\to \infty} \int_A \Theta^{\mu_0 \mres W_k}(\mu \mres W_k,x) \,\dif (\mu_0 \mres W_k)(x) + \lim_{k\to \infty} ((\mu \mres W_k)\mres Z_k)(A) \\
            &= \int_A \Theta^{\mu_0}(\mu,x) \,\dif \mu_0(x) + (\mu\mres Z)(A).
    \end{align*}
   
    Thus if $\mu \ll \mu_0$, then $\mu\mres Z = 0$ and part (i) holds, while in general part (ii) holds.
    Lastly, if $\mu$ also has the Symmetric Vitali Property, then by the result in the finite case we may take
    \[ Z_k := \{ x\in W_k : \Theta^{\mu_0 \mres W_k}(\mu \mres W_k,x) = \infty \} \qquad\forall \, k\in\Z^+ \]
    which means that
    \[ Z = \bigcup_{k=1}^\infty Z_k = \{ x \in X : \Theta^{\mu_0}(\mu,x) = \infty \} \]
    as desired in part (iii).
\end{proof}
