\section{Signed and Vector-Valued Measures, Decomposition Theorems}
\subsection{Motivation}

In the previous chapter, we studied generalizations of the fundamental theorem of calculus --- in particular, we found that 
\begin{itemize}
    \item (First Fundamental Theorem of Calculus) If $f : [a,b] \to \R$ is absolutely continuous, then $f$ is differentiable almost everywhere, $f' \in L^1([a,b])$, and
        \[ f(x) - f(a) = \int_a^x f'(t) \,\dif t \quad\text{ for all } x \in [a,b]. \]
    \item (Second Fundamental Theorem of Calculus) If $f \in L^1([a,b])$, then the function $F$ defined by
        \[ F(x) = \int_a^x f(t) \,\dif t \]
        is absolutely continuous, differentiable almost everywhere, and $F' = f$ almost everywhere.
\end{itemize}
In spirit, the Radon-Nikodym Theorem is a generalization of these results.

\subsection{Signed and Vector-Valued Measures}

\begin{definition}[Signed Measure]
    \label{def:signed_measure}
    A \textit{signed measure} on a measurable space $(X,\mathcal{A})$ is a function $\nu : \mathcal{A} \to \R$ such that
    for each countable collection $\{ A_j \}_{j=1}^\infty$ of pairwise disjoint sets in $\mathcal{A}$, we have
            \[ \nu\left( \bigcup_{j=1}^\infty A_j \right) = \sum_{j=1}^\infty \nu(A_j). \]  
\end{definition}
Note the \emph{by definition} a signed measure is finite on the whole space, i.e., $|\nu|(X) < \infty$.
That is, we do not allow any signed measures that take on infinite values.

In particular, many measures are \emph{not} signed measures, since they take on infinite values.
A measure $\mu$ is a signed measure if and only if it is finite on all measurable sets, i.e., $\mu(A) < \infty$ for all $A \in \mathcal{A}$.

The following is the key example of a signed measure.
\begin{example}[Signed Measure from an $L^1$ Function]
    \label{ex:signed_measure_from_L1_function}
    Let $(X,\mathcal{A},\mu)$ be a measure space, and let $f \in L^1(X,\mu)$.
    Define $\nu : \mathcal{A} \to \R$ by
    \[ \nu(A) = \int_A f \,\dif \mu. \]
    Then $\nu$ is a signed measure on $(X,\mathcal{A})$.

    We check this.
    Let $\{ A_j \}_{j=1}^\infty$ be a countable collection of pairwise disjoint sets in $\mathcal{A}$.
    Then
    \begin{align*}
        \nu\left( \bigcup_{j=1}^\infty A_j \right) &= \int_{\bigcup_{j=1}^\infty A_j} f \,\dif \mu \\
        &= \sum_{j=1}^\infty \int_{A_j} f \,\dif \mu \qquad\text{ by countable additivity of the integral } \\
        &= \sum_{j=1}^\infty \nu(A_j).
    \end{align*}
    Thus, $\nu$ is a signed measure.
\end{example}

\begin{remark}[Signed Measures as ``Charges'']
    \label{ref:signed_measure_as_charge}
    Some authors, particularly the older Russian literature, refer to signed measures as \textit{charges}.
    The idea is that in the case of an electrical charge distributed over a surface, each region of the surface can have either positive or negative charge, so the charge distribution is naturally modeled by a signed measure.
    This is a nice intuition to keep in mind.
\end{remark}

\begin{lemma}[Absolute Convergence for Disjoint Unions]
    \label{lem:signed_measure_absolute_convergence}
    Let $\nu$ be a signed measure on a measurable space $(X,\mathcal{A})$.
    Then for each countable collection $\{ A_j \}_{j=1}^\infty$ of pairwise disjoint sets in $\mathcal{A}$ we have
    \[ \sum_{j=1}^\infty | \nu(A_j) | < \infty. \]
\end{lemma}
\begin{proof}
    First note that $\nu(\varnothing) = 0$ since 
    \[ \nu(\varnothing) = \nu\left( \bigcup_{j=1}^\infty \varnothing \right) = \sum_{j=1}^\infty \nu(\varnothing) \]
    is a countable disjoint union of sets in $\mathcal{A}$, so 
    \[ \nu(\varnothing) = \sum_{j=1}^\infty \nu(\varnothing) \]
    and this can only hold if $\nu(\varnothing) = 0$.

    \vspace{2mm}

    Now let $\{ A_j \}_{j=1}^\infty$ be a countable collection of pairwise disjoint sets in $\mathcal{A}$.
    Then 
    \[ \nu\left( \bigcup_{ \{ j \,:\, \nu(A_j) > 0 \} } A_j \right) = \sum_{ \{ j \,:\, \nu(A_j) > 0 \} } \nu(A_j) = \sum_{ \{ k \,:\, \nu(A_k) > 0 \} } |\nu(A_k)| \]
    and
    \[ -\nu\left( \bigcup_{ \{ j \,:\, \nu(A_j) < 0 \} } A_j \right) = \sum_{ \{ j \,:\, \nu(A_j) < 0 \} } -\nu(A_j) = \sum_{ \{ k \,:\, \nu(A_k) < 0 \} } |\nu(A_k)|. \]
    Since $\nu(A) \in \R$ for each $A \in \mathcal{A}$, the right-hand side of both of these sums must be finite.
    Therefore the total sum
    \[ \sum_{j=1}^\infty | \nu(A_j) | < \infty \] 
    converges absolutely as desired.
\end{proof}

\begin{definition}[Complex Measure, Vector Measure]
    \label{def:vector_measure}
    Let $(X,\mathcal{A})$ be a measurable space.
    A \textit{complex measure} on $(X,\mathcal{A})$ is a function $\nu : \mathcal{A} \to \C$ such that for each countable collection $\{ A_j \}_{j=1}^\infty$ of pairwise disjoint sets in $\mathcal{A}$, we have
            \[ \nu\left( \bigcup_{j=1}^\infty A_j \right) = \sum_{j=1}^\infty \nu(A_j). \]  
    A \textit{vector measure} on $(X,\mathcal{A})$ taking values in $\R^m$ is a function $\nu : \mathcal{A} \to \R^m$ such that for each countable collection $\{ A_j \}_{j=1}^\infty$ of pairwise disjoint sets in $\mathcal{A}$, we have
            \[ \nu\left( \bigcup_{j=1}^\infty A_j \right) = \sum_{j=1}^\infty \nu(A_j). \]
\end{definition}
Technically we could define vector measures taking values in any Banach space, but we will only consider $\R^m$-valued vector measures.

The following exercise shows that vector and complex measures can be understood component-wise as signed measures.

\begin{exercise}[Components of a Vector Measure]
    \label{ex:vector_measure_components}
    Show that a function $\nu : \mathcal{A} \to \R^m$ is a vector measure if and only if each of its component functions $\nu_k : \mathcal{A} \to \R$ defined by
    \[ \nu_k(A) = \text{the $k$-th component of } \nu(A) \]
    is a signed measure.

    Similarly, show that a function $\nu : \mathcal{A} \to \C$ is a complex measure if and only if its real and imaginary parts are signed measures.
\end{exercise}
\begin{proof}
    Fix $m\geq 1$ and for each $1 \leq k \leq m$, let $\pi_k : \R^m \to \R$ be the projection onto the $k$-th coordinate, 
    \[ \pi_k(x_1,x_2,\ldots,x_m) = x_k. \]
    Suppose that $\nu : \mathcal{A} \to \R^m$ is a vector measure and let $\{ A_j \}_{j=1}^\infty$ be a countable collection of pairwise disjoint sets in $\mathcal{A}$.
    Then for each $1 \leq k \leq m$, we have
    \[  \nu_k\left( \bigcup_{j=1}^\infty A_j \right) = \pi_k\circ \nu \left( \bigcup_{j=1}^\infty A_j \right) = \pi_k\left( \sum_{j=1}^\infty \nu(A_j) \right) = \sum_{j=1}^\infty \nu_k(A_j). \]
    Thus, each component function $\nu_k$ is a signed measure.

    Conversely, suppose that each component function $\nu_k$ is a signed measure.
    Then for each countable collection $\{ A_j \}_{j=1}^\infty$ of pairwise disjoint sets in $\mathcal{A}$, we have
    \begin{align*}
        \nu\left( \bigcup_{j=1}^\infty A_j \right) &= \left( \nu_1\left( \bigcup_{j=1}^\infty A_j \right), \nu_2\left( \bigcup_{j=1}^\infty A_j \right), \ldots, \nu_m\left( \bigcup_{j=1}^\infty A_j \right) \right) \\
            &= \left( \sum_{j=1}^\infty \nu_1(A_j), \sum_{j=1}^\infty \nu_2(A_j), \ldots, \sum_{j=1}^\infty \nu_m(A_j) \right) \\
            &= \sum_{j=1}^\infty \left( \nu_1(A_j), \nu_2(A_j), \ldots, \nu_m(A_j) \right) \\
            &= \sum_{j=1}^\infty \nu(A_j).  
    \end{align*}
    Thus, $\nu$ is a vector measure as desired.

    \vspace{2mm}

    The complex measure case is similar, using the projections onto the real and imaginary parts.
\end{proof}

We generalize Example \ref{ex:signed_measure_from_L1_function} as follows.
\begin{example}[Vector Measure from a Vector-Valued $L^1$ Function]
    \label{ex:vector_measure_from_L1_function}
    Let $f \in L^1(X,\mu;\R^m)$ be a vector-valued $L^1$ function.
    That is, for each $1 \leq k \leq m$, the $k$-th component function $f_k$ is in $L^1(X,\mu)$.
    Define $\nu : \mathcal{A} \to \R^m$ by
    \[ \nu(A) = \int_A f \,\dif \mu. \]
    Then $\nu$ is a vector measure on $(X,\mathcal{A})$ by Exercise \ref{ex:vector_measure_components} and Example \ref{ex:signed_measure_from_L1_function}.
\end{example}

\begin{proposition}[Properties of Vector Measures]
    \label{prop:vector_measure_properties}
    Let $(X,\mathcal{A})$ be a measurable space and let $\nu$ be a vector measure or a complex measure on $(X,\mathcal{A})$.
    Then 
    \begin{itemize}
        \item $\nu(A\setminus B) = \nu(A) - \nu(B)$ for all $A,B \in \mathcal{A}$ with $B \subseteq A$,
        \item $\nu(A\cup B) = \nu(A) + \nu(B) - \nu(A \cap B)$ for all $A,B \in \mathcal{A}$,
        \item if $\{ A_j \}_{j=1}^\infty$ is a countable collection of sets in $\mathcal{A}$ with $A_j \subseteq A_{j+1}$ for all $j\geq 1$, then
            \[ \nu\left( \bigcup_{j=1}^\infty A_j \right) = \lim_{j\to\infty} \nu(A_j), \]
        \item if $\{ A_j \}_{j=1}^\infty$ is a countable collection of sets in $\mathcal{A}$ with $A_{j+1} \subseteq A_j$ for all $j\geq 1$, then
            \[ \nu\left( \bigcap_{j=1}^\infty A_j \right) = \lim_{j\to\infty} \nu(A_j). \]
    \end{itemize}
\end{proposition}
\begin{proof}
    By Exercise \ref{ex:vector_measure_components}, it suffices to prove these properties for signed measures.

    \vspace{2mm}

    (i) Let $A,B \in \mathcal{A}$ with $B \subseteq A$.
    Then
    \[ \nu(A) = \nu(B \cup (A\setminus B)) = \nu(B) + \nu(A\setminus B) \]
    since $B$ and $A\setminus B$ are disjoint, so rearranging gives the desired result.

    \vspace{2mm}

    (ii) Let $A,B \in \mathcal{A}$.
    Then
    \begin{align*}
        \nu(A\cup B) &= \nu(A \setminus B) + \nu(B) \\
            &= \nu(A) - \nu(A \cap B) + \nu(B)
    \end{align*}
    by part (i), giving the desired result.

    \vspace{2mm}

    The proofs of (iii) and (iv) are identical to as in \ref{prop:sequences_of_measurable_sets}. Note that the assumptions that $\nu(A_1)<\infty$ or $\nu(A_1)>-\infty$ are not needed here since $\nu$ is a signed measure and thus finite on all measurable sets.
\end{proof}

\subsection{Hahn and Jordan Decompositions, Total Variation}

In the case of an electrical charge distributed on a surface, we can divide the surface into regions of positive and negative charge.
This idea is formalized in the following decomposition theorems for signed measures.

\begin{definition}[Positive and Negative Sets]
    \label{def:negative_set_positive_set}
    Let $(X,\mathcal{A})$ be a measurable space and let $\nu$ be a signed measure on $(X,\mathcal{A})$.
    A set $A \in \mathcal{A}$ is called \textit{positive} (with respect to $\nu$) if for every measurable subset $E \subseteq A$, we have $\nu(E) \geq 0$.
    Similarly, a set $A \in \mathcal{A}$ is called \textit{negative} if for every measurable subset $E \subseteq A$, we have $\nu(E) \leq 0$.
\end{definition}

\begin{theorem}[Hahn Decomposition]
    \label{thm:hahn_decomposition}
    Let $(X,\mathcal{A})$ be a measurable space and let $\nu$ be a signed measure on $(X,\mathcal{A})$.
    Then there exists sets $A^+, A^- \in \mathcal{A}$ such that
    \begin{enumerate}[(a)]
        \item $A^+ \cup A^- = X$ and $A^+ \cap A^- = \varnothing$,
        \item $A^+$ is positive and $A^-$ is negative (with respect to $\nu$).
    \end{enumerate}
\end{theorem}

\begin{proof}
    As we remarked after the definition of signed measure, $\nu$ is finite on all measurable sets.
    Therefore we let 
    \[ a := \inf \{ \nu( A ) : A \in \mathcal{A}, A\text{ is negative } \} \]
    and know that $a \in \R$ is well-defined. 
    Let $\{ A_j^- \}_{j=1}^\infty$ be a sequence of negative sets such that
    \[ \lim_{j\to\infty} \nu(A_j^-) = a. \]
    Define the set
    \[ A^- := \bigcup_{j=1}^\infty A_j^- \]
    which is measurable since $\mathcal{A}$ is a $\sigma$-algebra, and satisfies $\nu(A^-) = a$ by Proposition \ref{prop:vector_measure_properties}.(iv).
    We also define the set
    \[ A^+ := X \setminus A^- \]
    which is measurable since $\mathcal{A}$ is a $\sigma$-algebra.

    We see that $A^-$ is negative by construction.
    It remains to show that $A^+$ is positive.
    Suppose for the sake of contradiction that $A^+$ is not positive.
    Then there exists a measurable set $E_0 \subseteq A^+$ such that $\nu(E_0) < 0$.
    But then the set $E_0$ cannot be negative --- otherwise the set $A^\star := A^- \cup E_0$ would be negative with
    \[ \nu(A^{\star}) = \nu(A^-) + \nu(E_0) < \nu(A^-) = a \]
    contradicting the definition of $a$.

    Therefore there is a smallest positive integer $k_1 \geq 1$ such that there exists a measurable set $E_1 \subseteq E_0$ with
    \[\nu(E_1) > \frac{1}{k_1}.\]
    Clearly we cannot have $E_0 = E_1$ since $\nu(E_0) < 0$.
    Now by applying the same reasoning to the nonempty set $E_0 \setminus E_1$, we can find a smallest positive integer $k_2 \geq 1$ and a measurable set $E_2 \subseteq E_0 \setminus E_1$ such that
    \[ \nu(E_2) > \frac{1}{k_2}. \]
    Continuing in this manner, we obtain a sequence of measurable sets $\{ E_j \}_{j=1}^\infty$ that are pairwise disjoint and satisfy
    \[ \nu(E_j) > \frac{1}{k_j} \]
    for some positive integer $k_j \geq 1$ for each $j \geq 1$.
    We let \[ F := E_0 \setminus \bigcup_{j=1}^\infty E_j \]
    and see that $F\neq \varnothing$ since $\nu(E_0) < 0$ but $\nu(E_j) > 0$ for all $j \geq 1$.
    Also see that $F$ is negative --- see that if there were a measurable set $G \subseteq F$ with $\nu(G) > 0$, then $G$ would be a subset of $E_0$ contradicting the definition of $E_1$.
    Hence the set \[ A^{\dagger} := A^- \cup F \] is negative with
    \[ \nu(A^{\dagger}) = \nu(A^-) + \nu(F) < \nu(A^-) = a \]
    contradicting the definition of $a$.
    Therefore $A^+$ is positive as desired.
\end{proof}

\begin{remark}[Non-uniqueness of the Hahn Decomposition]
    \label{rem:hahn_decomposition_nonunique}
    In general, the Hahn decomposition is not unique.
    We can always modify a Hahn decomposition $(A^+,A^-)$ by changing $A^+$ and $A^-$ on a set of $\nu$-measure zero to obtain a different Hahn decomposition.
    In fact, this is the only way to obtain a different Hahn decomposition as will be shown in the proof of the Jordan decomposition.
\end{remark}

\begin{corollary}[Jordan Decomposition]
    \label{thm:jordan_decomposition_signed_measure}
    Let $(X,\mathcal{A})$ be a measurable space and let $\nu$ be a signed measure on $(X,\mathcal{A})$.
    Then there exist unique measures $\nu^+,\nu^- : \mathcal{A} \to [0,\infty)$ such that
    \[ \nu = \nu^+ - \nu^-. \]
\end{corollary}

Maybe this reminds you of the Jordan decomposition \ref{thm:jordan_decomposition} for functions of bounded variation on an interval.

\begin{proof}
    \textit{Step 1:} We begin by investigating the uniqueness of the Hahn decomposition.
    \vspace{2mm}

    Assume that 
    \[ X = A^+_1 \cup A^-_1 \quad\text{ and }\quad X = A^+_2 \cup A^-_2 \]
    are two Hahn decompositions of $X$ with respect to $\nu$.
    Then we claim that 
    \[ \nu(E \cap A^+_1) = \nu(E \cap A_2^+) \quad\text{ and }\quad \nu(E \cap A^-_1) = \nu(E \cap A_2^-) \tag{$\star$} \]
    for each set $E \in \mathcal{A}$.

    To see this, let $E \in \mathcal{A}$ be arbitrary.
    Then we have 
    \[ E \cap ( A_1^- \setminus A_2^- ) \subset E \cap A_1^- \]
    and at the same time 
    \[ E \cap ( A_1^- \setminus A_2^- ) \subset E \cap A_2^+ \]
    so that 
    \[ \nu(E \cap ( A_1^- \setminus A_2^- )) \leq 0 \quad \text{ and }\quad \nu(E \cap ( A_1^- \setminus A_2^- )) \geq 0 \]
    which implies that
    \[ \nu(E \cap ( A_1^- \setminus A_2^- )) = 0. \]
    A symmetric argument shows that
    \[ \nu(E \cap ( A_2^- \setminus A_1^- )) = 0. \]
    Therefore we have
    \begin{align*}
        \nu( E \cap A_1^- ) &= \nu( E \cap ( A_1^- \setminus A_2^- ) ) + \nu( E \cap ( A_1^- \cap A_2^- ) ) \\
            &= 0 + \nu( E \cap ( A_2^- \setminus A_1^- ) ) + \nu( E \cap ( A_1^- \cap A_2^- ) ) \\
            &= \nu( E \cap A_2^- )
    \end{align*}
    as claimed. 
    A similar argument shows that $\nu(E \cap A_1^+) = \nu(E \cap A_2^+)$.

    \vspace{2mm}
    \textit{Step 2:} Now we prove the existence of the Jordan decomposition.
    \vspace{2mm}

    We define the measures $\nu^+$ and $\nu^-$ by
    \[ \nu^+(A) := \nu(A \cap A^+) \quad\text{ and }\quad \nu^-(A) := -\nu(A \cap A^-) \]
    for each $A \in \mathcal{A}$, where $(A^+,A^-)$ is a Hahn decomposition of $X$ with respect to $\nu$.
    It is clear that $\nu^+$ and $\nu^-$ are well-defined, and do not depend on the choice of Hahn decomposition by $(\star)$ in Step 1.
    
    Let's check that $\nu^+$ is a measure --- the proof for $\nu^-$ is similar.
    Let $\{ A_j \}_{j=1}^\infty$ be a countable collection of pairwise disjoint sets in $\mathcal{A}$.
    Then
    \begin{align*}
        \nu^+\left( \bigcup_{j=1}^\infty A_j \right) &= \nu\left( \left( \bigcup_{j=1}^\infty A_j \right) \cap A^+ \right) \\
            &= \nu\left( \bigcup_{j=1}^\infty (A_j \cap A^+) \right) \\
            &= \sum_{j=1}^\infty \nu( A_j \cap A^+ ) \\
            &= \sum_{j=1}^\infty \nu^+( A_j )
    \end{align*}
    as desired.

    Finally, we check that $\nu = \nu^+ - \nu^-$.
    Let $A \in \mathcal{A}$ be arbitrary.
    Then
    \begin{align*}
        \nu^+(A) - \nu^-(A) &= \nu(A \cap A^+) + \nu(A \cap A^-) \\
            &= \nu(A \cap (A^+ \cup A^-)) \\
            &= \nu(A)
    \end{align*}
    as desired.
\end{proof}

\begin{definition}[Positive and Negative Variation of a Signed Measure]
    \label{def:variation_of_signed_measure}
    Let $(X,\mathcal{A})$ be a measurable space and let $\nu$ be a signed measure on $(X,\mathcal{A})$.
    The \textit{positive variation} of $\nu$ is the measure $\nu^+$ from the Jordan decomposition, and the \textit{negative variation} of $\nu$ is the measure $\nu^-$ from the Jordan decomposition.
\end{definition}

\begin{exercise}[Positive plus Negative Variation is a Measure]
    \label{ex:total_variation_is_measure}
    Let $(X,\mathcal{A})$ be a measurable space and let $\nu$ be a signed measure on $(X,\mathcal{A})$.
    Show that $\nu^+ + \nu^-$ is a measure on $(X,\mathcal{A})$.
\end{exercise}
\begin{proof}
    Let $\{ A_j \}_{j=1}^\infty$ be a countable collection of pairwise disjoint sets in $\mathcal{A}$.
    Then
    \begin{align*}
        (\nu^+ + \nu^-)\left( \bigcup_{j=1}^\infty A_j \right) &= \nu^+\left( \bigcup_{j=1}^\infty A_j \right) + \nu^-\left( \bigcup_{j=1}^\infty A_j \right) \\
            &= \sum_{j=1}^\infty \nu^+( A_j ) + \sum_{j=1}^\infty \nu^-( A_j ) \\
            &= \sum_{j=1}^\infty ( \nu^+( A_j ) + \nu^-( A_j ) ) \\
            &= \sum_{j=1}^\infty (\nu^+ + \nu^-)( A_j )
    \end{align*}
    as desired.
\end{proof}

\begin{definition}[Total Variation of a Vector Measure]
    \label{def:total_variation_vector_measure}
    Let $(X,\mathcal{A})$ be a measurable space and let $\nu$ be a vector measure on $(X,\mathcal{A})$ taking values in $\R^m$.
    The \textit{total variation} of $\nu$ is the function $|\nu| : \mathcal{A} \to [0,\infty]$ defined by
    \[ |\nu|(A) := \sup \left\{ \sum_{j=1}^n \| \nu(A_j) \| : \{ A_j \}_{j=1}^n \text{ is a finite measurable partition of } A \right\} \]
    for each $A \in \mathcal{A}$, where $\| \cdot \|$ is the standard Euclidean norm on $\R^m$.
\end{definition}

\begin{lemma}
    \label{lem:properties_of_total_variation_vector_measure}
    Let $(X,\mathcal{A})$ be a measurable space and let $\nu$ be a vector measure on $(X,\mathcal{A})$ taking values in $\R^m$.
    Then
    \begin{enumerate}[(i)]
        \item $\|\nu (A) \| \leq |\nu|(A)$ for all $A \in \mathcal{A}$,
        \item $|\nu|(A) = \nu(A)$ for all $A \in \mathcal{A}$ if and only if $\nu$ is a measure,
        \item $|\nu|(A) = 0$ if and only if $\nu(C) = 0$ for all $C \subseteq A$, $C \in \mathcal{A}$
        \item if $\nu$ is a signed measure, then 
            \[ |\nu|(A) = \sup \{ |\nu(C_1)| + |\nu(C_2)| : C_1, C_2 \text{ are disjoint measurable subsets of } A \}. \]
    \end{enumerate}
\end{lemma}

\begin{proof}
    \begin{enumerate}[(i)]
        \item Let $A \in \mathcal{A}$ be arbitrary.
            Then the partition $\{ A \}$ is a finite measurable partition of $A$, so
            \[ |\nu|(A) = \sup \left\{ \sum_{j=1}^n \| \nu(A_j) \| : \{ A_j \}_{j=1}^n \text{ is a finite measurable partition of } A \right\} \geq \| \nu(A) \|. \]

        \item Assume that $\nu$ is a measure.
            Then for each finite measurable partition $\{ A_j \}_{j=1}^n$ of $A$, we have
            \[ \sum_{j=1}^n \| \nu(A_j) \| = \sum_{j=1}^n \nu(A_j) = \nu(A) \]
            so taking the supremum over all such partitions gives $|\nu|(A) = \nu(A)$.

            Conversely, assume that $|\nu|(A) = \nu(A)$ for all $A \in \mathcal{A}$.
            Let $\{ A_j \}_{j=1}^\infty$ be a countable collection of pairwise disjoint sets in $\mathcal{A}$.
            Then for each $n\geq 1$, we have
            \[ \sum_{j=1}^n \nu(A_j) = \nu\left( \bigcup_{j=1}^n A_j \right) = |\nu|\left( \bigcup_{j=1}^n A_j \right) \leq |\nu|\left( \bigcup_{j=1}^\infty A_j \right) = \nu\left( \bigcup_{j=1}^\infty A_j \right). \]
            Taking the limit as $n\to\infty$ gives
            \[ \sum_{j=1}^\infty \nu(A_j) \leq \nu\left( \bigcup_{j=1}^\infty A_j \right). \]
            The reverse inequality follows from Proposition \ref{prop:vector_measure_properties}.(ii), so $\nu$ is a measure.

        \item Assume that $|\nu|(A) = 0$.
            Then for each measurable subset $C \subseteq A$, we have
            \[ \| \nu(C) \| \leq |\nu|(C) \leq |\nu|(A) = 0 \]
            so that $\nu(C) = 0$.

            Conversely, assume that $\nu(C) = 0$ for all measurable subsets $C \subseteq A$.
            Then for each finite measurable partition $\{ A_j \}_{j=1}^n$ of $A$, we have
            \[ \sum_{j=1}^n \| \nu(A_j) \| = 0 \]
            so taking the supremum over all such partitions gives $|\nu|(A) = 0$.

        \item Assume that $\nu$ is a signed measure.
            Let $n\in \Z^+$ and let $A_1 , A_2, \ldots, A_n \in\mathcal{A}$ be disjoint subsets of $A$.
            Let \[ C_1 := \bigcup_{ \{ k\,:\, \nu(A_k) > 0 \} } A_k \quad\text{and}\quad C_2 := \bigcup_{ \{ k\,:\, \nu(A_k) < 0 \} } A_k. \]
            Then $C_1$ and $C_2$ are disjoint measurable subsets of $A$ with
            \[ |\nu(C_1)| + |\nu(C_2)| = \sum_{k=1}^n |\nu(A_k)|. \]
            Taking the supremum over all such collections $\{ A_j \}_{j=1}^n$ gives
            \[ |\nu|(A) \leq \sup \left\{ |\nu(C_1)| + |\nu(C_2)| : C_1, C_2 \text{ are disjoint measurable subsets of } A \right\}. \]
            The reverse inequality follows trivially. 
            Therefore we have the desired equality.
    \end{enumerate}
\end{proof}

We return to our key example.

\begin{proposition}[Total Variation of Vector Measure from $L^1$ Function]
    \label{prop:total_variation_of_vector_measure_from_L1_function}
    Let $(X,\mathcal{A},\mu)$ be a measure space.
    Let $f \in L^1(X,\mu;\R^m)$ be a vector-valued $L^1$ function, and let $\nu$ be the vector measure defined by
    \[ \nu(A) = \int_A f \,\dif \mu. \]
    Then the total variation of $\nu$ is given by
    \[ |\nu|(A) = \int_A \| f \| \,\dif \mu. \]
\end{proposition}

\begin{proof}
    Let $A \in \mathcal{A}$ be arbitrary.
    First, let $A_1, A_2, \ldots, A_n$ be a disjoint collection of measurable subsets of $A$.
    Then by the triangle inequality, we have
    \[ \sum_{k=1}^n \|\nu(A_k)\| = \sum_{k=1}^n \left\| \int_{A_k} f \,\dif \mu \right\| \leq \sum_{k=1}^n \int_{A_k} \| f \| \,\dif \mu = \int_A \| f \| \,\dif \mu. \]
    Hence taking the supremum over all such collections $\{ A_k \}_{k=1}^n$ gives
    \[ |\nu|(A) \leq \int_A \| f \| \,\dif \mu. \]

    To prove the reverse inequality, suppose first that $\nu$ is a signed measure (i.e., $m=1$).
    See that $\{ x\in A : f(x) > 0 \}$ and $\{ x\in A : f(x) < 0 \}$ are disjoint measurable subsets of $A$ and that
    \[ |\nu\left( \{ x\in A : f(x) > 0 \} \right)| + |\nu\left( \{ x\in A : f(x) < 0 \} \right)| = \int_{\{x\,:\,f(x) > 0\}} |f| \,\dif \mu + \int_{\{x\,:\,f(x) < 0\}} |f| \,\dif \mu = \int_A |f| \,\dif \mu. \]
    Therefore we have
    \[ |\nu|(A) \geq \int_A |f| \,\dif \mu \]
    which gives the desired result in this case by Lemma \ref{lem:properties_of_total_variation_vector_measure}.(iv).

    Now suppose that $\nu$ is a vector measure taking values in $\R^m$ for some $m\geq 1$.
    Let $\epsilon > 0$ be arbitrary.
    Since simple functions are dense in $L^1(X,\mu;\R^m)$ (\ref{prop:l1_approximation_by_simple_functions}), we can find a simple function $s : X \to \R^m$ such That
    \[ \| f - s \|_{L^1(X,\mu;\R^m)} < \frac{\epsilon}{2}. \]
    Write $s|_A$ in the form
    \[ s|_A = \sum_{j=1}^N c_j \chi_{E_j} \]
    where $E_1, E_2, \ldots, E_N$ are disjoint measurable subsets of $A$ and $c_1, c_2, \ldots, c_N \in \R^m$.
    Then we have
    \begin{align*}
        \sum_{j=1}^N\|\nu(E_j)\| &= \sum_{j=1}^N \left\| \int_{E_j} f \,\dif \mu \right\| \\
            &\geq \sum_{j=1}^N \left\| \int_{E_j} s \,\dif \mu \right\| - \sum_{j=1}^N \left\| \int_{E_j} (f - s) \,\dif \mu \right\| \\
            &= \sum_{j=1}^N \| c_j \| \mu(E_j) - \sum_{j=1}^N \left\|\int_{E_j}  (f - s)  \,\dif \mu \right\| \\
            &= \int_A \| s \| \,\dif \mu - \left\| \int_A (f - s ) \,\dif \mu \right\| \\
            &\geq \int_A \| s \| \,\dif \mu - \int_A \| f - s \| \,\dif \mu \\
            &\geq \int_A \| s \| \,\dif \mu - \frac{\epsilon}{2} \\
            &\geq \int_A \| f \| \,\dif \mu - \int_A \| f - s \| \,\dif \mu - \frac{\epsilon}{2} \\
            &\geq \int_A \| f \| \,\dif \mu - \epsilon
    \end{align*}
    which implies that 
    \[ |\nu|(A) \geq \int_A \| f \| \,\dif \mu - \epsilon. \]
    Since $\epsilon > 0$ was arbitrary, we obtain the desired result.
\end{proof}

\begin{lemma}[Total Variation is a Measure]
    \label{lem:total_variation_is_measure}
    Let $(X,\mathcal{A})$ be a measurable space and let $\nu$ be a vector measure on $(X,\mathcal{A})$ taking values in $\R^m$.
    Then the total variation $|\nu|$ is a measure on $(X,\mathcal{A})$.
\end{lemma}

\begin{proof}
    First note that $|\nu|(\varnothing) = 0$ since the only partition of $\varnothing$ is the empty partition and $\nu(\varnothing) = 0$.
    Now let $\{ A_j \}_{j=1}^\infty$ be a countable collection of disjoint sets in $\mathcal{A}$. Fix $m\in \Z^+$.
    Then for each $1 \leq j \leq m$, let $A_{j,1}, A_{j,2}, \ldots, A_{j,n_j}$ be a finite disjoint collection of measurable subsets of $A_j$.
    Then the collection
    \[ \{ A_{j,k} : 1 \leq j \leq m, 1 \leq k \leq n_j \} \]
    is a finite disjoint collection of measurable subsets of $\bigcup_{j=1}^\infty A_j$, so we have
    \[ \sum_{j=1}^m \sum_{k=1}^{n_j} |\nu(A_{j,k})| \leq |\nu|\left( \bigcup_{j=1}^\infty A_j \right). \]
    Taking the supremum over all such collections $\{ A_{j,k} \}_{k=1}^{n_j}$ gives
    \[ \sum_{j=1}^m |\nu|(A_j) \leq |\nu|\left( \bigcup_{j=1}^\infty A_j \right). \]
    Finally, taking the limit as $m\to\infty$ gives
    \[ \sum_{j=1}^\infty |\nu|(A_j) \leq |\nu|\left( \bigcup_{j=1}^\infty A_j \right). \]

    To prove the reverse inequality, let $E_1, E_2, \ldots, E_n$ be a finite disjoint collection of measurable subsets of $\bigcup_{j=1}^\infty A_j$.
    Then 
    \begin{align*}
        \sum_{j=1}^\infty |\nu| ( A_j) &\geq \sum_{j=1}^\infty \sum_{k=1}^n |\nu( E_k \cap A_j )| \\
            &= \sum_{k=1}^n \sum_{j=1}^\infty |\nu( E_k \cap A_j )| \\
            &\geq \sum_{k=1}^n \left| \sum_{j=1}^\infty \nu( E_k \cap A_j ) \right| \\
            &= \sum_{k=1}^n |\nu( E_k )|
    \end{align*}
    where we have used the definition of $|\nu|(A_j)$ in the first inequality, the triangle inequality in the second inequality, and the countable additivity of $\nu$ in the last equality.
    Taking the supremum over all such collections $\{ E_k \}_{k=1}^n$ gives
    \[ \sum_{j=1}^\infty |\nu| ( A_j) \geq |\nu|\left( \bigcup_{j=1}^\infty A_j \right). \]
    Therefore $|\nu|$ is a measure on $(X,\mathcal{A})$ as desired.
\end{proof}

\begin{exercise}[Total Variation of Signed Measure is Positive plus Negative Variation]
    \label{ex:total_variation_signed_measure}
    Let $(X,\mathcal{A})$ be a measurable space and let $\nu$ be a signed measure on $(X,\mathcal{A})$.
    Show that the total variation of $\nu$ is given by
    \[ |\nu| = \nu^+ + \nu^- \]
    where $\nu^+$ and $\nu^-$ are the positive and negative variations of $\nu$.
\end{exercise}

\begin{proof}
    Recall from Lemma \ref{lem:properties_of_total_variation_vector_measure}.(iv) that
    \[ |\nu|(A) = \sup \{ |\nu(C_1)| + |\nu(C_2)| : C_1, C_2 \text{ are disjoint measurable subsets of } A \} \]
    for each $A \in \mathcal{A}$.

    Let \[ X = A^+ \cup A^- \]
    be a Hahn decomposition of $X$ with respect to $\nu$.
    Then by definition of $\nu^+$ and $\nu^-$, we have
    \[ \nu^+(A) = \nu(A \cap A^+) \quad \text{and} \quad \nu^-(A) = -\nu(A \cap A^-) \]
    for each $A \in \mathcal{A}$.

    Let $A \in \mathcal{A}$ be arbitrary.
    Then we see that
    \[ \nu^+(A) + \nu^-(A) = \nu(A \cap A^+) - \nu(A \cap A^-) = |\nu(A\cap A^+)| + |\nu(A\cap A^-)| \leq |\nu|(A) \]
    since $A\cap A^+$ and $A\cap A^-$ are disjoint measurable subsets of $A$.
    To prove the reverse inequality, let $\epsilon > 0$ be arbitrary.
    Then by the definition of $|\nu|(A)$, there exist disjoint measurable subsets $C_1, C_2 \subseteq A$ such that
    \[ |\nu|(A) < |\nu(C_1)| + |\nu(C_2)| + \epsilon. \]
    Now we have
    \begin{align*}
        |\nu(C_1)| &\leq |\nu(C_1\cap A^+) + \nu(C\cap A^-)| \\
            &\leq \nu(C_1 \cap A^+) - \nu(C_1 \cap A^-)
    \end{align*}
    and similarly
    \[ |\nu(C_2)| \leq \nu(C_2 \cap A^+) - \nu(C_2 \cap A^-). \]
    Therefore we have
    \begin{align*}
        |\nu|(A) &< |\nu(C_1)| + |\nu(C_2)| + \epsilon \\
            &\leq \nu(C_1 \cap A^+) - \nu(C_1 \cap A^-) + \nu(C_2 \cap A^+) - \nu(C_2 \cap A^-) + \epsilon \\
            &= \nu( (C_1 \cup C_2) \cap A^+ ) - \nu( (C_1 \cup C_2) \cap A^- ) + \epsilon \\
            &\leq \nu(A \cap A^+) - \nu(A \cap A^-) + \epsilon \\
            &= \nu^+(A) + \nu^-(A) + \epsilon
    \end{align*}
    by disjoint additivity of $\nu$ and since $C_1 \cup C_2 \subseteq A$.
    Since $\epsilon > 0$ was arbitrary, obtain
    \[ |\nu|(A) \leq \nu^+(A) + \nu^-(A). \]
    Therefore we have shown that
    \[ |\nu|(A) = \nu^+(A) + \nu^-(A) \]
    for each $A \in \mathcal{A}$ as desired.
\end{proof}

\subsection{Absolute Continuity and Lebesgue Decomposition Theorem}

\begin{definition}[Absolutely Continuous]
    \label{def:absolute_continuity_of_measures}
    Let $(X,\mathcal{A},\mu)$ be a measure space, and let $\nu$ be a vector measure on $(X,\mathcal{A})$ taking values in $\R^m$.
    We say that $\nu$ is \textit{absolutely continuous} with respect to $\mu$, denoted $\nu \ll \mu$, if for each $A \in \mathcal{A}$ with $\mu(A) = 0$, we have $\nu(A) = 0$.
    That is,
    \[ A\in \mathcal{A} \ \ \text{and} \ \ \mu(A) = 0 \implies \nu(A) = 0. \]
    Similarly, if $\nu$ is a measure on $(X,\mathcal{A})$, we say that $\nu$ is absolutely continuous with respect to $\mu$ if
    \[ A\in \mathcal{A} \ \ \text{and} \ \ \mu(A) = 0 \implies \nu(A) = 0. \]
\end{definition}
Notice that we really have to include the second definition for measures since only finite measures are vector measures taking values in $\R$.

\begin{example}
    \label{ex:absolute_continuity_of_vector_measure_from_L1_function}
    Let $(X,\mathcal{A},\mu)$ be a measure space.
    Let $f \in L^1(X,\mu;\R^m)$ be a vector-valued $L^1$ function, and let $\nu$ be the vector measure defined by
    \[ \nu(A) = \int_A f \,\dif \mu. \]
    Then $\nu$ is absolutely continuous with respect to $\mu$.

    To see this, let $A \in \mathcal{A}$ be such that $\mu(A) = 0$.
    Then we have
    \[ \| \nu(A) \| \leq |\nu|(A) = \int_A \|f\| \,\dif \mu = 0 \]
    by Proposition \ref{prop:total_variation_of_vector_measure_from_L1_function}, so that $\nu(A) = 0$ as desired.
\end{example}

\begin{exercise}[Absolute Continuity of Variations]
    \label{ex:absolute_continuity_examples}
    Let $(X,\mathcal{A})$ be a measurable space.
    \begin{enumerate}[(i)]
        \item If $\nu$ is a signed measure, then $v^+ \ll |\nu|$ and $v^- \ll |\nu|$.
        \item If $\nu$ is a vector measure taking values in $\R^m$, then each component measure $\nu_k$ is absolutely continuous with respect to $|\nu|$ and also $\nu \ll |\nu|$.
    \end{enumerate}
\end{exercise}
\begin{proof}
    \begin{enumerate}
        \item Assume $\nu$ is a signed measure.
            Let $A \in \mathcal{A}$ be such that $|\nu|(A) = 0$.
            Then by Exercise \ref{ex:total_variation_signed_measure}, we have
            \[ \nu^+(A) + \nu^-(A) = |\nu|(A) = 0 \]
            so that $\nu^+(A) = 0$ and $\nu^-(A) = 0$.
            Since $A$ was an arbitrary measurable set with $|\nu|(A) = 0$, we conclude that $\nu^+ \ll |\nu|$ and $\nu^- \ll |\nu|$ as desired.
        
        \item Assume $\nu$ is a vector measure taking values in $\R^m$.
            Let $A \in \mathcal{A}$ be such that $|\nu|(A) = 0$.
            Then by Lemma \ref{lem:properties_of_total_variation_vector_measure}.(i) we have
            \[ | \nu_k(A) | \leq \| \nu(A) \| \leq |\nu|(A) = 0 \]
            for each $1 \leq k \leq m$, so that $\nu_k(A) = 0$ for each $1 \leq k \leq m$.
            Since $A$ was an arbitrary measurable set with $|\nu|(A) = 0$, we conclude that $\nu_k \ll |\nu|$ for each $1 \leq k \leq m$.
            Finally, since $\nu(A) = ( \nu_1(A), \nu_2(A), \ldots, \nu_m(A) )$, we also have $\nu(A) = 0$.
            Therefore we have $\nu \ll |\nu|$ as desired.
    \end{enumerate}
\end{proof}

\begin{definition}[Mutually Singular]
    \label{def:mutually_singular_measures}
    Let $(X,\mathcal{A})$ be a measurable space, and let $\nu$ and $\mu$ be two countably additive set functions on $\mathcal{A}$ of the same type, i.e. both measures, both signed measures, or both vector measures taking values in $\R^m$.
    We say that $\nu$ and $\mu$ are \textit{mutually singular}, denoted $\nu \perp \mu$, if there exist disjoint measurable sets $A,B \in \mathcal{A}$ such that
    \[ X = A \cup B, \qquad\text{and}\qquad \nu(E) = \nu( E \cap A ), \quad \mu(E) = \mu( E \cap B ) \]
    for all $E \in \mathcal{A}$.
\end{definition}

That is, the measure $\nu$ only "sees" the set $A$ while the measure $\mu$ only "sees" the set $B$, and the two sets do not overlap.

\begin{exercise}[Positive and Negative Variations of Signed Measure are Mutually Singular]
    \label{ex:positive_negative_variations_mutually_singular}
    Let $(X,\mathcal{A})$ be a measurable space and let $\nu$ be a signed measure.
    Show that $\nu^+$ and $\nu^-$ from the Jordan decomposition are mutually singular.
\end{exercise}
\begin{proof}
    By the proof of the Jordan decomposition (Theorem \ref{thm:jordan_decomposition}), let \[ X = A^+ \cup A^- \] be a Hahn decomposition of $X$ with respect to $\nu$.
    Then by definition of $\nu^+$ and $\nu^-$, we have
    \[ \nu^+(E) = \nu(E \cap A^+) \quad \text{and} \quad \nu^-(E) = -\nu(E \cap A^-) \]
    for each $E \in \mathcal{A}$.
    Therefore $\nu^+$ only "sees" the set $A^+$ while $\nu^-$ only "sees" the set $A^-$, and since $A^+$ and $A^-$ are disjoint, we conclude that $\nu^+$ and $\nu^-$ are mutually singular as desired.
\end{proof}

Absolute continuity and mutual singularity are in a sense "opposites" of each other, as shown by the following lemma.
\begin{lemma}[Absolutely Continuous and Mutually Singular implies Zero Measure]
    \label{ex:absolute_continuity_and_mutually_singular_implies_zero}
    Let $(X,\mathcal{A},\mu)$ be a measure space and let $\nu$ be a vector measure on $(X,\mathcal{A})$.
    Then $\nu \ll \mu$ and $\nu \perp \mu$ implies that $\nu = 0$. 
\end{lemma}
\begin{proof}
    Assume that $\nu \ll \mu$ and $\nu \perp \mu$.
    Then by definition of mutual singularity, there exist disjoint measurable sets $A,B \in \mathcal{A}$ such that
    \[ X = A \cup B, \qquad \nu(E) = \nu( E \cap A ), \quad \mu(E) = \mu( E \cap B ) \]
    for all $E \in \mathcal{A}$.

    Let $E \in \mathcal{A}$ be arbitrary.
    Then \[ \mu(E\cap A) = \mu((E\cap A)\cap B) = \mu(\varnothing) = 0. \]
    Since $\nu \ll \mu$, we have $\nu(E\cap A) = 0$.
    Therefore we have
    \[ \nu(E) = \nu(E\cap A) = 0. \]
    Since $E$ was an arbitrary measurable set, we conclude that $\nu = 0$ as desired.
\end{proof}

\begin{theorem}[Lebesgue Decomposition Theorem]
    \label{thm:lebsegue_decomposition}
    Let $(X,\mathcal{A},\mu)$ be a measure space, and let $\nu$ be a vector measure on $(X,\mathcal{A})$ taking values in $\R^m$.
    Then there exist unique vector measures $\nu_{\text{ac}}$ and $\nu_{\text{s}}$ on $(X,\mathcal{A})$ taking values in $\R^m$ such that
    $\nu = \nu_{\text{ac}} + \nu_{\text{s}}$ and
    \[ \nu_{\text{ac}} \ll \mu \quad\text{and}\quad \nu_{\text{s}} \perp \mu. \]
    The measures $\nu_{\text{ac}}$ and $\nu_{\text{s}}$ are called the \textit{absolutely continuous} and \textit{singular} parts of $\nu$ with respect to $\mu$, respectively.
\end{theorem}

\begin{proof}
    Let \[ b := \sup\{ |\nu|(A) : A \in \mathcal{A}, \mu(A) = 0 \}. \]
    Then for each $j \in \Z^+$, we can find a measurable set $A_j \in \mathcal{A}$ such that $\mu(A_j) = 0$ and
    \[ |\nu|(A_j) > b - \frac{1}{j}. \]
    Let \[ A := \bigcup_{j=1}^\infty A_j. \]
    Then $\mu(A) = 0$ since countable unions of $\mu$ measure zero sets have $\mu$ measure zero.
    Moreover, by countable subadditivity of $|\nu|$ (Lemma \ref{lem:total_variation_is_measure}), we have
    \[ |\nu|(A) \leq \sum_{j=1}^\infty |\nu|(A_j) \leq b. \]
    On the other hand, since $A_j \subseteq A$ for each $j\in \Z^+$, we have
    \[ |\nu|(A) \geq |\nu|(A_j) > b - \frac{1}{j} \]
    for each $j\in \Z^+$, so that taking the limit as $j\to\infty$ gives $|\nu|(A) \geq b$.
    Therefore we have shown that $|\nu|(A) = b$.

    Define vector measures $\nu_{\text{ac}}$ and $\nu_{\text{s}}$ on $(X,\mathcal{A})$ by
    \[ \nu_{\text{ac}}(E) := \nu(E \cap A^c), \quad \nu_{\text{s}}(E) := \nu(E \cap A) \]
    for each $E \in \mathcal{A}$.
    Then it is clear that $\nu = \nu_{\text{ac}} + \nu_{\text{s}}$ by disjoint additivity of $\nu$.

    If $E \in \mathcal{A}$, then 
    \[ \mu(E) = \mu(E\cap A) + \mu(E\cap A^c) = 0 + \mu(E\cap A^c) = \mu(E\cap A^c) \]
    since $\mu(A) = 0$.
    This shows that $\nu_{\text{s}} \perp \mu$.

    To see that $\nu_{\text{ac}} \ll \mu$, let $E \in \mathcal{A}$ be such that $\mu(E) = 0$.
    Then $\mu(E\cup A) = 0$, so by definition of $b$ we have
    \[ b \geq |\nu|(E\cup A) = |\nu|(E) + |\nu|(A) = |\nu|(E) + b = b \]
    which forces $|\nu|(E\setminus A) = 0$.
    Therefore
    \[ \nu_{\text{ac}}(E) = \nu(E\cap A^c) = \nu(E\setminus A) = 0. \]
    Since $E$ was an arbitrary measurable set with $\mu(E) = 0$, we conclude that $\nu_{\text{ac}} \ll \mu$.

    \vspace{2mm}
    \emph{Note:} The construction of $\nu_{\text{ac}}$ and $\nu_{\text{s}}$ shows that if $\nu$ is a (finite) measure, then so are $\nu_{\text{ac}}$ and $\nu_{\text{s}}$.
    That is, if $\nu$ never takes on negative values, then neither do $\nu_{\text{ac}}$ and $\nu_{\text{s}}$.
    \vspace{2mm}

    To prove uniqueness, suppose that there exist vector measures $\tilde{\nu}_{\text{ac}}$ and $\tilde{\nu}_{\text{s}}$ on $(X,\mathcal{A})$ taking values in $\R^m$ such that
    \[ \nu = \tilde{\nu}_{\text{ac}} + \tilde{\nu}_{\text{s}} \]
    and
    \[ \tilde{\nu}_{\text{ac}} \ll \mu \quad\text{and}\quad \tilde{\nu}_{\text{s}} \perp \mu. \]
    Then we have
    \[ \nu_{\text{ac}} + \nu_{\text{s}} = \tilde{\nu}_{\text{ac}} + \tilde{\nu}_{\text{s}} \]
    which implies that
    \[ \nu_{\text{ac}} - \tilde{\nu}_{\text{ac}} = \tilde{\nu}_{\text{s}} - \nu_{\text{s}}. \]
    Since the left-hand side is absolutely continuous with respect to $\mu$ and the right-hand side is mutually singular with respect to $\mu$, we conclude from Lemma \ref{ex:absolute_continuity_and_mutually_singular_implies_zero} that
    \[ \nu_{\text{ac}} - \tilde{\nu}_{\text{ac}} = 0 \quad\text{and}\quad \tilde{\nu}_{\text{s}} - \nu_{\text{s}} = 0. \]
    Therefore we have shown that $\nu_{\text{ac}} = \tilde{\nu}_{\text{ac}}$ and $\nu_{\text{s}} = \tilde{\nu}_{\text{s}}$, completing the proof.
\end{proof}
