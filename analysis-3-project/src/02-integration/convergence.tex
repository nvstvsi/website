\section{Modes of Convergence}

\subsection{Modes of Convergence}

If one has a sequence of real or complex numbers $\{a_n\}_{n=1}^\infty$, then it is unambiguous what it means for $\{a_n\}_{n=1}^\infty$ to converge to a limit $a$ --- it means that for each $\epsilon>0$, there exists $N \in \Z^+$ such that for all $n \geq N$, we have $|a_n - a| < \epsilon$.
The same works in $\R^n$ or $\C^n$ if we just replace the absolute value with a norm.

The same cannot be said for sequences of functions though. There are many different ways for a sequence of functions $\{f_n\}_{n=1}^\infty$ to converge to a function $f$, and each has its own utility.
In this note, we define and study several different modes of convergence for sequences of functions.
We hope to do this in a unified manner for functions which take values in $\R$, $\C$, or even $[-\infty,\infty]$.

To this end, let $\F$ be either $\R$ or $\C$, and let $\F_\infty$ be either $[-\infty,\infty]$ if $\F = \R$, or $\C \cup \{\infty\}$ if $\F = \C$.

\vspace{2mm}

We now define several different modes of convergence for sequences of functions.
The first two modes are standard in analysis, and are reviewed here for completeness.
The next five modes are specific to measure theory and integration.

\begin{definition}[Modes of Convergence]
    \label{def:modes_of_convergence}
    Let $X$ be a set, and let $\{f_n\}_{n=1}^\infty$ be a sequence of functions $f_n : X \to \F$, and let $f : X \to \F$ be another function.
    \begin{enumerate}
        \item (Pointwise Convergence) We say that $\{f_n\}_{n=1}^\infty$ converges \textbf{pointwise} to $f$ if for each $x \in X$ and each $\epsilon>0$, there exists $N \in \Z^+$ (which depends on both $x$ and $\epsilon$)
          such that for all $n \geq N$, we have $|f_n(x) - f(x)| < \epsilon$.
        \item (Uniform Convergence) We say that $\{f_n\}_{n=1}^\infty$ converges \textbf{uniformly} to $f$ if for each $\epsilon>0$, there exists $N \in \Z^+$ (which depends only on $\epsilon$)
          such that for all $n \geq N$ and all $x \in X$, we have $|f_n(x) - f(x)| < \epsilon$.
    \end{enumerate}

    Now suppose that $(X,\mu)$ is a measure space, and let $\{ f_n \}_{n=1}^\infty$ be a sequence of
    measurable functions $X \to \F_\infty$ which are finite almost everywhere, and let $f$ be a measurable function $X \to \F_\infty$ which is finite almost everywhere.
    Then we define the following five modes of convergence.

    \begin{enumerate}
        \setcounter{enumi}{2}
        \item (Pointwise a.e. Convergence) We say that $\{f_n\}_{n=1}^\infty$ converges \textbf{pointwise almost everywhere} to $f$ if for $\mu$-almost every $x \in X$, we have $f_n(x) \to f(x)$ as $n \to \infty$.
        \item ($L^\infty$ Convergence) We say that $\{f_n\}_{n=1}^\infty$ converges \textbf{in $L^\infty$ norm} to $f$ if for each $\epsilon>0$, there exists $N$ (which depends only on $\epsilon$)
          such that for all $n \geq N$, we have $|f_n(x) - f(x)| < \epsilon$ for $\mu$-almost every $x \in X$.
        \item (Almost Uniform Convergence) We say that $\{f_n\}_{n=1}^\infty$ converges \textbf{almost uniformly} to $f$ if for each $\epsilon>0$, there exists a measurable set $E \subseteq X$ with $\mu(E) < \epsilon$ such that $\{f_n\}_{n=1}^\infty$ converges uniformly to $f$ on $X \setminus E$.
        \item ($L^1$ Convergence) We say that $\{f_n\}_{n=1}^\infty$ converges \textbf{in $L^1$ norm} to $f$ if
          \[ \lim_{n \to \infty} \int_X |f_n - f| \,\dif\mu = 0. \]
        \item (Convergence in Measure) We say that $\{f_n\}_{n=1}^\infty$ converges \textbf{in measure} to $f$ if for each $\epsilon>0$,
          \[ \lim_{n \to \infty} \mu(\{ x \in X : |f_n(x) - f(x)| \geq \epsilon \}) = 0. \]
    \end{enumerate}
\end{definition}

Instead of saying that $\{f_n\}_{n=1}^\infty$ is defined as a sequence of functions $X \to \F_\infty$ which are finite almost everywhere, we will often just say that $\{f_n\}_{n=1}^\infty$ is a sequence of measurable functions $X \to \F$.
Similarly for the proposed limit function $f$.
This is because we can always redefine the functions on sets of measure zero to make them finite-valued without changing any of the modes of convergence defined above, or any integrals involving these functions, or any properties of these functions that hold almost everywhere.
Our mantra is that in measure theory, we only care about what happens almost everywhere.

\vspace{2mm}

The utility of the almost everywhere modes of convergence is that we can modify functions on sets of measure zero without changing their convergence.
Pointwise and uniform convergence are too rigid for this purpose. 

\begin{proposition}[Linearity of Convergence]
    Let $(X,\mu)$ be a measure space.
    Let $\{f_n\}_{n=1}^\infty$ and $\{g_n\}_{n=1}^\infty$ be sequences of measurable functions and let $f$ and $g$ be measurable functions.

    \begin{enumerate}
        \item The sequence $\{f_n\}_{n=1}^\infty$ converges to $f$ in one of the seven modes of convergence if and only if $\{|f_n - f|\}_{n=1}^\infty$ converges to $0$ in the same mode of convergence.
        \item If $\{f_n\}_{n=1}^\infty$ converges to $f$ in one of the seven modes of convergence, and $\{g_n\}_{n=1}^\infty$ converges to $g$ in the same mode of convergence, then $\{f_n + g_n\}_{n=1}^\infty$ converges to $f + g$ in the same mode of convergence.
        \item (Squeeze Theorem.) If $\{f_n\}_{n=1}^\infty$ converges to $0$ in one of the seven modes of convergence, and $|g_n| \leq f_n$ for all $n$, then $\{g_n\}_{n=1}^\infty$ converges to $0$ in the same mode of convergence.
    \end{enumerate}
\end{proposition}

\begin{proof}
    (i). 

    \vspace{2mm}
    \textit{Pointwise Convergence.}
    Assume that $\{f_n\}_{n=1}^\infty$ converges to $f$ pointwise.
    Let $x \in X$ and let $\epsilon>0$.
    Then there exists $N \in \Z^+$ such that for all $n \geq N$, we have $|f_n(x) - f(x)| < \epsilon$.
    But this is equivalent to saying that for all $n \geq N$, we have $| |f_n(x) - f(x)| - 0 | < \epsilon$.
    Since $x$ and $\epsilon$ were arbitrary, we have shown that $\{|f_n - f|\}_{n=1}^\infty$ converges pointwise to $0$.
    The converse is similar.

    \textit{Uniform Convergence.}
    Assume that $\{f_n\}_{n=1}^\infty$ converges to $f$ uniformly.
    Let $\epsilon>0$.
    Then there exists $N \in \Z^+$ such that for all $n \geq N$ and all $x \in X$, we have $|f_n(x) - f(x)| < \epsilon$.
    But this is equivalent to saying that for all $n \geq N$ and all $x \in X$, we have $| |f_n(x) - f(x)| - 0 | < \epsilon$.
    Since $\epsilon$ was arbitrary, we have shown that $\{|f_n - f|\}_{n=1}^\infty$ converges uniformly to $0$.
    The converse is similar.

    \textit{Pointwise a.e. Convergence.}
    Assume that $\{f_n\}_{n=1}^\infty$ converges to $f$ pointwise a.e.
    Then there exists a measurable set $E \subseteq X$ with $\mu(E) = 0$ such that for each $x \in X \setminus E$, we have $f_n(x) \to f(x)$ as $n \to \infty$.
    As we have already shown, this is equivalent to saying that for each $x \in X \setminus E$, we have $|f_n(x) - f(x)| \to 0$ as $n \to \infty$.
    Since $E$ is a measurable set with $\mu(E) = 0$, we have shown that $\{|f_n - f|\}_{n=1}^\infty$ converges pointwise a.e. to $0$.
    
    Assume that $\{ | f_n - f | \}_{n=1}^\infty$ converges to $0$ pointwise a.e.
    Then there exists a measurable set $E \subseteq X$ with $\mu(E) = 0$ such that for each $x \in X \setminus E$, we have $|f_n(x) - f(x)| \to 0$ as $n \to \infty$.
    As we have already shown, this is equivalent to saying that for each $x \in X \setminus E$, we have $f_n(x) \to f(x)$ as $n \to \infty$.
    Since $E$ is a measurable set with $\mu(E) = 0$, we have shown that $\{f_n\}_{n=1}^\infty$ converges pointwise a.e. to $f$.

    \textit{$L^\infty$ Convergence.}
    Assume that $\{f_n\}_{n=1}^\infty$ converges to $f$ in $L^\infty$ norm.
    Let $\epsilon>0$.
    Then there exists $N \in \Z^+$ such that for all $n \geq N$, we have $|f_n(x) - f(x)| < \epsilon$ for $\mu$-almost every $x \in X$.
    But this is equivalent to saying that for all $n \geq N$, we have $| |f_n(x) - f(x)| - 0 | < \epsilon$ for $\mu$-almost every $x \in X$.
    Since $\epsilon$ was arbitrary, we have shown that $\{|f_n - f|\}_{n=1}^\infty$ converges to $0$ in $L^\infty$ norm.
    The converse is similar.

    \textit{Almost Uniform Convergence.}
    Assume that $\{f_n\}_{n=1}^\infty$ converges to $f$ almost uniformly.
    Let $\epsilon>0$.
    Then there exists a measurable set $E \subseteq X$ with $\mu(E)< \epsilon$ such that $\{f_n\}_{n=1}^\infty$ converges uniformly to $f$ on $X \setminus E$.
    As we have already shown, this is equivalent to saying that $\{|f_n - f|\}_{n=1}^\infty$ converges uniformly to $0$ on $X \setminus E$.
    Since $E$ is a measurable set with $\mu(E) < \epsilon$, we have shown that $\{|f_n - f|\}_{n=1}^\infty$ converges almost uniformly to $0$.
    The converse is similar.

    \textit{$L^1$ Convergence.}
    Assume that $\{f_n\}_{n=1}^\infty$ converges to $f$ in $L^1$ norm.
    Then
    \[ \lim_{n \to \infty} \int_X |f_n - f| \,\dif\mu = 0. \]
    But this is equivalent to saying that
    \[ \lim_{n \to \infty} \int_X ||f_n - f| - 0| \,\dif\mu = 0. \]
    Thus $\{|f_n - f|\}_{n=1}^\infty$ converges to $0$ in $L^1$ norm.
    The converse is similar.

    \textit{Convergence in Measure.}
    Assume that $\{f_n\}_{n=1}^\infty$ converges to $f$ in measure.
    Let $\epsilon>0$.
    Then
    \[ \lim_{n \to \infty} \mu(\{ x \in X : |f_n(x) - f(x)| \geq \epsilon \}) = 0. \]
    But this is equivalent to saying that
    \[ \lim_{n \to \infty} \mu(\{ x \in X : ||f_n(x) - f(x)| - 0| \geq \epsilon \}) = 0. \]
    Thus $\{|f_n - f|\}_{n=1}^\infty$ converges to $0$ in measure.
    The converse is similar.

    \vspace{2mm}
    (ii). For pointwise and uniform convergence, this is known, so we prove it for the other five modes.

    \vspace{2mm}
    \textit{Pointwise a.e. Convergence.}
    Assume that $\{f_n\}_{n=1}^\infty$ converges to $f$ pointwise a.e. and $\{g_n\}_{n=1}^\infty$ converges to $g$ pointwise a.e.
    Then there exist measurable sets $E_f, E_g \subseteq X$ with $\mu(E_f) = 0$ and $\mu(E_g) = 0$ such that for each $x \in X \setminus E_f$, we have $f_n(x) \to f(x)$ as $n \to \infty$, and for each $x \in X \setminus E_g$, we have $g_n(x) \to g(x)$ as $n \to \infty$.
    Let $E = E_f \cup E_g$, which is a measurable set with $\mu(E) = 0$.
    For each $x \in X \setminus E$, we have $x \in X \setminus E_f$ and $x \in X \setminus E_g$, so
    we have $f_n(x) \to f(x)$ and $g_n(x) \to g(x)$ as $n \to \infty$.
    Thus for each $x \in X \setminus E$, we have
    \[ f_n(x) + g_n(x) \to f(x) + g(x) \]
    as $n \to \infty$.
    Since $E$ is a measurable set with $\mu(E) = 0$, we have shown that $\{f_n + g_n\}_{n=1}^\infty$ converges pointwise a.e. to $f + g$.

    \textit{$L^\infty$ Convergence.}
    Assume that $\{f_n\}_{n=1}^\infty$ converges to $f$ in $L^\infty$ norm and $\{g_n\}_{n=1}^\infty$ converges to $g$ in $L^\infty$ norm.
    Then for each $\epsilon>0$, there exist $N_f, N_g \in \Z^+$ such that for all $n \geq N_f$, we have $|f_n(x) - f(x)| < \epsilon/2$ for $\mu$-almost every $x \in X$, and for all $n \geq N_g$, we have $|g_n(x) - g(x)| < \epsilon/2$ for $\mu$-almost every $x \in X$.
    Let $N = \max(N_f, N_g)$.
    Then for all $n \geq N$ and for $\mu$-almost every $x \in X$, we have
    \[ |(f_n(x) + g_n(x)) - (f(x) + g(x))| \leq |f_n(x) - f(x)| + |g_n(x) - g(x)| < \epsilon/2 + \epsilon/2 = \epsilon. \]
    Since $\epsilon>0$ was arbitrary, we have shown that $\{f_n + g_n\}_{n=1}^\infty$ converges to $f + g$ in $L^\infty$ norm.

    \textit{Almost Uniform Convergence.}
    Assume that $\{f_n\}_{n=1}^\infty$ converges to $f$ almost uniformly and $\{g_n\}_{n=1}^\infty$ converges to $g$ almost uniformly.
    Let $\epsilon>0$.
    Then there exist measurable sets $E_f, E_g \subseteq X$ with $\mu(E_f) < \epsilon/2$ and $\mu(E_g) < \epsilon/2$ such that $\{f_n\}_{n=1}^\infty$ converges uniformly to $f$ on $X \setminus E_f$, and $\{g_n\}_{n=1}^\infty$ converges uniformly to $g$ on $X \setminus E_g$.
    Let $E = E_f \cup E_g$, which is a measurable set with $\mu(E) < \epsilon$.
    Since $\{f_n\}_{n=1}^\infty$ converges uniformly to $f$ on $X \setminus E_f$ and $\{g_n\}_{n=1}^\infty$ converges uniformly to $g$ on $X \setminus E_g$, we have that $\{f_n + g_n\}_{n=1}^\infty$ converges uniformly to $f + g$ on $X \setminus E$.
    Since $\epsilon>0$ was arbitrary, we have shown that $\{f_n + g_n\}_{n=1}^\infty$ converges almost uniformly to $f + g$.

    \textit{$L^1$ Convergence.}
    Assume that $\{f_n\}_{n=1}^\infty$ converges to $f$ in $L^1$ norm and $\{g_n\}_{n=1}^\infty$ converges to $g$ in $L^1$ norm.
    Then
    \[ \lim_{n \to \infty} \int_X |f_n - f| \,\dif\mu = \lim_{n \to \infty} \int_X |g_n - g| \,\dif\mu = 0. \]
    By the triangle inequality for integrals (Proposition \ref{prop:properties_of_lebesgue_integral}), we have
    \[ \int_X |(f_n + g_n) - (f + g)| \,\dif\mu \leq \int_X |f_n - f| \,\dif\mu + \int_X |g_n - g| \,\dif\mu \to 0 \quad\text{ as } n \to \infty. \]
    Thus $\{f_n + g_n\}_{n=1}^\infty$ converges to $f + g$ in $L^1$ norm.

    \textit{Convergence in Measure.}
    Assume that $\{f_n\}_{n=1}^\infty$ converges to $f$ in measure and $\{g_n\}_{n=1}^\infty$ converges to $g$ in measure.
    Let $\epsilon>0$. See that for each $n \in \Z^+$, we have
    \[  \mu(\{ x \in X : |(f_n(x) + g_n(x)) - (f(x) + g(x))| \geq \epsilon \}) \leq \mu(\{ x \in X : |f_n(x) - f(x)| \geq \epsilon/2 \}) + \mu(\{ x \in X : |g_n(x) - g(x)| \geq \epsilon/2 \}). \]
    Taking limits as $n \to \infty$ and using the fact that $\{f_n\}_{n=1}^\infty$ converges to $f$ in measure and $\{g_n\}_{n=1}^\infty$ converges to $g$ in measure, we have
    we see that the right-hand side tends to $0$ as $n \to \infty$.
    Thus
    \[ \lim_{n \to \infty} \mu(\{ x \in X : |(f_n(x) + g_n(x)) - (f(x) + g(x))| \geq \epsilon \}) = 0. \]
    Since $\epsilon>0$ was arbitrary, we have shown that $\{f_n + g_n\}_{n=1}^\infty$ converges to $f + g$ in measure.

    \vspace{2mm}
    (iii). 
    \vspace{2mm}

    \textit{Pointwise Convergence.}
    Assume that $\{f_n\}_{n=1}^\infty$ converges to $0$ pointwise and $|g_n| \leq f_n$ for all $n$.
    Let $x \in X$ and let $\epsilon>0$.
    Then there exists $N \in \Z^+$ such that for all $n \geq N$, we have $|f_n(x)| < \epsilon$.
    Since $|g_n(x)| \leq |f_n(x)|$, we have $|g_n(x)| < \epsilon$ for all $n \geq N$.
    Since $x$ and $\epsilon$ were arbitrary, we have shown that $\{g_n\}_{n=1}^\infty$ converges pointwise to $0$.

    \textit{Uniform Convergence.}
    Assume that $\{f_n\}_{n=1}^\infty$ converges to $0$ uniformly and $|g_n| \leq f_n$ for all $n$.
    Let $\epsilon>0$.
    Then there exists $N \in \Z^+$ such that for all $n \geq N$ and all $x \in X$, we have $|f_n(x)| < \epsilon$.
    Since $|g_n(x)| \leq |f_n(x)|$, we have $|g_n(x)| < \epsilon$ for all $n \geq N$ and all $x \in X$.
    Since $\epsilon$ was arbitrary, we have shown that $\{g_n\}_{n=1}^\infty$ converges uniformly to $0$.

    \textit{Pointwise a.e. Convergence.}
    Assume that $\{f_n\}_{n=1}^\infty$ converges to $0$ pointwise a.e. and $|g_n| \leq f_n$ for all $n$ and almost everywhere on $X$.
    Then there exists a measurable set $E_f \subseteq X$ with $\mu(E_f) = 0$ such that for each $x \in X \setminus E_f$, we have $f_n(x) \to 0$ as $n \to \infty$.
    There is also a measurable set $E_g \subseteq X$ with $\mu(E_g) = 0$ such that for each $n$, we have $|g_n(x)| \leq f_n(x)$ for each $x \in X \setminus E_g$.
    Let $E = E_f \cup E_g$, which is a measurable set with $\mu(E) = 0$.
    For each $x \in X \setminus E$, we have $x \in X \setminus E_f$ and $x \in X \setminus E_g$, so
    we have $f_n(x) \to 0$ as $n \to \infty$ and $|g_n(x)| \leq f_n(x)$ for all $n$.
    Thus we have $|g_n(x)| \to 0$ as $n \to \infty$ by the case for pointwise convergence.
    Since $E$ is a measurable set with $\mu(E) = 0$, we have shown that $\{g_n\}_{n=1}^\infty$ converges pointwise a.e. to $0$.

    \textit{$L^\infty$ Convergence.}
    Assume that $\{f_n\}_{n=1}^\infty$ converges to $0$ in $L^\infty$ norm and $|g_n| \leq f_n$ for all $n$ and almost everywhere on $X$.
    Let $\epsilon>0$.
    Then there exists $N \in \Z^+$ such that for all $n \geq N$, we have $|f_n(x)| < \epsilon$ for $\mu$-almost every $x \in X$.
    There is also a measurable set $E_g \subseteq X$ with $\mu(E_g) = 0$ such that for each $n$, we have $|g_n(x)| \leq f_n(x)$ for each $x \in X \setminus E_g$.
    Let $E = E_g$, which is a measurable set with $\mu(E) = 0$.
    For all $n \geq N$ and for $\mu$-almost every $x \in X$, we have
    $x \in X \setminus E$, so we have $|g_n(x)| \leq |f_n(x)| < \epsilon$.
    Since $\epsilon>0$ was arbitrary, we have shown that $\{g_n\}_{n=1}^\infty$ converges to $0$ in $L^\infty$ norm.

    \textit{Almost Uniform Convergence.}
    Assume that $\{f_n\}_{n=1}^\infty$ converges to $0$ almost uniformly and $|g_n| \leq f_n$ for all $n$ and almost everywhere on $X$.
    Let $\epsilon>0$.
    Then there exists a measurable set $E_f \subseteq X$ with $\mu(E_f) < \epsilon$ such that $\{f_n\}_{n=1}^\infty$ converges uniformly to $0$ on $X \setminus E_f$.
    There is also a measurable set $E_g \subseteq X$ with $\mu(E_g) = 0$ such that for each $n$, we have $|g_n(x)| \leq f_n(x)$ for each $x \in X \setminus E_g$.
    Let $E = E_f \cup E_g$, which is a measurable set with $\mu(E) < \epsilon$.
    Since $\{f_n\}_{n=1}^\infty$ converges uniformly to $0$ on $X \setminus E_f$ and $|g_n(x)| \leq |f_n(x)|$ for each $x \in X \setminus E_g$, we have that $\{g_n\}_{n=1}^\infty$ converges uniformly to $0$ on $X \setminus E$.
    Since $\epsilon>0$ was arbitrary, we have shown that $\{g_n\}_{n=1}^\infty$ converges almost uniformly to $0$.

    \textit{$L^1$ Convergence.}
    Assume that $\{f_n\}_{n=1}^\infty$ converges to $0$ in $L^1$ norm and $|g_n| \leq f_n$ for all $n$ and almost everywhere on $X$.
    Then
    \[ \lim_{n \to \infty} \int_X |f_n| \,\dif\mu = 0. \]
    Then monotonicity of the Lebesgue integral (Proposition \ref{prop:properties_of_lebesgue_integral}) implies that
    \[ \int_X |g_n| \,\dif\mu \leq \int_X |f_n| \,\dif\mu \to 0 \quad\text{ as } n \to \infty. \]
    Thus $\{g_n\}_{n=1}^\infty$ converges to $0$ in $L^1$ norm.

    \textit{Convergence in Measure.}
    Assume that $\{f_n\}_{n=1}^\infty$ converges to $0$ in measure and $|g_n| \leq f_n$ for all $n$ and almost everywhere on $X$.
    Let $\epsilon>0$. See that for each $n \in \Z^+$, we have
    \[  \mu(\{ x \in X : |g_n(x)| \geq \epsilon \}) \leq \mu(\{ x \in X : |f_n(x)| \geq \epsilon \}). \]
    Taking limits as $n \to \infty$ and using the fact that $\{f_n\}_{n=1}^\infty$ converges to $0$ in measure, we have
    we see that the right-hand side tends to $0$ as $n \to \infty$.
    Thus
    \[ \lim_{n \to \infty} \mu(\{ x \in X : |g_n(x)| \geq \epsilon \}) = 0. \]
    Since $\epsilon>0$ was arbitrary, we have shown that $\{g_n\}_{n=1}^\infty$ converges to $0$ in measure.
\end{proof}

\begin{proposition}[Easy Implications]
    Let $(X,\mu)$ be a measure space.
    Let $\{f_n\}_{n=1}^\infty$ and let $f$ be measurable functions.
    Then the following implications hold:
    \begin{enumerate}
        \item Uniform convergence $\implies$ pointwise convergence.
        \item Uniform convergence $\implies$ $L^\infty$ convergence.
        \item $L^\infty$ convergence $\implies$ almost uniform convergence.
        \item Almost uniform convergence $\implies$ pointwise a.e. convergence.
        \item Pointwise convergence $\implies$ pointwise a.e. convergence.
        \item $L^1$ convergence $\implies$ convergence in measure.
        \item Almost uniform convergence $\implies$ convergence in measure.
    \end{enumerate}
\end{proposition}

\begin{figure}
    \centering
    \includegraphics[width=0.6\textwidth]{figures/modes_of_1.png}
\end{figure}

\begin{proof}
    (1). Suppose that $\{f_n\}_{n=1}^\infty$ converges uniformly to $f$.
    Then for each $\epsilon>0$, there exists $N \in \Z^+$ such that for all $n \geq N$ and all $x \in X$, we have $|f_n(x) - f(x)| < \epsilon$.
    Let $x \in X$ and let $\epsilon>0$.
    Then for all $n \geq N$, we have $|f_n(x) - f(x)| < \epsilon$.
    Since $x$ and $\epsilon$ were arbitrary, we have shown that $\{f_n\}_{n=1}^\infty$ converges pointwise to $f$.

    \vspace{2mm}
    (2). Suppose that $\{f_n\}_{n=1}^\infty$ converges uniformly to $f$.
    Then for each $\epsilon>0$, there exists $N \in \Z^+$ such that for all $n \geq N$ and all $x \in X$, we have $|f_n(x) - f(x)| < \epsilon$.
    Since this holds for all $x \in X$, it holds for $\mu$-almost every $x \in X$.
    Since $\epsilon>0$ was arbitrary, we have shown that $\{f_n\}_{n=1}^\infty$ converges to $f$ in $L^\infty$ norm.

    \vspace{2mm}
    (3). Suppose that $\{f_n\}_{n=1}^\infty$ converges to $f$ in $L^\infty$ norm.
    Let $\epsilon>0$.
    Then there exists $N \in \Z^+$ such that for all $n \geq N$, we have $|f_n(x) - f(x)| < \epsilon$ for $\mu$-almost every $x \in X$.
    Let $E = \{ x \in X : |f_N(x) - f(x)| \geq \epsilon \}$.
    Then $E$ is a measurable set with $\mu(E) = 0$.
    For all $n \geq N$ and all $x \in X \setminus E$, we have $|f_n(x) - f(x)| < \epsilon$.
    Since $\epsilon>0$ was arbitrary, we have shown that $\{f_n\}_{n=1}^\infty$ converges almost uniformly to $f$.

    \vspace{2mm}
    (4). Suppose that $\{f_n\}_{n=1}^\infty$ converges almost uniformly to $f$.
    Then for each $\epsilon>0$, there exists a measurable set $E_\epsilon \subseteq X$ with $\mu(E_\epsilon) < \epsilon$ such that $\{f_n\}_{n=1}^\infty$ converges uniformly to $f$ on $X \setminus E_\epsilon$.
    Let $E = \bigcap_{m=1}^\infty E_{1/m}$.
    Then $E$ is a measurable set with $\mu(E) = 0$ by countable subadditivity.
    For each $x \in X \setminus E$, we have $x \in X \setminus E_{1/m}$ for some $m$, and since $\{f_n\}_{n=1}^\infty$ converges uniformly to $f$ on $X \setminus E_{1/m}$, we have $f_n(x) \to f(x)$ as $n \to \infty$.
    Thus $\{f_n\}_{n=1}^\infty$ converges pointwise a.e. to $f$.

    \vspace{2mm}
    (5). Suppose that $\{f_n\}_{n=1}^\infty$ converges pointwise to $f$.
    Then it is trivially true that $\{f_n\}_{n=1}^\infty$ converges pointwise a.e. to $f$.

    \vspace{2mm}
    (6). Suppose that $\{f_n\}_{n=1}^\infty$ converges to $f$ in $L^1$ norm.
    Let $\epsilon>0$. By Chebyshev's Inequality (Lemma \ref{lem:chebyshevs_inequality}), we have
    \[ \mu(\{ x \in X : |f_n(x) - f(x)| \geq \epsilon \}) \leq \frac{1}{\epsilon} \int_X |f_n - f| \,\dif\mu. \]
    Since $\{f_n\}_{n=1}^\infty$ converges to $f$ in $L^1$ norm, the right-hand side tends to $0$ as $n \to \infty$.
    Thus $\mu(\{|f_n-f| \geq \epsilon\}) \to 0$ as $n \to \infty$, so $\{f_n\}_{n=1}^\infty$ converges to $f$ in measure.

    \vspace{2mm}
    (7). Suppose that $\{f_n\}_{n=1}^\infty$ converges almost uniformly to $f$.
    Let $\epsilon>0$ and $\delta>0$. By $L^\infty$ convergence, there exists a measurable set $E \subseteq X$ with $\mu(E) < \delta$ such that $\{f_n\}_{n=1}^\infty$ converges uniformly to $f$ on $X \setminus E$.
    Thus there exists $N \in \Z^+$ such that for all $n \geq N$ and all $x \in X \setminus E$, we have 
    $|f_n(x) - f(x)| < \epsilon$.
    Thus for all $n \geq N$, we have
    \[ \{ x \in X : |f_n(x) - f(x)| \geq \epsilon \} \subseteq E, \]
    so $\mu(\{|f_n - f| \geq \epsilon\}) \leq \mu(E) < \delta$ for all $n \geq N$.
    Since $\delta>0$ was arbitrary, we have shown that $\mu(\{|f_n - f| \geq \epsilon\}) \to 0$ as $n \to \infty$, so $\{f_n\}_{n=1}^\infty$ converges to $f$ in measure.
\end{proof}

\begin{remark}[Extra Implications in Finite Measure Spaces]
    \label{rem:extra_implications_in_finite_measure_spaces}
    We remark that if the measure of the whole space is finite, then uniform convergence also implies $L^1$ convergence, and $L^\infty$ convergence also implies $L^1$ convergence.
    Also pointwise almost everywhere convergence implies almost uniform convergence by Egorov's Theorem (Theorem \ref{thm:egorovs_theorem}).

    \begin{figure}
        \centering
        \includegraphics[width=0.5\textwidth]{figures/modes_of_2.png}
    \end{figure}
    We prove these claims below.
    
    Let $(X,\mu)$ be a measure space with $\mu(X) < \infty$.
    Let $\{f_n\}_{n=1}^\infty$ and let $f$ be measurable functions such that $\{f_n\}_{n=1}^\infty$ converges to $f$ uniformly or in $L^\infty$ norm.
    Then we see that
    \[ \int_X |f_n - f| \, d\mu \leq \mu(X) \|f_n - f\|_\infty \]
    for each $n \in \Z^+$, and taking limits as $n \to \infty$ and using the uniform convergence or $L^\infty$ convergence of $\{f_n\}_{n=1}^\infty$ to $f$, we have
    \[ \lim_{n \to \infty} \int_X |f_n - f| \, d\mu = 0. \]
    Thus $\{f_n\}_{n=1}^\infty$ converges to $f$ in $L^1$ norm.

    Now suppose that $\{f_n\}_{n=1}^\infty$ converges to $f$ pointwise a.e.
    Then by Egorov's Theorem (Theorem \ref{thm:egorovs_theorem}), for each $\epsilon>0$, there exists a measurable set $E_\epsilon \subseteq X$ with $\mu(E_\epsilon) < \epsilon$ such that $\{f_n\}_{n=1}^\infty$ converges uniformly to $f$ on $X \setminus E_\epsilon$.
    Since $\epsilon>0$ was arbitrary, we have shown that $\{f_n\}_{n=1}^\infty$ converges almost uniformly to $f$.
    This completes the proof of the claims.
    Note that these implications do not hold in general if $\mu(X) = \infty$.
\end{remark}


\begin{example}[Escape to Horizontal Infinity]
    
\end{example}

\begin{example}[Escape to Width Infinity]
    
\end{example}

\begin{example}[Escape to Vertical Infinity]
    
\end{example}

\begin{example}[Typewriter Sequence]
    
\end{example}

\subsection{Uniqueness of Limits}
Even though the modes of convergence all differ from each other, they are all compatible in the sense that they never disagree about which function $f$ is the limit of a sequence $\{f_n\}_{n=1}^\infty$, 
outside of a set of measure zero.

\begin{theorem}[Uniqueness of Limits]
    Let $(X,\mu)$ be a measure space.
    Let $\{f_n\}_{n=1}^\infty$ be a sequence of measurable functions and let $f$ and $g$ be measurable functions.
    \label{thm:uniqueness_of_limits}
    If $\{f_n\}_{n=1}^\infty$ converges to $f$ in one of the seven modes of convergence, and $\{f_n\}_{n=1}^\infty$ converges to $g$ in another of the seven modes of convergence,
    then $f = g$ almost everywhere.
\end{theorem}



height width tail 

\subsection{The Case of Step Functions}

height width tail







\subsection{Topologies of Convergence}

Ok so as analysts, we are mainly interested in limits of sequences, but occasionally we want to talk topology.
It is useful after all. 

By definition, for each $1\leq p \leq \infty$, convergence in $L^p$ norm is the convergence on $L^p(X,\mu)$ induced by the topology of the $L^p$ norm.
Pointwise convergence is by definition the convergence induced by the product topology on $[-\infty,\infty]^X$.
Prove this last sentence? 

There is also a topology for uniform convergence, the uniform norm topology.

The others are more mysterious.

There is no topology for pointwise a.e. convergence.

is there a topology for almost uniform convergence? 

convergence in measure
see Folland exercise

\subsection{Variants of Convergence Theorems}
