\section{Inequality Party}

Get ready to party.

\subsection{H\"older's Inequality}

\begin{lemma}[Young's Inequality]
    \label{lem:youngs_inequality}
    Let $p,q \in (1,\infty)$ be such that $\frac{1}{p} + \frac{1}{q} = 1$.
    Then for all $a,b \geq 0$, we have
    \[ ab \leq \frac{a^p}{p} + \frac{b^q}{q}, \]
    with equality if and only if $a^p = b^q$.
\end{lemma}
\begin{proof}
    If $a=0$ or $b=0$, then the result is trivial, as the left side is zero and the right side is nonnegative; the right side is zero if and only if $a=b=0$, which agrees with the equality condition.

    Fix $b>0$. Define the function $f : (0,\infty) \to \R$ by
    \[ f(a) = \frac{a^p}{p} + \frac{b^q}{q} - ab. \]
    Then $f$ is differentiable and \[ f'(a) = a^{p-1} - b \qquad\forall a>0. \]

    See that $f'(a) = 0$ if and only if $a^{p-1} = b$, or equivalently, $a^p = b^q$.
    Since \[f''(a) = (p-1)a^{p-2} > 0, \qquad \forall a>0\] this critical point is a global minimum.
    Thus, for all $a>0$, we have
    \[ f(a) \geq f(b^{1/(p-1)}) = \frac{b^{p/(p-1)}}{p} + \frac{b^q}{q} - b^{(p-1)/(p-1)}b = \frac{b^q}{p} + \frac{b^q}{q} - b^q = 0 \]
    which is equivalent to
    \[ \frac{a^p}{p} + \frac{b^q}{q}- ab \geq 0 \]
    with equality if and only if $a^p = b^q$.

    Since $b>0$ was arbitrary, the result follows.
\end{proof}

\begin{proposition}[H\"older's Inequality]
    \label{prop:holders_inequality_1}
    Let $(X,\mu)$ be a measure space, and let $p,q \in [1,\infty]$ be such that $\frac{1}{p} + \frac{1}{q} = 1$.
    If $f \in L^p(X,\mu)$ and $g \in L^q(X,\mu)$, then $fg \in L^1(X,\mu)$ and
    \[ \|fg\|_{L^1} \leq \|f\|_{L^p} \|g\|_{L^q} \]
    with equality if and only if there are constants $c_1,c_2\geq 0$, not both zero, such that $c_1|f|^p = c_2|g|^q$ almost everywhere.
\end{proposition}
\begin{proof}
    \textit{Step 1:} Suppose first that $1 < p,q < \infty$.

    We will prove the result in the special case that both functions have unit norm.
    Suppose that $f \in L^p(X,\mu)$ and $g \in L^q(X,\mu)$ are such that $\|f\|_{L^p} = \|g\|_{L^q} = 1$.
    Then by \ref{lem:youngs_inequality}, we have
    \[ |f(x)g(x)| \leq \frac{|f(x)|^p}{p} + \frac{|g(x)|^q}{q} \qquad\text{for each } x \in X. \]
    By monotonicity of the integral, we have
    \begin{align*}
        \|fg\|_{L^1} = \int_X |f(x)g(x)| \,\dif\mu(x) &\leq \int_X \left( \frac{|f(x)|^p}{p} + \frac{|g(x)|^q}{q} \right) \dif\mu(x) \\
        &= \frac{1}{p} \int_X |f(x)|^p \,\dif\mu(x) + \frac{1}{q} \int_X |g(x)|^q \,\dif\mu(x) \\
        &= \frac{1}{p}\|f\|_{L^p}^p + \frac{1}{q}\|g\|_{L^q}^q = \frac{1}{p} + \frac{1}{q} = 1.
    \end{align*}
    Thus, in this special case, we have $\|fg\|_{L^1} \leq 1 = \|f\|_{L^p}\|g\|_{L^q}$.
    Moreover, equality holds if and only if $|f(x)|^p = |g(x)|^q$ for $\mu$-almost every $x \in X$.
    This proves the result in the special case that both functions have unit norm.

    Now, let $f \in L^p(X,\mu)$ and $g \in L^q(X,\mu)$ be arbitrary.
    If either $f$ or $g$ equals zero $\mu$-almost everywhere, then the result is trivial as both sides of the inequality are zero, so suppose that neither function is zero $\mu$-almost everywhere.
    Then $\|f\|_{L^p}, \|g\|_{L^q} > 0$, and so the functions $\frac{f}{\|f\|_{L^p}}$ and $\frac{g}{\|g\|_{L^q}}$ both have unit norm in their respective spaces.
    By the special case already proven, we have
    \[ \left\| \frac{f}{\|f\|_{L^p}} \frac{g}{\|g\|_{L^q}} \right\|_{L^1} \leq 1, \]
    which is equivalent to
    \[ \|fg\|_{L^1} \leq \|f\|_{L^p} \|g\|_{L^q} \]
    by homogeneity of the $L^1$ norm.
    Moreover, equality holds if and only if there is a constant $c>0$ such that $|f|^p = c|g|^q$ $\mu$-almost everywhere.
    This proves the result in the case that $1 < p,q < \infty$.

    \textit{Step 2:} Now, suppose that $p=1$ and $q=\infty$ (the case $p=\infty$ and $q=1$ is symmetric).
    Then we recall that the $L^\infty$-norm is defined as the essential supremum,
    \[ \|g\|_{L^\infty} := \esssup_{x \in X} |g(x)| := \inf\{ C : |g(x)| \leq C \text{ for } \mu\text{-almost every } x \in X \} \qquad\forall g \in L^\infty(X,\mu). \]

    Let $f\in L^1(X,\mu)$ and $g \in L^\infty(X,\mu)$ be arbitrary. Then 
    \begin{align*}
        \|fg\|_{L^1} = \int_X |f(x)g(x)| \,\dif\mu(x) &\leq \int_X |f(x)| \|g\|_{L^\infty} \,\dif\mu(x) \\
            &= \|g\|_{L^\infty} \int_X |f(x)| \,\dif\mu(x) = \|f\|_{L^1} \|g\|_{L^\infty}
    \end{align*}
    by monotonicity and homogeneity of the integral.
    Moreover, equality holds if and only if there is a constant $c \geq 0$ such that $|f| = c|g|$ $\mu$-almost everywhere.
    This proves the result in the case that $p=1$ and $q=\infty$.
\end{proof}

\begin{proposition}[Reverse H\"older Inequality]
    \label{prop:reverse_holders_inequality}
    Let $(X,\mu)$ be a measure space, and let $p<1$ be nonzero, and let $q$ be such that $\frac{1}{p} + \frac{1}{q} = 1$.
    Also suppose that $\mu(X)>0$.
    If $f,g: X\to \C$ are measurable functions such that $g\neq 0$ $\mu$-almost everywhere, $fg\in L^1(X,\mu)$, and $\int_X |g|^q \,\dif\mu < \infty$, then
    \[ \|fg\|_{L^1} \geq \left( \int_X |f|^p \,\dif\mu \right)^{\frac{1}{p}} \left( \int_X |g|^q \,\dif\mu \right)^{\frac{1}{q}}. \]
\end{proposition}
\begin{proof}
    Notice that 
    \[ 0<p<1 \iff q <0 \qquad\text{and}\qquad p<0 \iff 0 <q <1. \]
    Thus by symmetry we may assume without loss of generality that $0<p<1$ and $q<0$.

    Now let $f,g : X \to \C$ be measurable functions such that $g\neq 0$ $\mu$-almost everywhere,
    \[ \int_X |fg| \,\dif\mu < \infty \quad\text{ and }\quad \int_X |g|^q \,\dif\mu < \infty. \]
    Define $r:= \frac{1}{p} > 1$ and $r' := \frac{r}{r-1} > 1$ so that we have 
    \[ \frac{1}{r} + \frac{1}{r'} = 1. \]

    We estimate 
    \begin{align*}
        \| |fg|^{1/r} \|_{L^r} &= \left( \int_X |fg|^{1/r \cdot r} \,\dif\mu \right)^{1/r} \\
            &= \left( \int_X |fg| \,\dif\mu \right)^{1/r} = \left( \int_X |fg| \,\dif\mu \right)^p = \|fg\|_{L^1}^p < \infty 
    \end{align*}
    and 
    \begin{align*}
        \| |g|^{-1/r} \|_{L^{r'}} &= \left( \int_X |g|^{-1/r \cdot r'} \,\dif\mu \right)^{1/r'} \\
            &= \left( \int_X |g|^{-\frac{1}{r-1}} \,\dif\mu \right)^{1/r'} = \left( \int_X |g|^q \,\dif\mu \right)^{1-p} < \infty.
    \end{align*}
    by using our assumptions on $f$ and $g$.
    From this we can use H\"older's inequality (\ref{holders_inequality_1}) to estimate
    \begin{align*}
        \int_X |f|^p \,\dif\mu &= \int_X |fg|^{p} |g|^{-p} \,\dif\mu = \int_X |fg|^{1/r} |g|^{-1/r} \,\dif\mu \\
            &= \| |fg|^{1/r} |g|^{-1/r} \|_{L^1} \\
            &\leq \| |fg|^{1/r} \|_{L^r} \| |g|^{-1/r} \|_{L^{r'}} \\
            &= \|fg\|_{L^1}^p \left( \int_X |g|^q \,\dif\mu \right)^{1-p}.
    \end{align*}
    Because $\mu(X)>0$ and $g\neq 0$ $\mu$-almost everywhere, we have $\int_X |g|^q \,\dif\mu > 0$, and so we can divide both sides by $\left( \int_X |g|^q \,\dif\mu \right)^{1-p}$ to obtain
    \[ \int_X |f|^p \,\dif\mu \left( \int_X |g|^q \,\dif\mu \right)^{-(1-p)} \leq \|fg\|_{L^1}^p \]
    which is equivalent to
    \[ \left(\int_X |f|^p \,\dif\mu\right)^{1/p} \left( \int_X |g|^q \,\dif\mu \right)^{\frac{p-1}{p}} \leq \|fg\|_{L^1} \]
    by taking $p$-th roots of both sides.
    Since $\frac{p-1}{p} = \frac{1}{q}$, the desired result follows.
\end{proof}

\begin{proposition}
    \label{cor:holders_inequality_3}
    Let $(X,\mu)$ be a measure space, and let $0\lt r,p,q\leq \infty$ be such that
    \[ \frac{1}{r} = \frac{1}{p} + \frac{1}{q}. \]
    If $f \in L^p(X,\mu)$ and $g \in L^q(X,\mu)$, then $fg \in L^r(X,\mu)$ and
    \[ \|fg\|_{L^r} \leq \|f\|_{L^p} \|g\|_{L^q}. \]
\end{proposition}

\begin{proof}
    \textit{Case 1:} First suppose that $r=\infty$.
    Then 
    \[ \frac{1}{p} + \frac{1}{q} = 0 \]
    so we must have $p=q=\infty$. Thus if $f,g \in L^\infty(X,\mu)$, then we check that 
    \begin{align*}
        \|fg\|_{L^\infty} &= \esssup_{x\in X} |f(x)g(x)| \\
            &\leq \esssup_{x\in X} |f(x)| \cdot \esssup_{x\in X} |g(x)| = \|f\|_{L^\infty} \|g\|_{L^\infty}.
    \end{align*}
    This proves the result in the case that $r=\infty$.

    \vspace{3mm}
    \textit{Case 2:} Now suppose that $0\lt r<\infty$.
    Then we consider two subcases.
    \[ \text{a)} \quad q < \infty \text{ and } p = \infty \qquad\qquad \text{b)} \quad p,q<\infty. \]
    Note that both $p$ and $q$ cannot be infinite, as this would imply that $r=\infty$, which is covered in Case 1.
    \vspace{3mm}
    \textit{Subcase 2a:} Suppose that $q<\infty$ and $p=\infty$.
    Then we have \[ \frac{1}{r} = \frac{1}{\infty} + \frac{1}{q} = 0 + \frac{1}{q} = \frac{1}{q} \]
    so that $r=q$. 
    Let $f \in L^q(X,\mu)$ and $g \in L^\infty(X,\mu)$ be arbitrary.
    Then we estimate
    \begin{align*}
        \|fg\|_{L^r}^r = \int_X |f(x)g(x)|^r \,\dif\mu(x) &= \int_X |f(x)|^r |g(x)|^r \,\dif\mu(x) \\
            &\leq \int_X |f(x)|^r \|g\|_{L^\infty}^r \,\dif\mu(x) = \|g\|_{L^\infty}^r \|f\|_{L^r}^r.
    \end{align*}
    By taking $r$-th roots and using that $r=q$, we obtain
    \[ \|fg\|_{L^r} \leq \|f\|_{L^r} \|g\|_{L^\infty}. \]

    \vspace{3mm}
    \textit{Subcase 2b:} Now suppose that $p,q<\infty$.
    Then $\frac{p}{r}, \frac{q}{r} > 1$ and \[ \frac{1}{p/r} + \frac{1}{q/r} = 1 \]
    so we can apply H\"older's inequality (\ref{holders_inequality_1}) with exponents $\frac{p}{r}$ and $\frac{q}{r}$.

    Let $f \in L^p(X,\mu)$ and $g \in L^q(X,\mu)$ be arbitrary.
    Then $|f|^r \in L^{p/r}(X,\mu)$ and $|g|^r \in L^{q/r}(X,\mu)$, so we can use H\"older's inequality to estimate
    \begin{align*}
        \|fg\|_{L^r}^r &= \| |f|^r |g|^r \|_{L^1} \\
            &\leq \| |f|^r \|_{L^{p/r}} \| |g|^r \|_{L^{q/r}} \\
            &= \left( \int_X |f(x)|^{r \cdot p/r} \,\dif\mu(x) \right)^{r/p} \left( \int_X |g(x)|^{r \cdot q/r} \,\dif\mu(x) \right)^{r/q} \\
            &= \|f\|_{L^p}^r \|g\|_{L^q}^r.
    \end{align*}
    By taking $r$-th roots of both sides, we obtain the desired result.
\end{proof}

\begin{theorem}[Generalized Holder's Inequality]
    \label{thm:generalized_holders_inequality}
    Let $(X,\mu)$ be a measure space, and let $k \in \Z^+$.
    Also let $0\lt p,p_1,\ldots,p_k \leq \infty$ be such that 
    \[ \frac{1}{p} = \sum_{j=1}^k \frac{1}{p_j}. \]
    If $f_j \in L^{p_j}(X,\mu)$ for each $j=1,\ldots,k$, then $\prod_{j=1}^k f_j \in L^p(X,\mu)$ and
    \[ \left\| \prod_{j=1}^k f_j \right\|_{L^p} \leq \prod_{j=1}^k \|f_j\|_{L^{p_j}}. \]
\end{theorem}

\begin{proof}
    We proceed by induction on $k\in \Z^+$.
    The base case $k=1$ is trivial --- if $0< p,p_1 \leq \infty$ are such that $\frac{1}{p} = \frac{1}{p_1}$, then $p=p_1$, and so if $f_1 \in L^{p_1}(X,\mu)$, then $f_1 \in L^p(X,\mu)$ and $\|f_1\|_{L^p} = \|f_1\|_{L^{p_1}}$.

    Now suppose that the result holds for some $k \in \Z^+$, and let $0< p,p_1,\ldots,p_{k+1} \leq \infty$ be such that
    \[ \frac{1}{p} = \sum_{j=1}^{k+1} \frac{1}{p_j}. \]
    Define \[ q:= \left( \sum_{j=1}^{k+1} \frac{1}{p_j} \right)^{-1} \]
    so that $0< q\leq \infty$ and 
    \[ \frac{1}{p} = \frac{1}{q} + \frac{1}{p_{k+1}}. \]
    Let $f_j \in L^{p_j}(X,\mu)$ for each $j=1,\ldots,k+1$ be arbitrary.
    Then by the induction hypothesis, we have $ \prod_{j=1}^k f_j \in L^q(X,\mu) $, so the previous proposition (\ref{cor:holders_inequality_3}) implies that
    \begin{align*}
        \left\| \left( \prod_{j=1}^k f_j\right) \cdot f_{k+1} \right\| &\leq \left\| \prod_{j=1}^k f_j \right\|_{L^q} \|f_{k+1}\|_{L^{p_{k+1}}} \\
            &\leq \left( \prod_{j=1}^k \|f_j\|_{L^{p_j}} \right) \|f_{k+1}\|_{L^{p_{k+1}}} = \prod_{j=1}^{k+1} \|f_j\|_{L^{p_j}}.
    \end{align*}
    That is, 
    \[ \left\| \prod_{j=1}^{k+1} f_j \right\|_{L^p} \leq \prod_{j=1}^{k+1} \|f_j\|_{L^{p_j}} \]
    which completes the induction.
\end{proof}

\subsection{Minkowski's Inequality}

\begin{lemma}[Convexity Lemma]
    \label{lem:convexity_lemma}
    Suppose that $f : [0,\infty) \to [0,\infty)$ is a convex function such that $f(0)=0$.
    Then for all $t_1,t_2\geq 0$ we have
    \[ f(t_1) + f(t_2) \leq f(t_1 + t_2). \]
\end{lemma}
\begin{proof}
    Since $f$ is convex with $f(0)=0$, for each $\lambda \in [0,1]$ and each $x\geq 0$ we have
    \[ f(\lambda x) = f(\lambda x + (1-\lambda)0) \leq \lambda f(x) + (1-\lambda)f(0) = \lambda f(x) \]
    Let $t_1,t_2 \geq 0$ be arbitrary. If $t_1+t_2=0$, then the result is trivial, so suppose that $t_1+t_2>0$.
    Define $\lambda_0 := \frac{t_1}{t_1+t_2} \in (0,1)$.
    Then we estimate
    \[ f(t_1) = f(\lambda_0 (t_1+t_2)) \leq \lambda_0 f(t_1+t_2) \quad\text{and} \quad f(t_2) = f((1-\lambda_0)(t_1+t_2)) \leq (1-\lambda_0)f(t_1+t_2). \]
    By adding these two inequalities, we obtain
    \[ f(t_1) + f(t_2) \leq (\lambda_0 + (1-\lambda_0)) f(t_1+t_2) = f(t_1+t_2) \]
    which is the desired result.
\end{proof}

\begin{lemma}[Concavity Lemma]
    \label{lem:concavity_lemma}
    Suppose that $f : [0,\infty) \to [0,\infty)$ is a concave function such that $f(0)=0$.
    Then for all $t_1,t_2\geq 0$ we have
    \[ f(t_1 + t_2) \leq f(t_1) + f(t_2). \]
\end{lemma}
\begin{proof}
    Note that $-f$ is convex and $-f(0)=0$, so the result follows from \ref{lem:convexity_lemma}.
\end{proof}

\begin{proposition}
    Let $\{a_j\}_{j=1}^\infty \subseteq [0,\infty)$ be a sequence of nonnegative real numbers. 
    Then for all $0\leq p \leq 1$, we have
    \[ \left( \sum_{j=1}^\infty a_j \right)^p \leq \sum_{j=1}^\infty a_j^p. \]
    If $p>1$, then the reverse inequality holds.
\end{proposition}

\begin{proof}
    First note that if $p=1$, then both sides are equal, so the result is trivial.
    Also if $p=0$, then the left side is $1$ if the sequence is not identically zero, and the right side is the number of nonzero terms in the sequence, so the result holds.

    Let $N\in \Z^+$ be arbitrary.
    See that the function
    \[  f:[0,\infty) \to [0,\infty), \quad f(x) = x^p \]
    is twice differentiable with
    \[ f'(x) = p x^{p-1} \quad\text{ and thus }\quad f''(x) = p(p-1)x^{p-2}. \]
    Thus
    \[ f'(x) > 0 \iff p > 1 \quad\text{ and }\quad  f''(x)<0 \iff 0< p <1. \]
    That is, $f$ is convex if $p>1$ and concave if $0 < p < 1$, and in both cases we have $f(0)=0$.
    Thus by \ref{lem:convexity_lemma},\ref{lem:concavity_lemma}, we have
    \[ f\left( \sum_{j=1}^N a_j \right) \leq \sum_{j=1}^N f(a_j) \quad\text{if } 0 < p < 1 \]
    and
    \[ f\left( \sum_{j=1}^N a_j \right) \geq \sum_{j=1}^N f(a_j) \quad\text{if } p > 1. \]
    That is,
    \[ \left( \sum_{j=1}^N a_j \right)^p \leq \sum_{j=1}^N a_j^p \quad\text{if } 0 < p < 1 \]
    and
    \[ \left( \sum_{j=1}^N a_j \right)^p \geq \sum_{j=1}^N a_j^p \quad\text{if } p > 1. \]
    Since $N\in \Z^+$ was arbitrary, the result follows by taking limits as $N\to \infty$.
\end{proof}

\begin{theorem}[Minkowski's Inequality]
    \label{thm:minkowskis_inequality}
    Let $(X,\mu)$ be a measure space, and let $1 \leq p \leq \infty$.
    If $f,g \in L^p(X,\mu)$, then $f+g \in L^p(X,\mu)$ and
    \[ \|f+g\|_{L^p} \leq \|f\|_{L^p} + \|g\|_{L^p} \]
    with equality if and only if there is a constant $c\geq 0$ such that $|f| = c|g|$ almost everywhere.
\end{theorem}

\begin{proof}
    We remark that this has already been proven in the case that $p=1$ as part of the proof that $\|\cdot\|_{L^1}$ is a norm on $L^1(X,\mu)$.
    Also, the case that $p=\infty$ is easy to check directly --- if $f,g \in L^\infty(X,\mu)$, then
    \begin{align*}
        \|f+g\|_{L^\infty} = \esssup_{x\in X} |f(x)+g(x)| &\leq \esssup_{x\in X} |f(x)| + \esssup_{x\in X} |g(x)| \\
            &= \|f\|_{L^\infty} + \|g\|_{L^\infty}
    \end{align*}
    by the triangle inequality for complex numbers.

    Now suppose that $1 < p < \infty$ and let $f,g \in L^p(X,\mu)$ be arbitrary.
    It is not even obvious that $f+g \in L^p(X,\mu)$, so we will prove this first, with a more elementary inequality. 

    See that for each $x \in X$, we have
    \[ |f(x)+g(x)|^p \leq (|f(x)| + |g(x)|)^p \leq (2\max\{|f(x)|,|g(x)|\})^p \leq 2^p(|f(x)|^p + |g(x)|^p) \]
    by using the triangle inequality for complex numbers, the fact that $t \mapsto t^p$ is increasing on $[0,\infty)$, and the fact that $\max\{a,b\} \leq a+b$ for all $a,b\geq 0$.
    By monotonicity of the integral, we have
    \begin{align*}
        \|f+g\|_{L^p}^p = \int_X |f(x)+g(x)|^p \,\dif\mu(x) &\leq \int_X 2^p (|f(x)|^p + |g(x)|^p) \,\dif\mu(x) \\
            &= 2^p \left( \|f\|_{L^p}^p + \|g\|_{L^p}^p \right) < \infty
    \end{align*}
    since $f,g \in L^p(X,\mu)$.
    Thus $f+g \in L^p(X,\mu)$.

    We can now improve this estimate. Let $1< q < \infty$ be such that $\frac{1}{p} + \frac{1}{q} = 1$.
    See that 
    \begin{align*}
        \| f+g \|_{L^p}^p &= \int_X |f+g|^p \,\dif\mu = \int_X |f+g| |f+g|^{p-1} \,\dif\mu \\
            &\leq \int_X (|f|+|g|) |f+g|^{p-1} \,\dif\mu \qquad\qquad\qquad\text{by the triangle inequality} \\
            &= \int_X |f| |f+g|^{p-1} \dif\mu + \int_X |g| |f+g|^{p-1} \dif\mu \\
            &= \| |f| |f+g|^{p-1} \|_{L^1} + \| |g| |f+g|^{p-1} \|_{L^1} \\
            &\leq \|f\|_{L^p} \| |f+g|^{p-1} \|_{L^q} + \|g\|_{L^p} \| |f+g|^{p-1} \|_{L^q}
    \end{align*}
    by H\"older's inequality. 
    We should check that $|f+g|^{p-1} \in L^q(X,\mu)$ to justify the use of H\"older's inequality above --- we compute 
    \[ \| |f+g|^{p-1} \|_{L^q}^q = \int_X |f+g|^{(p-1)q} \,\dif\mu = \int_X |f+g|^{p-1 \cdot \frac{p}{p-1}} \,\dif\mu = \int_X |f+g|^p \,\dif\mu = \|f+g\|_{L^p}^p < \infty \]
    since we have already shown that $f+g \in L^p(X,\mu)$.
    Thus our use of H\"older's inequality is justified.

    Also the above computation shows that
    \[ \| |f+g|^{p-1} \|_{L^q} = \|f+g\|_{L^p}^{p/q} = \|f+g\|_{L^p}^{p-1} \]
    Thus our use of H\"older's inequality gives
    \[ \|f+g\|_{L^p}^p \leq \|f\|_{L^p} \|f+g\|_{L^p}^{p-1} + \|g\|_{L^p} \|f+g\|_{L^p}^{p-1}. \]
    If $\|f+g\|_{L^p} = 0$, then the result is trivial, so suppose that $\|f+g\|_{L^p} > 0$.
    Then we can divide both sides by $\|f+g\|_{L^p}^{p-1}$ to obtain
    \[ \|f+g\|_{L^p} \leq \|f\|_{L^p} + \|g\|_{L^p} \]
    which is the desired result.

    Finally, we check when equality holds.
    If there is a constant $c\geq 0$ such that $|f| = c|g|$ $\mu$-almost everywhere, then it is easy to check that equality holds in the above inequality.
    Conversely, if equality holds, then we must have equality in the application of H\"older's inequality above, which by \ref{prop:holders_inequality_1} implies that there is a constant $c\geq 0$ such that
    \[ |f(x)| = c |f(x)+g(x)|^{p-1} \quad\text{and}\quad |g(x)| = \frac{1}{c} |f(x)+g(x)|^{p-1} \]
    for $\mu$-almost every $x \in X$.
    If $c=0$, then $f=0$ $\mu$-almost everywhere, so we may assume that $c>0$.
    From this we see that
    \[ |f(x)|^p = c^p |f(x)+g(x)|^{p(p-1)} \quad\text{and}\quad |g(x)|^p = c^{-p} |f(x)+g(x)|^{p(p-1)} \]
    for $\mu$-almost every $x \in X$.
    By dividing these two equations, we obtain
    \[ |f(x)|^p = c^{2p} |g(x)|^p \]
    for $\mu$-almost every $x \in X$, which is the desired result.
\end{proof}

\begin{proposition}[Minkowski's Triangle Inequality]
    \label{prop:minkowskis_triangle_inequality}
    
\end{proposition}
\begin{proof}
    
\end{proof}



\begin{proposition}[Reverse Minkowski's Inequality]
    \label{prop:reverse_minkowskis_inequality}
    
\end{proposition}
\begin{proof}
    
\end{proof}


\begin{proposition}[Minkowski's Integral Inequality]
    \label{prop:minkowskis_integral_inequality}
    
\end{proposition}
\begin{proof}
    
\end{proof}

\begin{proposition}[Jensen's Inequality]
    \label{prop:jensen_inequality}

\end{proposition}

\subsection{Hanner's Inequalities}

\begin{lemma}[Hanner's Inequality Lemma $2\leq p < \infty$]
    \label{lem:hanners_inequality_lemma}
    Suppose $2\leq p < \infty$. Then
    \[ 2(|w|^p + |z|^p) \leq |w+z|^p + |w-z|^p \leq 2^{p-1}(|w|^p + |z|^p) \]
    for all $w,z \in \C$.
\end{lemma}

\begin{proof}
    Notice that the inequalities are symmetric in the variables $w$ and $z$, and that if either $w=0$ or $z=0$, then both inequalities are trivially true since $2 \leq 2 \leq 2^{p-1}$.
    
    \vspace{2mm}
    \textit{Step 1:} We prove the right inequality first. 

    Let $w,z \in \C$ be arbitrary nonzero complex numbers.
    Since the inequalities are symmetric in $w$ and $z$, we may assume without loss of generality that $|w| \geq |z| >0$.
    
    Since $p\geq 2$, the function 
    \[ \phi: [0,\infty) \to [0,\infty), \quad \phi(t) = t^{p/2} \]
    is convex and has $\phi(0) = 0$, so by \ref{lem:convexity_lemma}, we have
    \[ \phi(t_1) + \phi(t_2) \leq \phi( t_1 + t_2 ) \qquad \forall t_1,t_2 \geq 0 \]
    which says that
    \[ t_1^{p/2} + t_2^{p/2} \leq (t_1 + t_2)^{p/2} \tag{$\star$} \]
    for all $t_1,t_2 \geq 0$.
    But also convexity of $\phi$ implies that
    \[ \phi\left( \frac{1}{2}t_1 + \frac{1}{2}t_2 \right) \leq \frac{1}{2}\phi(t_1) + \frac{1}{2}\phi(t_2) \qquad \forall t_1,t_2\geq 0\] 
    which says that
    \[ \left( \frac{1}{2}t_1 + \frac{1}{2}t_2 \right)^{p/2} \leq \frac{1}{2}t_1^{p/2} + \frac{1}{2}t_2^{p/2} \tag{$\star\star$} \]
    for all $t_1,t_2 \geq 0$.
    Now we see that
    \begin{align*}
        \left| \frac{w+z}{2} \right|^p + \left| \frac{w-z}{2} \right|^p &= \left( \left| \frac{w+z}{2} \right|^2 \right)^{p/2} + \left( \left| \frac{w-z}{2} \right|^2 \right)^{p/2} \\
            &\leq \left( \left| \frac{w+z}{2} \right|^2 + \left| \frac{w-z}{2} \right|^2 \right)^{p/2} &&\text{by } (\star) \\
            &= \left( \frac{2|w|^2 + 2|z|^2}{4} \right)^{p/2} \\
            &= \left( \frac{1}{2}|w|^2 + \frac{1}{2}|z|^2 \right)^{p/2} \\
            &\leq \frac{1}{2}|w|^p + \frac{1}{2}|z|^p. &&\text{by } (\star\star)
    \end{align*}
    Multiplying both sides by $2^p$ gives us
    \[ |w+z|^p + |w-z|^p \leq 2^{p-1}(|w|^p + |z|^p) \]
    which is the right inequality.

    Since $z$ and $w$ were arbitrary nonzero complex numbers, and the inequality is trivially true if either $z=0$ or $w=0$, the right inequality holds for all $w,z \in \C$.

    \vspace{2mm}
    \textit{Step 2:} Now we prove the left inequality.
    Let $w,z \in \C$ be arbitrary nonzero complex numbers.

    Again, since the inequalities are symmetric in $w$ and $z$, we may assume without loss of generality that $|w| \geq |z| >0$.
    By using the right hand inequality we just proved with the complex numbers
    \[ a := \frac{w+z}{2}, \quad b := \frac{w-z}{2} \]
    we obtain
    \[ |a+b|^p + |a-b|^p \leq 2^{p-1}(|a|^p + |b|^p). \]
    But
    \[ a+b = w, \quad a-b = z, \quad |a|^p + |b|^p = \frac{|w+z|^p + |w-z|^p}{2^p}. \]
    Substituting these back into the inequality gives
    \[ |w|^p + |z|^p \leq 2^{p-1} \frac{|w+z|^p + |w-z|^p}{2^p}, \]
    which simplifies to
    \[ 2(|w|^p + |z|^p) \leq |w+z|^p + |w-z|^p. \]
    This completes the proof of the left inequality.
\end{proof}

\begin{lemma}[Hanner's Inequality Lemma $1 < p \leq 2$]
    \label{lem:hanners_inequality_lemma_2}
    \noindent Suppose $1 < p \leq 2$. Then
    \[ 2^{p-1}(|w|^p + |z|^p) \leq |w+z|^p + |w-z|^p \leq 2(|w|^p + |z|^p) \]
    for all $w,z \in \C$.
\end{lemma}

\begin{proof}
    Notice that this inequality is symmetric in the variables $w$ and $z$, and that if either $w=0$ or $z=0$, then both inequalities are trivially true since $2^{p-1} \leq 2 \leq 2$.

    \vspace{2mm}
    \textit{Step 1:} We prove the left inequality first.
    Let $w,z \in \C$ be arbitrary nonzero complex numbers.

    Since the inequalities are symmetric in $w$ and $z$, we may assume without loss of generality that $|w| \geq |z| >0$.
    Since $1 < p \leq 2$, the function
    \[ \psi: [0,\infty) \to [0,\infty), \quad \psi(t) = t^{p/2} \]
    is concave and has $\psi(0) = 0$, so by \ref{lem:concavity_lemma}, we have
    \[ \psi(t_1) + \psi(t_2) \geq \psi(t_1 + t_2) \qquad \forall t_1,t_2 \geq 0 \]
    which says that
    \[ t_1^{p/2} + t_2^{p/2} \geq (t_1 + t_2)^{p/2} \tag{$\star$} \]
    for all $t_1,t_2 \geq 0$.
    But also concavity of $\psi$ implies that
    \[ \psi\left( \frac{1}{2}t_1 + \frac{1}{2}t_2 \right) \geq \frac{1}{2}\psi(t_1) + \frac{1}{2}\psi(t_2) \qquad \forall t_1,t_2\geq 0\]
    which says that
    \[ \left( \frac{1}{2}t_1 + \frac{1}{2}t_2 \right)^{p/2} \geq \frac{1}{2}t_1^{p/2} + \frac{1}{2}t_2^{p/2} \tag{$\star\star$} \]
    for all $t_1,t_2 \geq 0$.
    Now we see that
    \begin{align*}
        \left| \frac{w+z}{2} \right|^p + \left| \frac{w-z}{2} \right|^p &= \left( \left| \frac{w+z}{2} \right|^2 \right)^{p/2} + \left( \left| \frac{w-z}{2} \right|^2 \right)^{p/2} \\
            &\geq \left( \left| \frac{w+z}{2} \right|^2 + \left| \frac{w-z}{2} \right|^2 \right)^{p/2} &&\text{by } (\star) \\
            &= \left( \frac{2|w|^2 + 2|z|^2}{4} \right)^{p/2} \\
            &= \left( \frac{1}{2}|w|^2 + \frac{1}{2}|z|^2 \right)^{p/2} \\
            &\geq \frac{1}{2}|w|^p + \frac{1}{2}|z|^p. &&\text{by } (\star\star)
    \end{align*}
    Multiplying both sides by $2^p$ gives us
    \[ |w+z|^p + |w-z|^p \geq 2^{p-1}(|w|^p + |z|^p) \]
    which is the left inequality.

    Since $z$ and $w$ were arbitrary nonzero complex numbers, and the inequality is trivially true if either $z=0$ or $w=0$, the left inequality holds for all $w,z \in \C$.

    \vspace{2mm}
    \textit{Step 2:} Now we prove the right inequality.
    Let $w,z \in \C$ be arbitrary nonzero complex numbers.

    Again, since the inequalities are symmetric in $w$ and $z$, we may assume without loss of generality that $|w| \geq |z| >0$.
    By using the left hand inequality we just proved with the complex numbers
    \[ a := \frac{w+z}{2}, \quad b := \frac{w-z}{2} \]
    we obtain
    \[ |a+b|^p + |a-b|^p \geq 2^{p-1}(|a|^p + |b|^p) \]
    which is equivalent to
    \[ 2^{-p}|w|^p + 2^{-p}|z|^p \geq 2^{p-1} \frac{|w+z|^p + |w-z|^p}{2^p} \]
    which simplifies to
    \[ 2(|w|^p + |z|^p) \geq |w+z|^p + |w-z|^p \]
    which is the desired right inequality.

    Since $z$ and $w$ were arbitrary nonzero complex numbers, and the inequality is trivially true if either $z=0$ or $w=0$, the right inequality holds for all $w,z \in \C$.
\end{proof}

\begin{theorem}[Hanner's Inequalities]
    \label{thm:hanners_inequalities}
    Let $(X,\mu)$ be a measure space. Suppose that $2 \leq p < \infty$.
    Then for all $f,g \in L^p(X,\mu)$, we have
    \[ 2(\|f\|_{L^p}^p + \|g\|_{L^p}^p) \leq \|f+g\|_{L^p}^p + \|f-g\|_{L^p}^p \leq 2^{p-1}(\|f\|_{L^p}^p + \|g\|_{L^p}^p). \]

    Suppose that $1 < p \leq 2$.
    Then for all $f,g \in L^p(X,\mu)$, we have
    \[ 2^{p-1}(\|f\|_{L^p}^p + \|g\|_{L^p}^p) \leq \|f+g\|_{L^p}^p + \|f-g\|_{L^p}^p \leq 2(\|f\|_{L^p}^p + \|g\|_{L^p}^p). \]
\end{theorem}
\begin{proof}
    Apply the Lemma \ref{lem:hanners_inequality_lemma} or \ref{lem:hanners_inequality_lemma_2} pointwise, integrate over $X$, and use the monotonicity and linearity of the integral.

    If you don't believe me ...

    Let $2 \leq p < \infty$ and let $f,g \in L^p(X,\mu)$ be arbitrary.
    Then by \ref{lem:hanners_inequality_lemma}, we have
    \[ 2(|f(x)|^p + |g(x)|^p) \leq |f(x)+g(x)|^p + |f(x)-g(x)|^p \leq 2^{p-1}(|f(x)|^p + |g(x)|^p) \]
    for all $x \in X$.
    By integrating over $X$ and using the linearity and monotonicity of the integral, we obtain
    \begin{align*}
        2\left( \int_X |f(x)|^p \,\dif\mu(x) + \int_X |g(x)|^p \,\dif\mu(x) \right) &\leq \int_X |f(x)+g(x)|^p \,\dif\mu(x) + \int_X |f(x)-g(x)|^p \,\dif\mu(x) \\
            &\leq 2^{p-1}\left( \int_X |f(x)|^p \,\dif\mu(x) + \int_X |g(x)|^p \,\dif\mu(x) \right)
    \end{align*} 
    which says exactly that
    \[ 2(\|f\|_{L^p}^p + \|g\|_{L^p}^p) \leq \|f+g\|_{L^p}^p + \|f-g\|_{L^p}^p \leq 2^{p-1}(\|f\|_{L^p}^p + \|g\|_{L^p}^p) \]
    as desired.

    The case that $1 < p \leq 2$ is similar, using \ref{lem:hanners_inequality_lemma_2} instead.
\end{proof}

\subsection{Hardy's Inequality}

\begin{theorem}[Hardy's Inequality for Integrals]
    \label{thm:hardys_inequality}
    Let $f: (0,\infty) \to [0,\infty)$ be a measurable function and let $1 \leq p < \infty$.
    Then 
    \[ \int_0^\infty \left( \frac{1}{x} \int_0^x f(t) \,\dif t \right)^p \,\dif x \leq \left( \frac{p}{p-1} \right)^p \int_0^\infty f(x)^p \,\dif x. \]
    with equality if and only if $f=0$ almost everywhere.
\end{theorem}

\begin{corollary}[Hardy's Inequality for Sums]
    \label{cor:hardys_inequality_for_sums}
    Let $\{a_n\}_{n=1}^\infty \subseteq [0,\infty)$ be a sequence of nonnegative real numbers, and let $1 \leq p < \infty$.
    Then
    \[ \sum_{n=1}^\infty \left( \frac{a_1 + a_2 + \cdots + a_n}{n} \right)^p \leq \left( \frac{p}{p-1} \right)^p \sum_{n=1}^\infty a_n^p. \]
    with equality if and only if $a_n = 0$ for all $n\in \Z^+$.
\end{corollary}