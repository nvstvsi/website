\section{The Lebesgue Integral}

Now that we have defined measure spaces and measurable functions, we would like to define the Lebesgue integral.
The Lebesgue integral is a generalization of the Riemann integral, and it fixes many of the deficiencies of the Riemann integral.

\noindent There are many different approaches to defining the Lebesgue integral, and they can all be shown to be equivalent.

\subsection{Integration of Nonnegative Functions with respect to a Measure}

Some people take an approach of defining the Lebesgue integral with a series of steps, but we will do it all at once.

\begin{definition}[Lower Lebesgue Sum, Lebesgue Integral of Nonnegative Function]
    \label{def:lebesgue_integral}
    Let $(X,\mu)$ be a measure space, and let $P$ be a finite collection of disjoint measurable sets such that $\bigcup_{A \in P} A = X$.
    We call such a collection a \textit{$\mu$-partition} of $(X,\mu)$.

    \vspace{2mm}

    \noindent If $f : X \to [0,\infty]$ is a nonnegative measurable function, then we define the \textit{lower Lebesgue sum} of $f$ with respect to the $\mu$-partition $P$ to be
    \[ L(f,P) = \sum_{A \in P} \inf_{x \in A} f(x) \mu(A). \]
    We then define the \textit{Lebesgue integral} of $f$ with respect to $\mu$ to be
    \[ \int_X f \, \dif \mu := \sup_{P} L(f,P), \]
    where the supremum is taken over all $\mu$-partitions $P$ of $(X,\mu)$.
\end{definition}

\begin{remark}[Notation for Lebesgue Integral]
    \label{rem:notation_for_lebesgue_integral}
We note that the symbol $\dif \mu$ is used to denote integration with respect to the measure $\mu$, and the symbol $\dif$ has no independent meaning here.
Sometimes people drop the $X$ in the integral sign and just write $\int f \, \dif \mu$ --- this is fine because the domain of integration is always the largest set in the $\sigma$-algebra on which $\mu$ is defined.
Sometimes people write the Lebesgue integral as $\int_X f(x) \, \dif \mu(x)$, and we will occasionally do this as well.

We also remark that when we are integrating with respect to the Lebesgue measure on $\R^n$, we will usually write $\dif x$ instead of $\dif \mu$;
this is by tradition and conveience. (There is a connection to differential forms, but we will not discuss that here.)
In this case, we will usually write $\int_{\R^n} f(x) \, \dif x$.
\end{remark}


\begin{lemma}[Integral of Characteristic Function]
    \label{lem:integral_of_characteristic_function}
    Let $(X,\mu)$ be a measure space, and let $A \subset X$ be a measurable set.
    Then
    \[ \int_X \Chi_A \, \dif \mu = \mu(A). \]
\end{lemma}

\begin{proof}
    Since $A$, $X \setminus A$ is a $\mu$-partition of $(X,\mu)$, we clearly have
    $L(\Chi_A, \{A, X \setminus A\}) = \mu(A) $.
    Thus \[ \int_X \Chi_A \, \dif \mu \geq \mu(A). \]

    On the other hand, let $P$ be a $\mu$-partition of $(X,\mu)$ into disjoint measurable sets $A_1, \ldots, A_n$.
    Then for each $j=1,2,\ldots,n$ we see that
    \[ \inf_{x \in A_j} \Chi_A(x) = \begin{cases}
        1 & A_j \subset A, \\
        0 & A_j \not\subset A
    \end{cases} \]
    which implies
    \[ \mu(A_j)\inf_{A_j} \Chi_A = \begin{cases}
        \mu(A_j) & A_j \subset A, \\
        0 & A_j \not\subset A.
    \end{cases} \]
    Then we have
    \[ L(\Chi_A, P) = \sum_{j:A_j \subset A} \mu(A_j) = \mu\left( \bigcup_{j:A_j \subset A} A_j \right) \leq \mu(A). \]
    This holds for any $\mu$-partition $P$, so we have
    \[ \int_X \Chi_A \, \dif \mu = \sup_P L(\Chi_A, P) \leq \mu(A). \]
    Combining the two inequalities gives the result.
\end{proof}

Now that we know the integral of characteristic functions, we can upgrade this to simple functions.

\begin{lemma}[Integral of Simple Function]
    \label{lem:integral_of_simple_function}
    Let $(X,\mu)$ be a measure space, and let $s : X \to [0,\infty)$ be a nonnegative simple function,
    \[ s = \sum_{j=1}^n c_j \Chi_{A_j}, \]
    where $c_1, \ldots, c_n \geq 0$ and $A_1, \ldots, A_n$ are disjoint measurable sets.
    Then we have
    \[ \int_X s \, \dif \mu = \sum_{j=1}^n c_j \mu(A_j). \]
\end{lemma}

Some authors take this as the definition of the Lebesgue integral for simple functions, and then use this to define the Lebesgue integral for nonnegative measurable functions.
We do not take this approach, because it is not clear that the right-hand side is well-defined (i.e. independent of the representation of $s$ as a sum of characteristic functions).

\begin{proof}
    Set $A_0 = X \setminus \bigcup_{i=1}^n A_i$.
    Then $A_0, A_1, \ldots, A_n$ is a $\mu$-partition of $(X,\mu)$, and we have
    \[ L(s, \{A_0, A_1, \ldots, A_n\}) = \sum_{j=1}^n c_j \mu(A_j). \]
    Thus
    \[ \int_X s \, \dif \mu \geq \sum_{j=1}^n c_j \mu(A_j). \]

    Now let $P$ be any $\mu$-partition of $(X,\mu)$ into disjoint measurable sets $B_1, \ldots, B_m$.
    Then we have
    \begin{align*}
        L(s,P) &= \sum_{j=1}^m \inf_{x \in B_j} s(x) \mu(B_j) \\
            &= \sum_{j=1}^m \min_{\{ i : B_j\cap A_i \neq \varnothing \}} c_i \,\mu(B_j) \\
            &\leq \sum_{j=1}^m \sum_{i=1}^n \mu(B_j\cap A_k) c_k \\
            &= \sum_{k=1}^n c_k \sum_{j=1}^m \mu(B_j \cap A_k) \\
            &= \sum_{k=1}^n c_k \mu(A_k).
    \end{align*}
    This holds for any $\mu$-partition $P$, so we have
    \[ \int_X s \, \dif \mu \leq \sum_{j=1}^n c_j \mu(A_j). \]
    Combining the two inequalities gives the result.
\end{proof}

\begin{lemma}[Lebesgue Integral is Order-Preserving]
    \label{lem:lebesgue_integral_is_order_preserving}
    Let $(X,\mu)$ be a measure space, and let $f,g : X \to [0,\infty]$ be nonnegative measurable functions such that $f(x) \leq g(x)$ for all $x \in X$.
    Then
    \[ \int_X f \, \dif \mu \leq \int_X g \, \dif \mu. \]
\end{lemma}

\begin{proof}
    Let $P$ be any $\mu$-partition of $(X,\mu)$.
    Then for each set $A \in P$ we have
    \[ \inf_{x \in A} f(x) \leq \inf_{x \in A} g(x), \]
    which implies
    \[ L(f,P) = \sum_{A \in P} \inf_{x \in A} f(x) \mu(A) \leq \sum_{A \in P} \inf_{x \in A} g(x) \mu(A) = L(g,P). \]
    This holds for any $\mu$-partition $P$, so we have
    \[ \int_X f \, \dif \mu \leq \int_X g \, \dif \mu. \]
\end{proof}

We finish this subsection by showing that that the integral of a general nonnegative measurable function can be approximated by integrals of simple functions.

\begin{lemma}[Integrals via Simple Functions]
    \label{lem:integrals_via_simple_functions}
    Let $(X,\mu)$ be a measure space, and let $f : X \to [0,\infty]$ be a nonnegative measurable function.
    Then \[ \int_X f \,\dif \mu = \sup\left\{ \int_X s \dif \mu : s \text{ is a simple function with } 0 \leq s \leq f \right\} \]
\end{lemma}

\begin{proof}
    First note that the inequality $\geq$ is immediate from Lemma \ref{lem:lebesgue_integral_is_order_preserving}.

    For the reverse inequality, we assume first that we are in the special case where $\inf_{A} f > 0$ for each measurable set $A \subseteq X$ with $\mu(A) > 0$.
    (This is a technical assumption that we will remove later.)
    
    Let $P$ be a $\mu$-partition of $(X,\mu)$ into disjoint nonempty measurable sets $A_1, \ldots, A_n$.
    For each $j=1,2,\ldots,n$, set $m_j = \inf_{x \in A_j} f(x)$. Then the function 
    \[ s = \sum_{j=1}^n m_j \Chi_{A_j} \]
    is a simple function with $0 \leq s \leq f$, and 
    \[ \int_X s \, \dif \mu = \sum_{j=1}^n m_j \,\mu(A_j) = \sum_{j=1}^\infty \,\inf_{x \in A_j} f(x) \,\mu(A_j) = L(f,P) \]
    where the first equality holds by Lemma \ref{lem:integral_of_simple_function}.
    Therefore taking the supremum over all $\mu$-partitions $P$ gives
    \[ \int_X f \, \dif \mu = \sup_P L(f,P) \leq \sup\left\{ \int_X s \dif \mu : s \text{ is a simple function with } 0 \leq s \leq f \right\}. \]

    Now we remove the technical assumption.
    Assume that there is a measurable set $A \subseteq X$ with $\mu(A) > 0$ and $\inf_A f = \infty$.
    Then we must have $f(x) = \infty$ for each $x \in A$.

    In this case, for arbitrary $t>0$ the function $t\Chi_A$ is a simple function with $0 \leq t\Chi_A \leq f$, and
    \[ \int_X t\Chi_A \, \dif \mu = t \mu(A) \]
    by Lemma \ref{lem:integral_of_characteristic_function}.
    Thus the supremum on the right-hand side is greater than $t\mu(A)$ for each $t>0$, and hence is infinite.
    Thus the inequality $\leq$ holds in this case as well, completeing the proof.
\end{proof}

\begin{example}[Sums are Integrals]
    \label{ex:sums_are_integrals}
    Consider the counting measure $\mu$ on $\N$, and let $\{a_k\}_{k=0}^\infty$ be a sequence of nonnegative real numbers.
    Define $a : \N \to [0,\infty)$ by $a(n) = a_n$.
    Then we have
    \[ \int a \,\dif \mu = \sum_{k=0}^\infty a_k. \]
\end{example}
\begin{proof}
    for each $n \in \N$, the set $\{n\}$ is measurable and has measure $\mu(\{n\}) = 1$.
    Thus for each $n \in \N$ the sum 
    \[ s_n := \sum_{j=0}^N a_j \Chi_{\{j\}} = \sum_{j=0}^N a_j \]
    is a simple function; the fact that the sequence $\{a_k\}_{k=0}^\infty$ is nonnegative implies that $0 \leq s_n \leq a$.
    By the previous lemma, we have
    \[ \int a \,\dif \mu \geq \int s_n \,\dif \mu = \sum_{j=0}^N a_j \]
    for each $N \in \N$; taking limits gives
    \[ \int a \,\dif \mu \geq \sum_{j=0}^\infty a_j. \]

    On the other hand, let $P$ be any $\mu$-partition of $(\N,\mu)$ into disjoint nonempty measurable sets $A_1, A_2, \ldots, A_m$.
    Then for each $j=1,2,\ldots,m$ we have
    \[ \inf_{n \in A_j} a(n) = \inf_{n \in A_j} a_n = \min_{n \in A_j} a_n \]
    since $A_j$ is a nonempty finite set.
    Thus we have
    \[ L(a,P) = \sum_{j=1}^m \inf_{n \in A_j} a(n) \mu(A_j) = \sum_{j=1}^m \min_{n \in A_j} a_n \cdot 1 \leq \sum_{j=1}^m \sum_{n \in A_j} a_n = \sum_{n=0}^\infty a_n. \]
    Taking the supremum over all $\mu$-partitions $P$ gives
    \[ \int a \,\dif \mu \leq \sum_{n=0}^\infty a_n. \]
    Combining the two inequalities gives the result.
\end{proof}

\subsection{Properties of the Lebesgue Integral}

Still only working with nonnegative measurable functions, we now prove some important properties of the Lebesgue integral.
One of the most important properties is the Monotone Convergence Theorem.

\begin{theorem}[Monotone Convergence Theorem]
    \label{thm:monotone_convergence_theorem}
    Let $(X,\mu)$ be a measure space, and let $\{f_k\}_{k=1}^\infty$ be a sequence of nonnegative measurable functions such that $f_k(x) \leq f_{k+1}(x)$ for all $x \in X$ and $k \in \N$.
    Define the function $f:X\to[0,\infty]$ by
    \[ f(x) = \lim_{k \to \infty} f_k(x) , \qquad \forall x\in X. \]
    Then 
    \[ \lim_{k \to \infty} \int_X f_k \, \dif \mu = \int_X f \, \dif \mu. \]    
\end{theorem}

In essence, this says that one can interchange limits and integrals $\lim_k \int f_k \dif\mu = \int \lim_k f_k \dif\mu$ for monotone sequences of nonnegative measurable functions.

\begin{proof}
    Note that the function $f$ is well-defined at each $x \in X$ because the sequence $\{f_k(x)\}_{k=1}^\infty$ is nondecreasing and bounded below by $0$;
    hence \[ f_k(x) \leq f(x) \qquad\forall x\in X,\forall k\geq 1.  \]
    Also note that $f$ is measurable because it is the pointwise limit of measurable functions.

    We remark that the limit $\lim_{k \to \infty} \int_X f_k \, \dif \mu$ exists because the sequence of integrals is nondecreasing (since $\{f_k\}_{k=1}^\infty$ is nondecreasing) and bounded below by $0$.
    Thus everything in the statement of the theorem is well-defined.

    By Lemma \ref{lem:lebesgue_integral_is_order_preserving}, we have
    \[ \int_X f_k \, \dif \mu \leq \int_X f \, \dif \mu \qquad\forall k\geq 1, \]
    and thus
    \[ \lim_{k \to \infty} \int_X f_k \, \dif \mu \leq \int_X f \, \dif \mu. \]

    To prove the reverse inequality, let $A_1, A_2, \ldots, A_n$ be disjoint measurable subsets of $X$, and let $c_1, c_2, \ldots, c_n \geq 0$ be such that
    \[ f(x) \geq \sum_{j=1}^m c_j \Chi_{A_j}(x) \qquad\forall x\in X. \]
    Let $t\in(0,1)$ be arbitrary.
    Then for each $k\in\Z^+$ we let
    \[ E^t_k := \left\{ x\in X : f_k(x) \geq t \sum_{j=1}^m c_j \Chi_{A_j}(x) \right\}. \]
    Then $\{E^t_k\}_{k=1}^\infty$ is an increasing sequence of measurable sets such that $\bigcup_{k=1}^\infty E^t_k = X$.
    By Proposition \ref{prop:sequences_of_measurable_sets}, we have
    \[ \lim_{k\to\infty} \mu(A_j\cap E^t_k) = \mu(A_j) \]
    for each $j=1,2,\ldots,m$.

    If $k\in \Z^+$, then 
    \[ f_k(x) \geq \sum_{j=1}^m tc_j \Chi_{A_j\cap E^t_k} (x) \]
    for each $x\in X$, and thus by Lemma \ref{lem:integral_of_simple_function} we have
    \[ \int_X f_k \, \dif \mu \geq t \sum_{j=1}^m c_j \mu(A_j\cap E^t_k). \]
    Taking limits as $k\to\infty$ gives
    \[ \lim_{k\to\infty} \int_X f_k \, \dif \mu \geq t \sum_{j=1}^m c_j \mu(A_j). \]
    Taking limits as $t\to 1^-$ gives
    \[ \lim_{k\to\infty} \int_X f_k \, \dif \mu \geq \sum_{j=1}^m c_j \mu(A_j). \]
    Taking the supremum over all choices of $m$, $c_1, \ldots, c_m$, and $A_1, \ldots, A_m$ gives
    \[ \lim_{k\to\infty} \int_X f_k \, \dif \mu \geq \int_X f \, \dif \mu. \]
    This completes the proof.
\end{proof}

In the Lemma \ref{lem:integral_of_simple_function}, we computed the integral of a simple function once it had been expressed as a sum of characteristic functions of disjoint measurable sets.
This expression is not unique, and the assumption that the sets are disjoint is inconvenient.
It turns out that the integral of a simple function can be computed from any of its expressions and they all give the same answer.

\begin{lemma}[Integral of Simple Functions, II.]
    \label{lem:integral_of_simple_function_2}
    Let $(X,\mu)$ be a measure space, and let $s : X \to [0,\infty)$ be a nonnegative simple function.
    If
    \[ s = \sum_{j=1}^m a_j \Chi_{A_j} = \sum_{k=1}^n b_k \Chi_{B_k} \]
    are two representations of $s$ as nonnegative linear combinations of characteristic functions, then
    \[ \sum_{j=1}^m a_j \mu(A_j) = \sum_{k=1}^n b_k \mu(B_k). \]
    In particular, \[ \int_X s\,\dif\mu = \sum_{j=1}^m a_j \mu(A_j) \] 
\end{lemma}

\begin{proof}
    Let $A_0 = X \setminus \bigcup_{j=1}^m A_j$ so that $A_0, A_1, \ldots, A_m$ is a $\mu$-partition of $(X,\mu)$.
    If $A_i \cap A_j \neq \varnothing$ for some $i \neq j$, then we can write
    \[ a_i\Chi_{A_i} + a_j\Chi_{A_j} = a_i\Chi_{A_i \setminus A_j} + a_j\Chi_{A_j \setminus A_i} + (a_i + a_j)\Chi_{A_i \cap A_j} \]
    and the sets $A_i \setminus A_j$, $A_j \setminus A_i$, and $A_i \cap A_j$ are disjoint measurable sets.

    Thus for $0\leq i < j \leq m$ such that $A_i \cap A_j \neq \varnothing$, we have
    \[ A_i = (A_i \setminus A_j) \cup (A_i \cap A_j) \quad\text{ and }\quad A_j = (A_j \setminus A_i) \cup (A_i \cap A_j) \]
    and each of these unions is disjoint; hence
    \[ \mu(A_i) = \mu(A_i \setminus A_j) + \mu(A_i \cap A_j) \quad\text{ and }\quad \mu(A_j) = \mu(A_j \setminus A_i) + \mu(A_i \cap A_j) \]
    by disjoint additivity of $\mu$, and it follows that
    \[ a_i \mu(A_i) + a_j \mu(A_j) = a_i \mu(A_i \setminus A_j) + a_j \mu(A_j \setminus A_i) + (a_i + a_j) \mu(A_i \cap A_j). \]
    Thus for each $0\leq i < j \leq m$ such that $A_i \cap A_j \neq \varnothing$, we can replace the sets $A_i, A_j$ by the disjoint sets $A_i \setminus A_j$, $A_j \setminus A_i$, and $A_i \cap A_j$, and replace the coefficients $a_i, a_j$ by $a_i, a_j, a_i + a_j$ respectively; 
    the linear combinations of the characteristic functions and the sums of the measures weighted by the coefficients remain unchanged.
    
    Repeating this process a finite number of times gives a representation of $s$ as a sum of characteristic functions of disjoint measurable sets.
    By relabeling, we have finitely many disjoint measurable sets $\tilde{A}_1, \ldots, \tilde{A}_N$, with nonnegative coefficients $\tilde{a}_1, \ldots, \tilde{a}_N \geq 0$, such that
    $s = \sum_{j=1}^N \tilde{a}_j \Chi_{\tilde{A}_j}.$

    The next step is to make the numbers $\tilde{a}_1, \ldots, \tilde{a}_N$ distinct.
    If $\tilde{a}_i = \tilde{a}_j$ for some $i \neq j$, then $\tilde{A}_i \cup \tilde{A}_j$ is a measurable set, and disjoint additivity of $\mu$ gives
    \[ \tilde{a}_i \mu(\tilde{A}_i) + \tilde{a}_j \mu(\tilde{A}_j) = \tilde{a}_i \mu(\tilde{A}_i \cup \tilde{A}_j). \]
    Thus for each $i \neq j$ such that $\tilde{a}_i = \tilde{a}_j$, we can replace the sets $\tilde{A}_i, \tilde{A}_j$ by the single set $\tilde{A}_i \cup \tilde{A}_j$, and replace the coefficients $\tilde{a}_i, \tilde{a}_j$ by the single coefficient $\tilde{a}_i$.

    Repeating this process a finite number of times gives a representation of $s$ as a sum of characteristic functions of disjoint measurable sets with distinct coefficients.
    By relabeling, we have finitely many disjoint measurable sets $\hat{A}_1, \ldots, \hat{A}_M$, with distinct nonnegative coefficients $\hat{a}_1, \ldots, \hat{a}_M \geq 0$, such that
    $s = \sum_{j=1}^M \hat{a}_j \Chi_{\hat{A}_j}.$

    Finally if $1\leq j\leq M$ is such that $\hat{A}_j = \varnothing$, then we can simply remove this term from the sum without changing anything.
    Thus we may assume that $\hat{A}_1, \ldots, \hat{A}_M$ are nonempty disjoint measurable sets, and $\hat{a}_1, \ldots, \hat{a}_M$ are distinct nonnegative numbers.
    The expression $s = \sum_{j=1}^M \hat{a}_j \Chi_{\hat{A}_j}$ is the \textit{standard form} of the simple function $s$.

    We can now do the same process to the other representation $s = \sum_{k=1}^n b_k \Chi_{B_k}$ to obtain a standard form $s = \sum_{k=1}^M \hat{b}_k \Chi_{\hat{B}_k}$.
    By construction these standard forms must have the same number of terms (the number of distinct values that $s$ takes), and for each $j=1,2,\ldots,M$ there exists $k_j \in \{1,2,\ldots,M\}$ such that $\hat{A}_j = \hat{B}_{k_j}$ and $\hat{a}_j = \hat{b}_{k_j}$.
    Thus we have 
    \[ \sum_{j=1}^m a_j \mu(A_j) = \sum_{j=1}^M \hat{a}_j \mu(\hat{A}_j) = \sum_{k=1}^M \hat{b}_k \mu(\hat{B}_k) = \sum_{k=1}^n b_k \mu(B_k) \]
    as desired.

    The fact that $\int_X s\,\dif\mu = \sum_{j=1}^m a_j \mu(A_j)$ follows from Lemma \ref{lem:integral_of_simple_function}.
\end{proof}

We can now use the Monotone Convergence Theorem to prove additivity of the Lebesgue integral.

\begin{proposition}[Additivity for Nonnegative Functions]
    \label{prop:additivity_for_nonnegative_functions}
    Let $(X,\mu)$ be a measure space, and let $f,g : X \to [0,\infty]$ be nonnegative measurable functions.
    Then
    \[ \int_X (f+g)\,\dif\mu = \int_X f\,\dif\mu + \int_X g\,\dif\mu. \]
\end{proposition}

\begin{proof}
    Note that for simple functions the result follows from Lemma \ref{lem:integral_of_simple_function_2}.
    Thus we approximate by such functions and use the Monotone Convergence Theorem.

    Let $\{f_k\}_{k=1}^\infty$ and $\{g_k\}_{k=1}^\infty$ be increasing sequences of nonnegative simple functions such that 
    \[ \lim_{k \to \infty} f_k = f \quad \text{and} \quad \lim_{k \to \infty} g_k = g \]
    pointwise on $X$; such sequences exist by Proposition \ref{prop:measurable_approx_by_simple_functions}.
    Then the monotone convergence theorem gives
    \begin{align*}
        \int_X (f+g) \,\dif \mu &= \lim_{k \to \infty} \int_X (f_k + g_k) \,\dif \mu \\
            &= \lim_{k \to \infty} \int_X f_k \,\dif \mu + \lim_{k \to \infty} \int_X g_k \,\dif \mu \\
            &= \int_X f\,\dif\mu + \int_X g\,\dif\mu.
    \end{align*}
    Here we have used the Monotone Convergence Theorem in the first and last equalities, and the additivity of the integral for simple functions in the second equality.
\end{proof}

\subsection{Integration of Real and Complex Valued Functions}

\begin{definition}[$f^+$, $f^-$]
    \label{def:f_plus_minus}
    Let $f : X \to [-\infty,\infty]$ be a function defined on a set $X$.
    We define the functions $f^+, f^- : X \to [0,\infty]$ by
    \[ f^+(x) := \max\{f(x),0\} \quad \text{ and } \quad f^-(x) := \max\{-f(x),0\} \qquad\forall x\in X. \]
\end{definition}

\begin{remark}
    \label{rem:f_plus_minus}
    Note that if $f$ is measurable, then $f^+$ and $f^-$ are nonnegative measurable functions, and that
    \[ f = f^+ - f^- \quad\text{and}\quad |f| = f^+ + f^-. \]
\end{remark}

\begin{definition}[Lebesgue Integral, Real-Valued Case]
    \label{def:lebesgue_integral_real_valued_case}
    Let $(X,\mu)$ be a measure space, and let $f : X \to [-\infty,\infty]$ be a measurable function 
    such that at least one of the integrals $\int_X f^+ \,\dif\mu$ or $\int_X f^- \,\dif\mu$ is finite.
    Then we define the \textit{Lebesgue integral} of $f$ with respect to $\mu$ to be
    \[ \int_X f \,\dif\mu := \int_X f^+ \,\dif\mu - \int_X f^- \,\dif\mu. \]
    We say that $f$ is \textit{integrable} if $\int_X f \,\dif\mu$ exists and is finite.
\end{definition}

See that if $f\geq 0$, then $f^- = 0$ and $f^+ = f$, so this definition agrees with the previous definition for nonnegative measurable functions.
Also notice that we allow $f$ to take the values $\pm\infty$, but if $f$ takes either value on a set of positive measure, then the integral will be infinite.
Also notice that 
\begin{align*}
    f \text{ is integrable } &\iff \int_X f \,\dif \mu = \int_X f^+ \,\dif\mu - \int_X f^- \,\dif\mu \in \R \\
        &\iff \int_X f^+ \,\dif\mu < \infty \text{ and } \int_X f^- \,\dif\mu < \infty \\
        &\iff \int_X (f^+ + f^-) \,\dif\mu = \int_X |f| \,\dif\mu < \infty.
\end{align*}
where in the second line we have used that at most one of $\int_X f^+ \,\dif\mu$ or $\int_X f^- \,\dif\mu$ is infinite, since otherwise the integral of $f$ would not be defined.
Thus a real-valued measurable function is integrable if $\int_X |f| \,\dif\mu$ is finite.

\begin{definition}[Lebesgue Integral, Complex-Valued Case]
    \label{def:lebesgue_integral_complex_valued_case}
    Let $(X,\mu)$ be a measure space, and let $f : X \to \C \cup \{ \infty \}$ be a measurable function such that $\int_X |f| \,\dif\mu$ is finite.
    Then we define the \textit{Lebesgue integral} of $f$ with respect to $\mu$ to be
    \[ \int_X f \,\dif\mu := \int_X \Re(f) \,\dif\mu + i \int_X \Im(f) \,\dif\mu. \]
    If the integral $\int_X f \,\dif\mu$ is defined, then we say that $f$ is \textit{integrable}.
\end{definition}

Similarly here we allow $f$ to take the value $\infty$, but if it does so on a set of positive measure, then the integral will not be defined (since $\int_X |f| \,\dif\mu$ will be infinite).
We should take note here that for complex-valued functions, we require absolute integrability in order for the integral to be defined.
Also if $f$ is real-valued, then this definition agrees with the previous definition for real-valued functions.

We would like to state and prove properties of the Lebesgue integral in the real-valued and complex-valued cases we have just defined. 
For efficiency, we let $\F$ denote either $\R$ or $\C$, and we let $\F_\infty$ denote either $[-\infty,\infty]$ or $\C \cup \{\infty\}$ respectively.

\begin{proposition}[Properties of the Lebesgue Integral]
    \label{prop:properties_of_the_lebesgue_integral}
    Let $(X,\mu)$ be a measure space. 
    \begin{enumerate}[(i)]
        \item (Homogeneity) If $f : X \to \F_\infty$ is a measurable function such that the integral $\int_X f \,\dif\mu$ is defined, and if $c \in \F$, then
            \[ \int_X cf \,\dif\mu = c \int_X f \,\dif\mu. \]
        \item (Additivity) If $f,g : X \to \F_\infty$ are measurable functions such that $\int_X |f| \,\dif\mu$ and $\int_X |g| \,\dif\mu$ are both finite, then
            \[ \int_X (f+g) \,\dif\mu = \int_X f \,\dif\mu + \int_X g \,\dif\mu. \]
        \item (Monotonicity) If $f,g : X \to [-\infty,\infty]$ are measurable functions such that $f \leq g$ and the integrals $\int_X f \dif\mu$ and $\int_X g \dif\mu$ are both defined, then
            \[ \int_X f \,\dif\mu \leq \int_X g \,\dif\mu. \]
        \item (Triangle Inequality) If $f : X \to \F_\infty$ is a measurable function such that $\int_X f\,\dif\mu$ is defined, then
            \[ \left| \int_X f \,\dif\mu \right| \leq \int_X |f| \,\dif\mu. \]
    \end{enumerate}
\end{proposition}

Notice the third property only applies to real-valued functions, since the complex numbers are not an ordered field.

\begin{proof}
    \textit{(i) Homogeneity.}
    \vspace{2mm}

    \textit{Case $\F = \R$.}
    First we consider the special case of nonnegative measurable functions and constants.
    Let $f : X \to [0,\infty]$ be a nonnegative measurable function, and let $c \geq 0$.
    Then for each $\mu$-partition $P$ of $(X,\mu)$ we have
    \[ L(cf,P) = \sum_{A \in P} \inf_{x \in A} cf(x) \mu(A) = c \sum_{A \in P} \inf_{x \in A} f(x) \mu(A) = c L(f,P). \]
    Taking the supremum over all $\mu$-partitions $P$ gives
    \[ \int_X cf \,\dif\mu = c \int_X f \,\dif\mu. \]

    Now we consider the general case. 
    Let $f : X \to [-\infty,\infty]$ be a measurable function such that the integral $\int_X f \,\dif\mu$ is defined.
    If $c \geq 0$ , then we have
    \begin{align*}
        \int_X cf \,\dif\mu &= \int_X (cf)^+ \,\dif\mu - \int_X (cf)^- \,\dif\mu \\
            &= \int_X c f^+ \,\dif\mu - \int_X c f^- \,\dif\mu \\
            &= c \left( \int_X f^+ \,\dif\mu - \int_X f^- \,\dif\mu \right) \\
            &= c \int_X f \,\dif\mu
    \end{align*}
    where we have used the result for nonnegative measurable functions in the third equality.

    If $c < 0$ then $-c>0$, and we have
    \begin{align*}
        \int_X cf \,\dif\mu &= \int_X (cf)^+ \,\dif\mu - \int_X (cf)^- \,\dif\mu \\
            &= \int_X (-c)f^- \,\dif\mu - \int_X (-c)f^+ \,\dif\mu \\
            &= -c \left( \int_X f^- \,\dif\mu - \int_X f^+ \,\dif\mu \right) \\
            &= (-c) \cdot \left( -\int_X f \,\dif\mu \right) \\
    \end{align*}
    which completes the proof of homogeneity.

    \vspace{2mm}
    \textit{Case $\F = \C$.}
    Let $f : X \to \C \cup \{\infty\}$ be a measurable function such that $\int_X |f| \,\dif\mu$ is finite, and let $c \in \C$.
    Then
    \begin{align*}
        \int_X cf \,\dif\mu &= \int_X \Re(cf) \,\dif\mu + i \int_X \Im(cf) \,\dif\mu \\
            &= \Re(c) \int_X \Re(f) \,\dif\mu - \Im(c) \int_X \Im(f) \,\dif\mu + i\left( \Re(c) \int_X \Im(f) \,\dif\mu + \Im(c) \int_X \Re(f) \,\dif\mu \right) \\
            &= c\left( \int_X \Re(f) \,\dif\mu + i\int_X \Im(f) \,\dif\mu \right) \\
            &= c \int_X f \,\dif\mu
    \end{align*}
    where we have used homogeneity for real-valued functions in the second equality.
    This completes the proof of homogeneity.

    \vspace{2mm}
    \textit{(ii) Additivity.}
    \vspace{2mm}

    \textit{Case $\F = \R$.}
    Let $f,g : X \to [-\infty,\infty]$ be measurable functions such that $\int_X |f| \,\dif\mu$ and $\int_X |g| \,\dif\mu$ are both finite.
    Then
    \[ (f+g)^+ - (f+g)^- = f + g = f^+ - f^- + g^+ - g^- \]
    which implies
    \[ (f+g)^+ + f^- + g^- = (f+g)^- + f^+ + g^+. \]
    Both sides of this equation are nonnegative measurable functions, so we can integrate both sides to obtain
    \[ \int_X (f+g)^+ \,\dif\mu + \int_X f^- \,\dif\mu + \int_X g^- \,\dif\mu = \int_X (f+g)^- \,\dif\mu + \int_X f^+ \,\dif\mu + \int_X g^+ \,\dif\mu \]
    by additivity of the Lebesgue integral for nonnegative measurable functions.

    Rearranging gives
    \[ \int_X (f+g)^+ \,\dif\mu - \int_X (f+g)^- \,\dif\mu = \int_X f^+ \,\dif\mu - \int_X f^- \,\dif\mu + \int_X g^+ \,\dif\mu - \int_X g^- \,\dif\mu \]
    The left side is not of the form $\infty-\infty$ since $(f+g)^+ \leq f^+ + g^+$ and $(f+g)^- \leq f^- + g^-$, and the
    assumption that $\int_X |f| \,\dif\mu$ and $\int_X |g| \,\dif\mu$ are both finite to ensure that the integrals $\int_X f^\pm \dif \mu$ and $\int_X g^\pm \dif \mu$ are all finite, 
    and thus both of the integrals on the left side are also finite.    
    
    This last equation is exactly the desired result, by definition of the Lebesgue integral for general measurable functions.

    \textit{Case $\F = \C$.} Let $f,g : X \to \C \cup \{\infty\}$ be measurable functions such that $\int_X |f| \,\dif\mu$ and $\int_X |g| \,\dif\mu$ are both finite.
    Then
    \begin{align*}
        \int_X (f+g) \,\dif\mu &= \int_X \Re(f+g) \,\dif\mu + i \int_X \Im(f+g) \,\dif\mu \\
            &= \int_X \Re(f) \,\dif\mu + \int_X \Re(g) \,\dif\mu + i \left( \int_X \Im(f) \,\dif\mu + \int_X \Im(g) \,\dif\mu \right) \\
            &= \left( \int_X \Re(f) \,\dif\mu + i \int_X \Im(f) \,\dif\mu \right) + \left( \int_X \Re(g) \,\dif\mu + i \int_X \Im(g) \,\dif\mu \right) \\
            &= \int_X f \,\dif\mu + \int_X g \,\dif\mu
    \end{align*}
    where we have used additivity for real-valued functions in the second equality.
    This completes the proof of additivity.

    \vspace{2mm}
    \textit{(iii) Monotonicity.}
    \vspace{2mm}

    Let $f,g : X \to [-\infty,\infty]$ be measurable functions such that $f \leq g$ almost everywhere with respect to $\mu$ and the integrals $\int_X f \dif\mu$ and $\int_X g \dif\mu$ are both defined.
    If $\int_X f \dif\mu = \pm\infty$ or $\int_X g \dif\mu = \pm\infty$, then we check this case by case:
    \begin{itemize}
        \item If $\int_X f \dif\mu = -\infty$ or $\int_X g \dif\mu = \infty$, then trivially $\int_X f \dif\mu \leq \int_X g \dif\mu$.
        \item If $\int_X f \dif\mu = \infty$, then we must have either $\int_X f^+ \dif\mu = \infty$ and $\int_X f^- \dif\mu < \infty$, or $\int_X f^+ \dif\mu < \infty$ and $\int_X f^- \dif\mu = \infty$.
            In the first subcase, we have $\int_X g^+ \dif\mu \geq \int_X f^+ \dif\mu = \infty$ and thus $\int_X g \dif\mu = \infty$; hence $\int_X f \dif\mu \leq \int_X g \dif\mu$.
            In the second subcase, we have $\int_X g^- \dif\mu \leq \int_X f^- \dif\mu = \infty$ and thus $\int_X g \dif\mu = -\infty$; hence $\int_X f \dif\mu \leq \int_X g \dif\mu$.
        \item If $\int_X g \dif\mu = -\infty$, then this is similar to the previous case.
    \end{itemize}

    Now assume that both $\int_X f \dif\mu$ and $\int_X g \dif\mu$ are finite.
    Then additivity and homogeneity with $c=-1$ give
    \[ \int_X (g-f) \dif\mu = \int_X g \dif\mu - \int_X f \dif\mu. \]
    Since $g-f \geq 0$, we have $\int_X (g-f) \dif\mu \geq 0$.
    Rearranging gives
    \[ \int_X g \dif\mu \geq \int_X f \dif\mu \]
    as desired. 

    \vspace{2mm}
    \textit{(iv) Triangle Inequality.}
    \vspace{2mm}
    
    \textit{Case $\F = \R$.}
    Let $f : X \to [-\infty,\infty]$ be a measurable function such that $\int_X f \dif\mu$ is defined.
    Then one of $\int_X f^+ \dif\mu$ or $\int_X f^- \dif\mu$ is finite.
    Thus 
    \begin{align*}
        \abs{ \int_X f \,\dif \mu } &= \abs{ \int_X f^+ \,\dif\mu - \int_X f^- \,\dif\mu } \\
            &\leq \int_X f^+ \,\dif\mu + \int_X f^- \,\dif\mu \\
            &= \int_X (f^+ + f^-) \,\dif\mu \\
            &= \int_X |f| \,\dif\mu
    \end{align*}
    where we have used the triangle inequality for real numbers in the second line, and additivity of the Lebesgue integral for nonnegative measurable functions in the third line.

    \vspace{2mm}
    \textit{Case $\F = \C$.}
    Let $f : X \to \C \cup \{\infty\}$ be a measurable function such that $\int_X |f| \dif\mu$ is finite.
    If $\int_X f \dif\mu = 0$ then the inequality is trivial, so assume that $\int_X f \dif\mu \neq 0$.
    Then let 
    \[ \alpha := \frac{\left|\int_X f \dif\mu\right|}{\int_X f \dif\mu}. \]
    We compute that
    \begin{align*}
        \left| \int_X f \dif\mu \right| = \alpha \int_X f \dif\mu &= \int_X \alpha f \dif\mu \\
            &= \int_X \Re(\alpha f) \dif\mu + i \int_X \Im(\alpha f) \dif\mu \\
            &= \int_X \Re(\alpha f) \dif\mu \\
            &\leq \int_X |\alpha f| \dif\mu \\
            &= \int_X |f| \dif\mu
    \end{align*}
    where we have use homogeneity for complex-valued functions in the second equality, and the fact that $ \alpha \int_X f \dif\mu$ is a nonnegative real number in the fourth equality, 
    and the triangle inequality for real-valued functions in the inequality.
    This completes the proof of the triangle inequality.
\end{proof}

\subsection{Integration of $\R^m$-Valued Functions}

Sometimes we wish to integrate functions that take values in $\R^m$ or some vector space other than $\R$ or $\C$.
This can be done component-wise using the Lebesgue integral we have already defined.

\begin{definition}[Lebesgue Integral, Vector-Valued Case]
    \label{def:lebesgue_integral_vector_valued_case}
    Let $(X,\mu)$ be a measure space, and let $f: X \to \R^m$ be a measurable function.
    We define the \textit{Lebesgue integral} of $f$ with respect to $\mu$ as the vector
    \[ \int_X f \,\dif \mu := \left( \int_X f_1 \,\dif \mu, \int_X f_2 \,\dif \mu, \ldots, \int_X f_m \,\dif \mu \right) \in \R^m \]
    where $f_j: X \to \R$ is the $j$-th component function of $f$ for each $1 \leq j \leq m$.
\end{definition}

\begin{proposition}[Properties of the Lebesgue Integral, Vector-Valued Case]
    \label{prop:properties_of_the_lebesgue_integral_vector_valued_case}
    Let $(X,\mu)$ be a measure space, and let $f,g: X \to \R^m$ be measurable functions.
    Then the following properties hold:
    \begin{enumerate}[(i)]
        \item (Linearity) For each $\lambda\in \R$ we have
            \[ \int_X (f + \lambda g) \,\dif \mu = \int_X f \,\dif \mu + \lambda \int_X g \,\dif \mu. \]
        \item (Triangle Inequality) We have
            \[ \left\| \int_X f \,\dif \mu \right\| \leq \int_X \| f \| \,\dif \mu. \]
    \end{enumerate}
\end{proposition}

\begin{proof}
    Letting $f_1,f_2,\ldots,f_m: X \to \R$ and $g_1,g_2,\ldots,g_m: X \to \R$ be the component functions of $f$ and $g$ respectively, we have
    \begin{align*}
        \int_X (f + \lambda g) \,\dif \mu &= \left( \int_X (f_1 + \lambda g_1) \,\dif \mu, \ldots, \int_X (f_m + \lambda g_m) \,\dif \mu \right) \\
            &= \left( \int_X f_1 \,\dif \mu + \lambda \int_X g_1 \,\dif \mu, \ldots, \int_X f_m \,\dif \mu + \lambda \int_X g_m \,\dif \mu \right) &&\text{by linearity for real-valued case} \\
            &= \left( \int_X f_1 \,\dif \mu, \ldots, \int_X f_m \,\dif \mu \right) + \lambda \left( \int_X g_1 \,\dif \mu, \ldots, \int_X g_m \,\dif \mu \right) \\
            &= \int_X f \,\dif \mu + \lambda \int_X g \,\dif \mu.
    \end{align*}

\vspace{2mm}

If $\int_X f \,\dif \mu = 0 \in \R^m$, the inequality is trivial, so we may assume that $\int_X f \,\dif \mu \neq 0$.

Let $v \in \R^m \setminus \{ 0 \}$ be arbitrary.
Then
\begin{align*}
    \left\langle v, \int_X f \,\dif \mu \right\rangle &= \sum_j v_j \int_X f_j \,\dif \mu \\
        &= \int_X \sum_j v_j f_j(x) \,\dif \mu(x) \\
        &=\int_X \langle v, f(x) \rangle \,\dif \mu(x) \\
        &\leq \int_X \| v \| \| f(x) \| \,\dif \mu(x) && \text{by Cauchy-Schwarz Inequality} \\
        &= \| v \| \int_X \| f(x) \| \,\dif \mu(x).
\end{align*}
By taking \[ v := \frac{ \int_X f \,\dif \mu }{\| \int_X f \,\dif \mu \|} \in \R^m \setminus \{0\} \]
which has norm $1$, we get
\[ \left\langle \frac{ \int_X f \,\dif \mu }{\| \int_X f \,\dif \mu \|}, \int_X f \,\dif \mu \right\rangle = \left\| \int_X f \,\dif \mu \right\| \]
and thus \[ \left\| \int_X f \,\dif \mu \right\| \leq \int_X \| f(x) \| \,\dif \mu(x). \]
\end{proof}