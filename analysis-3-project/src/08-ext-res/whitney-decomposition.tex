\section{Decomposition of Open Sets into Cubes}

In this section, we will construct a particularly nice decomposition of an open set into almost disjoint cubes.
We will then use this decomposition to construct a nice partition of unity for the open set, which will be a crucial tool for defining the extension operators in the next section.

\subsection{The Whitney Decomposition}

\begin{definition}[Dyadic Cubes]
    \label{def:dyadic_cubes}
For each $k \in \Z$, we let $\mathcal{D}_k$ be the collection of all dyadic closed cubes of the form
\[ Q = \{ (x_1, \ldots, x_n) \in\R^n :m_j 2^{-k} \leq x_j \leq (m_j+1) 2^{-k} \text{ for } j=1,\ldots,n \} \]
where $m_j \in \Z$ for each $j=1,\ldots,n$.

The set of all dyadic cubes is defined as $\mathcal{D} = \bigcup_{k\in \Z} \mathcal{D}_k$.
\end{definition}

\begin{exercise}[Basic Facts about Dyadic Cubes]
    \label{ex:dyadic_cubes}
    \begin{enumerate}[(i)]
        \item Let $k\in \Z$. 
            For each $Q \in \mathcal{D}_k$ there exists $2^n$ almost disjoint cubes $Q_1, \ldots, Q_{2^n} \in \mathcal{D}_{k+1}$ such that $Q = \bigcup_{j=1}^{2^n} Q_j$.

        \item If $Q,Q' \in \mathcal{D}$ are such that the interiors of $Q$ and $Q'$ have nonempty intersection, then either $Q \subseteq Q'$ or $Q' \subseteq Q$.
            In other words, the dyadic cubes are either almost disjoint or one is contained in the other.

        \item Let $k\in \Z$. For each $Q \in \mathcal{D}_k$ there are exactly $3^n - 1$ other cubes in $\mathcal{D}_k$ which intersect $Q$.            
    \end{enumerate}
\end{exercise}

\begin{proof}
    \begin{enumerate}
        \item Let $Q \in \mathcal{D}_k$ be given by
            \[ Q = \{ (x_1, \ldots, x_n) \in\R^n :m_j 2^{-k} \leq x_j \leq (m_j+1) 2^{-k} \text{ for } j=1,\ldots,n \}. \]
            For each $j=1,\ldots,n$, we define the intervals
            \[ I_j^0 = [m_j 2^{-k}, (m_j + \tfrac{1}{2}) 2^{-k}] \qquad \text{and} \qquad I_j^1 = [(m_j + \tfrac{1}{2}) 2^{-k}, (m_j + 1) 2^{-k}]. \]
            Then we can define the cube $Q_{(i_1, \ldots, i_n)}$ for each $i_j \in \{0,1\}$ by
            \[ Q_{(i_1, \ldots, i_n)} = I_1^{i_1} \times \cdots \times I_n^{i_n}. \]
            Then we have 
            \[Q = \bigcup_{i_1, \ldots, i_n \in \{0,1\}} Q_{(i_1, \ldots, i_n)}, \]
            and the collection of cubes $\{Q_{(i_1, \ldots, i_n)}\}_{i_j \in \{0,1\}}$ are pairwise almost disjoint and belong to $\mathcal{D}_{k+1}$.
            Furthermore, there are $2^n$ such cubes, as claimed.

        \item Let $Q,Q' \in \mathcal{D}$ be such that $Q \cap Q' \neq \varnothing$.
            Then there exist $k,k' \in \Z$ such that $Q \in \mathcal{D}_k$ and $Q' \in \mathcal{D}_{k'}$.
            Without loss of generality, we can assume that $k \leq k'$. 
            We write $Q$ and $Q'$ as
            \[ Q = \prod_{j=1}^n [m_j 2^{-k}, (m_j+1) 2^{-k}] \qquad \text{and} \qquad Q' = \prod_{j=1}^n [m'_j 2^{-k'}, (m'_j+1) 2^{-k'}] \]
            for some integers $m_1, \ldots, m_n$ and $m'_1, \ldots, m'_n$.

            Since the interiors of $Q$ and $Q'$ have nonempty intersection, there exists a point $x = (x_1, \ldots, x_n)$ in their common interior. 
            Then for each $j=1,\ldots,n$, we have
            \[ m_j 2^{-k} < x_j < (m_j + 1) 2^{-k} \qquad \text{and}\qquad  m'_j 2^{-k'} < x_j < (m'_j + 1) 2^{-k'}. \]
            Since $k \leq k'$, we have $2^{-k} \geq 2^{-k'}$, and so for each $j=1,\ldots,n$ the interval $[m_j 2^{-k}, (m_j+1) 2^{-k}]$ contains the interval $[m'_j 2^{-k'}, (m'_j + 1) 2^{-k'}]$.
            Therefore, we have $Q' \subseteq Q$, as claimed.

        \item Write
            \[ Q = \prod_{i=1}^n [m_i 2^{-k},(m_i+1)2^{-k}] \]
            for some $(m_1,\dots,m_n)\in\mathbb{Z}^n$. 
            If 
            \[ Q' = \prod_{i=1}^n [m_i' 2^{-k},(m_i'+1)2^{-k}] \in \mathcal{D}_k \]
            is such that $Q' \cap Q \neq \varnothing$, then for each $j=1,\dots,n$ we must have $|m_j - m_j'| \leq 1$, so that $m'_j \in \{m_j-1,m_j,m_j+1\}$.

            Thus if $Q' \in \mathcal{D}_k$ is such that $Q' \cap Q \neq \varnothing$, then there are $3^n$ possible choices for the vector $(m_1',\dots,m_n')$, and hence there are $3^n$ cubes in $\mathcal{D}_k$ which intersect $Q$.
            However, one of these cubes is $Q$ itself, so there are exactly $3^n - 1$ other cubes in $\mathcal{D}_k$ which intersect $Q$ as claimed.
    \end{enumerate}

\end{proof}

\begin{definition}[Adjacent Cubes]
    \label{def:adjacent_cubes}
    Two cubes $Q,Q' \subset \R^n$ are said to be adjacent if $Q$ and $Q'$ are almost disjoint but $\partial Q \cap \partial Q' \neq \varnothing$.
    In other words, $Q$ and $Q'$ are adjacent if their interiors are disjoint but they share a common boundary point.
\end{definition}

\begin{proposition}[Whitney Decomposition]
    \label{thm:whitney_decomposition}
    Let $A \subset \R^n$ be a non-empty proper closed subset.
    Then there exists a countable collection of almost disjoint dyadic closed cubes $\{Q_j\}_{j=1}^\infty$ such that
    \begin{enumerate}[(a)]
        \item $\displaystyle A^c = \bigcup_{j=1}^\infty Q_j$, and
        \item for each $j\in\Z^+$, we have \[ \diam Q_j \leq \dist(Q_j, A) \leq 4 \diam Q_j. \]
    \end{enumerate}
    As a consequence, note that
    \begin{enumerate}[(a)]
        \setcounter{enumi}{2}
        \item if $j,k\in\Z^+$ are such that $Q_j$ and $Q_k$ are adjacent, then 
            \[ \frac{1}{4} \leq \frac{\diam Q_j}{\diam Q_k} \leq 4. \]
    \end{enumerate}
    As a further consequence, note that
    \begin{enumerate}[(a)]
        \setcounter{enumi}{3}
        \item for each $j\in\Z^+$, there are at most $12^n - 4^n$ cubes in the collection which are adjacent to $Q_j$.
    \end{enumerate}
    The collection of dyadic cubes $\{Q_j\}_{j=1}^\infty$ is called a \textit{Whitney decomposition} of the open set $A^c$.
\end{proposition}

In other words, the key property (b) says the size of a cube in the Whitney decomposition is comparable to its distance from the set $A$.

\begin{proof}[Proof of (a) and (b)]
    For each $k \geq 1$, we let 
    \[ \Omega_k := \left\{ x\in A^c : 2\sqrt{n} 2^{-k} < \dist(x, A) \leq 4 \sqrt{n}\cdot 2^{-k} \right\} \]
    and see that $ \displaystyle A^c = \bigcup_{k=1}^\infty \Omega_k$.

    We define $\mathcal{F}$ to be the collection of all closed cubes $Q \subset \R^n$ such that there exists $k \geq 1$ such that $Q \in \mathcal{D}_k$ and $Q \cap \Omega_k \neq \varnothing$.
    We claim that property (b) holds for the cubes in $\mathcal{F}$.

    Let $Q \in \mathcal{F}$, and let $k\geq 1$ be such that $Q \in \mathcal{D}_k$ and $Q \cap \Omega_k \neq \varnothing$.
    Then choose $x \in Q \cap \Omega_k$, and observe that
    \begin{align*} 
        \diam Q = \sqrt{n}2^{-k} &\leq \dist(x, A) - \sqrt{n}2^{-k} \\
            &= \dist(x, A) - \diam Q \\
            &\leq \dist(Q,A) \\
            &\leq 4\sqrt{n}2^{-k} \\
            &= 4\diam Q 
    \end{align*}
    which gives \[ \diam Q \leq \dist(Q,A) \leq 4\diam Q \]
    as claimed.

    We also claim that
    \[ A^c = \bigcup_{Q\in \mathcal{F}} Q. \]
    Note that each cube in $\mathcal{F}$ is contained in $A^c$, since it has distance at least $\diam Q$ from $A$.
    Thus we have $\bigcup_{Q\in \mathcal{F}} Q \subseteq A^c$.

    Conversely, let $x \in A^c$ be given.
    Then there exists $k\geq 1$ such that $x \in \Omega_k$, and so there exists a cube $Q \in \mathcal{D}_k$ such that $x \in Q$.
    Since $x \in \Omega_k$, we have $\dist(x, A) > 2\sqrt{n} 2^{-k}$, and so $Q \cap \Omega_k \neq \varnothing$.
    Thus $Q \in \mathcal{F}$, and so $x \in \bigcup_{Q\in \mathcal{F}} Q$.
    Since $x$ was arbitrary, we have $A^c \subseteq \bigcup_{Q\in \mathcal{F}} Q$, and so $A^c = \bigcup_{Q\in \mathcal{F}} Q$ as claimed.

            \vspace{2mm}

    Thus the collection $\mathcal{F}$ satisfies properties (a) and (b).
    The problem is that the cubes in $\mathcal{F}$ are not almost disjoint.
    Recall that by Exercise \ref{ex:dyadic_cubes}, if $Q_1, Q_2 \in \mathcal{F}$ and $Q_1^\circ \cap Q_2^\circ \neq \varnothing$, then either $Q_1 \subseteq Q_2$ or $Q_2 \subseteq Q_1$.
    That is, two cubes in $\mathcal{F}$ either have disjoint interiors or one is contained in the other.
    
    Therefore for each cube $Q \in \mathcal{F}$, we let $Q^{\text{max}}$ be the maximal cube in $\mathcal{F}$ which contains $Q$; such a cube exists and is unique by the previous observation.
    By maximality and our previous observation, the maximal cubes have pairwise disjoint interiors, and hence are almost disjoint.
    Noting that the collection $\mathcal{F}$ is countable, we can enumerate the maximal cubes as $\{Q_j\}_{j=1}^\infty$.
    Then $\{Q_j\}_{j=1}^\infty$ is a Whitney cover of $A^c$ because
    \begin{itemize}
        \item the cubes $\{ Q_j\}_{j=1}^\infty$ are almost disjoint by construction;
        \item we have $\displaystyle A^c = \bigcup_{Q\in \mathcal{F}} Q = \bigcup_{j=1}^\infty Q_j$; and
        \item each cube $Q_j$ belongs to $\mathcal{F}$, and hence satisfies $\diam Q_j \leq \dist(Q_j, A) \leq 4\diam Q_j$.
    \end{itemize}
\end{proof}

\begin{proof}[Proof of (c) and (d)]
    \begin{enumerate}[(a)]
        \setcounter{enumi}{2}
        \item Suppose $j,k \geq 1$ are such that the cubes $Q_j$ and $Q_k$ are adjacent.
            Then $Q_j$ and $Q_k$ are almost disjoint, but $\partial Q_j \cap \partial Q_k \neq \varnothing$, and so $\dist(Q_j,Q_k) = 0$.
            Then we see that
            \[ \diam Q_j \leq \dist(Q_j, A) \leq \dist(Q_j, Q_k) + \dist(Q_k, A) \leq 0 + 4\diam Q_k \]
            which gives $\diam Q_j \leq 4\diam Q_k$.

            By symmetry, we also have $\diam Q_k \leq 4\diam Q_j$, and hence 
            \[\frac{1}{4} \leq \frac{\diam Q_j}{\diam Q_k} \leq 4\]
            as desired.

        \item Let $j\in\Z^+$ be arbitrary. 
            Note that the largest number of dyadic cubes adjacent to $Q_j$ is obtained when all of the adjacent cubes have the smallest possible size.
            By part (c), the smallest possible size of a cube adjacent to $Q_j$ is $\frac{1}{4}\diam Q_j$.
            
            We claim that for each $Q \in \mathcal{D}_k$ there are at most $4^n$ cubes in $\{Q_j\}_{j=1}^\infty$ with diameter $\geq \frac{1}{4}\diam Q$ which are contained in $Q$.

        \begin{proof}[Proof of Claim]
            Let $Q \in \mathcal{D}_k$ be arbitrary.
            Then each cube $Q_j$ with $\diam Q_j \geq \frac{1}{4}\diam Q$ which is contained in $Q$ must belong to $\mathcal{D}_k$, or $\mathcal{D}_{k-1}$, or $\mathcal{D}_{k-2}$.
            We cannot have $Q_j \in \mathcal{D}_{k-3}$ because then $\diam Q_j < \frac{1}{4}\diam Q$, and we cannot have $Q_j \in \mathcal{D}_{k+1}$ because then $\diam Q_j > 4\diam Q$ and hence $Q_j$ cannot be contained in $Q$.

            In the first case, there is a unique cube in $\mathcal{D}_k$ which is equal to $Q$, so there is at most one cube in $\{Q_j\}_{j=1}^\infty$ with diameter $\geq \frac{1}{4}\diam Q$ which is contained in $Q$ and belongs to $\mathcal{D}_k$.

            In the second case where $Q_j \in \mathcal{D}_{k-1}$, there are at most $2^n$ cubes in $\mathcal{D}_{k-1}$ which are contained in $Q$, and hence at most $2^n$ cubes in $\{Q_j\}_{j=1}^\infty$ with diameter $\geq \frac{1}{4}\diam Q$ which are contained in $Q$ and belong to $\mathcal{D}_{k-1}$.

            In the third case where $Q_j \in \mathcal{D}_{k-2}$, there are at most $4^n$ cubes in $\mathcal{D}_{k-2}$ which are contained in $Q$, and hence at most $4^n$ cubes in $\{Q_j\}_{j=1}^\infty$ with diameter $\geq \frac{1}{4}\diam Q$ which are contained in $Q$ and belong to $\mathcal{D}_{k-2}$.
        \end{proof}

        Combining the previous claim with exercise \ref{ex:dyadic_cubes} (c) shows that there are at most $4^n(3^n-1)=12^n - 4^n$ cubes in the collection which are adjacent to $Q_j$.
    \end{enumerate}
\end{proof}

\begin{corollary}
    \label{cor:whitney_cover_dilation}    
    Let $A \subseteq \R^n$ be a non-empty proper closed subset, and let $\{Q_j\}_{j=1}^\infty$ be the Whitney decomposition of $A^c$ given by Proposition \ref{thm:whitney_decomposition}.

    Fix $0 < \varepsilon < \frac{2}{5}$. 
    For each $j\in\Z^+$, let $Q_j^* = (1+\varepsilon) Q_j$ be the cube obtained by dilating $Q_j$ by a factor of $1+\varepsilon$ about its center.
    
\vspace{2mm}

    \noindent Then for each $j\in\Z^+$, we have $Q_j^* \subseteq A^c$.
    Also if $j,k\in\Z^+$ are such that $Q_j \cap Q_k = \varnothing$, then $Q_j^* \cap Q^*_k = \varnothing$.

    Furthermore, 
    \[ \Chi_{A^c} \leq \sum_{j=1}^\infty \Chi_{Q^*_j} \leq 2^n\Chi_{A^c}. \]
\end{corollary}

\begin{proof}
    Let $j\in\Z^+$ be arbitrary.
    Then we have $Q_j^* = (1+\varepsilon) Q_j$, and since each cube adjacent to $Q_j$ has diameter at least $\frac{1}{4}\diam Q_j$, we see that $Q_j^*$ is contained in the union of $Q_j$ and the cubes adjacent to $Q_j$.
    Since each cube adjacent to $Q_j$ is contained in $A^c$, we have $Q_j^* \subseteq A^c$ as claimed.

    Moreover, see that 
    \[ A^c = \bigcup_{j=1}^\infty Q_j \subseteq \bigcup_{j=1}^\infty Q_j^* \]
    implies that 
    \[ \Chi_{A^c} \leq \Chi_{\bigcup_{j=1}^\infty Q_j^*} = \sum_{j=1}^\infty \Chi_{Q_j^*} \]
    which is the first inequality.

    \vspace{2mm}

    Suppose now that $j,k\in\Z^+$ are such that $Q_j \cap Q_k = \varnothing$.
    Then $Q_j$ and $Q_k$ are not adjacent, and since $Q_j^*$ is contained in the union of $Q_j$ and the cubes adjacent to $Q_j$, we have $Q_j^* \cap Q_k = \varnothing$. 
    We claim that in fact $Q_j^* \cap Q_k^* = \varnothing$.
    
    \begin{proof}[Proof of Claim]
        Since $Q_j^*$ is contained in the union of $Q_j$ and the cubes adjacent to $Q_j$, and since $Q_k^*$ is contained in the union of $Q_k$ and the cubes adjacent to $Q_k$, we see that for the intersection $Q_j^* \cap Q_k^*$ to be nonempty, there must be a cube in the Whitney decomposition which is adjacent to both $Q_j$ and $Q_k$.
        Let $Q_l$ be a cube in the Whitney decomposition which is adjacent to both $Q_j$ and $Q_k$.
        Then by part (c) of Proposition \ref{thm:whitney_decomposition}, we have
        \[ \frac{1}{4} \leq \frac{\diam Q_j}{\diam Q_l} \leq 4 \qquad \text{and} \qquad \frac{1}{4} \leq \frac{\diam Q_k}{\diam Q_l} \leq 4. \]
        Without loss of generality, we can assume that $\diam Q_j \leq \diam Q_k$.
        Then we must have $\diam Q_k \leq 16 \diam Q_j$ by combining the previous inequalities, which implies that 
        \[ \diam Q_k = c\cdot \diam Q_j \qquad\text{for some } c \in \{ 1,2, 4, 8, 16\}. \]
        \textbf{DRAW PICTURE}

        Assume that $\diam Q_k = \diam Q_j$.
        Then $Q_j$ and $Q_k$ are non-adjacent dyadic cubes in $\mathcal{D}_N$ for some $N\in\Z$, and hence must have $\dist(Q_j, Q_k) \geq 2^{-N} = \ell(Q_j)$.
        (Why? The dyadic cubes in $\mathcal{D}_N$ are arranged in a grid, and two cubes in $\mathcal{D}_k$ are adjacent or separated by at least one cube in $\mathcal{D}_N$.)
        Hence, the cubes $Q_j^*$ and $Q_k^*$ have a positive distance
        \begin{align*}
            \dist(Q_j^* ,Q_k^*) &> \dist\left( \left(\frac{7}{5}Q_j\right) , \left(\frac{7}{5}Q_k\right) \right) \\
                &\geq \dist(Q_j ,Q_k) - \frac{1}{2}\frac{2}{5}\ell(Q_j) - \frac{1}{2}\frac{2}{5}\ell(Q_k) \\
                &\geq \ell(Q_j) - \frac{1}{5}\ell(Q_j) - \frac{1}{5}\ell(Q_k) \\
                &= \frac{3}{5}\ell(Q_j) > 0
        \end{align*}
        where the first inequality follows from the fact that $Q_j^* = (1+\varepsilon) Q_j$ and $1 + \varepsilon < \frac{7}{5}$, and the second inequality follows from the triangle inequality.

        Now assume that $\diam Q_k = 2 \diam Q_j$.
        Then $Q_j$ and $Q_k$ are non-adjacent dyadic cubes in $\mathcal{D}_{N+1}$ and $\mathcal{D}_{N}$ respectively for some $N\in\Z$, and hence must have $\dist(Q_j, Q_k) \geq 2^{-N-1} = \frac{1}{2}\ell(Q_k)$.
        (Why? Since $Q_k\in \mathcal{D}_{N}$ we see that $Q_k$ is a union of $2^n$ cubes in $\mathcal{D}_{N+1}$, and hence by identical reasoning as in the previous case with one of these cubes in $\mathcal{D}_{N+1}$ we see that $Q_j$ and $Q_k$ must be separated by at least one cube in $\mathcal{D}_{N+1}$, which has side length $\ell(Q_j) = \frac{1}{2}\ell(Q_k)$.)
        Hence the cubes $Q_j^*$ and $Q_k^*$ have a positive distance
        \begin{align*}
            \dist(Q_j^* ,Q_k^*) &> \dist\left( \left(\frac{7}{5}Q_j\right) , \left(\frac{7}{5}Q_k\right) \right) \\
                &\geq \dist(Q_j ,Q_k) - \frac{1}{2}\frac{2}{5}\ell(Q_j) - \frac{1}{2}\frac{2}{5}\ell(Q_k) \\
                &\geq \frac{1}{2}\ell(Q_k) - \frac{1}{5}\ell(Q_j) - \frac{1}{5}\ell(Q_k) \\
                &= \frac{3}{10}\ell(Q_k) - \frac{1}{10}\ell(Q_k) = \frac{1}{5}\ell(Q_k) > 0.
        \end{align*}

        Now assume that $\diam Q_k = 4 \diam Q_j$.
        Then $Q_j$ and $Q_k$ are non-adjacent dyadic cubes in $\mathcal{D}_{N+2}$ and $\mathcal{D}_{N}$ respectively for some $N\in\Z$, and hence must have $\dist(Q_j, Q_k) \geq 2^{-N-2} = \frac{1}{4}\ell(Q_k)$.
        (Why? Since $Q_k\in \mathcal{D}_{N-2}$ we see that $Q_k$ is a union of $2^{2n}$ cubes in $\mathcal{D}_{N}$, and hence by identical reasoning as in the previous case with one of these cubes in $\mathcal{D}_{N}$ we see that $Q_j$ and $Q_k$ must be separated by at least one cube in $\mathcal{D}_{N}$, which has side length $\ell(Q_j)$.)
        Hence the cubes $Q_j^*$ and $Q_k^*$ have a positive distance
        \begin{align*}
            \dist(Q_j^* ,Q_k^*) &> \dist\left( \left(\frac{7}{5}Q_j\right) , \left(\frac{7}{5}Q_k\right) \right) \\
                &\geq \dist(Q_j ,Q_k) - \frac{1}{2}\frac{2}{5}\ell(Q_j) - \frac{1}{2}\frac{2}{5}\ell(Q_k) \\
                &\geq \frac{1}{4}\ell(Q_k) - \frac{1}{5}\ell(Q_j) - \frac{1}{5}\ell(Q_k) \\
                &= \frac{5}{20} \ell(Q_k) - \frac{1}{20}\ell(Q_k) - \frac{4}{20}\ell(Q_k) = 0.
        \end{align*}

        In the cases that $\diam Q_k = 8 \diam Q_j$ or $\diam Q_k = 16 \diam Q_j$, the cubes $Q_j$ and $Q_k$ must have distance at least $ \frac{1}{4}\ell(Q_k)$, because the smallest cube adjacent to $Q_k$ has side length at least $\frac{1}{4}\ell(Q_k)$.
        In these cases, the cubes $Q^*_j$ and $Q^*_k$ have a positive distance, as 
        \begin{align*}
            \dist(Q_j^* ,Q_k^*) &\geq \dist\left( \left(\frac{7}{5}Q_j\right) , \left(\frac{7}{5}Q_k\right) \right) \\
                &\geq \dist(Q_j ,Q_k) - \frac{1}{2}\frac{2}{5}\ell(Q_j) - \frac{1}{2}\frac{2}{5}\ell(Q_k) \\
                &\geq \frac{1}{4}\ell(Q_k) - \frac{1}{5}\ell(Q_j) - \frac{1}{5}\ell(Q_k) \\
                &= \frac{1}{20}\ell(Q_k) - \frac{1}{5}\ell(Q_j) 
        \end{align*}
        and in the case $\diam Q_k = 8 \diam Q_j$ we have $\ell(Q_j) = \frac{1}{8}\ell(Q_k)$, and in the case $\diam Q_k = 16 \diam Q_j$ we have $\ell(Q_j) = \frac{1}{16}\ell(Q_k)$, so that in either case we have $\dist(Q_j^* ,Q_k^*) > 0$.

        This case analysis shows that if $Q_j \cap Q_k = \emptyset$, then $Q_j^*$ and $Q_k^*$ are disjoint, as claimed.
    \end{proof}

    It remains to prove the final inequality.

    \vspace{2mm}

    We claim that the maximum number of pairwise adjacent cubes in the Whitney decomposition such that their dilations by a factor of $1+\varepsilon$ contain a common point is at most $2^n$.

    \begin{proof}[Proof of Claim]
        Assume that there are cubes $Q_{j_1}, \ldots, Q_{j_M}$ in the Whitney decomposition, and that the intersection of their dilations is nonempty, i.e. $Q_{j_1}^* \cap \cdots \cap Q_{j_M}^* \neq \emptyset$.
        By the contrapositive of the previous claim, we see that the intersection of any pair of cubes in the collection $\{Q_{j_1}, \ldots, Q_{j_M}\}$ is nonempty.
        If $M < 2^n$, then we are done.
        The only way to have $M \geq 2^n$ is if the cubes $Q_{j_1}, \ldots, Q_{j_M}$ all have a common corner vertex, and hence are all adjacent to each other, and this forces $M = 2^n$.
    \end{proof}

    In particular, for each point $x \in A^c$, there are at most $2^n$ cubes in the collection $\{Q_j^*\}_{j=1}^\infty$ which contain $x$, and hence
    \[ \sum_{j=1}^\infty \Chi_{Q_j^*}(x) \leq 2^n \]
    for each $x \in A^c$ as claimed.

    If $x \in A$, then $x$ cannot belong to any of the cubes in the Whitney decomposition (as their union equals $A^c$), and hence for each $j\in\Z^+$ we have $x \notin Q_j^*$, since $Q_j^*$ is contained in the union of $Q_j$ and the cubes adjacent to $Q_j$.
    Thus, for each $x \in A$, we have $\sum_{j=1}^\infty \Chi_{Q_j^*}(x) = 0$, so that
    \[ \sum_{j=1}^\infty \Chi_{Q_j^*}(x) = 0 = 2^n \Chi_{A^c}(x) \]
    as claimed. 

    This completes the proof of the final inequality, and hence the proof of the corollary.
\end{proof}