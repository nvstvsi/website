\section{Hardy-Littlewood Maximal Function and Lebesgue Differentiation Theorem}

\subsection{Differentiation of Integrals}

Suppose $f: [a,b]\to \R$ is an integrable function, and we let 
\[ F(x) := \int_a^x f(t) \,\dif t \qquad \forall x \in [a,b]. \]
If we want to differentiate the function $F$ at a point $x \in (a,b)$, we must use the difference quotient
\[ \frac{F(x+h) - F(x)}{h} = \frac{1}{h} \int_x^{x+h} f(t) \,\dif t. \]
We pause for a moment to note that this is precisely the average value of $f$ on the interval $(x,x+h)$ when $h > 0$, and on the interval $(x+h,x)$ when $h < 0$.
If we expect anything like the Fundamental Theorem of Calculus to hold, we would want this average value to converge to $f(x)$ as $h \to 0$.
That is, we would want
\[ \frac{1}{|I|} \int_I f(t) \,\dif t \to f(x) \quad \text{as } |I| \to 0, \]
where $I$ is any interval containing $x$, and $|I|$ denotes its length.
We can ask that $x$ be the center of $I$, as $I$ shrinks to $x$, or that only $x\in I$ as $|I| \to 0$.

Moving into higher dimensions, we can ask a similar question.
If $f$ is an integrable function on an open set $U \subseteq \R^n$, can we hope to have
\[ \frac{1}{\vol B} \int_B f(y) \,\dif y \to f(x) \quad \text{as } \vol B \to 0, \]
where $B$ is any ball containing $x$, and $\vol B$ denotes its volume?
This is known as the \textbf{averaging problem}.

\begin{exercise}[Continuity and the Averaging Problem]
    \label{ex:continuity_and_averaging}
    Let $f: U\to \R$ be an integrable function on an open set $U \subseteq \R^n$, and let $x \in U$ be such that $f$ is continuous at $x$.
    Show that
    \[ \frac{1}{\vol B} \int_B f(y) \,\dif y \to f(x) \quad \text{as } \vol B \to 0, \]
    where $B$ is any ball containing $x$.
\end{exercise}
\begin{proof}
    Let $\varepsilon > 0$ be arbitrary.
    Since $f$ is continuous at $x$, there exists $\delta > 0$ such that
    \[ |f(y) - f(x)| < \varepsilon \quad \forall y \in B_n(x,\delta). \]
    Now, let $B$ be any ball containing $x$ with volume $\vol B < \vol B_n(0,\delta)$.
    Then $B \subseteq B_n(x,\delta)$, and we have
    \begin{align*}
        \left| \frac{1}{\vol B} \int_B f(y) \,\dif y - f(x) \right|
            &= \left| \frac{1}{\vol B} \int_B (f(y) - f(x)) \,\dif y \right| \\
            &\leq \frac{1}{\vol B} \int_B |f(y) - f(x)| \,\dif y \\
            &< \varepsilon.
    \end{align*}
    Since $\varepsilon > 0$ was arbitrary, the result follows.
\end{proof}

\subsection{The Hardy-Littlewood Maximal Function}

\begin{definition}
    \label{def:l1_loc_functions}
    Let $U \subseteq \R^n$ be open.
    A function $f : U \to \R$ is said to be \textit{locally integrable} on $U$ if for every compact set $K \subseteq U$, we have $f \in L^1(K)$.
    The collection of all locally integrable functions on $U$ is denoted by $L^1_{\text{loc}}(U)$, where we identify functions that are equal almost everywhere.
\end{definition}
That is, an element of $L^1_{\text{loc}}(U)$ is formally an equivalence class of functions that are equal almost everywhere, but we will often abuse notation and refer to a particular representative of this equivalence class as the function itself.

Later we will generalize this definition to arbitrary Borel outer measures, but for now this will suffice.

\begin{definition}
    \label{def:hardy_littlewood_maximal_function}
    Let $f \in L^1_{\text{loc}}(\R^n)$.
    The \textit{(uncentered) Hardy-Littlewood maximal function} of $f$ is the function $M f : \R^n \to [0,\infty]$ defined by
    \[ M f(x) := \sup_{ B \ni x } \frac{1}{\vol B} \int_B |f(y)| \,\dif y, \]
    where the supremum is taken over all balls $B$ containing $x$.

    \vspace{2mm}

    \noindent Similarly the \textit{(centered) Hardy-Littlewood maximal function} of $f$ is the function $M_c f : \R^n \to [0,\infty]$ defined by
    \[ M_c f(x) := \sup_{B_n(x,r)} \frac{1}{\vol B_n(x,r)} \int_{B_n(x,r)} |f(y)| \,\dif y, \]
    where the supremum is taken over all balls $B_n(x,r)$ centered at $x$.
\end{definition}

For many pruposes, we can work interchangeably with either the centered or uncentered maximal function, as they are pointwise comparable.

\begin{exercise}[Pointwise Comparison of Centered and Uncentered Maximal Functions]
    \label{ex:pointwise_comparison_maximal_functions}
    Let $f \in L^1_{\text{loc}}(\R^n)$.
    Show that for each $x \in \R^n$, we have
    \[ M_c f(x) \leq M f(x) \leq 2^n M_c f(x). \]
\end{exercise}
\begin{proof}
    Let $x \in \R^n$ be arbitrary.
    For any $r > 0$, we have
    \[ \frac{1}{\vol B_n(x,r)} \int_{B_n(x,r)} |f(y)| \,\dif y \leq M f(x), \]
    since $B_n(x,r)$ is a ball containing $x$.
    Taking the supremum over all $r > 0$, we obtain $M_c f(x) \leq M f(x)$.

    Conversely, let $B$ be any ball containing $x$.
    Then let $r := \text{radius}(B)$, and see that $B \subseteq B_n(x,2r)$.
    Thus,
    \begin{align*}
        \frac{1}{\vol B} \int_B |f(y)| \,\dif y
            &\leq \frac{1}{\vol B_n(x,2r)} \int_{B_n(x,2r)} |f(y)| \,\dif y \\
            &= \frac{1}{2^n \vol B_n(x,r)} \int_{B_n(x,2r)} |f(y)| \,\dif y \\
            &\leq 2^n M_c f(x).
    \end{align*}
    Taking the supremum over all balls $B$ containing $x$, we obtain $M f(x) \leq 2^n M_c f(x)$.
\end{proof}

\begin{exercise}[Sublinearity of the Hardy-Littlewood Maximal Function]
    \label{ex:sublinearity_of_maximal_function}
    Let $f,g \in L^1_{\text{loc}}(\R^n)$ and $\alpha \in \R$.
    Show that
    \[ M(f + g)(x) \leq M f(x) + M g(x) \quad \text{and} \quad M(\alpha f)(x) = |\alpha| M f(x) \quad \forall x \in \R^n. \]
\end{exercise}
\begin{proof}
    It's easy. 
    Let $x \in \R^n$ be arbitrary.
    For each ball $B$ containing $x$, we have
    \begin{align*}
        \frac{1}{\vol B} \int_B |f(y) + g(y)| \,\dif y
            &\leq \frac{1}{\vol B} \int_B |f(y)| \,\dif y + \frac{1}{\vol B} \int_B |g(y)| \,\dif y \\
            &\leq M f(x) + M g(x).
    \end{align*}
    Taking the supremum over all balls $B$ containing $x$, we obtain $M(f + g)(x) \leq M f(x) + M g(x)$.

    Similarly, for each ball $B$ containing $x$, we have
    \[
        \frac{1}{\vol B} \int_B |\alpha f(y)| \,\dif y = |\alpha| \frac{1}{\vol B} \int_B |f(y)| \,\dif y \leq |\alpha| M f(x).
    \]
    Taking the supremum over all balls $B$ containing $x$, we obtain $M(\alpha f)(x) = |\alpha| M f(x)$.
\end{proof}

\begin{proposition}[Properties of the Hardy-Littlewood Maximal Function]
    \label{prop:maximal_function_properties}
    Let $f \in L^1_{\text{loc}}(\R^n)$.
    Then the following properties hold:
    \begin{enumerate}[(i)]
        \item $M f$ is measurable,
        \item if $f\in L^1(\R^n)$, then $M f(x) < \infty$ for almost every $x \in \R^n$, and
        \item if $f\in L^1(\R^n)$, then for each $\alpha > 0$ the function $M f$ satisfies
            \[ \mathcal{L}^n(\{ x \in \R^n : M f(x) > \alpha \}) \leq \frac{3^n}{\alpha} \|f\|_{L^1(\R^n)} \]
    \end{enumerate}
\end{proposition}

The inequality in part (iii) is known as the \textit{Hardy-Littlewood Maximal Inequality}, or the \textit{weak $(1,1)$ estimate} for the maximal function.
The notion of weak estimates will be made precise in later chapters.

\begin{proof}
(i) Let $\alpha > 0$ be arbitrary.
    Then we see that the set $\{ x \in \R^n : M f(x) > \alpha \}$ is open because if $x$ is in this set, then
    \[ \sup_{r > 0} \frac{1}{\vol B_n(x,r)} \int_{B_n(x,r)} |f(y)| \,\dif y > \alpha. \]
    Thus there exists a ball $B$ containing $x$ such that
    \[ \frac{1}{\vol B} \int_B |f(y)| \,\dif y > \alpha. \]
    and hence for each $\bar{x} \in B$ we also have $M f(\bar{x}) > \alpha$.
    Since $\alpha > 0$ was arbitrary, we conclude that $M f$ is measurable.

\vspace{2mm}

Now we assume that $f \in L^1(\R^n)$, i.e. $\| f \|_{L^1(\R^n)} < \infty$.
We will prove parts (iii) and then show that part (ii) follows from part (iii).

\vspace{2mm}

(iii) Let $\alpha > 0$ be arbitrary, and let
    \[ E_\alpha := \{ x \in \R^n : M f(x) > \alpha \}. \]
    For each $x \in E_\alpha$, there exists a ball $B_x$ containing $x$ such that
    \[ \frac{1}{\vol B_x} \int_{B_x} |f(y)| \,\dif y > \alpha. \]
    Thus, we have
    \[ E_\alpha \subseteq \bigcup_{x \in E_\alpha} B_x. \]

    Fix a compact subset $K \subseteq E_\alpha$.
    Then the collection $\{ B_x : x \in K \}$ is an open cover of $K$, so there exists a finite subcover $\{ B_j \}_{j=1}^N$ such that
    \[ K \subseteq \bigcup_{j=1}^N B_j. \]
    By the Vitali Covering Lemma (\ref{lem:finite_3r_covering_lemma}), there exists a disjoint subcollection $\{ B_{j_k} \}_{k=1}^M$ such that
    \[ K \subseteq \bigcup_{k=1}^M 3 B_{j_k}. \]
    Therefore, we have
    \begin{align*}
        \mathcal{L}^n(K) &\leq \sum_{k=1}^M \mathcal{L}^n(3 B_{j_k}) \\
            &= \sum_{k=1}^M 3^n \mathcal{L}^n(B_{j_k}) \\
            &< \sum_{k=1}^M \frac{3^n}{\alpha} \int_{B_{j_k}} |f(y)| \,\dif y \\
            &= \frac{3^n}{\alpha} \int_{\bigcup_{k=1}^M B_{j_k}} |f(y)| \,\dif y \\
            &\leq \frac{3^n}{\alpha} \|f\|_{L^1(\R^n)}
    \end{align*}
    since the balls $B_{j_k}$ are disjoint.
    Since $K \subseteq E_\alpha$ was an arbitrary compact subset, we conclude that
    \[ \mathcal{L}^n(E_\alpha) \leq \frac{3^n}{\alpha} \|f\|_{L^1(\R^n)}. \]

(ii) See that for each $\alpha > 0$, we have
    \[ \{ x \in \R^n : M f(x) = \infty \} \subseteq \{ x\in \R^n : M f(x) > \alpha \} \]
    so we must have that
    \[ \mathcal{L}^n(\{ x \in \R^n : M f(x) = \infty \}) \leq \mathcal{L}^n(\{ x\in \R^n : M f(x) > \alpha \}) \leq \frac{3^n}{\alpha} \|f\|_{L^1(\R^n)} \]
    for all $\alpha > 0$ by part (iii).
    Letting $\alpha \to \infty$, we conclude that $\mathcal{L}^n(\{ x \in \R^n : M f(x) = \infty \}) = 0$.
\end{proof}

\subsection{The Lebesgue Differentiation Theorem}

We will now show that the estimate in Proposition \ref{prop:maximal_function_properties}(iii) allows us to solve the averaging problem for integrable functions.

\begin{theorem}[Lebesgue Differentiation Theorem]
    \label{thm:lebesgue_differentiation_theorem}
    Let $f \in L^1_{\text{loc}}(\R^n)$.
    Then for almost every $x \in \R^n$, we have
    \[ \lim_{\substack{ \vol B \to 0 \\ x\in B }} \frac{1}{\vol B} \int_B f(y) \,\dif y = f(x). \]
\end{theorem}

\begin{proof}
    \textit{Step 1:} First we assume that $f \in L^1(\R^n)$.

    It suffices to show that for each $\alpha > 0$ the set
    \[  E_\alpha := \left\{ x \in \R^n : \limsup_{\substack{ \vol B \to 0 \\ x\in B }} \left| \frac{1}{\vol B} \int_B f(y) \,\dif y - f(x) \right| > 2\alpha \right\} \]
    has measure zero.
    Indeed, if this is the case, then we see that $ E := \bigcup_{m=1}^\infty E_{1/m}$ has measure zero, and for each $x \notin E$ we have
    \[ \limsup_{\substack{ \vol B \to 0 \\ x\in B }} \left| \frac{1}{\vol B} \int_B f(y) \,\dif y - f(x) \right| < 2 \cdot \frac{1}{m} \quad \forall m \in \N, \]
    which implies that
    \[ \lim_{\substack{ \vol B \to 0 \\ x\in B }} \frac{1}{\vol B} \int_B f(y) \,\dif y = f(x). \]

    So, let $\alpha > 0$ be arbitrary, and let $\epsilon > 0$.
    Since $f \in L^1(\R^n)$, there exists $g \in C_c(\R^n)$ such that
    \[ \| f - g \|_{L^1(\R^n)} < \epsilon \]
    since $C_c(\R^n)$ is dense in $L^1(\R^n)$ by Lemma \ref{thm:density_of_Cc_in_Lp_on_lch_space}.
    Since $g$ is continuous almost everywhere, by Exercise \ref{ex:continuity_and_averaging} we have
    \[ \lim_{\substack{ \vol B \to 0 \\ x\in B }} \frac{1}{\vol B} \int_B g(y) \,\dif y = g(x) \quad \text{for each } x \in \R^n. \]
    Now
    \[ \frac{1}{\vol B} \int_B f(y) \,\dif y - f(x) = \frac{1}{\vol B} \int_B (f(y) - g(y)) \,\dif y + \frac{1}{\vol B} \int_B g(y) \,\dif y - g(x) + g(x) - f(x) \]
    implies that
    \[ \limsup_{ \substack{ \vol B \to 0 \\ x\in B } } \left| \frac{1}{\vol B} \int_B f(y) \,\dif y - f(x) \right| \leq M(f-g) + |g(x) - f(x)| \]
    by taking the limit superior (the middle term goes to zero by continuity of $g$).
    See that
    \[ E_\alpha \subseteq \{ x : M(f-g)(x) > \alpha \} \cup \{ x : |g(x) - f(x)| > \alpha \} \]
    and Chebyshev's inequality \ref{lem:chebyshevs_inequality} implies that
    \[ \mathcal{L}^n( \{ x : |f(x) - g(x)| > \alpha \} ) \leq \frac{1}{\alpha} \| f-g \|_{L^1(\R^n)} \]
    and the weak $(1,1)$ estimate from Proposition \ref{prop:maximal_function_properties}(iii) implies that
    \[ \mathcal{L}^n( \{ x : M(f-g)(x) > \alpha \} ) \leq \frac{3^n}{\alpha} \| f-g \|_{L^1(\R^n)}. \]
    Therefore, we have
    \[ \mathcal{L}^n(E_\alpha) \leq \frac{(3^n + 1)}{\alpha} \| f-g \|_{L^1(\R^n)} < \frac{(3^n + 1)}{\alpha} \epsilon. \]
    Since $\epsilon > 0$ was arbitrary, we conclude that $\mathcal{L}^n(E_\alpha) = 0$ and thus the result holds when $f \in L^1(\R^n)$.

    \vspace{2mm}
    \textit{Step 2:} Now we consider the general case when $f \in L^1_{\text{loc}}(\R^n)$.
    For each $m \in \Z^+$, let $f_m := f \cdot \chi_{B_n(0,m)}$.
    Then $f_m \in L^1(\R^n)$ for each $m \in \N$, so by Step 1 we have
    \[ \lim_{\substack{ \vol B \to 0 \\ x\in B }} \frac{1}{\vol B} \int_B f_m(y) \,\dif y = f_m(x) \quad \text{for almost every } x \in \R^n. \]
    Let $A_m$ denote the set of measure zero where this limit fails for $f_m$.
    Then for each $x \notin \bigcup_{m=1}^\infty A_m$ and all $m\in \Z^+$ such that $\| x \| < m$, we have $f_m(x) = f(x)$ and
    \[ \lim_{\substack{ \vol B \to 0 \\ x\in B }} \frac{1}{\vol B} \int_B f(y) \,\dif y = \lim_{\substack{ \vol B \to 0 \\ x\in B }} \frac{1}{\vol B} \int_B f_m(y) \,\dif y = f_m(x) = f(x) \]
    That is, the set $\bigcup_{m=1}^\infty A_m$ has measure zero, and for each $x$ not in this set we have
    \[ \lim_{\substack{ \vol B \to 0 \\ x\in B }} \frac{1}{\vol B} \int_B f(y) \,\dif y = f(x) \]
    as desired.
\end{proof}

\begin{corollary}[Maximal Function Bounds Function Almost Everywhere]
    \label{cor:maximal_function_bounds_function}
    Let $f \in L^1_{\text{loc}}(\R^n)$.
    Then for almost every $x \in \R^n$, we have $|f(x)| \leq M f(x)$ for almost every $x \in \R^n$.
\end{corollary}

\begin{proof}
    We have $|f| \in L^1_{\text{loc}}(\R^n)$ since $f \in L^1_{\text{loc}}(\R^n)$.
    By Theorem \ref{thm:lebesgue_differentiation_theorem}, for almost every $x \in \R^n$ we have
    \[ \lim_{\substack{ \vol B \to 0 \\ x\in B }} \frac{1}{\vol B} \int_B |f(y)| \,\dif y = |f(x)|. \]
    Since
    \[ M f(x) = \sup_{ B \ni x } \frac{1}{\vol B} \int_B |f(y)| \,\dif y \geq \lim_{\substack{ \vol B \to 0 \\ x\in B }} \frac{1}{\vol B} \int_B |f(y)| \,\dif y, \]
    we conclude that $|f(x)| \leq M f(x)$ for almost every $x \in \R^n$.
\end{proof}

\begin{corollary}[Measure Theoretic Interior and Exterior]
    \label{cor:measure_theory_interior_and_exterior}
    Let $A \subseteq \R^n$ be a Lebesgue measurable set.
    Then for almost every $x \in A$, we have
    \[ \lim_{ r\to 0^+ } \frac{\mathcal{L}^n( B_n(x,r) \cap A )}{\mathcal{L}(B_n(x,r))} = 1 \]
    and for almost every $x \in \R^n \setminus A$, we have
    \[ \lim_{ r\to 0^+ } \frac{\mathcal{L}^n( B_n(x,r) \cap A )}{\mathcal{L}(B_n(x,r))} = 0. \]
\end{corollary}
Traditionally, the sets of points $x\in \R^n$ where the above limit equals $1$ are called the \textit{measure theoretic interior} of $A$, and the set of points where the limit equals $0$ are called the \textit{measure theoretic exterior} of $A$.
Later, we will study a kind of measure theoretic boundary as well.

\begin{proof}
    Since $A \subseteq \R^n$ is Lebesgue measurable, we have $\chi_A \in L^1_{\text{loc}}(\R^n)$.
    By Theorem \ref{thm:lebesgue_differentiation_theorem}, for almost every $x \in \R^n$ we have
    \[ \lim_{ r\to 0^+ } \frac{1}{\mathcal{L}(B_n(x,r))} \int_{B_n(x,r)} \chi_A(y) \,\dif y = \chi_A(x). \]
    If $x \in A$ is such a point, then $\chi_A(x) = 1$ and hence
    \[ \lim_{ r\to 0^+ } \frac{\mathcal{L}^n( B_n(x,r) \cap A )}{\mathcal{L}(B_n(x,r))} = 1. \]
    If $x \in \R^n \setminus A$ is such a point, then $\chi_A(x) = 0$ and hence
    \[ \lim_{ r\to 0^+ } \frac{\mathcal{L}^n( B_n(x,r) \cap A )}{\mathcal{L}(B_n(x,r))} = 0. \]
\end{proof}

\begin{definition}[Lebesgue Point]
    \label{def:lebesgue_point}
    Let $f \in L^1_{\text{loc}}(\R^n)$.
    A point $x \in \R^n$ is called a \textit{Lebesgue point} of $f$ if
    \[ \lim_{\substack{ \vol B \to 0 \\ x\in B }} \frac{1}{\vol B} \int_B |f(y) - f(x)| \,\dif y = 0. \]
    The set of all Lebesgue points of $f$ is called the \textit{Lebesgue set} of $f$.
\end{definition}
\begin{remark}
    By Exercise \ref{ex:continuity_and_averaging}, every point where $f$ is continuous is a Lebesgue point of $f$.
    Furthermore, the triangle inequality implies thatif $x$ is a Lebesgue point of $f$, then
    \[ \lim_{\substack{ \vol B \to 0 \\ x\in B }} \frac{1}{\vol B} \int_B f(y) \,\dif y = f(x). \]
\end{remark}

\begin{corollary}
    \label{cor:lebesgue_points_almost_everywhere}
    Let $f \in L^1_{\text{loc}}(\R^n)$.
    Then almost every point $x \in \R^n$ is a Lebesgue point of $f$.
\end{corollary}
\begin{proof}
    For each rational number $ r \in \Q $, consider the function $g_r := |f - r|$.
    Then $g_r \in L^1_{\text{loc}}(\R^n)$, so by Theorem \ref{thm:lebesgue_differentiation_theorem}, there exists a set $N_r \subseteq \R^n$ of measure zero such that for each $x \in \R^n \setminus N_r$, we have
    \[ \lim_{\substack{ \vol B \to 0 \\ x\in B }} \frac{1}{\vol B} \int_B g_r(y) \,\dif y = g_r(x) = |f(x) - r|. \]
    Let $N := \bigcup_{r \in \Q} N_r$ which has measure zero since $\Q$ is countable.

    Now let $x \in \R^n \setminus N$ be arbitrary and assume that $f(x)$ is finite.
    (Note that by Proposition \ref{prop:maximal_function_properties}(ii) and \ref{cor:maximal_function_bounds_function}, $f(x)$ is finite for almost every $x \in \R^n$.)
    Let $\epsilon > 0$ be arbitrary, and choose $r \in \Q$ such that $|f(x) - r| < \epsilon/2$.
    Because 
    \[ \frac{1}{\vol B} \int_B | f(y) - f(x) | \,\dif y \leq \frac{1}{\vol B} \int_B | f(y) - r | \,\dif y + |f(x) - r| \]
    taking the limit superior as $\vol B \to 0$ and $x \in B$, we have
    \[ \limsup_{\substack{ \vol B \to 0 \\ x\in B }} \frac{1}{\vol B} \int_B | f(y) - f(x) | \,\dif y \leq \lim_{\substack{ \vol B \to 0 \\ x\in B }} \frac{1}{\vol B} \int_B | f(y) - r | \,\dif y + |f(x) - r| = 2|f(x) - r| < \epsilon \]
    since $x \notin N_r$.
    Since $\epsilon > 0$ was arbitrary, we conclude that
    \[ \lim_{\substack{ \vol B \to 0 \\ x\in B }} \frac{1}{\vol B} \int_B | f(y) - f(x) | \,\dif y = 0, \]
    so $x$ is a Lebesgue point of $f$.
\end{proof}

\begin{remark}[A Precise Representative]
    \label{rem:precise_representative}
    Let $f : \R^n \to \R$ be a locally integrable function.
    By Corollary \ref{cor:lebesgue_points_almost_everywhere}, almost every point $x \in \R^n$ is a Lebesgue point of $f$.
    Thus, we can define a new function $f^* : \R^n \to \R$ by
    \[ f^*(x) := \begin{cases} \lim_{\substack{ \vol B \to 0 \\ x\in B }} \frac{1}{\vol B} \int_B f(y) \,\dif y & \text{if } x \text{ is a Lebesgue point of } f, \\ 0 & \text{otherwise}. \end{cases} \]
    Then $f^*$ is equal to $f$ almost everywhere, and has the property that
    \[ \lim_{\substack{ \vol B \to 0 \\ x\in B }} \frac{1}{\vol B} \int_B f^*(y) \,\dif y = f^*(x) \quad \text{for every } x \in \R^n. \]
    We call $f^*$ the \textit{precise representative} of $f$.

    Recall that $L^1_{\text{loc}}(\R^n)$ is defined as the set of equivalence classes of functions that are equal almost everywhere.
    Thus, if $f,g$ are two representatives of the same equivalence class in $L^1_{\text{loc}}(\R^n)$, then their precise representatives $f^*,g^*$ are equal everywhere.
    In particular, the precise representative for an equivalence class in $L^1_{\text{loc}}(\R^n)$ is unique.
    This is nice, as there is some ambiguity in working with equivalence classes of functions, and a canonical representative removes this ambiguity.
    Thus, if we ever need to evaluate things pointwise, we can always use the precise representative.
\end{remark}

So far the theory has been developed for averaging over balls. 
One can generalize this of course, to other families of sets. 
One way to capture this is the following definition.

\begin{definition}
    \label{def:shrink_regularly}
    A collection of sets $\{ U_i \}_{i \in I}$ in $\R^n$ is said to \textit{shrink regularly} to a point $x \in \R^n$ if there exists a constant $c > 0$ such that for each $i \in I$, there exists a ball $B_i$ containing $x$ with
    \[ U_i \subseteq B_i \quad \text{and} \quad \vol B_i \leq c \cdot \vol U_i. \]
\end{definition}

Thus the sets $U_i$ are all contained in balls that are not too much larger in volume.
For instance, if $x\in \R^n$ then the collection of cubes centered at $x$ shrinks regularly to $x$.
With this definition, we can generalize the Lebesgue Differentiation Theorem as follows.

\begin{corollary}
    \label{cor:lebesgue_differentiation_theorem_shrink_regularly}
    Let $f \in L^1_{\text{loc}}(\R^n)$, and let $\{ U_i \}_{i \in I}$ be a collection of sets that shrink regularly to $x \in \R^n$.
    If $x$ is a Lebesgue point of $f$, then
    \[ \lim_{\substack{ \mathcal{L}^n(U_i) \to 0 \\ x\in U_i }} \frac{1}{\mathcal{L}^n(U_i)} \int_{U_i} f(y) \,\dif y = f(x). \]
\end{corollary}
\begin{proof}
    Let $x \in \R^n$ be a Lebesgue point of $f$.
    Since $\{ U_i \}_{i \in I}$ shrinks regularly to $x$, there exists a constant $c > 0$ such that for each $i \in I$, there exists a ball $B_i$ containing $x$ with
    \[ U_i \subseteq B_i \quad \text{and} \quad \vol B_i \leq c \cdot \vol U_i. \]
    Then
    \[ \frac{1}{\mathcal{L}^n(U_i)} \int_{U_i} |f(y) - f(x)| \,\dif y \leq \frac{1}{c\,\vol B_i} \int_{B_i} |f(y) - f(x)| \,\dif y. \]
    Taking the limit as $\vol B_i \to 0$, the right-hand side goes to zero since $x$ is a Lebesgue point of $f$.
    Thus,
    \[ \lim_{\substack{ \mathcal{L}^n(U_i) \to 0 \\ x\in U_i }} \frac{1}{\mathcal{L}^n(U_i)} \int_{U_i} |f(y) - f(x)| \,\dif y = 0, \]
    which implies the result.
\end{proof}