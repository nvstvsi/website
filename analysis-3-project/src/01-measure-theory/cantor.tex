\section{Cantor Sets}

\subsection{The Standard Cantor Set and Fat Cantor Sets}

The \textbf{standard Cantor set} $C$ is constructed by starting with the closed interval $[0,1]$ and iteratively removing the open middle third of each remaining interval.
The resulting set $C$ is compact, nonempty, perfect, totally disconnected, and has Lebesgue measure zero.

\begin{exercise}[Cantor Set Properties]
    \label{ex:properties_of_cantor_set}
    Prove the Cantor set $C$ is compact, nonempty, perfect, totally disconnected, and has uncountably many points.

    Also show that $C$ is nowhere dense in $[0,1]$, and thus contains no intervals.
\end{exercise}

The fact that the Cantor set has Lebesgue measure zero was shown in Example \ref{ex:cantor_set}.

\begin{proof}
    By definition, the Cantor set $C$ is the intersection of a decreasing sequence of nonempty closed sets, so $C$ is itself a closed set.
    Since $C$ is a closed subset of the compact set $[0,1]$, it follows that $C$ is compact.

    To see that $C$ is nonempty, notice that if $n\geq 0$, then $0$ and $1$ are in the set $C_n$ obtained after $n$ steps of the construction, so $0$ and $1$ are in $C$.
    More generally, for each $n\geq 0$, the endpoints of each of the $2^n$ intervals in $C_n$ are in $[0,1], C_1, \ldots, C_n$, because they have not been removed in any of the first $n$ steps of the construction;
    because we are only removing the middle thirds open interval, these endpoints are never removed in any subsequent steps of the construction, so they are all in $C$.
    That is, if $[a_{n,k}, b_{n,k}]$ is an interval in $C_n$ for some $n\geq 0$ and some $k\in\{1,\ldots,2^n\}$, then $a_{n,k}, b_{n,k} \in C$.
    Thus we have shown that $C$ actually contains at least countably many points.

    Let $E$ be the set of all endpoints of the intervals in $C_n$ for all $n\geq 0$; that is,
    \[ E = \bigcup_{n=0}^\infty \{ a_{n,k}, b_{n,k} : k=1,\ldots,2^n \}. \]

    To show that $C$ is perfect, let $x \in C$ and let $\epsilon > 0$.
    Fix $n\geq 0$ large enough that $3^{-n} < \epsilon$.
    Then $x\in C_n$, so $x$ is in one of the $2^n$ closed intervals in $C_n$, say $I_{n,k} := [a_{n,k}, b_{n,k}]$ for some $k\in\{1,\ldots,2^n\}$.
    Then the set $ E \cap I_{n,k} $ contains infinitely many points because infinitely many intervals are removed from $I_{n,k}$ in subsequent steps of the construction of the Cantor set, and the end points of each of these removed intervals are in $E \cap I_{n,k}$.
    By choice of $n$, we have $I_{n,k} \subseteq (x - \epsilon, x + \epsilon)$, so $E \cap (x - \epsilon, x + \epsilon)$ contains infinitely many points.
    Since $\epsilon > 0$ was arbitrary, we conclude that $x$ is a cluster point of $E$ --- recall this means each open set containing $x$ contains infinitely many points of $E$.
    Since $x\in C$ was arbitrary, we conclude that every point in $C$ is a cluster point of $E$, which means that $C$ is perfect.

    Finally to show that $C$ is totally disconnected, let $x\in C$ and let $\epsilon > 0$.
    Fix $n\geq 0$ large enough that $3^{-n} < \epsilon$.
    Then $x\in C_n$, so $x$ is in one of the $2^n$ closed intervals in $C_n$, say $I_{n,k} := [a_{n,k}, b_{n,k}]$ for some $k\in\{1,\ldots,2^n\}$.
    The compliment $C_n \setminus I_{n,k}$ is a union of finitely many closed intervals, so it is closed.
    Thus $I_{n,k}$ is open in the relative topology on $C_n$, as its compliment in $C_n$ is closed.
    As a result, the set $C \cap I_{n,k}$ is both open and closed in the relative topology on $C$.
    That is, $C\cap I_{n,k}$ is a clopen set in $C$ which is contained in the open interval $(x - \epsilon, x + \epsilon)$.
    Since $\epsilon > 0$ and $x\in C$ were arbitrary, we conclude that $C$ is totally disconnected.

    Since $C$ is compact, it must be complete. Since every complete, nonempty, perfect metric space is uncountable, we conclude that $C$ is uncountable.

    \vspace{2mm}

    To see that $C$ contains no intervals, suppose to the contrary that $C$ contains an interval $(a,b)$.
    Then $(a,b) \subseteq C_n$ for all $n\geq 0$. Let $N\geq 0$ be large enough that $3^{-N} < b-a$.
    Then $(a,b)$ cannot be contained in any of the $2^N$ closed intervals in $C_N$, each of which has length $3^{-N}$, a contradiction.
    Thus $C$ contains no intervals.

    To see that $C$ is nowhere dense in $[0,1]$, suppose that $U$ is a nonempty open subset of $[0,1]$ such that $C$ is dense in $U$.
    That is, we have $\overline{ C \cap U} \supseteq U$. Then $U$ contains an interval $(a,b)$, and 
    \[ C = \overline{C} \supseteq \overline{ C\cap U } \supseteq U \supseteq (a,b)  \]
    which contradicts the fact that $C$ contains no intervals.
    Thus $C$ is nowhere dense in $[0,1]$.
\end{proof}

We can do this construction in a slightly different way,  to obtain a \textbf{fat Cantor set}, which is like the standard Cantor set but has positive Lebesgue measure.
Let's give an example. 

\begin{example}[Fat Cantor Set]
    \label{ex:fat_cantor_set}
Start with the interval $[0,1]$ and remove the open middle $1/4$ from it, leaving the two closed intervals $[0,3/8]$ and $[5/8,1]$, each of length $3/8$.
Call the union of these two intervals $C_1$.
Next, remove the open middle $1/16$ from each of these two intervals, leaving four closed intervals, each of length $3/16$.
Call the union of these four intervals $C_2$.
Continue this process indefinitely, removing the open middle $1/4^n$ from each of the $2^n$ intervals in $C_n$ to form $C_{n+1}$.
The resulting set
\[ C^{\text{fat}} = \bigcap_{n=0}^\infty C_n \]
is a fat Cantor set.
The set $C^{\text{fat}}$ is closed as it is an intersection of closed sets, and it is totally disconnected (it contains no intervals) by construction.
Moreover, the Lebesgue measure of $C^{\text{fat}}$ is positive.
To see this, note that the total length of the intervals removed at each step is
\[ \sum_{n=1}^\infty 2^{n-1} \cdot \frac{1}{4^n} = \sum_{n=1}^\infty \frac{1}{2^{n+1}} = \frac{1}{2}, \]
so the Lebesgue measure of $C^{\text{fat}}$ is
\[ \mathcal{L}^1(C^{\text{fat}}) = 1 - \frac{1}{2} = \frac{1}{2}. \]
\end{example}

In general, we can construct a fat Cantor set with any desired Lebesgue measure in the interval $(0,1)$ by adjusting the lengths of the intervals removed at each step.

\begin{definition}[Fat Cantor Set]
    \label{def:fat_cantor_set}
    Let $\{a_n\}_{n=1}^\infty$ be a sequence of positive numbers such that $\sum_{n=1}^\infty a_n < 1$.
    A \textbf{fat Cantor set} $F$ can be constructed by starting with the interval $F_0 = [0,1]$, letting $F_1$ be the set obtained by removing the open middle interval of length $a_1$ from $F_0$, and letting $F_2$ be the set obtained by removing the open middle intervals of length $a_2$ from each of the two intervals in $F_1$, and so on.
    For $n \geq 1$, the set $F_n$ is obtained by removing the open middle intervals of length $a_n$ from each of the $2^{n-1}$ intervals in $F_{n-1}$; by induction, $F_n$ consists of $2^n$ closed intervals of total length $1 - \sum_{k=1}^n a_k$.
    Then the fat Cantor set $F$ is defined as
    \[ F = \bigcap_{n=0}^\infty F_n. \]
\end{definition}

By construction, a fat Cantor set $F$ is closed.
Moreover, the Lebesgue measure of $F$ is
\[ \mathcal{L}^1(F) = 1 - \sum_{n=1}^\infty a_n, \]
which is positive by our assumption on the sequence $\{a_n\}_{n=1}^\infty$.

These sets are just like the standard Cantor set in many ways.

\begin{exercise}
    \label{ex:fat_cantor_set_properties}
    Prove that a fat Cantor set $F$ is compact, nonempty, perfect, totally disconnected, has uncountably many points, contains no intervals, and is nowhere dense in $[0,1]$.
\end{exercise}
\begin{proof}
    The set $F$ is closed as it is an intersection of closed sets, and since $F$ is a closed subset of the compact set $[0,1]$, it follows that $F$ is compact.

    To see that $F$ is nonempty, notice that if $n\geq 0$, then $0$ and $1$ are in the set $F_n$ obtained after $n$ steps of the construction, so $0$ and $1$ are in $F$.
    More generally, for each $n\geq 0$, the endpoints of each of the $2^n$ intervals in $F_n$ are in $[0,1], F_1, \ldots, F_n$, because they have not been removed in any of the first $n$ steps of the construction;
    because we are only removing open middle intervals, these endpoints are never removed in any subsequent steps of the construction, so they are all in $F$.
    That is, if $[a_{n,k}, b_{n,k}]$ is an interval in $F_n$ for some $n\geq 0$ and some $k\in\{1,\ldots,2^n\}$, then $a_{n,k}, b_{n,k} \in F$.
    Thus we have shown that $F$ actually contains at least countably many points.
    
    We show that $F$ is perfect, meaning every point of $F$ is a limit point of $F$.
    Let $x \in F$ and let $\epsilon > 0$.     
    Then for each $n\geq 0$ the set $F_n$ consists of $2^n$ closed intervals of length $(1 - \sum_{k=1}^n a_k)/2^n$, and since $\sum_{k=1}^\infty a_k$ converging to a number less than $1$, it follows that $(1 - \sum_{k=1}^n a_k)/2^n \to 0$ as $n \to \infty$.
    As a result, we can fix $N \geq 0$ so that $(1 - \sum_{k=1}^N a_k)/2^N < \epsilon$.
    That is, the length of each of the $2^N$ intervals in $F_N$ is $<\epsilon$.
    Since $x \in F \subseteq F_N§$, the point $x$ lies in one of the $2^n$ intervals in $F_n$.
    Since this interval has length less than $\epsilon$, both of its endpoints are in $F$ (as shown above) and lie within $\epsilon$ of $x$.
    But this interval also has infinitely many other endpoints of intervals removed in subsequent steps of the construction of $F$, and these endpoints are also in $F$ and lie within $\epsilon$ of $x$.
    In other words, the set $F \cap (x - \epsilon, x + \epsilon)$ contains infinitely many points.
    Since $\epsilon > 0$ was arbitrary, we conclude that $x$ is a cluster point of $F$.
    Since $x\in F$ was arbitrary, we conclude that every point in $F$ is a cluster point of $F$, which means that $F$ is perfect.

    Since $F$ is compact, it must be complete. Since every complete, nonempty, perfect metric space is uncountable, we conclude that $F$ is uncountable.

    To see that $F$ is totally disconnected, let $x\in F$ and let $\epsilon > 0$.
    Again fix $N \geq 0$ so that each interval in $F_N$ has $<\epsilon$.
    Then $x\in F_N$, so $x$ is in one of the $2^N$ closed intervals in $F_N$, say $I_{N,k} := [a_{N,k}, b_{N,k}]$ for some $k\in\{1,\ldots,2^N\}$.
    The compliment $F_N \setminus I_{N,k}$ is a union of finitely many closed intervals, so it is closed.
    Thus $I_{N,k}$ is open in the relative topology on $F_N$, as its compliment in $F_N$ is closed.
    As a result, the set $F \cap I_{N,k}$ is both open and closed in the relative topology on $F$.
    That is, $F\cap I_{N,k}$ is a clopen set in $F$ which is contained in the open interval $(x - \epsilon, x + \epsilon)$.
    Since $\epsilon > 0$ and $x\in F$ were arbitrary, we conclude that $F$ is totally disconnected.

    Finally, to see that $F$ contains no intervals, suppose to the contrary that $F$ contains an interval $(a,b)$.
    Then $(a,b) \subseteq F_n$ for all $n\geq 0$. Let $N\geq 0$ be large enough that each interval in $F_N$ has length $<b-a$.
    Then $(a,b)$ cannot be contained in any of the $2^N$ closed intervals in $F_N$, each of which has length $<b-a$, a contradiction.
    Thus $F$ contains no intervals.

    To see that $F$ is nowhere dense in $[0,1]$, suppose that $U$ is a nonempty open subset of $[0,1]$ such that $F$ is dense in $U$.
    That is, we have $\overline{ F \cap U} \supseteq U$. Then $U$ contains an interval $(a,b)$, and since $F$ contains no intervals, this is a contradiction.
    Thus $F$ is nowhere dense in $[0,1]$.
\end{proof}

\subsection{The Devil's Staircase}

We can also use the Cantor set to construct a pretty wild function.

\begin{proposition}
    \label{prop:singular_function}
    There exist continuous increasing functions $f : [0,1] \to [0,1]$ such that $f(0) = 0$ and $f(1) = 1$, but $f'(x) = 0$ for almost every $x\in[0,1]$.
\end{proposition}

There are actually many such functions, but one of the most famous is the \textbf{Devil's Staircase}, which is constructed using the standard Cantor set.
Another famous example which we do not explore here is the Minkowski question mark function.

\begin{proof}[Construction of the Devil's Staircase]
    \label{ex:devils_staircase}
Let $C = \bigcup_{n=0}^\infty C_n$ be the standard Cantor set, where $C_n$ is the union of $2^n$ closed intervals of length $3^{-n}$ for each $n\geq 0$.

Define a sequence of functions $\{f_k\}_{k=0}^\infty$ as follows:
Let $f_0(x) = x$ for all $x\in[0,1]$.
For each $k \geq 1$, define a function $f_k$ by
\[ f_k(x) = \begin{cases}
    f_{k-1}(3x)/2 & \text{if } x \in [0,1/3], \\
    1/2 & \text{if } x \in (1/3,2/3), \\
    1/2 + f_{k-1}(3x - 2)/2 & \text{if } x \in [2/3,1].
\end{cases} \]
In other words, $f_k$ is obtained from $f_{k-1}$ by compressing the graph of $f_{k-1}$ horizontally by a factor of $3$, compressing it vertically by a factor of $2$, and placing two copies of this compressed graph on the left and right thirds of the interval $[0,1]$, with a flat segment at height $1/2$ in the middle third.

We claim that for each $k \geq 0$, the function $f_k$ is continuous, increasing, $f_k(0) = 0$ and $f_k(1) = 1$, and thus maps $[0,1]$ onto $[0,1]$.

\begin{proof}[Proof of Claim]
We can prove this by induction on $k$.
Since 
\[ \lim_{ x\to \frac{1}{3}^- } f_0(x) = \frac{1}{2}f_0(1) = \frac{1}{2} \] 
and
\[ \lim_{ x\to \frac{2}{3}^+ } f_0(x) = \frac{1}{2} + \frac{1}{2}f_0(0) = \frac{1}{2}, \]
we see that $f_1$ is continuous.
Since $f_0$ is increasing, we have $f_1(x) \leq f_1(y)$ for all $x,y \in [0,1]$ with $x < y$.
Moreover, $f_1(0) = 0$ and $f_1(1) = 1$, so $f_1$ maps $[0,1]$ onto $[0,1]$.

Now suppose that for some $k\geq 1$, the function $f_k$ is continuous, increasing, and $f_k(0) = 0$ and $f_k(1) = 1$.
Then we see that
\[ \lim_{ x\to \frac{1}{3}^- } f_{k+1}(x) = \frac{1}{2}f_k(1) = \frac{1}{2} \]
since $f_k(1) = 1$ by the induction hypothesis, and
\[ \lim_{ x\to \frac{2}{3}^+ } f_{k+1}(x) = \frac{1}{2} + \frac{1}{2}f_k(0) = \frac{1}{2} \]
since $f_k(0) = 0$ by the induction hypothesis.
Thus $f_{k+1}$ is continuous.
Since $f_k$ is increasing, we have $f_{k+1}(x) \leq f_{k+1}(y)$ for all $x,y \in [0,1]$ with $x < y$.
Moreover, $f_{k+1}(0) = 0$ and $f_{k+1}(1) = 1$ by the induction hypothesis, so $f_{k+1}$ maps $[0,1]$ onto $[0,1]$.
By induction, we conclude that for each $k\geq 0$, the function $f_k$ is continuous, increasing, and maps $[0,1]$ onto $[0,1]$.
\end{proof}

Now we claim that the sequence of functions $\{f_k\}_{k=0}^\infty$ satisfies
\[ |f_k(x) - f_{k-1}(x)| \leq 2^{-k} \quad \forall x\in[0,1], \ k\geq 1. \]

\begin{proof}[Proof of Claim]
To see this, again we use induction on $k$.

For $k=1$, we have
\[ |f_1(x) - f_0(x)| = \begin{cases}
    |f_0(3x)/2 - x| & \text{if } 0 \leq x \leq 1/3, \\
    |1/2 - x| & \text{if } 1/3 < x < 2/3, \\
    |1/2 + f_0(3x - 2)/2 - x| & \text{if } 2/3 \leq x \leq 1.
\end{cases} \]
If $0 \leq x \leq \frac{1}{3}$, then
\[ |f_1(x) - f_0(x)| = \left|\frac{3x}{2} - x\right| = \frac{x}{2} \leq \frac{1}{6} < \frac{1}{2}. \]
If $\frac{1}{3} < x < \frac{2}{3}$, then
\[ |f_1(x) - f_0(x)| = |1/2 - x| \leq \max\{ |1/2 - 1/3|, |1/2 - 2/3| \} = 1/6 < 1/2. \]
If $\frac{2}{3} \leq x \leq 1$, then
\[ |f_1(x) - f_0(x)| = \left| \frac{1}{2} + \frac{3x - 2}{2} - x \right| = \left| \frac{1}{2} - \frac{x}{2} \right| \leq 1 - \frac{1}{2}. \] 
Thus for all $x\in[0,1]$, we have
\[ |f_1(x) - f_0(x)| \leq \frac{1}{2} = 2^{-1}. \]
This establishes the base case.

Now suppose that for some $k\geq 1$, we have
\[ |f_k(x) - f_{k-1}(x)| \leq 2^{-k} \quad \forall x\in[0,1]. \]

If $\frac{1}{3} < x < \frac{2}{3}$, then
\[ |f_{k+1}(x) - f_k(x)| = \left|\frac{1}{2} - \frac{1}{2} \right| = 0 \]

If $0 \leq x \leq \frac{1}{3}$, then
\[ |f_{k+1}(x) - f_k(x)| = \left|\frac{1}{2}f_k(3x) - f_k(x)\right| = \frac{1}{2} \left| f_k(3x) - f_{k-1}(3x) \right| \leq \frac{1}{2} \cdot 2^{-k} = 2^{-(k+1)}. \]
If $\frac{1}{3} \leq x \leq \frac{2}{3}$, then 
\[ |f_{k+1}(x) - f_k(x)| = \left|\frac{1}{2} - \frac{1}{2} \right| = 0. \]
If $\frac{2}{3} \leq x \leq 1$, then
\[ |f_{k+1}(x) - f_k(x)| = \left| \frac{1}{2} + \frac{1}{2}f_k(3x - 2) - f_k(x) \right| = \frac{1}{2} \left| f_k(3x - 2) - f_{k-1}(3x - 2) \right| \leq \frac{1}{2} \cdot 2^{-k} = 2^{-(k+1)}. \]
Thus for all $x\in[0,1]$, we have
\[ |f_{k+1}(x) - f_k(x)| \leq 2^{-(k+1)}. \]
By induction, we conclude that for all $k\geq 1$ and all $x\in[0,1]$, we have
\[ |f_k(x) - f_{k-1}(x)| \leq 2^{-k}. \]
\end{proof}

In other words, for each $k\geq 1$ we have
\[ \|f_k - f_{k-1}\|_\infty = \sup_{x\in[0,1]} |f_k(x) - f_{k-1}(x)| \leq 2^{-k}. \]
By Exercise \ref{ex:sufficient_condition_cauchy_sequence}, this implies that the sequence of functions $\{f_k\}_{k=0}^\infty$ is a Cauchy sequence in $C^0([0,1])$ with the supremum norm $\|\cdot\|_\infty$.
Since $C^0([0,1])$ is complete with respect to the supremum norm, there exists a continuous function $f:[0,1] \to \R$ such that
\[ \|f_k - f\|_\infty \to 0 \quad \text{as } k \to \infty. \]
In other words, the sequence $\{f_k\}_{k=0}^\infty$ converges uniformly to $f$ on $[0,1]$.

The function $f$ is called the \textbf{Devil's staircase} or the \textbf{Cantor function}.

By construction, the function $f$ is continuous.
We claim that $f$ is increasing and that $f(0) = 0$ and $f(1) = 1$, so $f$ maps $[0,1]$ onto $[0,1]$.
To see this, let $x,y \in [0,1]$ with $x < y$.
Since each $f_k$ is increasing, we have $f_k(x) \leq f_k(y)$ for all $k\geq 0$.
Taking the limit as $k \to \infty$, we get
\[ f(x) = \lim_{k\to\infty} f_k(x) \leq \lim_{k\to\infty} f_k(y) = f(y). \]
Thus $f$ is increasing.
Moreover, since $f_k(0) = 0$ and $f_k(1) = 1$ for all $k\geq 0$, we have
\[ f(0) = \lim_{k\to\infty} f_k(0) = 0 \quad\text{and}\quad f(1) = \lim_{k\to\infty} f_k(1) = 1. \]
Thus $f$ maps $[0,1]$ onto $[0,1]$.

Our final claim is that the function $f$ is differentiable at each point in $[0,1] \setminus C$ with derivative zero, and that $f$ is not differentiable at any point in the Cantor set $C$.
To see this, let $x \in [0,1] \setminus C$.
Then $x$ is in the interior of one of the intervals removed in the construction of $C$, say at the $N$-th step; for each $k \geq N$, the function $f_k$ is constant on this interval, so $f$ is also constant on this interval.
Thus $f$ is differentiable at $x$ with derivative zero.
That is, the derivative of $f$ exists and is equal to zero at each point in $[0,1] \setminus C$.
\end{proof}

\begin{figure}
    \centering
    \includegraphics[width = 0.5\textwidth]{figures/devil-step.png}
    \caption{The Devil's Staircase}
    \label{fig:devil_staircase}
\end{figure}

\begin{exercise}[Sufficient Condition for Cauchy Sequence]
    \label{ex:sufficient_condition_cauchy_sequence}
    Let $(X,d)$ be a metric space, let $\{ x_k \}_{k=1}^\infty$ be a sequence of points in $X$, and let $\{ r_k \}_{k=1}^\infty$ be a sequence of positive real numbers such that $\sum_{k=1}^\infty r_k < \infty$.
    If \[ d(x_{k+1}, x_k) \leq r_k \quad \forall k \in \mathbb{Z}^+, \]
    then $\{ x_n \}_{n=1}^\infty$ is a Cauchy sequence.
\end{exercise}

\begin{proof}
    Let $\epsilon > 0$.
    Since $\sum_{k=1}^\infty r_k < \infty$, there exists $N \in \mathbb{Z}^+$ such that
    \[ \sum_{k=N+1}^\infty r_k < \epsilon. \]
    Then for all $m\geq n > N$, we have
    \[ d(x_m, x_n) \leq d(x_m, x_{m-1}) + d(x_{m-1}, x_{m-2}) + \cdots + d(x_{n+1}, x_n) \leq \sum_{k=n}^{m-1} r_k \leq \sum_{k=N+1}^\infty r_k < \epsilon. \]
    Thus $\{ x_n \}_{n=1}^\infty$ is a Cauchy sequence.
\end{proof}

\begin{exercise}[Symmetry of the Cantor Function]
    \label{ex:symmetry_of_cantor_function}
    Show that the Cantor function $f$ satisfies 
    \[ f(1-x) = 1 - f(x) \quad \forall x \in [0,1] \]
    and \[ f\left( \frac{x}{3} \right) = \frac{f(x)}{2} \quad \forall x \in [0,1] \]
    and \[ f\left( \frac{2+x}{3} \right) = \frac{1}{2} + \frac{f(x)}{2} \quad \forall x \in [0,1]. \]
\end{exercise}

\begin{proof}
 Let $x \in [0,1]$.
 
\textit{First identity:}
We claim that $f_k(1-x) = 1 - f_k(x)$ for all $k\geq 0$ and all $x\in[0,1]$.
We prove this by induction on $k$.

For $k=0$, we have
\[ f_0(1-x) = 1 - x = 1 - f_0(x) \]
for each $x\in[0,1]$.

Now suppose that for some $k\geq 0$, we have $f_k(1-x) = 1 - f_k(x)$ for each $x\in[0,1]$.
Then we show the same is true for $f_{k+1}$. We consider three cases.

If $0 \leq x \leq \frac{1}{3}$, then $1-x \in [\frac{2}{3},1]$ and
\begin{align*}
    f_{k+1}(1-x) &= \frac{1}{2} + \frac{1}{2}f_k(3(1-x) - 2) \\
        &= \frac{1}{2} + \frac{1}{2}f_k(1 - 3x) \\
        &= \frac{1}{2} + \frac{1}{2}(1 - f_k(3x)) \\
        &= 1 - \frac{1}{2}f_k(3x) \\
        &= 1 - f_{k+1}(x).
\end{align*}

If $\frac{1}{3} < x < \frac{2}{3}$, then $1-x \in (\frac{1}{3},\frac{2}{3})$ and
\[ f_{k+1}(1-x) = \frac{1}{2} = 1 - \frac{1}{2} = 1 - f_{k+1}(x). \]

If $\frac{2}{3} \leq x \leq 1$, then $1-x \in [0,\frac{1}{3}]$ and by using the first case we have
\[ f_{k+1}(x) = f_{k+1}(1 - (1-x)) = 1 - f_{k+1}(1-x) \]
which rearranges to
\[ f_{k+1}(1-x) = 1 - f_{k+1}(x). \]

By induction, we conclude that for all $k\geq 0$ and all $x\in[0,1]$, we have
\[ f_k(1-x) = 1 - f_k(x). \]

Taking the limit as $k \to \infty$, we get
\[ f(1-x) = \lim_{k\to\infty} f_k(1-x) = \lim_{k\to\infty} (1 - f_k(x)) = 1 - f(x) \]
for each $x\in[0,1]$, so the first identity holds.

\vspace{2mm}
\textit{Second Identity:}
Suppose that $x \in [0,1]$.
Then $\frac{x}{3}$ so we have
\[ f\left( \frac{x}{3} \right) = \lim_{k\to\infty} f_k\left( \frac{x}{3} \right) = \lim_{k\to\infty} \frac{f_{k-1}(x)}{2} = \frac{f(x)}{2}. \]
Since $x\in[0,1]$ was arbitrary, the second identity holds.

\vspace{2mm}
\textit{Third Identity:}
Suppose that $x \in [0,1]$.
Then $\frac{2+x}{3} \in [\frac{2}{3},1]$ so we have
\begin{align*}
    f\left( \frac{2+x}{3} \right) &= \lim_{k\to\infty} f_k\left( \frac{2+x}{3} \right) \\
        &= \lim_{k\to\infty} \left( \frac{1}{2} + \frac{1}{2}f_{k-1}(x) \right) \\
        &= \frac{1}{2} + \frac{{f(x)}}{2}.
\end{align*}
Since $x\in[0,1]$ was arbitrary, the third identity holds.

\end{proof}